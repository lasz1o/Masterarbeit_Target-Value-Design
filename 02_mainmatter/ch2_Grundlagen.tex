\chapter{Grundlagen}
\label{ch:grundlagen}
% Ziel: Ziel: Der Leser soll die Struktur, die Rollenverteilung und die prozessualen Schwachstellen der traditionellen Projektabwicklung verstehen. Dieses Kapitel etabliert das "Problem", für das Ihre Arbeit eine Lösung präsentiert.

\section{Konventionelle Projektabwicklung in der deutschen Baupraxis}
\label{sec: 2.1}
Ziel:

\clearpage

\section{Integrierte Projektabwicklungsmodelle als alternativer Ansatz}
\label{sec: 2.2}

% Ziel: Der Leser soll eine Einführung in die Entstehung und Entwicklung integrierter Projekabwicklung bekommen.

\subsection{Der Lean-Ansatz als Fundament integrierter Projektabwicklung}
\label{sec:2.2.1}

% Ziel: Vermittlung der Lean-Pilosophie als Ursprungsgedanke und der Weiterentwicklung über Lean Construction zu kollaborativen Projekabwicklungsmodellen

\textit{Einleitung - Hinführung zur Notwendigkeit neuer Ansätze nach der Kritik an der konventionellen Projektabwicklung (aus Abschnitt 2.1). Die Philosophie des Lean Managements bietet hierfür das Fundament.}

Die im vorangegangenen Abschnitt aufgezeigten Schwachstellen konventioneller, sequenzieller Projektabwicklung sind keineswegs ein neues Phänomen oder eine reine Eigenheit der Bauwirtschaft. Vergleichbare Herausforderungen zeigten sich bereits nach dem Zweiten Weltkrieg in der japanischen Automobilindustrie. Die dort vorherrschenden Produktionsmethoden waren von Ineffizienz und Verschwendung geprägt, was angesichts knapper Ressourcen nicht tragbar war. Als Reaktion darauf entwickelte der Ingenieur Taiichi Ohno bei Toyota einen grundlegend neuen Managementansatz, der heute als das Toyota-Produktionssystem bekannt ist und das Fundament der Lean-Philosophie bildet.

Ursprung des Ansatzes (Lean Management) Skizzieren
- Toyota-Produktionssystem\autocite[]{ohno_toyota-produktionssystem_2013} als Ursprung
- Kernideen und Prinzipien : Wertmaximierung, Verschwendungsreduktion, People First (checken), Kaizen

Transfer auf die Baubranche durch Lean Construction
- Landesweite Aufmerksamkeit in Japan nach der Ölkrise 1973
- Während sich die 
- Internationale Aufmerksamkeit durch 

Ursprung und Entwicklung von Projetct Alliancing, IPD und IPA bis heute

\subsection{Prinzipien und Merkmale der Integrierten Projektabwicklung (IPA)}
\label{sec:2.2.2}

\subsection{Internationale Erfolge und Status Quo in Deutschland}
\label{sec:2.2.3}

\clearpage

\section{Target Value Design (TVD) als Kernmethode}
\label{sec: 2.3}

%Ziel:

\subsection{TVD - Prozess oder Methode}
\label{sec: 2.3.1}

% =======================================================
% FÜR ABSCHNITT 2.3.1: TVD - Prozess oder Methode
% =======================================================
%https://service.tu-dortmund.de/group/intra/ausstattung-lokaler-arbeitsplatze
%   - Ziel: Eine präzise und trennscharfe Definition von TVD für die
%     vorliegende Arbeit herleiten. Es soll geklärt werden, ob TVD als
%     Prozess, als Methode oder als beides zu verstehen ist, um eine
%     eindeutige Basis für die nachfolgende Analyse zu schaffen.
%
%   - Einstieg: Feststellung, dass die Begriffe "Prozess" und "Methode"
%     in der Literatur zu TVD oft unscharf oder synonym verwendet werden.
%
%   - Begriffsdefinition "Methode": Ein systematisches, geplantes Vorgehen,
%     das auf Prinzipien, Regeln und Werkzeugen basiert, um ein Ziel zu erreichen.
%
%   - Begriffsdefinition "Prozess": Eine logische und zeitliche Abfolge von
%     miteinander verknüpften Aktivitäten zur Umwandlung eines Inputs in einen Output.
%
%   - Anwendung auf TVD:
%     - TVD ist mehr als ein reiner Prozess, da es eine bestimmte Denkweise
%       ("Target First"), Prinzipien (Kollaboration) und Werkzeuge voraussetzt.
%     - Diese übergeordnete Denkweise wird jedoch erst durch einen konkreten,
%       strukturierten Ablauf (Prozess) in der Praxis anwendbar.
%
%   - Fazit & Positionierung: Für diese Arbeit wird TVD als eine
%     *MANAGEMENT-METHODE* verstanden, die durch einen
%     _STRUKTURIERTEN PROZESS_ operationalisiert und umgesetzt wird.
%     Diese Definition legitimiert die nachfolgende "prozessorientierte
%     Darstellung" der Methode.
%
% =======================================================

Zur Hinleitung auf die in \cref{ch:methodik} beschriebene, methodische Vorgehensweise soll an dieser Stelle auf das in dieser Arbeit zu Grunde liegende Verständnis von \ac{TVD} als Prozess bzw. Methode eingegangen werden.\\
In der Literatur ist die Abgrenzung zwischen Prozess und Methode in der aktuellen Diskussion um TVD unscharf. Folgt man dem Ergebnis einer einfachen Stichwortsuche in einschlägigen wissenschaftlichen Datenbanken (z.B. reseachgate.net) so wird schnell deutlich, das in der überwiegenden Mehrheit der Literatur (ca. 70\%) Target Value Design als Prozess verstanden wird.
Glenn Ballard, der als einer Gründer des \ac{TVD} gilt, bezeichnet letzteres hingegen häufig als Managementansatz bzw. Management-Methode \autocite[]{}.

In der allgemeinen Literatur wird ein Prozess als 