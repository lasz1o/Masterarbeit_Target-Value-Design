\chapter{Grundlagen}
\label{ch:grundlagen}

Das folgende Kapitel legt das theoretische Fundament für die Untersuchung einer wertorientierten Projektsteuerung im Bauwesen. Um die Funktionsweise und den Mehrwert des \textit{Target Value Designs} (TVD) im weiteren Verlauf der Arbeit bewerten zu können, ist ein dreistufiges Verständnis der theoretischen Zusammenhänge erforderlich. Zunächst werden die konventionellen Strukturen der deutschen Baupraxis analysiert, um die systemimmanenten Defizite aufzuzeigen. Darauf aufbauend werden die \textit{Integrierte Projektabwicklung} (IPA) als organisatorischer Rahmen und schließlich das TVD als methodisches Instrument zur aktiven Steuerung von Kosten und Werten eingeführt.

\section{Konventionelle Projektabwicklung in Deutschland}
\label{sec:2.1}

Die deutsche Bauwirtschaft ist durch tief verwurzelte Traditionen und eine hohe regulatorische Dichte geprägt, die das Bild der Projektabwicklung seit Jahrzehnten bestimmen. Bevor alternative Steuerungsansätze diskutiert werden können, muss dargelegt werden, warum die etablierten Mechanismen bei zunehmender Projektkomplexität an ihre Grenzen stoßen. Der folgende Abschnitt dekonstruiert die konventionelle Praxis, indem er die Rollenbilder der Akteure, die prozessualen Fesseln der Honorar- und Vergabeordnungen sowie die daraus resultierenden ökonomischen Fehlanreize analysiert.

\subsection{Akteure, Rollenbilder und die Trennung von Planung und Bau}
\label{sec:2.1.1}

Die konventionelle Projektabwicklung in Deutschland ist historisch durch eine strikte Trennung von Planungs- und Ausführungsleistungen geprägt, ein Modell, das international als \textit{Design-Bid-Build} (DBB) bekannt ist. Diese Struktur basiert auf einem spezifischen Rollenverständnis der beteiligten Akteure, welches tief in den berufsrechtlichen und regulatorischen Rahmenbedingungen (insb. HOAI und AHO) verwurzelt ist.

\minisec{Das Treuhänder-Modell des Planers}
Zentrales Element der deutschen Planungskultur ist die treuhänderische Stellung des Architekten und der Ingenieure gegenüber dem Bauherrn. Gemäß dem Leitbild der Kammern agiert der Planer als unabhängiger Sachwalter des Auftraggebers, der dessen Interessen gegenüber den bauausführenden Unternehmen vertritt \autocite[vgl.]{Girmscheid2016}. Dieses Rollenmodell impliziert eine strikte Neutralität gegenüber den ausführenden Gewerken, was häufig dazu führt, dass eine frühzeitige Kooperation zwischen Planern und Bauunternehmen unterbleibt. Insbesondere im öffentlichen Vergaberecht wird die Trennung durch das Verbot der \enquote{Vorbefasstheit} zementiert, um Wettbewerbsverzerrungen zu vermeiden \autocite[siehe]{VOB_A}. Die Folge ist jedoch eine prozessuale Isolation der Planungsphase von den Realitäten der Bauausführung.

\minisec{Sequenzialität und die \enquote{V-Struktur} der Kommunikation}
Die organisatorische Struktur konventioneller Projekte folgt einer bilateralen Logik, die in der Literatur oft als \enquote{V-Struktur} bezeichnet wird. Der Bauherr schließt isolierte Einzelverträge mit den Planungsbeteiligten (LPH 1–4/5) und erst zu einem deutlich späteren Zeitpunkt mit den bauausführenden Unternehmen (LPH 6–9). Zwischen den beiden Gruppen – den Trägern des \enquote{Entwurfswissens} und den Trägern des \enquote{Fertigungswissens} – existiert keine vertragliche oder prozessuale Verbindung \autocite[vgl.]{Sommer2016}. 

\begin{figure}[htbp]
    \centering
    \fcolorbox{gray!50}{white}{%
        \includegraphics[width=\dimexpr\textwidth-2\fboxsep-2\fboxrule\relax, height=0.5\textwidth]{05_figures/V_Struktur_Silos.jpg}
    }
    \caption[V-Struktur der Kommunikation]{Bilaterale Vertrags- und Kommunikationsstrukturen (V-Struktur) in der konventionellen Projektabwicklung (Quelle: Eigene Darstellung in Anlehnung an Habib 2020).}
    \label{fig:v_struktur}
\end{figure}

Diese Fragmentierung führt zu einer sequenziellen Informationsweitergabe, bei der die Planung oft ohne Berücksichtigung spezifischer Bauverfahren, Logistikketten oder aktueller Marktkapazitäten der Unternehmen abgeschlossen wird. Habib beschreibt dieses Phänomen treffend als einen Paradigmenverlust des Wissens, da das Know-how derer, die das Projekt realisieren, in der Phase der größten Beeinflussbarkeit der Kosten (Entwurfsphase) faktisch ausgeschlossen ist \autocite[in Anlehnung an]{Habib2020}.

\minisec{Informationsasymmetrie und die \enquote{Knowledge Gap}}
Ein wesentliches Defizit dieser Rollentrennung ist die entstehende Informationsasymmetrie. Während der Planer die funktionale und gestalterische Qualität definiert, verfügt erst der Bauunternehmer über die detaillierte Kenntnis der realen Kostenstrukturen und Bauprozesse. In der konventionellen Abwicklung kollidieren diese Welten erst nach der Vergabe. Erweist sich die Planung als nicht baubar oder ökonomisch ineffizient, sind aufwendige Umplanungen (\textit{Rework}) die Folge, welche wiederum die Zeitpläne und Budgets belasten \autocite[vgl.]{Ballard2025}. Die MacLeamy-Kurve illustriert hierbei eindrücklich, dass die Fähigkeit zur Kostenbeeinflussung mit fortschreitender Projektdauer rapide sinkt, während die Kosten für Änderungen exponentiell ansteigen \autocite{MacLeamy2004}.

\begin{figure}[htbp]
    \centering
    \fcolorbox{gray!50}{white}{%
        \includegraphics[width=\dimexpr\textwidth-2\fboxsep-2\fboxrule\relax, height=0.5\textwidth]{05_figures/Knowledge_Gap.jpg}
    }
    \caption[Knowledge Gap]{Die Wissenslücke (\textit{Knowledge Gap}) zwischen Planungs- und Ausführungsexpertise im konventionellen Prozess (Quelle: Eigene Darstellung).}
    \label{fig:knowledge_gap}
\end{figure}

\minisec{Haftungssilos und Misstrauenskultur}
Die Fragmentierung wird durch das deutsche Haftungsrecht verstärkt. Da jeder Akteur primär für seinen abgegrenzten Leistungsbereich haftet, entsteht ein System von \enquote{Haftungssilos}. Anstatt Probleme interdisziplinär an den Schnittstellen zu lösen, konzentrieren sich die Akteure auf die eigene Risikoabsicherung und Exkulpanz \autocite[vgl.]{Kyrein2002}. In der Praxis manifestiert sich dies in einer Flut von Behinderungsanzeigen und Bedenkenanmeldungen nach VOB/B, die eher der Dokumentation des gegenseitigen Versagens als dem Projekterfolg dienen. Dieses feindselige Klima (\textit{Adversarial Relationship}) verhindert die für moderne Steuerungskonzepte wie das Target Value Design notwendige Radikale Transparenz und das gegenseitige Vertrauen.

Zusammenfassend lässt sich festhalten, dass die traditionellen Rollenbilder in Deutschland eine organisatorische Barriere bilden, die den Fluss von Informationen und Werten unterbricht. Diese systemische Trennung von Geist und Hand ist die primäre Ursache für die mangelnde Steuerbarkeit komplexer Bauvorhaben.

\subsection{ Die regulatorische Rahmenbedingungen HOAI und VOB}
\label{sec:2.1.2}

Während die Rollenbilder der Akteure das soziale Gefüge definieren, bilden die Honorarordnung für Architekten und Ingenieure (HOAI) sowie die Vergabe- und Vertragsordnung für Bauleistungen (VOB) das regulatorische Gerüst der konventionellen Projektabwicklung. Diese Regelwerke begünstigen eine sequenzielle Prozesslogik, die iterativen Optimierungsprozessen strukturell entgegensteht.

\minisec{Die HOAI als Prozessstandard}
Trotz der Novellierung im Jahr 2021, durch welche die verbindlichen Mindest- und Höchstsätze infolge der Rechtsprechung des Europäischen Gerichtshofes entfielen, bleibt die HOAI als \enquote{empfohlene} Honorarstruktur das maßgebliche Prozessmodell der deutschen Baupraxis. Die Unterteilung in neun Leistungsphasen (LPH) schafft eine starre zeitliche Abfolge, die in der Literatur häufig als Wasserfall-Modell charakterisiert wird \autocite[vgl.]{Girmscheid2016}.

\begin{figure}[htbp]
    \centering
    % \includegraphics[width=\textwidth]{Abbildung_HOAI_Phasen.pdf}
    \caption{Sequenzielle Abfolge der HOAI-Leistungsphasen (Eigene Darstellung)}
    \label{fig:hoai_phasen}
\end{figure}

Die Phasen lassen sich grob in drei Blöcke unterteilen:
\begin{itemize}
    \item \textbf{Planungsphase (LPH 1--4):} Von der Grundlagenermittlung bis zur Genehmigungsplanung wird das Projekt primär abstrakt und theoretisch definiert.
    \item \textbf{Vorbereitungsphase (LPH 5--7):} Die Ausführungsplanung sowie die Vorbereitung und Mitwirkung bei der Vergabe bilden die Schnittstelle zum Markt.
    \item \textbf{Realisierungsphase (LPH 8--9):} Die Objektüberwachung und Dokumentation begleiten die physische Errichtung.
\end{itemize}

Diese Phasentrennung führt dazu, dass Planungsergebnisse zu starren Meilensteinen \enquote{eingefroren} werden müssen, um die darauf basierenden Honoraransprüche und Genehmigungsschritte zu sichern. Nachträgliche Anpassungen oder die Rückkehr zu bereits abgeschlossenen Phasen für Optimierungszwecke werden systemisch als \enquote{Störung} oder kostenpflichtige \enquote{Umplanung} gewertet, anstatt als integraler Teil der Wertschöpfung begriffen zu werden \autocite[siehe]{HOAI_Text}.

\begin{figure}[htbp]
    \centering
     \fcolorbox{gray!50}{white}{%
    \includegraphics[width=\dimexpr\textwidth-2\fboxsep-2\fboxrule\relax, height=0.5\textwidth]{05_figures/HOAI vs. TVD Phasen Vergleich .jpg}%
    }
    
    \caption[Vergleich HOAI und TVD]{Vergleich der HOAI-Leistungsphasen mit dem idealtypischen TVD-Phasenmodell (Quelle: Eigene Darstellung).}
    \label{fig:phasenvergleich}
\end{figure}

\minisec{Starrheit der Kostensteuerung nach DIN 276}
Die konventionelle Kostenermittlung orientiert sich an der DIN 276, welche Kosten nach Kostengruppen (Bauteilen) strukturiert (z.B. KG 300 Bauwerk - Baukonstruktion). Diese bauteilorientierte Sichtweise erschwert eine interdisziplinäre Funktionsoptimierung. In der konventionellen Praxis dient die DIN 276 primär als Instrument zur ex-post-Dokumentation. Da die Kostenverantwortung innerhalb der gewerkspezifischen Planungs-Silos verbleibt, können Budgetüberschreitungen oft erst dann festgestellt werden, wenn der Planungsgrad bereits so weit fortgeschritten ist, dass korrigierende Eingriffe hohe Änderungskosten verursachen \autocite[vgl.]{Becker2022}.

\minisec{Vertragsarten und das Risiko des Bau-Solls}
Das vertragliche Fundament bildet in der Regel die VOB/B, welche die Geschäftsbedingungen für die Ausführung regelt. Hierbei dominieren zwei Vertragsarten, die jeweils unterschiedliche Risikoprofile aufweisen:

\begin{itemize}
    \item \textbf{Einheitspreisvertrag (EPV):} Die Vergütung erfolgt auf Basis tatsächlich erbrachter Massen und festgelegter Einheitspreise aus einem Leistungsverzeichnis (LV). Hier trägt der Bauherr das Mengenrisiko. Bei lückenhafter Planung bietet dieser Vertragstyp für Auftragnehmer Anreize zur Mengenmaximierung und zum Nachtragsmanagement \autocite[vgl.]{VOB_B}.
    \item \textbf{Pauschalpreisvertrag (PPV):} Das Honorar wird für eine vorab definierte Gesamtleistung fixiert. Während dies dem Bauherrn vermeintliche Preissicherheit bietet, verlagert es das Risiko unvorhergesehener Massenänderungen auf den Auftragnehmer. Dies führt in der Praxis häufig zu hohen Risikoaufschlägen in der Kalkulation und einer aggressiven Abwehrhaltung gegenüber jeglichen Änderungen im Projektverlauf.
\end{itemize}

Beide Vertragsformen basieren auf der Fiktion eines zum Zeitpunkt der Vergabe abschließend definierten \enquote{Bau-Solls}. Diese Annahme zwingt die Beteiligten dazu, den Entwurf frühzeitig festzuschreiben, was Flexibilität unterbindet und die Basis für die in der Bauwirtschaft weit verbreitete Konfliktkultur legt.

\minisec{Vergaberechtliche Restriktionen}
Ergänzt wird dieses Korsett durch das Vergaberecht (insb. GWB und VgV), welches insbesondere öffentliche Auftraggeber zur Losvergabe und zum Preiswettbewerb zwingt. Diese regulatorischen Vorgaben erschweren die frühzeitige Einbindung von Ausführungswissen (\textit{Early Contractor Involvement}), da eine Zusammenarbeit vor der formalen Vergabe oft als wettbewerbsverzerrende \enquote{Vorbefasstheit} gewertet wird \autocite[vgl.]{Reformkommission2015}. Damit wird das in 2.1.1 beschriebene Wissens-Vakuum in der Entwurfsphase rechtlich institutionalisiert.

\begin{figure}[htbp]
    \centering
    \fcolorbox{gray!50}{white}{%
        \includegraphics[width=\dimexpr\textwidth-2\fboxsep-2\fboxrule\relax, height=0.5\textwidth]{05_figures/Claim_Circle_Teufelskreis.jpg}
    }
    \caption[Teufelskreis des Nachtragsmanagements]{Ökonomische Fehlanreize und der Teufelskreis des konventionellen Claim-Managements (Quelle: Eigene Darstellung in Anlehnung an Girmscheid 2016).}
    \label{fig:claim_circle}
\end{figure}

Zusammenfassend lässt sich festhalten, dass HOAI und VOB ein starres System schaffen, das auf Misstrauen und exakter Abgrenzung von Leistungen basiert. Eine proaktive, wertorientierte Steuerung, die Flexibilität und Kooperation voraussetzt, ist innerhalb dieser regulatorischen Leitplanken kaum realisierbar.

\subsection{Die Konsequenzen: Fragmentierung und ökonomische Fehlanreize}
\label{sec:2.1.3}

Die in den vorangegangenen Abschnitten beschriebene Trennung der Akteure (\cref{sec:2.1.1}) sowie die starre prozessuale Einbindung durch HOAI und VOB (\cref{sec:2.1.2}) führen in der Summe zu einer systemischen Dysfunktionalität. Diese manifestiert sich sowohl in ökonomischen Kennzahlen als auch in einer spezifischen, oft projektkritischen Kommunikationskultur. In der Literatur wird dieser Zustand häufig als Krise der Bauwirtschaft charakterisiert, die einen grundlegenden Paradigmenwechsel erforderlich macht \autocite[vgl.]{Habib2020}.

\minisec{Ökonomie der Nachträge und Claim-Management}
Ein zentrales Merkmal konventioneller Abwicklungsmuster ist die Verschiebung des unternehmerischen Fokus weg von der reinen Bauleistung hin zur Identifikation von Planungs- und Ausschreibungsmängeln. In einem Marktumfeld, das durch das Billigstbieterprinzip und geringe Margen geprägt ist, wird das sogenannte \textbf{Claim-Management} für viele Bauunternehmen zu einer existenziellen Notwendigkeit \autocite[vgl.]{Girmscheid2016}.

Da das vertragliche \enquote{Bau-Soll} aufgrund der sequenziellen Planung (LPH 1--5) zum Zeitpunkt der Vergabe selten fehlerfrei definiert ist, entstehen zwangsläufig Abweichungen während der Ausführung. In der konventionellen Logik werden diese Abweichungen nicht als gemeinsame Chance zur Optimierung begriffen, sondern als Grundlage für Nachtragsforderungen genutzt. Dieser Mechanismus bindet erhebliche Ressourcen auf Seiten aller Beteiligten für die Prüfung, Abwehr und Verhandlung von Mehrkosten, anstatt diese Kapazitäten in die wertschöpfende Planung oder Ausführung zu investieren.

\minisec{Die Stagnation der Produktivität und die Komplexitätsfalle}
Die organisatorische Fragmentierung verhindert zudem den Aufbau nachhaltiger Lernkurven und den Transfer von Prozesswissen über Projektgrenzen hinweg. Während andere Industriezweige durch Standardisierung und integrierte Wertschöpfungsketten signifikante Produktivitätssteigerungen erzielen konnten, stagniert die Wertschöpfung pro Erwerbstätigenstunde im deutschen Baugewerbe seit Jahrzehnten. \autocite[vgl.]{McKinsey2017}\,\autocite[vgl.]{Reformkommission2015}

Dieser Rückstand wird durch die massiv gestiegene technologische Komplexität moderner Bauwerke verschärft. Ein Gebäude der Gegenwart ist nicht mehr mit der Baustruktur der 1970er-Jahre vergleichbar; insbesondere die technische Gebäudeausrüstung (TGA) hat eine \enquote{Explosion} der Kosten- und Schnittstellenanteile erfahren. Während die Technik in den 1960er- und 1970er-Jahren oft nur 10\,\% bis 15\,\% der Gesamtkosten ausmachte, liegt die TGA-Quote (Kostengruppe 400 nach DIN 276) bei komplexen Hochbauten wie Kliniken oder modernen Bürogebäuden heute häufig zwischen 40\,\% und 50\,\% \autocite[siehe]{Sommer2016}\, \autocite[vgl.]{BKI2024}. Das konventionelle Modell versucht jedoch, diese hochgradig vernetzte Technik mit denselben sequenziellen und gewerkeorientierten Methoden zu steuern, die für den vergleichsweise simplen Mauerwerksbau des letzten Jahrhunderts konzipiert wurden. Da die Planung der TGA oft isoliert erfolgt, verpuffen potenzielle Effizienzgewinne aus Innovationen wie dem \textit{Building Information Modeling} (BIM) an den starren vertraglichen Schnittstellen der Leistungsphasen \autocite[vgl.]{Becker2022}.

\minisec{Kulturelle Folgen: Die \enquote{Adversarial Culture}}
Die vielleicht schwerwiegendste Konsequenz ist die Entstehung einer tief verwurzelten Misstrauenskultur, die international als \textit{Adversarial Culture} bezeichnet wird. Die konventionelle Struktur basiert auf der Annahme gegensätzlicher Interessen: Der Gewinn des einen Akteurs wird häufig als der Verlust des anderen wahrgenommen (Nullsummenspiel).

Die \textit{Reformkommission Bau von Großprojekten} stellte bereits 2015 fest, dass diese Atmosphäre der Konfrontation zu einer massiven Risikoaversion führt. Anstatt Risiken gemeinsam proaktiv zu steuern, steht der Versuch im Vordergrund, Risiken vertraglich auf andere Parteien abzuwälzen. Tritt ein Schadensfall ein, steht nicht die Problemlösung, sondern die Entschuldung im Vordergrund. Diese Defensivhaltung untergräbt die für eine wertorientierte Projektsteuerung notwendige Transparenz und verhindert die offene Kommunikation über Fehler und Verbesserungspotenziale. \autocite[vgl.]{Habib2020}\,\autocite[vgl.]{Reformkommission2015}

Zusammenfassend lässt sich feststellen, dass die konventionelle Projektabwicklung in Deutschland in einer strukturellen Sackgasse verharrt. Die Trennung von Planung und Bau, die Starrheit der Regelwerke und die daraus resultierenden ökonomischen Fehlanreize schaffen ein Umfeld, in dem eine zielgerichtete Steuerung von Kosten und Werten systematisch erschwert wird. Diese Analyse bildet die Grundlage für die Forderung nach alternativen, kollaborativen Abwicklungsmodellen, wie sie im folgenden \cref{sec: 2.2} vorgestellt werden.

\clearpage









\section{Integrierte Projektabwicklungsmodelle als alternativer Ansatz}
\label{sec: 2.2}

Die im vorangegangenen Abschnitt aufgezeigten Schwachstellen der konventionellen Projektabwicklung, insbesondere die organisatorische Fragmentierung und die daraus resultierenden Effizienzverluste, sind kein einzigartiges Problem der deutschen Bauwirtschaft. Historisch betrachtet lassen sich Parallelen zur stationären Industrie ziehen, wo ähnliche Herausforderungen zur Entwicklung der Lean-Philosophie führten.

\subsection{Lean-Management als Wegbereiter kollaborativer Modelle}
\label{sec:2.2.1}

Der Ursprung dieses Denkansatzes liegt in der japanischen Automobilindustrie der Nachkriegszeit. Da Toyota das amerikanische Modell der Massenproduktion aufgrund begrenzter Ressourcen nicht adaptieren konnte, entwickelte Taiichi Ohno das Toyota-Produktionssystem (TPS). Dessen Kernziel war die radikale Eliminierung jeglicher Verschwendung (\textit{Muda}) durch eine konsequente Fluss-Orientierung der Produktion \autocite[vgl. S. 33]{ohno_toyota-produktionssystem_2013}.

Die Übertragung dieser Prinzipien auf das Bauwesen wurde 1992 durch Lauri Koskela wissenschaftlich fundiert. In seinem wegweisenden Report \textit{Application of the New Production Philosophy to Construction} etablierte er die sogenannte \textbf{TFV-Theorie} (Transformation, Flow, Value). Er legte dar, dass Bauproduktion nicht nur als Transformation von Material zu Bauwerk verstanden werden darf, sondern vor allem unter den Aspekten des \enquote{Flusses} (\textit{Flow}) und der \enquote{Wertschöpfung} (\textit{Value}) gesteuert werden muss \autocite[vgl. S. 15]{koskela_application_1992}. Während die Transformation oft lokal effizient ist, liegen in den komplexen Übergängen zwischen Gewerken und Phasen massive Anteile an nicht-wertschöpfenden Tätigkeiten wie Warten oder Nacharbeiten \autocite[vgl. S. 16]{haghsheno_lean_2019}.

Die Etablierung von \textit{Lean Construction} führte zur Entwicklung operativer Methoden (wie dem \textit{Last Planner System}), um diesen Prozessfluss zu stabilisieren. In der Praxis zeigte sich jedoch eine systemische Grenze: Die Optimierung des Gesamtflusses steht oft im Widerspruch zu den lokalen Gewinninteressen der Einzelakteure, die durch konventionelle, bilaterale Verträge motiviert sind \autocite[vgl. S. 12]{becker_integrierte_2022}. Solange Verträge auf der Abwälzung von Risiken basieren, können Lean-Methoden ihre volle Wirkung nicht entfalten.

Die historische Antwort auf dieses Dilemma lieferte der Offshore-Sektor: Beim \textit{BP Andrew Project} in den 1990er-Jahren wurde erstmals eine Allianz gebildet, bei der alle Partner am finanziellen Gesamterfolg beteiligt waren (\textit{Pain-Share/Gain-Share}), anstatt Risiken bilateral abzuwälzen \autocite[vgl. S. 2]{ross_introduction_2000}. Dieses als \textit{Project Alliancing} bekannt gewordene Modell bewies, dass die vertragliche Integration der Schlüssel zur massiven Kostenreduktion ist. In der Folge etablierte sich das Modell schnell in Australien und Finnland – hier insbesondere durch die erfolgreiche Anwendung im öffentlichen Infrastrukturbau \autocite[vgl.]{lahdenpera_making_2012}. Inspiriert durch diese Erfolge prägte das American Institute of Architects (AIA) 2007 den Begriff \textit{Integrated Project Delivery} (IPD) für den Hochbau \autocite[vgl.]{AIA_2007__integrated}. Im deutschsprachigen Raum hat sich hierfür die Bezeichnung \textbf{Integrierte Projektabwicklung (IPA)} etabliert, die die Anwendung dieser kollaborativen Prinzipien und Vertragsstrukturen im Kontext der deutschen Bauwirtschaft beschreibt \autocite[vgl. S. 28]{haghsheno_lean_2019}.

\clearpage

% \subsection{Prinzipien und Merkmale der Integrierten Projektabwicklung}
% \label{sec:2.2.2}

% Da der Begriff der Integrierten Projektabwicklung in Deutschland – anders als die klassischen Modelle der VOB – nicht gesetzlich definiert ist, erfordert die wissenschaftliche Einordnung eine präzise Abgrenzung anhand von branchenweit akzeptierten Merkmalen. Das \ac{GLCI} definiert IPA hierbei als ein ganzheitliches Projektabwicklungsmodell, das vertragliche, kulturelle und ökonomische Aspekte in einem kohärenten System integriert \autocite[vgl. S. 28]{haghsheno_lean_2019}.

% Zur Veranschaulichung dieser Systematik hat sich in der deutschsprachigen Literatur das Bild des \enquote{Hauses der IPA} etabliert. Dieses Modell visualisiert das Zusammenspiel der verschiedenen Ebenen: Das \textbf{Fundament} bildet der Mehrparteienvertrag, der alle Schlüsselakteure rechtlich bindet. Die \textbf{Säulen} bestehen aus dem gemeinsamen Risikomanagement (ökonomische Anreize) und der kooperativen Zusammenarbeit (Kultur und Methoden). Das \textbf{Dach} symbolisiert schließlich die gemeinsamen Projektziele (\enquote{Best for Project}), die über den individuellen Einzelinteressen stehen \autocite[vgl. S. 29]{haghsheno_lean_2019}.

% Um die Beliebigkeit des Begriffs einzuschränken und eine klare Abgrenzung zu unverbindlichen Partnering-Modellen zu schaffen, hat das IPA-Zentrum im Jahr 2022 eine präzisere Definition erarbeitet. Diese unterscheidet zwischen wesensgebenden Charakteristika (wie einer kooperativen Grundhaltung) und harten \textbf{konstitutiven Modellbestandteilen}. Entscheidend für die wissenschaftliche Einordnung ist, dass ein Projekt erst dann dem IPA-Standard entspricht, wenn alle konstitutiven Bestandteile kumulativ vorliegen \autocite[vgl. S. 3]{ipa_zentrum_integrierte_2022}.

% Für das Verständnis des späteren \textit{Target Value Designs} (TVD) sind insbesondere folgende konstitutive Elemente hervorzuheben \autocite[vgl. S. 4--9]{ipa_zentrum_integrierte_2022}:

% \begin{description}
%     \item[Mehrparteienvertrag:] Mindestens der Bauherr, die maßgeblichen Planungsbeteiligten und die schlüsselfertigen Ausführungsunternehmen schließen einen einzigen, gemeinsamen Vertrag. Dies löst die bilaterale \enquote{V-Struktur} der konventionellen Abwicklung auf und schafft eine gemeinsame Rechtsgrundlage.
    
%     \item[Vergütungsmodell und Risikopool:] Die Vergütung erfolgt über ein Drei-Säulen-Modell: (1) Erstattung der tatsächlichen Selbstkosten, (2) ein fixer Gemeinkostenzuschlag und (3) ein variabler Gewinnanteil. Letzterer fließt in einen gemeinsamen Risikopool. Bei Kostenüberschreitungen wird dieser Pool zur Deckung genutzt (\textit{Pain-Share}), während Unterschreitungen als Bonus an alle Partner ausgeschüttet werden (\textit{Gain-Share}) \autocite[vgl. S. 19]{becker_integrierte_2022}.
    
%     \item[Open Book Accounting:] Diese radikale finanzielle Transparenz ist die conditio sine qua non der IPA. Alle Kostenstrukturen müssen offengelegt werden, da ohne diese Datenbasis eine gemeinsame Kostensteuerung im TVD-Prozess methodisch nicht operabel wäre.
    
%     \item[Einstimmigkeitsprinzip:] Entscheidungen im integrierten Führungsgremium (\textit{Senior Management Team}) müssen konsensual getroffen werden. Dies zwingt die Parteien zur sachlichen Argumentation und verhindert die im konventionellen System übliche Dominanz einzelner Akteure.
    
%     \item[Haftungsbeschränkung (\textit{No-Blame-Culture}):] Die Partner verzichten weitgehend auf gegenseitige Haftungsansprüche bei Fehlern (ausgenommen Vorsatz und grobe Fahrlässigkeit). Dies fördert eine offene Fehlerkultur, in der die Behebung von Defiziten Vorrang vor der Schuldzuweisung hat.
% \end{description}

% Haghsheno et al. ordnen diese vielfältigen Elemente in einem wissenschaftlichen Strukturierungsansatz drei logischen Ebenen zu, die sich gegenseitig bedingen \autocite[vgl. S. 80]{haghsheno_strukturierungsansatz_2022}:
% \begin{enumerate}
%     \item \textbf{Kommerzielle und rechtliche Rahmenbedingungen:} Der Mehrparteienvertrag und das Anreizsystem bilden das unverzichtbare Gerüst.
%     \item \textbf{Organisation und Kultur:} Die integrierte Teamstruktur und das verhaltensbezogene Mindset (\enquote{Best for Project}).
%     \item \textbf{Prozesse und Methoden:} Die operative Ebene, auf der konkrete Werkzeuge zum Einsatz kommen.
% \end{enumerate}

% Diese Systematik ist zentral für den weiteren Verlauf dieser Arbeit: Während IPA den rechtlichen und kulturellen Rahmen (Ebene 1 und 2) spannt, ist das \textbf{Target Value Design} als die entscheidende Methode auf der Prozessebene (Ebene 3) zu verorten, um die Projektziele innerhalb dieses Rahmens aktiv zu steuern.

\subsection{Prinzipien und Merkmale der Integrierten Projektabwicklung}
\label{sec:2.2.2}

Die Integrierte Projektabwicklung (IPA) stellt weit mehr dar als eine bloße vertragliche Modifikation; sie ist ein ganzheitliches \textit{Project Delivery Model}, das eine radikale Abkehr von der fragmentierten Logik des konventionellen Bauwesens vollzieht. Da IPA im deutschen Rechtsraum nicht durch ein einzelnes Gesetz kodifiziert ist, erfolgt die wissenschaftliche Definition über die kumulative Präsenz spezifischer Strukturmerkmale. Das Ziel ist die Schaffung einer \enquote{Schicksalsgemeinschaft}, in der der Erfolg des Einzelnen untrennbar mit dem Erfolg des Gesamtsystems verknüpft ist \autocite[vgl.]{haghsheno_lean_2019}.

\subsubsection{Das Fundament: Das Haus der IPA}
Zur Systematisierung dieses komplexen Gefüges hat sich in der deutschsprachigen Fachliteratur das Modell des \enquote{Hauses der IPA} etabliert. Dieses dient als visuelle und funktionale Landkarte für die Einordnung der verschiedenen Ebenen des Modells.

\begin{figure}[htbp]
    \centering
    \fcolorbox{gray!50}{white}{%
        \includegraphics[width=\dimexpr\textwidth-2\fboxsep-2\fboxrule\relax, height=0.5\textwidth]{05_figures/Haus_der_IPA.jpg}%
    }
    \caption[Das Haus der IPA]{Das Haus der IPA zur Darstellung der systemischen Abhängigkeiten (Quelle: Eigene Darstellung in Anlehnung an Haghsheno 2019).}
    \label{fig:haus_der_ipa}
\end{figure}

Das \textbf{Fundament} dieses Hauses bildet der \textbf{Mehrparteienvertrag}, welcher die rechtliche Basis für alle weiteren Interaktionen darstellt. Ohne diese formale Klammer blieben kooperative Ansätze unverbindlich. Die \textbf{Säulen} bestehen aus der \textbf{Organisation und Kultur} (integrierte Teams, No-Blame) sowie dem \textbf{kommerziellen Rahmen} (ökonomische Anreizsysteme). Diese stützen das \textbf{Dach}, welches die gemeinsamen \textbf{Projektziele} symbolisiert. Die zentrale Erkenntnis dieses Modells ist die Interdependenz: Versagt eine Säule, etwa durch den Rückfall in opportunistisches Verhalten aufgrund falscher ökonomischer Anreize, verliert das gesamte System seine Stabilität \autocite[vgl.]{haghsheno_lean_2019}.

\subsubsection{Die acht Charakteristika der IPA}
Bevor die harten vertraglichen Bestandteile analysiert werden, müssen die acht wesensgebenden Charakteristika definiert werden, die den kulturellen und prozessualen Rahmen spannen. Diese dienen als \enquote{Leitplanken} für das Projektverhalten \autocite[vgl.]{ipa_zentrum_integrierte_2022}:

\begin{itemize}
    \item \textbf{Projektziele vor Einzelzielen:} Jede Entscheidung wird unter der Prämisse \enquote{Best for Project} getroffen.
    \item \textbf{Frühzeitige Einbindung:} Planungs- und Ausführungsexpertise verschmelzen bereits in der Konzeptphase (ECI).
    \item \textbf{Gemeinsame Entscheidungsfindung:} Abkehr von der hierarchischen Bauherren-Dominanz hin zum Konsensprinzip.
    \item \textbf{Integrierte Teamstruktur:} Auflösung von Firmenidentitäten zugunsten einer gemeinsamen Projektidentität.
    \item \textbf{Gemeinsames Risiko und Gewinn:} Ökonomische Synchronisation der Partner.
    \item \textbf{Haftungsbeschränkung:} Etablierung einer angstfreien Fehlerkultur (\textit{No-Blame}).
    \item \textbf{Finanzielle Transparenz:} Offenlegung aller Kostenstrukturen (\textit{Open Book}).
    \item \textbf{Kollaborative Werkzeuge:} Nutzung von BIM und Lean-Methoden als gemeinsame Arbeitsebene.
\end{itemize}

\subsubsection{Konstitutive Modellbestandteile und vertragliche Vertiefung}
Um die Beliebigkeit des Begriffs IPA zu vermeiden, definiert das IPA-Zentrum (2022) konstitutive Bestandteile, die zwingend vorhanden sein müssen. Für das Target Value Design sind insbesondere die folgenden Elemente von substanzieller Bedeutung.

\paragraph{Der Mehrparteienvertrag und der deutsche Mustervertrag}
Das Herzstück der IPA ist der Mehrparteienvertrag (MPV). Im Gegensatz zur konventionellen Welt, in der der Bauherr Dutzende bilaterale Verträge schließt, unterzeichnen hier alle Schlüsselakteure (Bauherr, Architekt, Fachplaner, Hauptunternehmer) ein einziges Dokument. In Deutschland wurde hierfür durch das \textit{IPA-Zentrum} ein \textbf{IPA-Muster-Mehrparteienvertrag} veröffentlicht, der als Standard für die rechtliche Umsetzung gilt. 

Dieser Vertrag ist als Projektvertrag \textit{sui generis} zu verstehen, der oft gesellschaftsrechtliche Elemente (GbR-Strukturen) nutzt, um die gesamthändische Bindung der Partner zu erzwingen. Er regelt die \textbf{Integrierte Projektorganisation (IPO)}, die sich meist in ein \textit{Senior Management Team} (SMT) für strategische Entscheidungen und ein \textit{Project Management Team} (PMT) für die operative Steuerung gliedert. Diese vertragliche Fusion verhindert, dass Informationen an Schnittstellen verloren gehen, da es keine rechtlichen \enquote{Mauern} zwischen Planung und Bau mehr gibt \autocite[vgl.]{ipa_zentrum_integrierte_2022}.

\paragraph{Das Drei-Säulen-Vergütungsmodell und der Risikopool}
Die ökonomische Logik der IPA bricht mit dem klassischen Preiswettbewerb. Die Vergütung der Partner basiert auf drei Säulen \autocite[vgl.]{becker_integrierte_2022}:
\begin{enumerate}
    \item \textbf{Selbstkostenerstattung:} Alle tatsächlich angefallenen, projektbezogenen Kosten werden dem Partner eins zu eins erstattet. Dies minimiert das existenzielle Risiko für die Unternehmen.
    \item \textbf{Fixe Gemeinkosten:} Ein vorab vereinbarter Zuschlag deckt die allgemeinen Verwaltungskosten der Partner.
    \item \textbf{Risiko-Gewinn-Pool:} Der kalkulierte Gewinn der Partner wird in einen gemeinsamen Pool eingebracht. Dieser Pool ist \textit{ring-fenced}, er gehört dem Projekt. Entstehen Kostenüberschreitungen, werden diese primär aus dem Pool gedeckt (\textit{Pain-Share}). Werden die Zielkosten unterschritten, wird der Überschuss nach einem vereinbarten Schlüssel an alle Partner ausgezahlt (\textit{Gain-Share}). 
\end{enumerate}
Dieser Mechanismus ist der direkte \textbf{ökonomische Enabler für TVD}: Da der eigene Gewinn nur durch die Optimierung des Gesamtprojekts gesichert werden kann, verschwindet der Anreiz für Nachträge und Claims vollständig.

\paragraph{Open Book Accounting und finanzielle Transparenz}
Eng verknüpft mit der Selbstkostenerstattung ist das Prinzip des \textit{Open Book Accounting}. Jeder Partner gewährt dem Bauherrn und dem Team vollen Einblick in seine Kalkulationen, Lohnabrechnungen und Nachunternehmerverträge. Diese \enquote{gläserne Baustelle} ist für das Target Value Design unverzichtbar, da Zielkostensteuerungen nur auf Basis von Realwerten und nicht auf Basis von spekulativen Marktpreisen funktionieren können. Das Open Book schafft das notwendige Vertrauen, um Kostenverschiebungen zwischen den Gewerken (\enquote{Geld fließt dorthin, wo es den größten Nutzen bringt}) ohne Widerstände zu ermöglichen.

\paragraph{Einstimmigkeitsprinzip und Governance}
Entscheidungen im SMT müssen einstimmig getroffen werden. Dies mag in der Theorie nach Stillstand klingen, wirkt in der IPA jedoch als Beschleuniger. Da alle Partner durch den Gewinnpool finanziell am zügigen Projektfortschritt interessiert sind, erzwingt das Einstimmigkeitsprinzip eine hohe Qualität der Argumentation und eine schnelle Konsensfindung. Es beendet die konventionelle Praxis, bei der der Bauherr Entscheidungen einseitig trifft, deren Konsequenzen dann oft zeitverzögert als Claims auf ihn zurückfallen.

\paragraph{Haftungsbeschränkung und No-Blame-Kultur}
Ein konstitutiver Pfeiler ist der weitgehende Verzicht auf gegenseitige Haftungsansprüche (\textit{Liability Waiver}). Nach deutschem Recht (insb. § 276 BGB) kann die Haftung für Vorsatz nicht ausgeschlossen werden. Der IPA-Mustervertrag sieht jedoch vor, dass die Partner für einfache Fahrlässigkeit innerhalb des Teams nicht haften. Dies ist die Geburtsstunde einer echten \textbf{No-Blame-Kultur}. Wenn ein Planungs- oder Ausführungsfehler auftritt, liegt der Fokus sofort auf der Behebung (\enquote{What went wrong, not who is to blame}). Diese angstfreie Umgebung ist die psychologische Voraussetzung für die iterativen Optimierungsprozesse des TVD, da Fehler als Lernchancen begriffen werden, anstatt sie aus Regressangst zu verschleiern \autocite[vgl.]{ipa_zentrum_integrierte_2022}.


Diese Einordnung ist zentral für den weiteren Verlauf dieser Arbeit: Während IPA den vertraglichen und kulturellen Rahmen (Ebene 1 und 2) spannt, ist \textbf{Target Value Design} die entscheidende Methode auf der Prozessebene (Ebene 3), um die Kostenziele innerhalb dieses Rahmens aktiv zu steuern.












% IPA als ist das organisatorische und vertragliche "Betriebssystem", um die Lean-Philosophie in Bauprojekten praktisch umzusetzen.\\

% Schlüsseleigenschaften und IPA-Charakteristika:\\
% - Frühzeitige Integration aller Schlüsselpartner (Planer und Ausführende) von Beginn an.\\
% Ein Mehrparteienvertrag, der alle Partner an einen Tisch bindet.\\
% Ein gemeinsames Risikomanagement (Chancen-/Risikopool).\\
% Eine kollaborative Kultur und gemeinsame Entscheidungsfindung.\\
















\subsection{Internationale Erfolge und Status Quo in Deutschland}
\label{sec:2.2.3}

\begin{figure}[htbp]
    \centering
    \fcolorbox{gray!50}{white}{%
        \includegraphics[width=\dimexpr\textwidth-2\fboxsep-2\fboxrule\relax]{05_figures/MacLeamy-Curve.png}%
    }
    \caption[MacLeamy-Kurve]{MacLeamy-Kurve, die den relativen Grad an Kosten und Aufwand über den Lebenszyklus eines Bauwerks zeigt. \autocite[]{Quelle: https://education.buildingsmart.org/de/a-new-way-of-working} }
    \label{fig:macleamycurve}
\end{figure}


International: Hohe Erfolgsquoten im angelsächsischen Raum (USA, UK, Australien mit "Project Alliancing") bei der Einhaltung von Kosten und Terminen (Quelle z.B. cook2014).\\

Deutschland: Das Thema gewinnt stark an Fahrt: es gibt eine wachsende Zahl an Pilotprojekten (Quelle: IPA-Report 2024 von haghsheno2025).\\

Eine zentrale Kernmethode zur Kosten- und Wertsteuerung innerhalb dieser IPA-Modelle ist das Target Value Design.\\

\clearpage

\section{Target Value Design (TVD) als Kernmethode}
\label{sec: 2.3}

\subsection{Begriffliche Einordnung und Definition}
\label{sec: 2.3.1}

% =======================================================
% FÜR ABSCHNITT 2.3.1: TVD - Prozess oder Methode
% =======================================================
%   - Ziel: Eine präzise und trennscharfe Definition von TVD für die
%     vorliegende Arbeit herleiten. Es soll geklärt werden, ob TVD als
%     Prozess, als Methode oder als beides zu verstehen ist, um eine
%     eindeutige Basis für die nachfolgende Analyse zu schaffen.
%
%   - Einstieg: Feststellung, dass die Begriffe "Prozess" und "Methode"
%     in der Literatur zu TVD oft unscharf oder synonym verwendet werden.
%
%   - Begriffsdefinition "Methode": Ein systematisches, geplantes Vorgehen,
%     das auf Prinzipien, Regeln und Werkzeugen basiert, um ein Ziel zu erreichen.
%
%   - Begriffsdefinition "Prozess": Eine logische und zeitliche Abfolge von
%     miteinander verknüpften Aktivitäten zur Umwandlung eines Inputs in einen Output.
%
%   - Anwendung auf TVD:
%     - TVD ist mehr als ein reiner Prozess, da es eine bestimmte Denkweise
%       ("Target First"), Prinzipien (Kollaboration) und Werkzeuge voraussetzt.
%     - Diese übergeordnete Denkweise wird jedoch erst durch einen konkreten,
%       strukturierten Ablauf (Prozess) in der Praxis anwendbar.
%
%   - Fazit & Positionierung: Für diese Arbeit wird TVD als eine
%     *MANAGEMENT-METHODE* verstanden, die durch einen
%     _STRUKTURIERTEN PROZESS_ operationalisiert und umgesetzt wird.
%     Diese Definition legitimiert die nachfolgende "prozessorientierte
%     Darstellung" der Methode.
%
% =======================================================

Zur Hinleitung auf die in \cref{ch:methodik} beschriebene, methodische Vorgehensweise soll an dieser Stelle auf das in dieser Arbeit zu Grunde liegende Verständnis von \ac{TVD} als Prozess bzw. Methode eingegangen werden.\\
In der Literatur ist die Abgrenzung zwischen Prozess und Methode in der aktuellen Diskussion um TVD unscharf. Folgt man dem Ergebnis einer einfachen Stichwortsuche in einschlägigen wissenschaftlichen Datenbanken (z.B. reseachgate.net) so wird schnell deutlich, das in der überwiegenden Mehrheit der Literatur (ca. 70\%) Target Value Design als Prozess verstanden wird.
Glenn Ballard, der als einer Gründer des \ac{TVD} gilt, bezeichnet letzteres hingegen häufig als Managementansatz bzw. Management-Methode \autocite[]{}.\\
% In der allgemeinen Literatur wird ein Prozess als

- Definition Prozess Vs. Methode: Verständnis als Management Methode die durch einen sturkturierten Prozess umgesetzt wird -> legitimiert die nachfolgende prozessoroentierte Darstelung.\\

\subsection{Historische Entwicklung}
\label{sec: 2.3.2}

Inhalt: Herleitung aus den drei historischen Strängen (vgl. Ballard, 2025, Kap. 3 ):\\
Manufacturing: Target Costing bei Toyota (Produktionssystem) .\\
UK: Cost Planning der Quantity Surveyors (1950er) .\\
Finnland: Haahtela-Modell (1980er) .\\
USA: Die Synthese zu "Lean TVD" (ab 2000er) .\\
Ziel: Kosten als Inputgröße statt Outputgröße der Planung.\\

\subsection{Handlungslogik und Kernprinzipien}
\label{sec: 2.3.3}

Target First: Kosten sind eine Design-Voraussetzung, kein Ergebnis.\\

Collaboration: Alle sitzen früh in einem Boot (Cross-funktionale Teams).\\

Optimize the Whole: Projektziel vor Einzelziel.\\

Continuous Improvement: Der Prozess ist nie "fertig", sondern wird iterativ verbessert.\\



\subsection{Methoden und Werkzeuge}
\label{sec: 2.3.4}

Die operative Umsetzung der Zielkostenerreichung erfordert spezifische Methoden, die über die reine Kooperationsbereitschaft (vgl. \cref{sec: 2.2}) hinausgehen. Nach Ballard (2012) stützt sich der TVD-Prozess auf ein integriertes Set an Werkzeugen, welche die Designstrategie, die Entscheidungsfindung und die technische Datenverarbeitung steuern.\autocite[vgl. S. 15]{ballard_target_2012}

\subsection*{Set-Based-Design}
Set-Based Design (SBD): Definition als paralleles Entwickeln von Alternativen bis zum "Last Responsible Moment" (vgl. Ballard, 2025, S. 27; LCI 2016).

\subsection*{Choosing By Advantages}
Choosing By Advantages (CBA): Definition als Entscheidungsmethode basierend auf Vorteilen, nicht Gewichtungen

\subsection*{A3-Methode bzw. Report}
A3 Thinking / Reports: Definition als Problemlösungs- und Konsens-Werkzeug.

\subsection*{Last Planner System}
Last Planner System (LPS) in der Planung: Nutzung zur Steuerung des Design-Workflows (nicht nur Bau).

\subsection*{Building Information Modelling}
Im TVD-Kontext wird \ac{BIM} zur Datenbank für die Kostenkalkulation.\\
Zweck: "Wir müssen alle 3 Wochen wissen, was es kostet."
Funktion: Automatisierter Massenauszug (Quantity Take-Off). Wenn Sie eine Wand im Modell verschieben, ändert sich sofort die m²-Zahl in der Datenbank.\\
Ziel: Geschwindigkeit..\\


\subsection{Akteure und Organisationsstruktur}
\label{sec: 2.3.5}

- Unterschied zu Target Costing (Industrie) und Value Engineering (klassische Planung).\\

- Einordnung in IPA-Kontext.\\

- Relevanz: Warum TVD nicht nur Kostentechnik, sondern Führungsprinzip ist.\\

- TVD als prozessorientierte Managementmethode mit systematischer Kosten- und Wertsteuerung.\\

Zur detaillierten Untersuchung dieses Prozesses im Kontext der deutschen Baupraxis wird im folgenden Kapitel das methodische Vorgehen beschrieben. Dabei wird zunächst das Forschungsdesign erläutert und anschließend dargelegt, wie der idealtypische TVD-Prozess systematisch analysiert und modelliert wird.

\clearpage