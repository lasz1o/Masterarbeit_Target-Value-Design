\chapter{Grundlagen}
\label{ch:grundlagen}
Ziel: Dieses Kapitel legt die Wissensbasis

\section{Konventionelle Projektabwicklung in der deutschen Baupraxis}
\label{sec: 2.1}
Ziel:

\clearpage

\section{Integrierte Projektabwicklungsmodelle als alternativer Ansatz}
\label{sec: 2.2}

\subsection{Der Lean-Ansatz als Fundament integrierter Projektabwicklung}
\label{sec:2.2.1}

Ursprung des Ansatzes (Lean Management) Skizzieren
- Toyota-Produktionssystem\autocite[]{ohno_toyota-produktionssystem_2013} als Ursprung
- Kernideen und Prinzipien : Wertmaximierung, Verschwendungsreduktion, People First (checken), Kaizen
Transfer auf die Baubranche durch Lean Construction
Ursprung und Entwicklung von Projetct Alliancing, IPD und IPA bis heute

\subsection{Prinzipien und Merkmale der Integrierten Projektabwicklung (IPA)}
\label{sec:2.2.2}

\subsection{Internationale Erfolge und Status Quo in Deutschland}
\label{sec:2.2.3}

\clearpage

\section{Target Value Design (TVD) als Kernmethode}
\label{sec: 2.3}
Ziel: