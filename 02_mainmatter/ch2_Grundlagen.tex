\chapter{Grundlagen}
\label{ch:2}

Das folgende Kapitel legt das theoretische Fundament für die Untersuchung einer wertorientierten Projektsteuerung im Bauwesen. Um die Funktionsweise und den Mehrwert des \textit{Target Value Designs} (TVD) im weiteren Verlauf der Arbeit bewerten zu können, ist ein dreistufiges Verständnis der theoretischen Zusammenhänge erforderlich. Zunächst werden die konventionellen Strukturen der deutschen Baupraxis analysiert, um die systemimmanenten Defizite aufzuzeigen. Darauf aufbauend werden die \textit{Integrierte Projektabwicklung} (IPA) als organisatorischer Rahmen und schließlich das TVD als methodisches Instrument zur aktiven Steuerung von Kosten und Werten eingeführt.

\section{Konventionelle Projektabwicklung in Deutschland}
\label{sec:2.1}

Die deutsche Bauwirtschaft ist durch tief verwurzelte Traditionen und eine hohe regulatorische Dichte geprägt, die das Bild der Projektabwicklung seit Jahrzehnten bestimmen. Bevor alternative Steuerungsansätze diskutiert werden können, muss dargelegt werden, warum die etablierten Mechanismen bei zunehmender Projektkomplexität an ihre Grenzen stoßen. Der folgende Abschnitt dekonstruiert die konventionelle Praxis, indem er die Rollenbilder der Akteure, die prozessualen Fesseln der Honorar- und Vergabeordnungen sowie die daraus resultierenden ökonomischen Fehlanreize analysiert.

\subsection{Akteure, Rollenbilder und die Trennung von Planung und Bau}
\label{sec:2.1.1}

Die konventionelle Projektabwicklung in Deutschland ist historisch durch eine strikte Trennung von Planungs- und Ausführungsleistungen geprägt – ein Modell, das international als \textit{Design-Bid-Build} (DBB) bekannt ist \autocite[Vgl.][S. 435]{girmscheid_projektabwicklung_2016}. Diese Struktur basiert auf einem spezifischen Rollenverständnis der beteiligten Akteure, welches zu einer Trennung von Entwurfs- und Fertigungswissen führt \autocite[Vgl.][S. 12]{sommer_projektmanagement_2016} und tief in den regulatorischen Rahmenbedingungen der HOAI sowie der AHO verwurzelt ist \autocite[Vgl.][S. 43 ff.]{girmscheid_projektabwicklung_2016}.

\minisec{Das Treuhänder-Modell des Planers}
Zentrales Element der deutschen Planungskultur ist die treuhänderische Stellung des Architekten und der Ingenieure gegenüber dem Bauherrn. Gemäß dem Leitbild der Kammern agiert der Planer als unabhängiger Sachwalter des Auftraggebers, der dessen Interessen gegenüber den bauausführenden Unternehmen vertritt \autocite[Vgl.][S. 43 ff., 89--91]{girmscheid_projektabwicklung_2016}. Dieses Rollenmodell impliziert eine strikte Neutralität gegenüber den ausführenden Gewerken, was häufig dazu führt, dass eine frühzeitige Kooperation zwischen Planern und Bauunternehmen unterbleibt. 

Insbesondere im öffentlichen Vergaberecht wird diese Trennung durch das Verbot der \enquote{Vorbefasstheit} zementiert, um Wettbewerbsverzerrungen zu vermeiden \autocite[Vgl. \S~7 Abs.~2 VOB/A; siehe auch][S. 15]{frohlich_vob_2003}. Die rechtliche Systematik sieht vor, dass die Bedingungen für den Wettbewerb (VOB/A) strikt von der späteren Vertragsdurchführung (VOB/B) isoliert bleiben müssen \autocite[Vgl.][S. 4]{frohlich_vob_2003}\, \autocite[Vgl.][S. 4]{springer_fachmedien_wiesbaden_vob2019_2020}. Die Folge ist jedoch eine prozessuale Isolation der Planungsphase von den Realitäten der Bauausführung, da baubetriebliche Expertise im Sinne der Chancengleichheit erst nach Abschluss der Planung eingeholt werden darf \autocite[Vgl.][S. 653]{girmscheid_projektabwicklung_2016}.

\minisec{Sequenzialität und die \enquote{V-Struktur} der Kommunikation}
Die organisatorische Struktur konventioneller Projekte folgt einer bilateralen Logik, die in der Literatur oft als \enquote{V-Struktur} bezeichnet wird. Der Bauherr schließt isolierte Einzelverträge mit den Planungsbeteiligten (LPH 1–4/5) und erst zu einem deutlich späteren Zeitpunkt mit den bauausführenden Unternehmen (LPH 6–9). Zwischen den beiden Gruppen, den Trägern des \enquote{Entwurfswissens} und den Trägern des \enquote{Fertigungswissens}, existiert keine vertragliche oder prozessuale Verbindung \autocite[Vgl.][ S. 12]{sommer_projektmanagement_2016}. 

Diese Fragmentierung der Projektorganisation führt zu einer sequenziellen Informationsweitergabe, bei der Planungsentscheidungen isoliert durch einzelne Fachdisziplinen getroffen und erst zeitverzögert an nachgelagerte Akteure weitergegeben werden. In der traditionellen Projektabwicklung erfolgt die Entscheidungsfindung dabei schrittweise und spezialisiert, wodurch das Wissen der ausführenden Unternehmen systematisch erst nach Abschluss zentraler Entwurfsentscheidungen wirksam wird. Ballard und Howell beschreiben dieses Vorgehen als ein Paradigma, in dem Entscheidungen sequentiell von Spezialisten getroffen und anschließend „über die Wand geworfen“ werden, anstatt nachgelagerte Akteure frühzeitig in upstream-Entscheidungen einzubinden \autocite[Vgl.][S.1 ff]{ballard_lean_2003}.

Die Folge ist ein struktureller Ausschluss ausführungsspezifischen Wissens in jener Projektphase, in der die Möglichkeit zur Kostenbeeinflussung am größten ist. Habib fasst dieses Defizit im Kontext der konventionellen Projektabwicklung als einen Verlust entscheidungsrelevanten Produktionswissens zusammen, da das Know-how der späteren Realisierungsakteure in der Entwurfsphase faktisch nicht berücksichtigt wird \autocite[Vgl.][S. 46 f.]{habib_alternative_2020}.

\minisec{Informationsasymmetrie und die \enquote{Knowledge Gap}}
Ein zentrales Defizit der funktionalen Trennung von Planung und Ausführung besteht in der daraus resultierenden Informationsasymmetrie zwischen den Projektbeteiligten. Während Planer primär die funktionalen und gestalterischen Anforderungen eines Bauwerks definieren, liegt das detaillierte Wissen über reale Kostenstrukturen, Bauverfahren und Produktionsprozesse maßgeblich bei den ausführenden Unternehmen. In der konventionellen Projektabwicklung werden diese beiden Wissensdomänen erst nach Abschluss der Vergabe zusammengeführt. Erweist sich die zuvor erarbeitete Planung in dieser Phase als technisch nicht realisierbar oder ökonomisch ineffizient, sind umfangreiche Anpassungen und Umplanungen (\textit{Rework}) erforderlich, die regelmäßig zu zusätzlichen Kosten- und Terminabweichungen führen \autocite[Vgl.][S.1 ff]{ballard_lean_2003}.  

Die zeitliche Entkopplung von Entscheidungs- und Ausführungswissen wird durch die MacLeamy-Kurve (\cref{fig:macleamycurve}in \cref{sec:2.2.2} in anschaulich verdeutlicht. Diese zeigt, dass die Möglichkeit zur Einflussnahme auf die Projektkosten mit zunehmendem Projektfortschritt stark abnimmt, während die Kosten von Änderungen im Gegenzug überproportional ansteigen \autocite[Vgl.][S. 3 ff]{curt_collaboration_2004}.


\minisec{Informationsasymmetrie und die \enquote{Knowledge Gap}}
Ein wesentliches Defizit dieser Rollentrennung ist die entstehende Informationsasymmetrie. Während der Planer die funktionale und gestalterische Qualität definiert, verfügt erst der Bauunternehmer über die detaillierte Kenntnis der realen Kostenstrukturen und Bauprozesse. In der konventionellen Abwicklung kollidieren diese Welten erst nach der Vergabe. Erweist sich die Planung als nicht baubar oder ökonomisch ineffizient, sind aufwendige Umplanungen (\textit{Rework}) die Folge, welche wiederum die Zeitpläne und Budgets belasten \autocite[Vgl.][S.]{ballard_lean_2003}. Die MacLeamy-Kurve illustriert hierbei eindrücklich, dass die Fähigkeit zur Kostenbeeinflussung mit fortschreitender Projektdauer rapide sinkt, während die Kosten für Änderungen exponentiell ansteigen. \autocite[Vgl.][S. 5]{curt_collaboration_2004}

\minisec{Haftungssilos und Misstrauenskultur}
Die Fragmentierung wird durch das deutsche Haftungsrecht verstärkt. Da jeder Akteur primär für seinen abgegrenzten Leistungsbereich haftet, entsteht ein System von \enquote{Haftungssilos}. Anstatt Probleme interdisziplinär an den Schnittstellen zu lösen, konzentrieren sich die Akteure auf die eigene Risikoabsicherung und Exkulpanz \autocite[Vgl.][S. 263]{schwab_konfliktkompetenz_2019}. In der Praxis manifestiert sich dies in einer Flut von Behinderungsanzeigen und Bedenkenanmeldungen nach VOB/B, die eher der Dokumentation des gegenseitigen Versagens als dem Projekterfolg dienen. Dieses feindselige Klima (\textit{adversarial relationship}) verhindert die für moderne Steuerungskonzepte wie das Target Value Design notwendige radikale Transparenz und das gegenseitige Vertrauen \autocite[Vgl.][S. 7 f]{latham_constructing_1994}.

Zusammenfassend lässt sich festhalten, dass die traditionellen Rollenbilder in Deutschland eine organisatorische Barriere bilden, die den Fluss von Informationen und Werten unterbricht. Diese systemische Trennung von Geist und Hand ist die primäre Ursache für die mangelnde Steuerbarkeit komplexer Bauvorhaben.

\subsection{ Die regulatorische Rahmenbedingungen HOAI und VOB}
\label{sec:2.1.2}

Während die Rollenbilder der Akteure das soziale Gefüge definieren, bilden die Honorarordnung für Architekten und Ingenieure (HOAI) sowie die Vergabe- und Vertragsordnung für Bauleistungen (VOB) das regulatorische Gerüst der konventionellen Projektabwicklung. Diese Regelwerke begünstigen eine sequenzielle Prozesslogik, die iterativen Optimierungsprozessen strukturell entgegensteht.

\minisec{Die HOAI als Prozessstandard}
Trotz der Novellierung im Jahr 2021, durch welche die verbindlichen Mindest- und Höchstsätze infolge der Rechtsprechung des Europäischen Gerichtshofes entfielen, bleibt die HOAI als \enquote{empfohlene} Honorarstruktur das maßgebliche Prozessmodell der deutschen Baupraxis. Die Unterteilung in neun Leistungsphasen (LPH) schafft eine starre zeitliche Abfolge, die in der Literatur häufig als Wasserfall-Modell charakterisiert wird \autocite[Vgl. S.435 f]{girmscheid_projektabwicklung_2016}.

\begin{figure}[htbp]
    \centering
    % \includegraphics[width=\textwidth]{Abbildung_HOAI_Phasen.pdf}
    \caption{Sequenzielle Abfolge der HOAI-Leistungsphasen (Eigene Darstellung)}
    \label{fig:hoai_phasen}
\end{figure}

Die Phasen lassen sich grob in drei Blöcke unterteilen:
\begin{itemize}
    \item \textbf{Planungsphase (LPH 1--4):} Von der Grundlagenermittlung bis zur Genehmigungsplanung wird das Projekt primär abstrakt und theoretisch definiert.
    \item \textbf{Vorbereitungsphase (LPH 5--7):} Die Ausführungsplanung sowie die Vorbereitung und Mitwirkung bei der Vergabe bilden die Schnittstelle zum Markt.
    \item \textbf{Realisierungsphase (LPH 8--9):} Die Objektüberwachung und Dokumentation begleiten die physische Errichtung.
\end{itemize}

Diese Phasentrennung führt dazu, dass Planungsergebnisse zu starren Meilensteinen \enquote{eingefroren} werden müssen, um die darauf basierenden Honoraransprüche und Genehmigungsschritte zu sichern. Nachträgliche Anpassungen oder die Rückkehr zu bereits abgeschlossenen Phasen für Optimierungszwecke werden systemisch als \enquote{Störung} oder kostenpflichtige \enquote{Umplanung} gewertet, anstatt als integraler Teil der Wertschöpfung begriffen zu werden \autocite[Vgl.][§~34 i.\,V.\,m.~Anl.~10]{hoai_honorarordnung_2021}.

\minisec{Starrheit der Kostensteuerung nach DIN 276}
Die konventionelle Kostenermittlung orientiert sich an der DIN 276, welche Kosten nach Kostengruppen (Bauteilen) strukturiert (z.B. KG 300 Bauwerk - Baukonstruktion).\autocite[Vgl.][S. 13 ]{din_276_din_2018} Diese bauteilorientierte Sichtweise erschwert eine interdisziplinäre Funktionsoptimierung. In der konventionellen Praxis dient die DIN 276 primär als Instrument zur ex-post-Dokumentation. Da die Kostenverantwortung innerhalb der gewerkspezifischen Planungs-Silos verbleibt, können Budgetüberschreitungen oft erst dann festgestellt werden, wenn der Planungsgrad bereits so weit fortgeschritten ist, dass korrigierende Eingriffe hohe Änderungskosten verursachen. \autocite[Vgl.][S. 3 f]{becker_integrierte_2022}

\minisec{Vertragsarten und das Risiko des Bau-Solls}
Das vertragliche Fundament bildet in der Regel die VOB/B, welche die Geschäftsbedingungen für die Ausführung regelt. Hierbei dominieren zwei Vertragsarten, die jeweils unterschiedliche Risikoprofile aufweisen:

\begin{itemize}
    \item \textbf{Einheitspreisvertrag (EPV):} Die Vergütung erfolgt auf Basis tatsächlich erbrachter Massen und festgelegter Einheitspreise aus einem Leistungsverzeichnis (LV). Hier trägt der Bauherr das Mengenrisiko. Bei lückenhafter Planung bietet dieser Vertragstyp für Auftragnehmer Anreize zur Mengenmaximierung und zum Nachtragsmanagement \autocite[Vgl.][§~2~Abs.~2]{vobb_din_2016}.
    \item \textbf{Pauschalpreisvertrag (PPV):} Das Honorar wird für eine vorab definierte Gesamtleistung fixiert. Während dies dem Bauherrn vermeintliche Preissicherheit bietet, verlagert es das Risiko unvorhergesehener Massenänderungen auf den Auftragnehmer. Dies führt in der Praxis häufig zu hohen Risikoaufschlägen in der Kalkulation und einer aggressiven Abwehrhaltung gegenüber jeglichen Änderungen im Projektverlauf.\autocite[Vgl.][§~2~Abs.~7]{vobb_din_2016}
\end{itemize}

Beide Vertragsformen basieren auf der Fiktion eines zum Zeitpunkt der Vergabe abschließend definierten \enquote{Bau-Solls}. Diese Annahme zwingt die Beteiligten dazu, den Entwurf frühzeitig festzuschreiben, was Flexibilität unterbindet und die Basis für die in der Bauwirtschaft weit verbreitete Konfliktkultur legt.

\minisec{Vergaberechtliche Restriktionen}
Ergänzt wird dieses Korsett durch das Vergaberecht (insb. GWB und VgV), welches insbesondere öffentliche Auftraggeber zur Losvergabe und zum Preiswettbewerb zwingt. Diese regulatorischen Vorgaben erschweren die frühzeitige Einbindung von Ausführungswissen (\textit{Early Contractor Involvement}), da eine Zusammenarbeit vor der formalen Vergabe oft als wettbewerbsverzerrende \enquote{Vorbefasstheit} gewertet wird. \autocite[Vgl.][§~97~Abs.~1~u.~2]{noauthor_gesetz_2013}\,\autocite[Vgl.][§~7~Abs.~1~u.~2]{vgv_verordnung_2016}Damit wird das in 2.1.1 beschriebene Wissens-Vakuum in der Entwurfsphase rechtlich institutionalisiert.

Zusammenfassend lässt sich festhalten, dass HOAI und VOB ein starres System schaffen, das auf Misstrauen und exakter Abgrenzung von Leistungen basiert. Eine proaktive, wertorientierte Steuerung, die Flexibilität und Kooperation voraussetzt, ist innerhalb dieser regulatorischen Leitplanken kaum realisierbar.

\subsection{Die Konsequenzen: Fragmentierung und ökonomische Fehlanreize}
\label{sec:2.1.3}

Die in den vorangegangenen Abschnitten beschriebene Trennung der Akteure (\cref{sec:2.1.1}) sowie die starre prozessuale Einbindung durch HOAI und VOB (\cref{sec:2.1.2}) führen in der Summe zu einer systemischen Dysfunktionalität. Diese manifestiert sich sowohl in ökonomischen Kennzahlen als auch in einer spezifischen, oft projektkritischen Kommunikationskultur. In der Literatur wird dieser Zustand häufig als Krise der Bauwirtschaft charakterisiert, die einen grundlegenden Paradigmenwechsel erforderlich macht. \autocite[Vgl.][S.~7 ff]{latham_constructing_1994}\, \autocite[Vgl.][S.~42--44]{habib_alternative_2020}

\minisec{Ökonomie der Nachträge und Claim-Management}
Ein zentrales Merkmal konventioneller Abwicklungsmuster ist die Verschiebung des unternehmerischen Fokus weg von der reinen Bauleistung hin zur Identifikation von Planungs- und Ausschreibungsmängeln. In einem Marktumfeld, das durch das Billigstbieterprinzip und geringe Margen geprägt ist, wird das sogenannte \textbf{Claim-Management} für viele Bauunternehmen zu einer existenziellen Notwendigkeit. \autocite[Vgl.][S.~203 ff]{girmscheid_projektabwicklung_2016}

Da das vertragliche \enquote{Bau-Soll} aufgrund der sequenziellen Planung (LPH 1--5) zum Zeitpunkt der Vergabe selten fehlerfrei definiert ist, entstehen zwangsläufig Abweichungen während der Ausführung. In der konventionellen Logik werden diese Abweichungen nicht als gemeinsame Chance zur Optimierung begriffen, sondern als Grundlage für Nachtragsforderungen genutzt. Dieser Mechanismus bindet erhebliche Ressourcen auf Seiten aller Beteiligten für die Prüfung, Abwehr und Verhandlung von Mehrkosten, anstatt diese Kapazitäten in die wertschöpfende Planung oder Ausführung zu investieren. \autocite[Vgl.][S.~234 ff]{love_forensic_2008}

\minisec{Die Stagnation der Produktivität und die Komplexitätsfalle}
Die organisatorische Fragmentierung verhindert zudem den Aufbau nachhaltiger Lernkurven und den Transfer von Prozesswissen über Projektgrenzen hinweg. Während andere Industriezweige durch Standardisierung und integrierte Wertschöpfungsketten signifikante Produktivitätssteigerungen erzielen konnten, stagniert die Wertschöpfung pro Erwerbstätigenstunde im deutschen Baugewerbe seit Jahrzehnten.\autocite[Vgl.][S.~7 f]{bmvi_abschlussbericht_2015}

Dieser Rückstand wird durch die massiv gestiegene technologische Komplexität moderner Bauwerke verschärft. Ein Gebäude der Gegenwart ist nicht mehr mit der Baustruktur der 1970er-Jahre vergleichbar; insbesondere die technische Gebäudeausrüstung (TGA) hat eine \enquote{Explosion} der Kosten- und Schnittstellenanteile erfahren. Während die Technik in den 1960er- und 1970er-Jahren oft nur 10\,\% bis 15\,\% der Gesamtkosten ausmachte, liegt die TGA-Quote (Kostengruppe 400 nach DIN 276) bei komplexen Hochbauten wie Kliniken oder modernen Bürogebäuden heute häufig zwischen 40\,\% und 50\,\%. \autocite[Vgl.][S.~113 ff]{sommer_projektmanagement_2016} Das konventionelle Modell versucht jedoch, diese hochgradig vernetzte Technik mit denselben sequenziellen und gewerkeorientierten Methoden zu steuern, die für den vergleichsweise simplen Mauerwerksbau des letzten Jahrhunderts konzipiert wurden. Da die Planung der TGA oft isoliert erfolgt, verpuffen potenzielle Effizienzgewinne aus Innovationen wie dem \textit{Building Information Modeling} (BIM) an den starren vertraglichen Schnittstellen der Leistungsphasen \autocite[Vgl.][S. 58 ff]{becker_integrierte_2022}.

\minisec{Kulturelle Folgen: Die \enquote{Adversarial Culture}}
Die vielleicht schwerwiegendste Konsequenz ist die Entstehung einer tief verwurzelten Misstrauenskultur, die international als \textit{Adversarial Culture} bezeichnet wird. Die konventionelle Struktur basiert auf der Annahme gegensätzlicher Interessen: Der Gewinn des einen Akteurs wird häufig als der Verlust des anderen wahrgenommen (Nullsummenspiel).

Die \textit{Reformkommission Bau von Großprojekten} stellte bereits 2015 fest, dass diese Atmosphäre der Konfrontation zu einer massiven Risikoaversion führt. Anstatt Risiken gemeinsam proaktiv zu steuern, steht der Versuch im Vordergrund, Risiken vertraglich auf andere Parteien abzuwälzen. Tritt ein Schadensfall ein, steht nicht die Problemlösung, sondern die Entschuldung im Vordergrund. Diese Defensivhaltung untergräbt die für eine wertorientierte Projektsteuerung notwendige Transparenz und verhindert die offene Kommunikation über Fehler und Verbesserungspotenziale. \autocite[Vgl.][S.~42--44]{habib_alternative_2020}\,\autocite[Vgl.][S.~7 f]{bmvi_abschlussbericht_2015}


Zusammenfassend lässt sich feststellen, dass die konventionelle Projektabwicklung in Deutschland in einer strukturellen Sackgasse verharrt. Die Trennung von Planung und Bau, die Starrheit der Regelwerke und die daraus resultierenden ökonomischen Fehlanreize schaffen ein Umfeld, in dem eine zielgerichtete Steuerung von Kosten und Werten systematisch erschwert wird. Diese Analyse bildet die Grundlage für die Forderung nach alternativen, kollaborativen Abwicklungsmodellen, wie sie im folgenden \cref{sec: 2.2} vorgestellt werden.

\clearpage

\section{Integrierte Projektabwicklungsmodelle als alternativer Ansatz}
\label{sec: 2.2}

Die im vorangegangenen Abschnitt aufgezeigten Schwachstellen der konventionellen Projektabwicklung, insbesondere die organisatorische Fragmentierung und die daraus resultierenden Effizienzverluste, sind kein einzigartiges Problem der deutschen Bauwirtschaft. Historisch betrachtet lassen sich Parallelen zur stationären Industrie ziehen, wo ähnliche Herausforderungen zur Entwicklung der Lean-Philosophie führten.

\subsection{Lean-Management als Wegbereiter kollaborativer Modelle}
\label{sec:2.2.1}

Der Ursprung dieses Denkansatzes liegt in der japanischen Automobilindustrie der Nachkriegszeit. Da Toyota das amerikanische Modell der Massenproduktion aufgrund begrenzter Ressourcen nicht adaptieren konnte, entwickelte Taiichi Ohno das Toyota-Produktionssystem (TPS). Dessen Kernziel war die radikale Eliminierung jeglicher Verschwendung (\textit{Muda}) durch eine konsequente Fluss-Orientierung der Produktion \autocite[Vgl.][ S. 33]{ohno_toyota-produktionssystem_2013}.

Die Übertragung dieser Prinzipien auf das Bauwesen wurde 1992 durch Lauri Koskela wissenschaftlich fundiert. In seinem wegweisenden Report \textit{Application of the New Production Philosophy to Construction} etablierte er die sogenannte \textbf{TFV-Theorie} (Transformation, Flow, Value). Er legte dar, dass Bauproduktion nicht nur als Transformation von Material zu Bauwerk verstanden werden darf, sondern vor allem unter den Aspekten des \enquote{Flusses} (\textit{Flow}) und der \enquote{Wertschöpfung} (\textit{Value}) gesteuert werden muss \autocite[Vgl.][ S. 15]{koskela_application_1992}. Während die Transformation oft lokal effizient ist, liegen in den komplexen Übergängen zwischen Gewerken und Phasen massive Anteile an nicht-wertschöpfenden Tätigkeiten wie Warten oder Nacharbeiten \autocite[Vgl.][ S. 16]{haghsheno_lean_2019}.

Die Etablierung von \textit{Lean Construction} führte zur Entwicklung operativer Methoden (wie dem \textit{Last Planner System}), um diesen Prozessfluss zu stabilisieren. In der Praxis zeigte sich jedoch eine systemische Grenze: Die Optimierung des Gesamtflusses steht oft im Widerspruch zu den lokalen Gewinninteressen der Einzelakteure, die durch konventionelle, bilaterale Verträge motiviert sind \autocite[Vgl.][ S. 12]{becker_integrierte_2022}. Solange Verträge auf der Abwälzung von Risiken basieren, können Lean-Methoden ihre volle Wirkung nicht entfalten.

Die historische Antwort auf dieses Dilemma lieferte der Offshore-Sektor: Beim \textit{BP Andrew Project} in den 1990er-Jahren wurde erstmals eine Allianz gebildet, bei der alle Partner am finanziellen Gesamterfolg beteiligt waren (\textit{Pain-Share/Gain-Share}), anstatt Risiken bilateral abzuwälzen \autocite[Vgl.] [S. 2]{ross_introduction_2000}. Dieses als \textit{Project Alliancing} bekannt gewordene Modell bewies, dass die vertragliche Integration der Schlüssel zur massiven Kostenreduktion ist. In der Folge etablierte sich das Modell schnell in Australien und Finnland – hier insbesondere durch die erfolgreiche Anwendung im öffentlichen Infrastrukturbau \autocite[Vgl.][S. 58]{lahdenpera_making_2012}. Inspiriert durch diese Erfolge prägte das American Institute of Architects (AIA) 2007 den Begriff \textit{Integrated Project Delivery} (IPD) für den Hochbau. \autocite[Vgl.][S. 2 f]{aia_integrated_2007} Im deutschsprachigen Raum hat sich hierfür die Bezeichnung \textbf{Integrierte Projektabwicklung (IPA)} etabliert, die die Anwendung dieser kollaborativen Prinzipien und Vertragsstrukturen im Kontext der deutschen Bauwirtschaft beschreibt \autocite[Vgl.][S. 28]{haghsheno_lean_2019}.

\clearpage


\subsection{Prinzipien und Merkmale der Integrierten Projektabwicklung}
\label{sec:2.2.2}

Da die Integrierte Projektabwicklung in Deutschland kein gesetzlich definierter Begriff ist, bedarf es einer präzisen Abgrenzung zu konventionellen oder reinen Partnering-Modellen. Wissenschaftlich und praktisch hat sich hierfür eine Definition über spezifische Strukturmerkmale durchgesetzt. Ein Projekt gilt erst dann als IPA-Projekt, wenn die konstitutiven Elemente vertraglich und organisatorisch verankert sind und somit eine verbindliche \enquote{Schicksalsgemeinschaft} entsteht.

\minisec{Das Haus der IPA}
Zur Veranschaulichung der Wirkungszusammenhänge hat sich in der deutschsprachigen Literatur das Bild des \enquote{Hauses der IPA} etabliert. Es ordnet die essentiellen Bestandteile einer logischen Struktur zu: \autocite[Vgl.] [S. 73]{haghsheno_lean_2019}

\begin{figure}[htbp]
    \centering
    \fcolorbox{gray!50}{white}{%
        \includegraphics[width=\dimexpr\textwidth-2\fboxsep-2\fboxrule\relax, height=0.5\textwidth]{05_figures/Haus_der_IPA.jpg}
    }
    \caption[Das Haus der IPA]{Das Haus der IPA zur Darstellung der systemischen Abhängigkeiten (Quelle: Eigene Darstellung in Anlehnung an Haghsheno 2019).}
    \label{fig:haus_der_ipa}
\end{figure}

\begin{itemize}
    \item Das \textbf{Fundament} bildet der Mehrparteienvertrag, der die rechtliche Basis schafft.
    \item Die \textbf{Säulen} bestehen aus dem gemeinsamen Risikomanagement (ökonomische Anreize) und der kooperativen Zusammenarbeit (Kultur und Methoden).
    \item Das \textbf{Dach} symbolisiert die gemeinsamen Projektziele (\enquote{Best for Project}), die über den Einzelinteressen der Firmen stehen.
\end{itemize}

\subsubsection{Wesentliche Charakteristika nach IPA-Zentrum}
Um eine einheitliche Einordnung zu ermöglichen, definiert das IPA-Zentrum acht Charakteristika, die kumulativ vorliegen müssen, um von einem IPA-Projekt zu sprechen. Ihnen sind insgesamt 21 konstitutive Modellbestandteile zugeordnet \autocite[Vgl.][ S. 4 ff]{bold_integrierte_2022}:

\begin{enumerate}
    \item \textbf{Etablierung eines Mehrparteiensystems:} Einbindung aller Schlüsselbeteiligten in ein gemeinsames Vertragswerk.
    \item \textbf{Frühzeitige Einbindung mittels Kompetenzwettbewerb:} Auswahl der Partner basierend auf Qualifikation (statt reinem Preis) zu einem Zeitpunkt, der eine gemeinsame Zielkostenermittlung ermöglicht.
    \item \textbf{Gemeinsames Risikomanagement:} Identifikation, Bewertung und Tragung von Risiken im Kollektiv.
    \item \textbf{Gemeinsame Entscheidungen:} Prinzip der Einstimmigkeit und \enquote{Best for Project}-Ausrichtung in einer integrierten Organisation.
    \item \textbf{Anreizsystem im Rahmen eines Vergütungsmodells:} Erstattung der Selbstkosten plus ein erfolgsabhängiges Honorar (\textit{Pain-Share/Gain-Share}).
    \item \textbf{Einsatz kollaborativer Arbeitsmethoden:} Nutzung transparenzfördernder Methoden wie BIM und Lean Construction.
    \item \textbf{Lösungsorientierte Konfliktbearbeitung:} Interne Mechanismen zur schnellen Beilegung und Verzicht auf gegenseitige Haftungsansprüche (\textit{No-Blame}).
    \item \textbf{Kooperative Haltung der Beteiligten:} Ausrichtung auf gemeinsame Werte wie Offenheit, Vertrauen und Fehlertoleranz.
\end{enumerate}

Um das Zusammenspiel von Target Value Design und IPA zu verstehen, müssen die operativen und vertraglichen Bestandteile, speziell das Anreizsystem, im Folgenden detailliert betrachtet werden.

\minisec{Integrierte Projektorganisation (IPO) und Governance}
Die IPA bricht die klassische Trennung zwischen Bauherr, Planer und Unternehmer auf. Stattdessen wird eine integrierte Projektorganisation (IPO) etabliert, die firmenübergreifend agiert. Im deutschen Mustervertragswesen hat sich hierfür eine dreigliedrige Governance-Struktur durchgesetzt: \autocite[Vgl.][ S. 44 ff.]{cheng_ipa-handlungsleitfaden_2020}
\begin{itemize}
    \item \textbf{\ac{SMT}:} Das strategische Lenkungsgremium, besetzt mit Geschäftsführern der Allianzpartner. Hier werden Grundsatzentscheidungen (z.\,B. Zielkostenanpassungen) einstimmig getroffen.
    \item \textbf{\ac{PMT}):} Die operative Projektleitung, die den Alltag steuert, Ressourcen koordiniert und die Zielerreichung überwacht.
    \item \textbf{\ac{PIT}:} Die operative Arbeitsebene. Hier arbeiten Planer und Ausführende in interdisziplinären Clustern (z.\,B. PIT Hülle, PIT TGA) zusammen. Dies ist der Ort, an dem TVD operativ stattfindet.
\end{itemize}

\minisec{Frühzeitige Einbindung (ECI) und der MacLeamy-Effekt}
Ein konstitutives Merkmal der IPA ist die vertraglich gesicherte Einbindung der ausführenden Unternehmen (\enquote{Makers}) bereits in der frühen Entwurfsphase (LPH 2/3). Die theoretische Begründung für diese Abkehr vom sequenziellen Prozess liefert die MacLeamy-Kurve (\cref{fig:macleamycurve}).

\begin{figure}[H]
    \centering
    \fcolorbox{gray!50}{white}{%
        \includegraphics[width=\dimexpr\textwidth-2\fboxsep-2\fboxrule\relax]{05_figures/MacLeamy-Curve.png}%
    }
    \caption[MacLeamy-Kurve]{MacLeamy-Kurve: Verschiebung des Planungsaufwands zur Nutzung der maximalen Beeinflussbarkeit.\protect\footnotemark}
    \label{fig:macleamycurve}
\end{figure}
\footnotetext{Quelle: \url{https://education.buildingsmart.org/de/a-new-way-of-working}, eigene Darstellung in Anlehnung an MacLeamy (2004).}

Die Grafik verdeutlicht das Dilemma konventioneller Prozesse (dargestellt als Kurve 3): Der Schwerpunkt des Ressourcenaufwands liegt hier oft erst in der Werkplanungs- und Bauphase. Zu diesem späten Zeitpunkt ist die Fähigkeit zur Kostenbeeinflussung (Kurve 1) jedoch bereits drastisch gesunken, während die Kosten für Änderungen (Kurve 2) exponentiell ansteigen. 

IPA adressiert dieses strukturelle Defizit durch ein gezieltes \enquote{Frontloading} (Kurve 4). Durch das \textit{Early Contractor Involvement} (ECI) werden Entscheidungen und Planungsaufwände bewusst in die frühen Phasen vorverlagert. Damit wird das in Kapitel \ref{sec:2.1.1} beschriebene \enquote{Wissens-Vakuum} geschlossen: Die Expertise der Ausführung fließt genau dann in das Projekt ein, wenn sie mit minimalen Kosten den maximalen Wertbeitrag leisten kann. \autocite[Vgl.][S. 4]{curt_collaboration_2004}

\minisec{Der Mehrparteienvertrag als Fundament}
Rechtlich wird die Allianz durch einen Mehrparteienvertrag besiegelt. Im Gegensatz zur bilateralen V-Struktur unterzeichnen alle Schlüsselpartner (Bauherr, Planer, Hauptgewerke) denselben Vertrag. Dies schafft eine rechtliche \enquote{Klammer}, die opportunistisches Verhalten unterbindet, da Vertragsbrüche oder Klagen nicht gegen einen \enquote{Gegner}, sondern gegen das eigene Team gerichtet wären \autocite[vgl. S. 5]{bold_integrierte_2022}.

\minisec{Das kommerzielle Modell: Pain/Gain-Share Mechanismus}
Das Herzstück der \ac{IPA} ist das Inzentivierungsmodell, das die ökonomischen Interessen aller Beteiligten synchronisiert (\textit{Alignment of Interests}). Anstatt Gewinne durch Nachträge oder Mengensteigerungen zu maximieren, partizipieren alle Partner am Gesamterfolg des Projekts. Dieses Modell basiert typischerweise auf einer Drei-Säulen-Struktur (3-Limb-Model): \autocite[Vgl.][S. 33 ff.]{becker_integrierte_2022}

\begin{enumerate}
    \item \textbf{Limb 1 -- Erstattung der Selbstkosten:} Das Fundament der Vergütung bilden die direkten projektbezogenen Kosten (Personal, Material, Nachunternehmer, Geräte). Diese werden den Partnern zu 100\,\% transparent erstattet. Dies sichert die Liquidität der Unternehmen, bietet jedoch keinerlei Gewinnmarge.
    
    \item \textbf{Limb 2 -- Gemeinkostenzuschlag:} Auf die Selbstkosten wird ein fixierter Zuschlag für die allgemeinen Verwaltungskosten (Overhead) der Unternehmen gezahlt. Zusammen mit Limb 1 deckt dies die operativen Kosten (\enquote{Cost to survive}), beinhaltet aber noch keinen Unternehmergewinn.
    
    \item \textbf{Limb 3 -- Risiko-Gewinn-Pool:} Der entscheidende Hebel ist die dritte Säule. Der kalkulierte Gewinn (\textit{Fee}) aller Partner wird nicht garantiert ausgezahlt, sondern fließt zunächst in einen virtuellen Risikotopf, den sogenannten \textit{Risk Pool}. Dieser Pool dient als Puffer für das Projektrisiko.
\end{enumerate}

\definecolor{tu-green-dark}{RGB}{79,134,38}

\begin{tikzpicture}[
    box/.style={
        rectangle,
        minimum width=2cm,
        text width=1.8cm,
        align=center,
        font=\footnotesize
    },
    small box/.style={
        box,
        minimum height=0.8cm,
        fill=tu-green-dark,
        text=white
    },
    large box/.style={
        box,
        minimum height=4cm,
        fill=tu-green,
        text=white
    },
    dashed arrow/.style={
        ->,
        dashed,
        thick,
        black!60
    }
]

% Fall 1: Basis-Zielkosten
\node[small box] (fall1-top) at (0,0) {Gewinn\\CRP};
\node[large box, below=0cm of fall1-top] (fall1-bottom) {Direkte\\Kosten};
\node[below=0.2cm of fall1-bottom, font=\footnotesize\bfseries] (fall1-label) {Fall 1: Basis-\\Zielkosten};

% Fall 2: Mehrkosten
\node[small box, right=0.8cm of fall1-top] (fall2-top) {Gewinn\\CRP};
\node[large box, below=0cm of fall2-top] (fall2-bottom) {Direkte\\Kosten};
\node[below=0.2cm of fall2-bottom, font=\footnotesize\bfseries] (fall2-label) {Fall 2: Mehrkosten};

% Fall 3: Kosteneinsparung
\node[small box, right=0.8cm of fall2-top] (fall3-top1) {Bonusteilung};
\node[small box, below=0cm of fall3-top1] (fall3-top2) {Bonusteilung};
\node[small box, below=0cm of fall3-top2] (fall3-top3) {Gewinn\\CRP};
\node[large box, below=0cm of fall3-top3] (fall3-bottom) {Direkte\\Kosten};
\node[below=0.2cm of fall3-bottom, font=\footnotesize\bfseries] (fall3-label) {Fall 3: Kosteneinsparung};

% Fall 4: Tatsächliche Kosten
\node[large box, minimum height=6.4cm, right=0.8cm of fall3-top1] (fall4) {Direkte\\Kosten};
\node[below=0.2cm of fall4, font=\footnotesize\bfseries] (fall4-label) {Fall 4: Tatsächliche\\Kosten};

% Pfeile
\draw[dashed arrow] (fall1-top.east) -- (fall2-top.west);
\draw[dashed arrow] (fall2-top.east) -- (fall3-top3.west);
\draw[dashed arrow] (fall3-top1.east) -- (fall4.north west);

% Grauer Rahmen um alles (vergrößert)
\begin{scope}[on background layer]
    \draw[gray!50, thick, rounded corners=3pt] 
        ([shift={(-0.5,0.8)}]fall1-top.north west) rectangle 
        ([shift={(0.5,-0.5)}]fall4-label.south east);
\end{scope}

% Beschriftung unten mit Label
\node[below=0.5cm of fall2-label, font=\footnotesize] (caption) {
    \textbf{Abb. 3.5} \quad Vergütungsstufen in der IPA, Vgl. Becker, 2022, S.22
};

% Label für Referenzierung
\label{fig:ipa_verguetung}

\end{tikzpicture}

Wie die \textit{Abbildung 3.5} verdeutlicht, ist die finale Auszahlung dieses Pools an die Einhaltung der Zielkosten (\textit{Target Cost}) gekoppelt. Dabei greift der Pain-Share/Gain-Share-Mechanismus:
\begin{itemize}
    \item \textbf{Gain-Share (Chance):} Gelingt es der Allianz durch Methoden wie TVD, die tatsächlichen Kosten unter die Zielkosten zu senken, wächst der Pool an. Dieser Überschuss wird am Projektende nach einem definierten Schlüssel (z.\,B. 50/50) zwischen Bauherr und Allianzpartnern aufgeteilt. Die Marge der Unternehmen steigt über den ursprünglich kalkulierten Wert.
    \item \textbf{Pain-Share (Risiko):} Übersteigen die tatsächlichen Kosten den Zielwert, müssen diese Mehrkosten aus dem Risk Pool gedeckt werden. Die Gewinne der Partner schmelzen ab, um das Budget des Bauherrn zu schützen. Erst wenn der gesamte Pool aufgebraucht ist (d.\,h. alle Partner arbeiten zum Selbstkostenpreis), greift in der Regel eine Verlustabsicherung durch den Bauherrn (\textit{Catastrophic Loss Protection}).
\end{itemize}

Durch diesen Mechanismus entsteht eine direkte wirtschaftliche Abhängigkeit: Kein Partner kann auf Kosten anderer gewinnen. Optimierungen im Prozess (Limb 1) schützen und vergrößern direkt den eigenen Gewinn (Limb 3).

\minisec{Open Book Accounting}
Voraussetzung für dieses Vergütungsmodell ist radikale Transparenz. Alle Partner müssen ihre Kalkulationen offenlegen (\textit{Open Book}). Nur durch diese Transparenz ist eine gemeinsame Steuerung der Kosten im TVD-Prozess überhaupt möglich, da Entscheidungen auf echten Kosten (\textit{Actual Cost}) und nicht auf taktischen Preisen basieren \autocite[Vgl.][ S. 9]{bold_integrierte_2022}.

\minisec{Einstimmigkeitsprinzip}
Entscheidungen im \ac{SMT} müssen einstimmig getroffen werden. Da durch den Pain/Gain-Mechanismus alle im selben ökonomischen Boot sitzen, verhindert dies Blockaden: Jeder Partner hat ein intrinsisches Interesse an einer Lösung, da Stillstand Geld kostet \autocite[Vgl.][ S. 7]{bold_integrierte_2022}.

\minisec{Haftungsbeschränkung (No-Blame)}
Um eine offene Fehlerkultur zu ermöglichen, verzichten die Partner vertraglich auf die Geltendmachung von Schadensersatzansprüchen untereinander (außer bei Vorsatz/grober Fahrlässigkeit). Dies verhindert die im konventionellen Bauen übliche \enquote{Absicherungs-Bürokratie} und lenkt die Energie auf die Problemlösung statt auf die Schuldsuche \autocite[Vgl.][ S. 10]{bold_integrierte_2022}.





\subsection{Internationale Erfolge und Status Quo in Deutschland}
\label{sec:2.2.3}

Die Wirksamkeit integrierter Abwicklungsmodelle ist international empirisch gut belegt. Insbesondere im anglo-amerikanischen Raum und in Skandinavien hat sich gezeigt, dass die vertragliche Synchronisation der Interessen zu messbaren Verbesserungen der Projektperformance führt.

\minisec{Empirische Evidenz aus dem internationalen Raum}
Umfangreiche Untersuchungen im US-amerikanischen Markt belegen die ökonomische Überlegenheit des integrierten Ansatzes. Eine Analyse von 47 Projekten zeigte, dass Vorhaben, die konsequent kollaborativ gesteuert wurden (mittels Target Value Design), im Durchschnitt \textbf{15 bis 20\,\% unter dem Marktpreis} realisiert werden konnten. Im Gegensatz zu konventionellen Projekten, die häufig unter Kostenüberschreitungen leiden, blieben diese Projekte nicht nur im Budget, sondern unterschritten es signifikant, ohne Abstriche bei der Qualität oder den Terminen zu machen \autocite[Vgl.][S. 176]{do_target_2014}.

Auch in Australien, wo das Modell als \textit{Project Alliancing} bei großen Infrastrukturmaßnahmen etabliert ist, bestätigen Auswertungen, dass die \textit{Adversarial Culture} erfolgreich durch eine Partnerschaft auf Augenhöhe ersetzt werden konnte. Dies führte in der Praxis dazu, dass Rechtsstreitigkeiten zwischen den Vertragspartnern nahezu vollständig eliminiert wurden \autocite[Vgl.][ S. 60]{lahdenpera_making_2012}.

\paragraph{Status Quo der Implementierung in Deutschland}
In Deutschland verzeichnet die Anwendung von IPA derzeit ein deutliches Wachstum. Während das Modell international – insbesondere in den USA – primär durch private Bauherren entwickelt wurde, stellt sich die Situation in Deutschland anders dar: Hier fungieren vor allem öffentliche Auftraggeber und Sektorenauftraggeber als wesentliche Treiber der Entwicklung. Insbesondere Infrastrukturbauprojekte bilden aktuell größten Anteil an IPA-Projekten und fungieren als wesentlicher Treiber dieser Entwicklung.\autocite[Vgl.][S. 10]{haghsheno_ipa-report_2025}

\begin{figure}[htbp]
    \centering
    % Die fcolorbox erzeugt den Rahmen
    \fcolorbox{gray!50}{white}{%
        % Die Minipage zwingt die Box auf die exakte Textbreite (abzüglich Rahmenstärke)
        \begin{minipage}{\dimexpr\textwidth-2\fboxsep-2\fboxrule\relax}
            \centering
            \begin{tikzpicture}
                \begin{axis}[
                    ybar,
                    % Diagramm füllt die Minipage aus
                    width=0.9\linewidth, 
                    height=7cm,
                    symbolic x coords={2023, 2024, 2025},
                    xtick=data,
                    nodes near coords,
                    nodes near coords align={vertical},
                    ymin=0,
                    enlarge x limits=0.25,
                    ylabel={Anzahl IPA-Projekte (kumuliert)},
                    xlabel={Jahr / Erhebungszeitraum},
                    ymajorgrids=true,
                    axis on top=false,
                    bar width=1.8cm,
                ]
                    % HIER wird die Farbe explizit zugewiesen:
                    \addplot[fill=tu-green, draw=tu-green!80!black] coordinates {(2023, 17) (2024, 35) (2025, 43)};
                \end{axis}
            \end{tikzpicture}
        \end{minipage}%
    }
    \caption[Entwicklung der IPA-Projektanzahl]{Dynamischer Anstieg der identifizierten IPA-Projekte in Deutschland (Quelle: Eigene Darstellung in Anlehnung an IPA-Report 2023--2025).}
    \label{fig:ipa_hochlauf}
\end{figure}

Der aktuelle \textit{IPA-Report 2025} belegt anhand dieser wachsenden Zahl von Pilotprojekten, dass die Adaption des Modells auf das deutsche Rechtssystem grundsätzlich machbar ist, wenngleich sie spezifische vertragsrechtliche Lösungen (oft als Verträge eigener Art) erfordert. Die untersuchten Projekte zeigen eine deutlich höhere Zufriedenheit der Beteiligten sowie eine verbesserte Risikosteuerung im Vergleich zu konventionellen Referenzprojekten. \autocite[Vgl.][S. 11]{haghsheno_ipa-report_2025}

Dennoch ist die Implementierung kein Selbstläufer: Der Report identifiziert insbesondere die kulturelle Transformation der Akteure und die notwendige Anpassung vergaberechtlicher Prozesse als zentrale Herausforderungen, die einer flächendeckenden Standardisierung noch entgegenstehen.
\minisec{Fazit und Überleitung}
Zusammenfassend bietet IPA den notwendigen vertraglichen und organisatorischen Rahmen, um die systemischen Defizite des Status Quo zu überwinden. Doch der Rahmen allein garantiert noch keine Kostensicherheit. Um die kommerziellen Potenziale des Modells (Pain/Gain-Share) tatsächlich zu heben, bedarf es einer operativen Methode, die die Kosten proaktiv steuert. Diese Methode ist das \textbf{Target Value Design}, dessen Funktionsweise im folgenden Kapitel detailliert untersucht wird.



\clearpage

\section{Target Value Design (TVD) als Kernmethode}
\label{sec:2.3}

Nachdem im vorangegangenen Kapitel der vertragliche und organisatorische Rahmen der Integrierten Projektabwicklung (IPA) aufgespannt wurde, widmet sich dieser Abschnitt dem \textbf{Target Value Design (TVD)} als der zentralen operativen Methode zur Kosten- und Wertsteuerung innerhalb dieses Rahmens. Während IPA die Umgebung schafft, liefert TVD das konkrete Instrumentarium, um die Projektziele aktiv zu erreichen.

\subsection{Begriffliche Einordnung und Definition}
\label{sec:2.3.1}

Zur Hinleitung auf die methodische Vorgehensweise dieser Arbeit ist zunächst eine präzise Begriffsbestimmung erforderlich. In der aktuellen baubetrieblichen Diskussion und Literatur wird der Begriff TVD oft unscharf verwendet; die Grenze zwischen einer bloßen Prozessbeschreibung und einer methodischen Lehre verschwimmt.

Eine Analyse einschlägiger Datenbanken zeigt, dass ein Großteil der Publikationen TVD primär als \enquote{Prozess} beschreibt – also als eine logische Abfolge von Aktivitäten. Glenn Ballard, einer der Begründer des Ansatzes, fasst den Begriff jedoch weiter und bezeichnet TVD als \enquote{Management-Methode}, die eine fundamentale Änderung der Denkweise erfordert. \autocite[Vgl.][S.~2--4]{ballard_target_2025}

Um eine eindeutige Basis für die vorliegende Arbeit zu schaffen, wird folgende Differenzierung vorgenommen:
\begin{itemize}
    \item Eine \textbf{Methode} ist ein systematisches, planmäßiges Vorgehen zur Erreichung eines Ziels, basierend auf festgelegten Prinzipien und Regeln.
    \item Ein \textbf{Prozess} ist die operative Umsetzung dieser Methode in eine zeitliche und logische Abfolge von Aktivitäten (Input $\rightarrow$ Transformation $\rightarrow$ Output).
\end{itemize}

Für diese Arbeit wird TVD folglich als eine \textbf{Management-Methode} definiert, die auf spezifischen Prinzipien (wie \enquote{Target First}) basiert und durch einen \textbf{strukturierten Prozess} operationalisiert wird. Diese Definition legitimiert die in Kapitel 4 folgende, detaillierte Prozessmodellierung: Sie ist die Übersetzung der hier beschriebenen Methode in anwendbare Handlungsanweisungen.

\subsection{Historische Entwicklung}
\label{sec:2.3.2}

Das heutige Verständnis von Target Value Design ist das Ergebnis einer Synthese aus drei historischen Entwicklungssträngen, die Ballard (2025) identifiziert \autocite[Vgl.][Kap. 1]{ballard_target_2025}:

\begin{enumerate}
    \item \textbf{Manufacturing (Japan):} Den Ursprung bildet das \textit{Target Costing} (Genka Kikaku) der japanischen Fertigungsindustrie, insbesondere bei Toyota. Hier wurde das Prinzip etabliert, dass der Marktpreis die erlaubten Kosten diktiert (\enquote{Price minus Profit equals Cost}) und nicht die Addition der Herstellungskosten den Preis bestimmt.
    \item \textbf{Cost Planning (UK):} In den 1950er Jahren entwickelten britische \textit{Quantity Surveyors} Methoden zur frühzeitigen Kostenplanung, um Budgets nicht erst nach dem Entwurf zu prüfen, sondern den Entwurf daran auszurichten.
    \item \textbf{Haahtela-Modell (Finnland):} In den 1980er Jahren entwickelten Yrjön und Kari Haahtela ein Modell, das \enquote{Allowable Costs} (zulässige Kosten) wissenschaftlich fundiert aus den funktionalen Anforderungen der Nutzer herleitete, noch bevor eine Zeichnung existierte.
\end{enumerate}

In den USA, insbesondere durch das \textit{Lean Construction Institute} (LCI), wurden diese Stränge ab den frühen 2000er Jahren mit den Prinzipien der kollaborativen Zusammenarbeit verknüpft. Aus dieser Synthese entstand das Target Value Design als spezifische Adaption für die Anwendung im Bauwesen.

\subsection{Handlungslogik und Kernprinzipien}
\label{sec:2.3.3}

Die Methode TVD unterscheidet sich vom klassischen Entwurfsprozess durch eine Umkehrung der Handlungslogik. Ballard hat diese Logik in fünf \enquote{Cardinal Points} zusammengefasst, die als unverhandelbare Grundsätze gelten \autocite[Vgl.][S. 15]{ballard_target_2012}:

\begin{itemize}
    \item \textbf{Target First (Kosten als Designparameter):} Die Zielkosten sind keine resultierende Größe des Entwurfs, sondern eine feste Eingangsvoraussetzung (\enquote{Constraint}). Es wird entworfen, um das Budget zu treffen, anstatt das Budget an den Entwurf anzupassen.
    \item \textbf{Collaboration (Cross-funktionale Teams):} Planung erfolgt nicht in isolierten Silos, sondern gemeinsam mit den ausführenden Unternehmen (siehe \cref{sec:2.2.2}).
    \item \textbf{Optimize the Whole (Gesamtoptimum):} Entscheidungen werden im Sinne des Projektziels getroffen. Einsparungen in einem Gewerk dürfen nicht zu Mehrkosten in einem anderen führen, die den Gesamtwert schmälern.
    \item \textbf{Continuous Improvement (Iterative Steuerung):} Der Entwurf wird in kurzen Zyklen immer wieder gegen die Zielkosten geprüft und optimiert, nicht erst am Ende einer Leistungsphase.
    \item \textbf{Work on Value:} Der Fokus liegt darauf, den Wert für den Kunden (Value) innerhalb der Kostengrenzen zu maximieren, nicht darauf, die Kosten durch Qualitätsverlust zu minimieren.
\end{itemize}

\subsection{Methoden und Werkzeuge}
\label{sec:2.3.4}

Die operative Umsetzung der Zielkostenerreichung erfordert spezifische Werkzeuge, die die Entscheidungsfindung und Datenverarbeitung steuern. Ballard (2012) beschreibt TVD daher als ein System, das sich auf ein Set integrierter Methoden stützt \autocite[Vgl.][S. 55]{hill_target_2017}:

\minisec{Set-Based Design (SBD)}
Im Gegensatz zum klassischen \textit{Point-Based Design}, bei dem früh eine Lösung gewählt und detailliert wird, hält SBD mehrere Lösungsalternativen (Sets) so lange wie möglich parallel offen. Dies verhindert negative Iterationen und ermöglicht es, Entscheidungen erst dann zu treffen, wenn ausreichende Informationen vorliegen (\enquote{Last Responsible Moment}).\autocite[Vgl.][S.~9--12]{tommelein_target_2016}

\minisec{Choosing By Advantages (CBA)}
CBA ist die bevorzugte Entscheidungsmethode im TVD. Entscheidungen zwischen Alternativen werden nicht basierend auf subjektiven Gewichtungen, sondern ausschließlich anhand der Bedeutung von Vorteilen getroffen. Dies entkoppelt die Diskussion über Kosten von der Diskussion über Qualität und führt zu transparenteren Ergebnissen.\autocite[Vgl.][S.~27--31]{hill_target_2017}


\minisec{A3-Thinking / Reports}
Zur Dokumentation von Problemlösungen und Entscheidungen wird die A3-Methode genutzt. Sie zwingt das Team, komplexe Sachverhalte (Problem, Analyse, Lösung, Maßnahmen) auf einem einzigen Blatt Papier (DIN A3) kondensiert und visuell darzustellen, was die Konsensfindung im \textit{Big Room} beschleunigt.\autocite[Vgl.][S.~41--43]{hill_target_2017}


\minisec{Last Planner System (LPS) in der Planung}
Während das Last Planner System (LPS) in der Praxis häufig ausschließlich mit der Produktionssteuerung auf der Baustelle assoziiert wird, kommt ihm im Kontext von Target Value Design eine zentrale Rolle bereits in der Planungsphase zu. LPS dient hier nicht primär der Terminsteuerung, sondern der gezielten Organisation und Stabilisierung des immateriellen Informations- und Entscheidungsflusses innerhalb des Planungsteams.  

Im Mittelpunkt steht die Frage, welche Informationen, Entscheidungen oder planerischen Ergebnisse zu welchem Zeitpunkt benötigt werden, um nachgelagerte Arbeitsschritte zuverlässig ausführen zu können. Durch die strukturierte Abstimmung von Aufgaben, Abhängigkeiten und Verfügbarkeiten wird Planung als ein kollaborativer Produktionsprozess verstanden, dessen Leistungsfähigkeit aktiv gesteuert werden kann.  

Insbesondere die Prinzipien der zuverlässigen Zusagen, der transparenten Aufgabenklärung und der kontinuierlichen Rückkopplung ermöglichen es, Planungsunsicherheiten frühzeitig sichtbar zu machen und Engpässe im Entscheidungsprozess gezielt aufzulösen. Damit schafft LPS die organisatorischen Voraussetzungen, um iterative Entwurfs- und Kostenzyklen im Sinne des Target Value Design stabil und reproduzierbar umzusetzen, anstatt Planungsfortschritt dem Zufall oder individuellen Arbeitsweisen zu überlassen. \autocite[Vgl.][S.~85--90]{ballard_target_2025}\,\autocite[Vgl.][S.~1--4]{ballard_lean_2003}

\minisec{Building Information Modeling (BIM)}
Im TVD-Kontext fungiert BIM primär als Datenbank für die kontinuierliche Kostenkalkulation. Der Zweck ist die Geschwindigkeit des Feedbacks: \enquote{Wir müssen alle drei Wochen wissen, was es kostet.} Durch modellbasierte Massenermittlung (\textit{Quantity Take-Off}) führen geometrische Änderungen am Entwurf unmittelbar zu aktualisierten Kostenprognosen, was die Iterationszyklen drastisch verkürzt.\autocite[Vgl.][S.~392--395]{zimina_target_2012}

\subsection{Akteure und Abgrenzung}
\label{sec:2.3.6}

Abschließend ist Target Value Design organisatorisch und methodisch von verwandten Ansätzen der Kostensteuerung abzugrenzen. Im Unterschied zum klassischen \textbf{Value Engineering (VE)}, das in der Praxis häufig reaktiv zur Anwendung kommt, wenn Kostenüberschreitungen bereits eingetreten sind, setzt TVD bewusst in einer frühen Projektphase an. Während VE typischerweise auf nachträgliche Kostensenkung durch Reduktion von Qualitäten oder Leistungsumfängen abzielt (\enquote{Cost Cutting}), verfolgt TVD einen präventiven Ansatz, bei dem Kosten-, Qualitäts- und Nutzungsziele von Beginn an integrativ entwickelt und kontinuierlich eingehalten werden.\autocite[vgl.][S.~15]{hill_target_2017}

Ebenso unterscheidet sich TVD grundlegend vom \textbf{Target Costing} der stationären Industrie. Während dort – etwa im Produktionssystem von Toyota – Zielkosten häufig durch hierarchische Vorgaben und Preisdrückung entlang der Lieferkette durchgesetzt werden, basiert TVD im Bauwesen auf einer kooperativen Organisationslogik. Diese ist eingebettet in die Integrierte Projektabwicklung (IPA), die eine gemeinsame Verantwortung für Kosten, Qualität und Termine vorsieht und damit einen strukturellen Bruch mit traditionellen, funktional getrennten Rollenbildern darstellt \autocite[Vgl.][S.~7--9]{tommelein_target_2016}.

Zentral für die operative Umsetzung von TVD ist die Arbeit in interdisziplinären \textbf{Clustern}. Diese Cluster stellen temporäre, leistungsbezogene Arbeitseinheiten dar, die sich jeweils auf einen funktionalen Teilbereich des Projekts beziehen, beispielsweise Tragwerk, Gebäudehülle, technische Gebäudeausrüstung oder Nutzungseinheiten. In einem Cluster sind alle für die jeweilige Teilleistung relevanten Akteure vertreten, typischerweise bestehend aus Planern, ausführenden Unternehmen, Fachingenieuren sowie – sofern erforderlich – Vertretern des Auftraggebers oder des Betriebs.\autocite[Vgl.][S.~49--60]{hill_target_2017}


Die Clusterstruktur verfolgt das Ziel, Entscheidungswissen, Ausführungswissen und Kostenverantwortung räumlich und organisatorisch zusammenzuführen. Anstatt Kostenfragen isoliert durch einzelne Kalkulatoren oder Planungsdisziplinen zu behandeln, wird die Verantwortung für die Einhaltung der Zielkosten kollektiv im Cluster verankert. Entscheidungen über Entwurf, Materialien, Bauverfahren oder technische Lösungen erfolgen damit nicht sequenziell, sondern integrativ und unter kontinuierlicher Rückkopplung zwischen Kosten, Nutzen und Umsetzbarkeit.\autocite[Vgl.][S.~36--38]{hill_target_2017}


TVD ist vor diesem Hintergrund nicht als reine Kostentechnik zu verstehen, sondern als ein übergeordnetes Führungs- und Organisationsprinzip. Es verlagert die Verantwortung für wirtschaftliche Entscheidungen vom Individuum auf das interdisziplinäre Team und schafft damit die strukturelle Voraussetzung, Zielkosten nicht nur zu definieren, sondern aktiv und dauerhaft einzuhalten.\autocite[Vgl.][S.~15--16]{hill_target_2017}
