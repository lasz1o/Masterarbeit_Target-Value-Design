\chapter{Grundlagen}
\label{ch:grundlagen}


\section{Konventionelle Projektabwicklung in der deutschen Baupraxis}
\label{sec: 2.1}
% Ziel: Der Leser soll die Struktur, die Rollenverteilung und die prozessualen Schwachstellen der traditionellen Projektabwicklung verstehen. Dieses Kapitel etabliert das "Problem", für das die Arbeit eine Lösung präsentiert.

% Zielsetzung: Den Leser in das Thema einführen und die prägenden Regelwerke der konventionellen Projektabwicklung in Deutschland benennen.\\
% - Feststellung: Die deutsche Baupraxis wird maßgeblich durch HOAI und VOB geformt.\\
% - Grundprinzip herausarbeiten: Strikte Trennung von Planungs- und Ausführungsleistungen.\\
% - Charakterisierung des Modells: Sequenzieller Ablauf in Phasen.\\

\subsection{Das lineare Prozessmodell der HOAI}
\label{sec:2.1.1}

In der deutschen Baupraxis wird der Projektablauf primär durch die Leistungsphasen der Honorarordnung für Architekten und Ingenieure (HOAI) strukturiert. Dieses Modell folgt der Logik der sequenziellen Abwicklung, die im internationalen Kontext als \textit{Design-Bid-Build} (DBB) bezeichnet wird \autocite[vgl.]{Habib2020}. Kennzeichnend für diese Phasenlogik ist die Annahme, dass ein Projekt durch das sukzessive Abarbeiten vordefinierter Meilensteine – von der Grundlagenermittlung bis zur Objektbetreuung – erfolgreich gesteuert werden kann.

Diese Linearität führt jedoch zu einer massiven \textbf{prozessualen Fragmentierung}. In der Literatur wird dieses Vorgehen oft als \enquote{Wasserfall-Modell} kritisiert, bei dem Informationen lediglich einseitig von einer Phase in die nächste weitergegeben werden \autocite[siehe]{Habib2020}. Für eine effiziente Kosten- und Wertsteuerung ergeben sich daraus drei zentrale Defizite:

\begin{itemize}
    \item \textbf{Das Wissens-Paradoxon (Knowledge Gap):} Die HOAI sieht vor, dass die Planung (LPH 1–4) weitestgehend abgeschlossen sein muss, bevor die ausführenden Unternehmen in den Leistungsphasen 6 und 7 (Vorbereitung und Mitwirkung bei der Vergabe) eingebunden werden \autocite[vgl.]{Habib2020}. Zu diesem Zeitpunkt sind Schätzungen zufolge bereits ca. 80\,\% der Lebenszykluskosten eines Gebäudes durch Entwurfsentscheidungen fixiert \autocite[siehe]{Ballard2025}. Das spezifische Fertigungs- und Marktwissen derer, die das Gebäude letztlich realisieren, bleibt somit in der Phase der größten Beeinflussbarkeit ungenutzt.
    
    \item \textbf{Reaktive statt proaktive Steuerung:} Die Kostenermittlungen erfolgen im konventionellen Modell oft \enquote{post-mortem}. Eine Kostenberechnung am Ende der Entwurfsplanung (LPH 3) dient häufig nur der Dokumentation des Status Quo, statt als aktives Design-Werkzeug zu fungieren. Weichen die Kosten vom Budget ab, führt dies in der linearen Logik unweigerlich zu aufwendigen und teuren Umplanungen (\textit{Rework}), da der Prozess nicht auf iterative Feedbackschleifen ausgelegt ist \autocite[vgl.]{Ballard2025}.
    
    \item \textbf{Schnittstellenverluste und mangelnde Baubarkeit:} Da die Planer in der Entwurfsphase keine Kenntnis über die spezifischen Bauverfahren oder Subunternehmerketten der späteren Auftragnehmer haben, entstehen abstrakte Planungen. Diese müssen in der Ausführungsphase häufig korrigiert werden, was zu Zeitverlusten und Kostensteigerungen führt \autocite[vgl.]{Habib2020}.
\end{itemize}

Zusammenfassend lässt sich festhalten, dass die Phasenlogik der HOAI ein \textbf{Steuerungsdefizit} zementiert: Sie erzwingt weitreichende Entscheidungen zu einem Zeitpunkt, an dem das Wissen über deren Konsequenzen minimal ist, und verhindert Korrekturen zu einem Zeitpunkt, an dem sie noch kostengünstig möglich wären \autocite[Hier ggf. weitere Quelle zu Projektmanagement-Grundlagen einfügen]{...}. Dieses strukturelle Problem bildet die wesentliche Motivation für den Einsatz integrativer Ansätze wie dem Target Value Design, um die Trennung von Planung und Bau zu überwinden.

\subsection{Das Prozessmodell - HOAI Leistungsphasen}
\label{sec: 2.1.1}
Zielsetzung: Das abstrakte "Phasenmodell" mit dem konkreten Ablauf der HOAI-Leistungsphasen füllen und dessen starre, lineare Struktur verdeutlichen.\\
- Prozessstruktur: Wird durch die neun Leistungsphasen der HOAI definiert.\\
- Darstellung des Ablaufs: Inhärent linear und sequenziell (LPH 1 -> LPH 9).\\
- Kritischen Punkt identifizieren: Wesentliche Kosten- und Qualitätsentscheidungen fallen in den frühen Planungsphasen (insb. LPH 2-3).\\
- Problem benennen: Zu diesem frühen Zeitpunkt fehlt die vertraglich gebundene Expertise der Ausführung.\\
- Empfehlung: Visualisierung des linearen Ablaufs durch ein einfaches Diagramm.\\

\subsection{Die Organisationsstruktur - getrennte Silos}
\label{sec: 2.1.2}
Zielsetzung: Die prozessuale Trennung auf die organisatorische Ebene übertragen und das Problem der isolierten Akteure (Silos) einführen.\\
- Organisationsstruktur beschreiben: Klassische Dreieckskonstellation (Bauherr, Planer, Ausführender) mit bilateralen Verträgen.\\
- Problem der "Silos" formulieren: Jeder Akteur ist primär auf die Erfüllung des eigenen Vertrags fokussiert.\\
- Kommunikationswege: Oft nur indirekt über den Bauherrn, nicht disziplinübergreifend.\\

\subsection{Die Konsequenzen - strukturelle Schwachstellen}
\label{sec: 2.1.3}
Zielsetzung: Die negativen, systemischen Folgen der beschriebenen Struktur klar benennen und mit Belegen untermauern. Dies ist der Kern der Problemanalyse.\\
- Schwachstellen auflisten und erläutern:\\
- Späte Einbindung von ausführendem Know-how.\\
- Informationsverluste an den Schnittstellen.\\
- Entstehung eines adversialen Projektklimas (Fokus auf Nachtragsmanagement).\\
- Direkte Verbindung zu Kosten- und Terminüberschreitungen herstellen\\

- Zusammenfassen: Das konventionelle Modell bietet zwar einen klaren rechtlichen Rahmen, weist jedoch inhärente Schwächen auf, die Effizienz und Kooperation behindern.
- Schlussfolgerung: Angesichts aktueller Herausforderungen ist das Modell zunehmend reformbedürftig.\\
- Überleitung formulieren: Hinweis auf die Entwicklung alternativer, international etablierter Projektabwicklungsmodelle als Reaktion auf diese Defizite.

\clearpage









\section{Integrierte Projektabwicklungsmodelle als alternativer Ansatz}
\label{sec: 2.2}

Die im vorangegangenen Abschnitt aufgezeigten Schwachstellen der konventionellen Projektabwicklung – insbesondere die organisatorische Fragmentierung und die daraus resultierenden Effizienzverluste – sind kein isoliertes Problem der deutschen Bauwirtschaft. Historisch betrachtet lassen sich Parallelen zur stationären Industrie ziehen, wo ähnliche Herausforderungen zur Entwicklung der Lean-Philosophie führten.

\subsection{Lean-Management als Wegbereiter kollaborativer Modelle}
\label{sec:2.2.1}

Der Ursprung dieses Denkansatzes liegt in der japanischen Automobilindustrie der Nachkriegszeit. Da Japan, anders als die USA, kaum über natürliche Ressourcen verfügte und der Inlandsmarkt klein war, konnte Toyota das amerikanische Modell der Massenproduktion (Fordismus) mit seinen großen Lagerbeständen nicht kopieren. Aus dieser Not heraus entwickelte Taiichi Ohno das Toyota-Produktionssystem (TPS). Sein Kernziel war die radikale Eliminierung jeglicher Verschwendung (\textit{Muda}) durch eine konsequente Fluss-Orientierung der Produktion.\autocite[vgl. S. 33]{ohno_toyota-produktionssystem_2013}

Lange Zeit galt die Übertragbarkeit dieser Prinzipien auf das Bauwesen als schwierig, da Bauprojekte als Unikate mit stationärer Fertigung verstanden wurden. Dieser theoretische Konflikt wurde 1992 durch Lauri Koskela aufgelöst. In seinem wegweisenden Report \textit{Application of the New Production Philosophy to Construction} legte er dar, dass die Bauproduktion nicht nur als Transformation von Inputs zu Outputs (Material zu Bauwerk) verstanden werden darf. Vielmehr muss sie auch unter den Aspekten des „Flusses“ (Flow) und der „Wertschöpfung“ (Value) betrachtet werden.\autocite[vgl. S. 15]{koskela_application_1992} Während die Transformation in der Bauwirtschaft oft effizient ist, liegen im „Fluss“ – also den Übergängen zwischen Gewerken und Planungsphasen – massive Anteile an nicht-wertschöpfenden Tätigkeiten wie Warten oder Nacharbeiten.\autocite[vgl. S. 16]{haghsheno_lean_2019}

Die Etablierung von \textit{Lean Construction} führte zur Entwicklung diverser Methoden (wie dem Last Planner System), um diesen Prozessfluss zu stabilisieren. In der Praxis zeigte sich jedoch eine systemische Grenze: Die Optimierung des Gesamtprozesses (Flow) steht oft im Widerspruch zu den lokalen Optimierungsinteressen einzelner Akteure, die durch bilaterale Verträge und reine Einheitspreisvergütung motiviert sind.\autocite[vgl. S. 12]{becker_integrierte_2022} Solange Verträge die Trennung fördern, können Lean-Methoden ihre volle Wirkung nicht entfalten.

Die historische Antwort auf dieses Dilemma lieferte nicht die Bauindustrie, sondern der Offshore-Sektor. In den 1990er Jahren stand British Petroleum (BP) vor der Herausforderung, das Ölfeld „Andrew“ in der Nordsee zu erschließen. Mit konventionellen Verträgen waren die Kosten zu hoch, um das Projekt wirtschaftlich durchzuführen. BP initiierte daher einen radikalen Modellwechsel: Anstatt Risiken auf Auftragnehmer abzuwälzen, wurde eine Allianz gebildet, in der alle Partner am finanziellen Erfolg oder Misserfolg (Pain-Share/Gain-Share) beteiligt waren.\autocite[vgl. S. 2]{ross_introduction_2000} Dieses als \textit{Project Alliancing} bekannt gewordene Modell bewies, dass die vertragliche Integration der Schlüssel zur Kostenreduktion ist.

Der Erfolg des Andrew-Projekts blieb nicht unbeachtet. Insbesondere in Australien etablierte sich das Modell unter dem Begriff \textit{Project Alliancing} (PA) schnell als Standard für komplexe öffentliche Infrastrukturprojekte.\autocite[vgl. S. 1]{ross_introduction_2000} Inspiriert durch die australischen Erfahrungen adaptierte die US-amerikanische Bauwirtschaft, getrieben durch private Bauherren im Gesundheitswesen wie Sutter Health, diese Prinzipien für den Hochbau. Dort prägte das American Institute of Architects (AIA) im Jahr 2007 den Begriff \textit{Integrated Project Delivery} (IPD) und definierte ihn erstmals vertraglich.\autocite[vgl.]{becker_integrierte_2022}\, \autocite[vgl.]{AIA_2007__integrated}

Im deutschsprachigen Raum hat sich für diesen Ansatz der Begriff \textbf{Integrierte Projektabwicklung (IPA)} etabliert. Hierbei handelt es sich um die direkte Entsprechung des angloamerikanischen IPD-Modells. IPA ist folglich keine Neuentwicklung, sondern beschreibt die Anwendung derselben kollaborativen Prinzipien und Vertragsstrukturen im Kontext der deutschen Bauwirtschaft.\autocite[vgl. S. 28]{haghsheno_lean_2019}

\clearpage





\subsection{Prinzipien und Merkmale der Integrierten Projektabwicklung}
\label{sec:2.2.2}

Da der Begriff der Integrierten Projektabwicklung in Deutschland – anders als die VOB-basierten Modelle – nicht gesetzlich definiert ist, erfordert die wissenschaftliche Einordnung eine klare Abgrenzung anhand von branchenweit akzeptierten Merkmalen. Das \ac{GLCI} definiert IPA hierbei nicht nur als Vertragsart, sondern als ein ganzheitliches Projektabwicklungsmodell, das vertragliche, kulturelle und ökonomische Aspekte in einem einzigen System integriert.\autocite[vgl. S. 28]{haghsheno_lean_2019}

Zur Veranschaulichung dieser Systematik hat sich in der deutschsprachigen Literatur das Bild des „Hauses der IPA“ etabliert. Dieses Modell visualisiert das Zusammenspiel der verschiedenen Ebenen:\autocite[vgl. S. 29]{haghsheno_lean_2019}
\begin{itemize}
    \item Das \textbf{Fundament} bildet der Mehrparteienvertrag, der alle Schlüsselakteure rechtlich bindet.
    \item Die \textbf{Säulen} bestehen aus dem gemeinsamen Risikomanagement (ökonomische Anreize) und der kooperativen Zusammenarbeit (Kultur und Methoden).
    \item Das \textbf{Dach} symbolisiert die gemeinsamen Projektziele („Best for Project“), die über den Einzelinteressen stehen.
\end{itemize}

Um dieses Modell in der Praxis greifbar und abgrenzbar zu machen – insbesondere gegenüber reinen „Partnering“-Modellen, die oft unverbindlich bleiben –, hat das IPA-Zentrum im Jahr 2022 eine präzisere Definition erarbeitet. Diese unterscheidet zwischen den wesensgebenden \textit{Charakteristika} (z.\,B. Ausrichtung auf den Projekterfolg, kooperative Kultur) und den harten \textit{konstitutiven Modellbestandteilen}. Entscheidend ist hierbei, dass ein Projekt nur dann als IPA-Projekt gilt, wenn alle konstitutiven Modellbestandteile gleichzeitig vorliegen.\autocite[vgl. S. 3]{ipa_zentrum_integrierte_2022}

Um die Beliebigkeit des Begriffs einzuschränken, definiert das IPA-Zentrum eine Liste von konstitutiven Modellbestandteilen. Für die vorliegende Arbeit und das Verständnis von Target Value Design sind insbesondere folgende Elemente hervorzuheben:\autocite[vgl. S. 4--9]{ipa_zentrum_integrierte_2022}

\begin{description}
    \item[Mehrparteienvertrag]
    Mindestens der Bauherr, die wichtigsten Planer und die schlüsselfertigen Ausführungsunternehmen schließen einen einzigen, gemeinsamen Vertrag. Dies verhindert die vertraglichen Schnittstellenprobleme der konventionellen Abwicklung.

    \item[Vergütungsmodell und Risikopool]
    Die Vergütung erfolgt in drei Bausteinen: (1) Die Erstattung der direkten Selbstkosten, (2) ein Gemeinkostenzuschlag und (3) der Gewinn. Der Gewinnanteil ist jedoch nicht garantiert, sondern fließt in einen gemeinsamen Risikopool. Bei Kostenüberschreitungen wird dieser Pool genutzt, um Verluste zu decken („Pain-Share“); bei Unterschreitungen wird der Überschuss an alle ausgezahlt („Gain-Share“).\autocite[vgl. S. 19]{becker_integrierte_2022}

    \item[Open Book Accounting]
    Transparenz ist zwingend: Alle Kosten müssen offengelegt werden. Ohne diese „gläserne Baustelle“ ist eine gemeinsame Kostensteuerung (TVD) methodisch nicht durchführbar.

    \item[Einstimmigkeitsprinzip]
    Entscheidungen im Führungsgremium (Senior Management Team) müssen einstimmig getroffen werden. Dies zwingt die Parteien zur Lösungsfindung und verhindert Machtdemonstrationen des Bauherrn.

    \item[Haftungsverzicht]
    Die Parteien verzichten weitgehend auf gegenseitige Haftungsansprüche bei Fehlern (ausgenommen Vorsatz). Dies fördert eine offene Fehlerkultur, in der Probleme sofort gemeldet statt vertuscht werden.
\end{description}

Haghsheno et al. ordnen diese vielfältigen Elemente in einem wissenschaftlichen Strukturierungsansatz drei logischen Ebenen zu, die sich gegenseitig bedingen:\autocite[vgl. S. 80]{haghsheno_strukturierungsansatz_2022}
\begin{enumerate}
    \item \textbf{Kommerzielle und rechtliche Rahmenbedingungen:} Hierzu zählen der Mehrparteienvertrag und das Anreizsystem (Pain/Gain). Sie bilden das unverzichtbare Gerüst.
    \item \textbf{Organisation und Kultur:} Dies umfasst die integrierte Teamstruktur und das verhaltensbezogene Mindset („Best for Project“).
    \item \textbf{Prozesse und Methoden:} Auf dieser operativen Ebene kommen konkrete Werkzeuge zum Einsatz, um die Ziele zu erreichen.
\end{enumerate}

Diese Einordnung ist zentral für den weiteren Verlauf dieser Arbeit: Während IPA den vertraglichen und kulturellen Rahmen (Ebene 1 und 2) spannt, ist \textbf{Target Value Design} die entscheidende Methode auf der Prozessebene (Ebene 3), um die Kostenziele innerhalb dieses Rahmens aktiv zu steuern.












% IPA als ist das organisatorische und vertragliche "Betriebssystem", um die Lean-Philosophie in Bauprojekten praktisch umzusetzen.\\

% Schlüsseleigenschaften und IPA-Charakteristika:\\
% - Frühzeitige Integration aller Schlüsselpartner (Planer und Ausführende) von Beginn an.\\
% Ein Mehrparteienvertrag, der alle Partner an einen Tisch bindet.\\
% Ein gemeinsames Risikomanagement (Chancen-/Risikopool).\\
% Eine kollaborative Kultur und gemeinsame Entscheidungsfindung.\\
















\subsection{Internationale Erfolge und Status Quo in Deutschland}
\label{sec:2.2.3}

\begin{figure}[htbp]
    \centering
    \fcolorbox{gray!50}{white}{%
        \includegraphics[width=\dimexpr\textwidth-2\fboxsep-2\fboxrule\relax]{05_figures/MacLeamy-Curve.png}%
    }
    \caption[MacLeamy-Kurve]{MacLeamy-Kurve, die den relativen Grad an Kosten und Aufwand über den Lebenszyklus eines Bauwerks zeigt. \autocite[]{Quelle: https://education.buildingsmart.org/de/a-new-way-of-working} }
    \label{fig:macleamycurve}
\end{figure}


International: Hohe Erfolgsquoten im angelsächsischen Raum (USA, UK, Australien mit "Project Alliancing") bei der Einhaltung von Kosten und Terminen (Quelle z.B. cook2014).\\

Deutschland: Das Thema gewinnt stark an Fahrt: es gibt eine wachsende Zahl an Pilotprojekten (Quelle: IPA-Report 2024 von haghsheno2025).\\

Eine zentrale Kernmethode zur Kosten- und Wertsteuerung innerhalb dieser IPA-Modelle ist das Target Value Design.\\

\clearpage

\section{Target Value Design (TVD) als Kernmethode}
\label{sec: 2.3}

\subsection{Begriffliche Einordnung und Definition}
\label{sec: 2.3.1}

% =======================================================
% FÜR ABSCHNITT 2.3.1: TVD - Prozess oder Methode
% =======================================================
%   - Ziel: Eine präzise und trennscharfe Definition von TVD für die
%     vorliegende Arbeit herleiten. Es soll geklärt werden, ob TVD als
%     Prozess, als Methode oder als beides zu verstehen ist, um eine
%     eindeutige Basis für die nachfolgende Analyse zu schaffen.
%
%   - Einstieg: Feststellung, dass die Begriffe "Prozess" und "Methode"
%     in der Literatur zu TVD oft unscharf oder synonym verwendet werden.
%
%   - Begriffsdefinition "Methode": Ein systematisches, geplantes Vorgehen,
%     das auf Prinzipien, Regeln und Werkzeugen basiert, um ein Ziel zu erreichen.
%
%   - Begriffsdefinition "Prozess": Eine logische und zeitliche Abfolge von
%     miteinander verknüpften Aktivitäten zur Umwandlung eines Inputs in einen Output.
%
%   - Anwendung auf TVD:
%     - TVD ist mehr als ein reiner Prozess, da es eine bestimmte Denkweise
%       ("Target First"), Prinzipien (Kollaboration) und Werkzeuge voraussetzt.
%     - Diese übergeordnete Denkweise wird jedoch erst durch einen konkreten,
%       strukturierten Ablauf (Prozess) in der Praxis anwendbar.
%
%   - Fazit & Positionierung: Für diese Arbeit wird TVD als eine
%     *MANAGEMENT-METHODE* verstanden, die durch einen
%     _STRUKTURIERTEN PROZESS_ operationalisiert und umgesetzt wird.
%     Diese Definition legitimiert die nachfolgende "prozessorientierte
%     Darstellung" der Methode.
%
% =======================================================

Zur Hinleitung auf die in \cref{ch:methodik} beschriebene, methodische Vorgehensweise soll an dieser Stelle auf das in dieser Arbeit zu Grunde liegende Verständnis von \ac{TVD} als Prozess bzw. Methode eingegangen werden.\\
In der Literatur ist die Abgrenzung zwischen Prozess und Methode in der aktuellen Diskussion um TVD unscharf. Folgt man dem Ergebnis einer einfachen Stichwortsuche in einschlägigen wissenschaftlichen Datenbanken (z.B. reseachgate.net) so wird schnell deutlich, das in der überwiegenden Mehrheit der Literatur (ca. 70\%) Target Value Design als Prozess verstanden wird.
Glenn Ballard, der als einer Gründer des \ac{TVD} gilt, bezeichnet letzteres hingegen häufig als Managementansatz bzw. Management-Methode \autocite[]{}.\\
% In der allgemeinen Literatur wird ein Prozess als

- Definition Prozess Vs. Methode: Verständnis als Management Methode die durch einen sturkturierten Prozess umgesetzt wird -> legitimiert die nachfolgende prozessoroentierte Darstelung.\\

\subsection{Historische Entwicklung}
\label{sec: 2.3.2}

Inhalt: Herleitung aus den drei historischen Strängen (vgl. Ballard, 2025, Kap. 3 ):\\
Manufacturing: Target Costing bei Toyota (Produktionssystem) .\\
UK: Cost Planning der Quantity Surveyors (1950er) .\\
Finnland: Haahtela-Modell (1980er) .\\
USA: Die Synthese zu "Lean TVD" (ab 2000er) .\\
Ziel: Kosten als Inputgröße statt Outputgröße der Planung.\\

\subsection{Handlungslogik und Kernprinzipien}
\label{sec: 2.3.3}

Target First: Kosten sind eine Design-Voraussetzung, kein Ergebnis.\\

Collaboration: Alle sitzen früh in einem Boot (Cross-funktionale Teams).\\

Optimize the Whole: Projektziel vor Einzelziel.\\

Continuous Improvement: Der Prozess ist nie "fertig", sondern wird iterativ verbessert.\\



\subsection{Methoden und Werkzeuge}
\label{sec: 2.3.4}

Die operative Umsetzung der Zielkostenerreichung erfordert spezifische Methoden, die über die reine Kooperationsbereitschaft (vgl. \cref{sec: 2.2}) hinausgehen. Nach Ballard (2012) stützt sich der TVD-Prozess auf ein integriertes Set an Werkzeugen, welche die Designstrategie, die Entscheidungsfindung und die technische Datenverarbeitung steuern.\autocite[vgl. S. 15]{ballard_target_2012}

\subsection*{Set-Based-Design}
Set-Based Design (SBD): Definition als paralleles Entwickeln von Alternativen bis zum "Last Responsible Moment" (vgl. Ballard, 2025, S. 27; LCI 2016).

\subsection*{Choosing By Advantages}
Choosing By Advantages (CBA): Definition als Entscheidungsmethode basierend auf Vorteilen, nicht Gewichtungen

\subsection*{A3-Methode bzw. Report}
A3 Thinking / Reports: Definition als Problemlösungs- und Konsens-Werkzeug.

\subsection*{Last Planner System}
Last Planner System (LPS) in der Planung: Nutzung zur Steuerung des Design-Workflows (nicht nur Bau).

\subsection*{Building Information Modelling}
Im TVD-Kontext wird \ac{BIM} zur Datenbank für die Kostenkalkulation.\\
Zweck: "Wir müssen alle 3 Wochen wissen, was es kostet."
Funktion: Automatisierter Massenauszug (Quantity Take-Off). Wenn Sie eine Wand im Modell verschieben, ändert sich sofort die m²-Zahl in der Datenbank.\\
Ziel: Geschwindigkeit..\\


\subsection{Akteure und Organisationsstruktur}
\label{sec: 2.3.5}

- Unterschied zu Target Costing (Industrie) und Value Engineering (klassische Planung).\\

- Einordnung in IPA-Kontext.\\

- Relevanz: Warum TVD nicht nur Kostentechnik, sondern Führungsprinzip ist.\\

- TVD als prozessorientierte Managementmethode mit systematischer Kosten- und Wertsteuerung.\\

Zur detaillierten Untersuchung dieses Prozesses im Kontext der deutschen Baupraxis wird im folgenden Kapitel das methodische Vorgehen beschrieben. Dabei wird zunächst das Forschungsdesign erläutert und anschließend dargelegt, wie der idealtypische TVD-Prozess systematisch analysiert und modelliert wird.

\clearpage