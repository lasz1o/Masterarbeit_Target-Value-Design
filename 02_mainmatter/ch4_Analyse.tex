\chapter{Analyse und Modellierung}
\label{ch:4}

\section{Das Phasenmodell}
\label{sec:4.1}

Mit Blick in die Literatur fällt auf, dass eine zusammenhängende Darstellung des \ac{TVD}-Gesamtprozesses oder die Gliederung in ein Phasenmodell wie es hierzulande durch HOAI und AHO häufig der Fall ist, auffällig selten stattfindet. Der Fokus liegt mehr auf Prinzipien, Rollen, Arbeitsweisen und Entscheidungslogiken als auf einem formalisierten Prozessmodell.\\
Die Gründe dafür liegen mutmaßlich in den Kernideen von \ac{TVD} selbst und der Art und Weise wie es aufgrund letzterer aktuell in der Praxis integriert wird. Wesentlich sind hierfür die folgenden Punkte:\\

- Philosophischer Charakter:TVD ist in erster Linie ein Denk- und Managementprinzip, nicht ein „Planungsphasen-System“.\\
- Integration statt eigener Prozess:
In der Praxis wird TVD in bestehende Planungsmodelle eingebettet (USA: IPD + Designphasen; DE: HOAI-Gefüge).
Dadurch verliert es seine trennscharfe Prozesssicht.\\
- Iterative, zyklische Natur: TVD basiert auf ständigen Rückkopplungs- und Entscheidungszyklen. Lineare Abbildungen werden dem nicht gerecht.\\
- Qualität, Tiefe und Sequenz der Anwendung hängen stark vom Projektumfeld ab (Beteiligte, Vertrag, Risikoteilung, Vergabeform etc.).
Ergo: Kein einheitlicher Ablauf kann überall passen.\\
- Die US-Forschung liefert eher Richtimpulse als standardisierte Prozessmodelle; europäische bzw. deutsche Normung fehlt nahezu vollständig.\\

Wenn kein eindeutiges TVD-Phasenmodell existiert, muss ein idealtypisches konstruieren, um überhaupt messen zu können, wo deutsche Praxis kompatibel ist und wo nicht..
TVD lebt zwar von Iteration und Flexibilität, aber zur Bewertung der Anschlussfähigkeit an HOAI, AHO, Projektstufen etc. wird eine klar definierte Prozesslogik benötigt.
Das idealtypisches Phasenmodell erhebt kein Anspruch auf Normierung ist, sondern dient im Rahmen dieser Untersuchung als ein analytisches Hilfsmittel.\\

Für die Modellierung des idealtypischen TVD-Prozesses wird in dieser Arbeit ein 4-Phasen-Modell zugrunde gelegt. Diese Struktur stellt eine Synthese aus dem 4-Phasen-Modell des LCI Practitioner Guidebooks (2016) und dem Prozess-Schema nach Ballard (2012)  dar.
Während das LCI-Modell 'Design' und 'Construction' in einer übergeordneten Phase ('Value Delivery') zusammenfasst , nimmt das hier entwickelte Modell eine explizite Trennung in Phase 2 (Design) und Phase 3 (Umsetzung) vor. Diese Differenzierung erfolgt im Hinblick auf die spätere Transferanalyse (Kapitel 5), um eine präzise Gegenüberstellung mit der in Deutschland strukturell verankerten Trennung von Planungs- und Ausführungsphasen (gemäß HOAI) zu ermöglichen, ohne dabei die iterative Logik des TVD aufzugeben.\\

Erster Entwurf/Wahl des Phasenmodells\\

\begin{itemize}[leftmargin=2em]
    \item \textbf{Phase 0}: Phase 0: Projektinitiierung \& Definition (Project Definition)
    \item \textbf{Phase 1}: Validierung \& Zielsetzung (Validation \& Target Setting)
    \item \textbf{Phase 2}: Wertorientierte Planung (Value Delivery / Design)
    \item \textbf{Phase 3}: Realisierung \& Abschluss (Implementation \& Learning)
\end{itemize}

\clearpage

\section{Phase 0 - Projektdefinition und Wertbestimmung}
\label{sec:4.2}
Die Phase der Projektdefinition und Wertbestimmung (Phase 0) bildet das fundamentale Ausgangsniveau des Target Value Design, in dem die Projektziele und der finanzielle Rahmen (Allowable Cost) ausschließlich auf Basis des Business Case und losgelöst von spezifischen Designlösungen festgelegt werden\autocite[S. 2]{ballard_target_2012}.
Der Ablauf gliedert sich dabei in drei wesentliche Schritte: die Formulierung eines belastbaren Business Case, die Ableitung der Allowable Cost sowie einen initialen Marktabgleich zur Verifizierung der generellen Machbarkeit.
Am Ausgangspunkt dieser Phase steht der Business Case, dessen Entstehung abhängig von den Interessen und Zielen des Bauherren stark variieren kann\autocite[S.20]{hill_target_2017}.

Wer sind die Akteure? Bauherr + Dienstleister

Funktionsprogramm oder Leistungsprogramm 

Ableitung der Allowable Cost als das was der Bauherr für den erwarteten Nutzen/Wert bereit ist zu zahlen\\
Kostenschätzung - Estimated Cost (EC) meist basierend auf einem Benchmarking (historische Daten) oder parametrischen (Modell-)Simulationen \autocite[S.5]{ballard_target_2025}.\\
Beide Ansätze sind ähnlich effektiv, wenn sie dem Projekt entsprechend gewählt und angewandt werden so Ballard (2025). Dementsprechend kann die Kostenschätzung mittels Modellsimulation bei Projekten mit verhältnismäßig hohem Innovationsgrad bessere Ergebnisse erzielen, als der Ansatz mit historischen Kostendaten und letztere umgekehrt bei häufiger Umgesetzten Projekttypen bzw. Designs eine verlässliche Kostenbasis liefern\autocite[S.5]{ballard_target_2025}.\\.

Nachdem die beiden Größen AC und EC hinreichend genau definiert wurden erfolgt eine erste Validierung der Machbarkeit in Form einer einfachen Gap-Analyse. Dabei werden die Allowable Cost (AC) den Estimated Costs (EC) gegenübergestellt:

\begin{itemize}[leftmargin=2em]
    \item \textbf{Szenario A (EC $<$ AC)}: Das Projekt ist in der aktuellen Form nicht machbar. Anpassung der Ziele (Scope) oder Abbruch ("Revise or stop")
    \item \textbf{Szenario B (EC $\le$ AC)}: Das Projekt ist machbar. Übergang zur Validierung.
\end{itemize}

Im Szenario A sind die Estimated Costs aus der Kostenschätzung höher als die vom Bauherren festgelegten Allowable Costs für das Projekt mit den zuvor definierten Anforderungen.

Bauherr muss einschätzen EC = AC möglich?

ja, EC = AC
nein, Revision (Anpassung Scope) oder Abbruch. 








 \clearpage
 
\section{Phase 1 - Zielkostenentwicklung}
\label{sec:4.3}
eng. Target Costing \& Target Definition



















\clearpage

\section{Phase 2 - Zielkostenentwicklung}
\label{sec:4.3}
eng. Target Costing \& Target Definition





















\clearpage

\section{Phase 3 - Zielkostenentwicklung}
\label{sec:4.3}
eng. Target Costing \& Target Definition