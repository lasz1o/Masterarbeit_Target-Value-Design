\chapter{Analyse und Modellierung}
\label{ch:4}
\section{Das Phasenmodell}
\label{sec:4.1}

Eine Betrachtung der einschlägigen Fachliteratur zeigt, dass Target Value Design (TVD) im Gegensatz zur deutschen \ac{HOAI} nur selten als lineares Prozessmodell mit starren Phasen dargestellt wird. Der Fokus der Forschung liegt primär auf den zugrundeliegenden Prinzipien, Rollenbildern und Entscheidungslogiken sowie auf der kulturellen Transformation der Zusammenarbeit.\autocite[vgl.]{zimina_target_2012} \autocite[vgl.]{ballard_target_2025}

Diese Abwesenheit eines einheitlichen und normierten Phasenmodells lässt sich auf die inhärente Logik des \ac{TVD}-Ansatzes zurückführen. Zum einen versteht sich \ac{TVD} in erster Linie als Managementphilosophie und nicht als formales Leistungsbild. Weiterhin wird \ac{TVD} in der Praxis meist als methodische Ergänzung in bestehende lokale Vertragsmodelle wie das US-amerikanische \ac{IPD} oder britische \acl{PA}-Modelle integriert\autocite[vgl.]{z.B. Fischer et al. 2017, Mossman oder LCI Guide}, wodurch es sich der jeweiligen Prozessstruktur anpasst, anstatt eine eigene Struktur vorzugeben. Zudem widerspricht eine rein lineare Abwicklung der zyklischen sowie iterativen Natur des TVD-Prozesses, der maßgeblich von Rückkopplungsschleifen bestimmt wird.

Um im Rahmen dieser Arbeit dennoch untersuchen zu können, wie kompatibel TVD mit den stark formalisierten deutschen Abläufen ist, muss ein idealtypisches Phasenmodell konstruiert werden. Dieses Modell dient nicht als neuer Standardvorschlag für die Praxis. Es fungiert vielmehr als analytisches Hilfsmittel, um Unterschiede und Gemeinsamkeiten überhaupt messbar zu machen.

Für die weitere Untersuchung wird ein Modell mit vier Phasen zugrunde gelegt. Es verbindet das Prozess-Schema nach Ballard \autocite[vgl.]{ballard_target_2012} mit dem Phasenmodell des \acl{LCI}.\autocite[vgl.]{hill_target_2017} Diese Struktur weist zudem deutliche Parallelen zum deutschen Modell der Integrierten Projektabwicklung (IPA) auf. Ein wesentlicher Unterschied in der Modellierung liegt jedoch in der Gewichtung der Validierung: Während das hier genutzte TVD-Modell der Validierung aufgrund ihrer zentralen Steuerungsfunktion eine eigenständige Phase widmet (Phase 1), verortet das IPA-Modell diesen Prozessschritt meist integrativ zu Beginn der Planungsphase. Die zugrundeliegende Logik, dass keine Realisierung ohne gesicherte Zielkosten erfolgt, ist beiden Ansätzen jedoch gemein.\autocite[vgl. S.8]{haghsheno_ipa-report_2025}

Während das Modell des Lean Construction Institute die Phasen Design (Planung) und Construction (Ausführung) häufig unter dem Begriff \enquote{Value Delivery} zusammenfasst\autocite[vgl.]{hill_target_2017}, nimmt das hier entwickelte Modell eine bewusste analytische Trennung zwischen Phase 2 (Design) und Phase 3 (Realisierung) vor. Diese Unterscheidung erfolgt aus methodischen Gründen. Sie ermöglicht im späteren Transferkapitel einen präzisen Vergleich mit der in Deutschland durch \acs{VOB} und \ac{HOAI} verankerten Trennung von Planung und Ausführung (vgl. Abbildung \ref{fig:phasenvergleich}). Gleichzeitig bleibt die für \ac{TVD} essenzielle Phase der Validierung als eigenständiger Abschnitt erhalten.

Das zugrunde gelegte Modell gliedert sich folglich in:
\begin{itemize}[leftmargin=2em]
    \item \textbf{Phase 0}: Projektinitiierung (Project Definition)
    \item \textbf{Phase 1}: Validierung und Zielsetzung (Validation)
    \item \textbf{Phase 2}: Wertorientierte Planung (Design to Targets)
    \item \textbf{Phase 3}: Realisierung und Abschluss (Project Delivery)
\end{itemize}

\begin{figure}[htbp]
    \centering
    \includegraphics[width=1.0\textwidth]{05_figures/HOAI vs. TVD Phasen Vergleich .jpg}
    
    \caption[Vergleich HOAI und TVD]{Vergleich der HOAI-Leistungsphasen mit dem idealtypischen TVD-Phasenmodell (Quelle: Eigene Darstellung).}
    \label{fig:phasenvergleich}
\end{figure}

\clearpage

\section{Phase 0 - Projektdefinition und Wertbestimmung (Project Definition)}
\label{sec:4.2}

Die Phase der Projektdefinition (Phase 0) bildet das fundamentale Ausgangsniveau des Target Value Design. Anders als in konventionellen Planungsabläufen, bei denen der Entwurf oft frühzeitig auf physische Lösungsansätze fokussiert, konzentriert sich Phase 0 ausschließlich auf die Definition des \enquote{Warum} (Business Case) und des finanziellen \enquote{Wie viel} (Allowable Cost). 

Ziel dieser Phase ist es, den Wert aus Sicht des Bauherrn so präzise zu definieren, dass er in den nachfolgenden Phasen als harter Steuerungsmaßstab dienen kann. Design-Lösungen oder spezifische architektonische Ausformulierungen spielen zu diesem Zeitpunkt noch keine Rolle. Die Phase ist durch eine klare Dominanz der Auftraggeber-Sphäre geprägt: Der Bauherr muss, ggf. unterstützt durch Analysten, festlegen, welchen Nutzen das Projekt stiften soll und welches Investment dafür wirtschaftlich tragbar ist, noch bevor das eigentliche Planungsteam konstituiert wird \autocite[vgl.]{ballard_target_2025}.

Der Ablauf gliedert sich dabei in drei wesentliche Schritte: die Formulierung eines belastbaren Wertversprechens, die Ableitung des finanziellen Rahmens sowie einen initialen Marktabgleich zur Verifizierung der generellen Machbarkeit.

\subsection{Definition des Wertversprechens (Business Case \& Value)}
\label{sec:4.2.1}
Am Ausgangspunkt dieser Phase steht der Business Case, dessen Entstehung abhängig von den Interessen und Zielen des Bauherren stark variieren kann \autocite[vgl.]{hill_target_2017}. Während in der klassischen Bedarfsplanung (z.B. nach DIN 18205) häufig das \enquote{Raumprogramm} (Output) im Vordergrund steht, fordert der TVD-Ansatz eine Fokussierung auf den \enquote{Nutzen} (Outcome). Es gilt, das reine Leistungsprogramm um die Dimension der Nutzerwerte zu erweitern.

Diese Anforderungen werden als \textbf{Conditions of Satisfaction (CoS)} definiert. Sie beschreiben die Bedingungen, unter denen der Bauherr das Projekt als Erfolg wertet. Um späteren \enquote{Design Churn} (unnötige Planungsänderungen) zu vermeiden, ist hierbei eine strikte Priorisierung notwendig \autocite[vgl.]{zimina_target_2012}:
\begin{itemize}
    \item \textbf{Needs (Muss-Kriterien):} Diese Aspekte sind unverzichtbar für die Erfüllung des Business Case (z.B. Patientensicherheit im Krankenhaus, gesetzliche Vorgaben). Ein Nichterfüllen führt zum Scheitern des Projekts.
    \item \textbf{Wants (Wunsch-Kriterien):} Diese Aspekte steigern den Wert, sind jedoch verhandelbar (z.B. spezifische Ausstattungsqualitäten), falls das Budget dies erfordert.
\end{itemize}
Diese Unterscheidung bildet die Grundlage für spätere Value-Engineering-Entscheidungen im Designprozess.

\subsection{Ermittlung des finanziellen Rahmens (AC vs. EC)}
\label{sec:4.2.2}
Parallel zur Wertdefinition erfolgt die Festlegung der finanziellen Eckpfeiler. Hierbei prallen im TVD-Prozess zwei methodisch getrennte Sichtweisen aufeinander: die finanzielle Tragfähigkeit des Bauherrn und die Realität des Baumarktes.

\textbf{1. Allowable Cost (AC): Die Sicht des Bauherrn}\\
Die Allowable Cost repräsentieren den Betrag, den der Bauherr maximal investieren \textit{darf}, damit sein Business Case (z.B. Renditeerwartung, Refinanzierung) positiv bleibt. Diese Größe wird \enquote{top-down} ermittelt und ist explizit \textit{keine} Kostenschätzung des Gebäudes, sondern eine betriebswirtschaftliche Setzung \autocite[vgl.]{ballard_target_2012}.

\textbf{2. Estimated Cost (EC): Die Sicht des Marktes}\\
Um die Realisierbarkeit der Anforderungen zu prüfen, wird den AC eine Marktprognose gegenübergestellt (Estimated Cost). Da in Phase 0 noch kein detaillierter Entwurf vorliegt, kommen hierbei stochastische oder empirische Methoden zum Einsatz. Nach Ballard et al. (2025) sind hierbei zwei Ansätze zu unterscheiden, deren Eignung vom Innovationsgrad des Projekts abhängt:
\begin{itemize}
    \item \textbf{Benchmarking (Historische Daten):} Bei Standard-Typologien (z.B. Wohnungsbau) liefern Datenbanken aus abgewickelten Projekten verlässliche Referenzwerte.
    \item \textbf{Parametrische Simulation:} Bei komplexen Projekten mit hohem Innovationsgrad, für die keine historischen Vergleichsdaten existieren, erzielen modellbasierte Simulationen oft präzisere Ergebnisse als reine Kennzahlenvergleiche \autocite[vgl.]{ballard_target_2025}.
\end{itemize}

\subsection{Initiale Machbarkeitsanalyse (The Gap)}
\label{sec:4.2.3}
Der Abschluss der Phase 0 bildet das erste entscheidende \enquote{Gate} im TVD-Prozess. Hierbei werden die Allowable Cost (AC) und die Estimated Costs (EC) in einer Gap-Analyse gegenübergestellt ($\Delta = EC - AC$). Das Ergebnis bestimmt das weitere Vorgehen:

\begin{itemize}
    \item \textbf{Szenario A ($EC \gg AC$):} Die Marktkosten übersteigen das Budget massiv (z.B. > 30\%). Das Projekt ist in der aktuellen Form unrealistisch. Es folgt die Entscheidung \enquote{Revise or Stop}: Entweder müssen die Anforderungen (CoS) drastisch reduziert oder der Business Case verworfen werden.
    \item \textbf{Szenario B ($EC > AC$):} Die Kosten liegen moderat über dem Budget (z.B. 5--15\%). Dies ist der \enquote{Idealfall} für den Start eines TVD-Projekts. Die Lücke definiert das \enquote{Stretch Goal}, das durch Innovation und Effizienzsteigerung im folgenden Designprozess geschlossen werden soll.
    \item \textbf{Szenario C ($EC \leq AC$):} Der Marktpreis liegt unter dem Budget. In diesem Fall sollten die Allowable Costs nach unten korrigiert werden, um unnötige Ausgaben (Verschwendung) zu vermeiden.
\end{itemize}

Erst wenn eine prinzipielle Machbarkeit (Szenario B oder C) bestätigt ist, wird das Projekt für die nächste Phase freigegeben und die Rekrutierung des integrierten Teams eingeleitet.


% Die Phase der Projektdefinition und Wertbestimmung (Phase 0) bildet das fundamentale Ausgangsniveau des Target Value Design, in dem die Projektziele und der finanzielle Rahmen (Allowable Cost) ausschließlich auf Basis des Business Case und losgelöst von spezifischen Designlösungen festgelegt werden\autocite[S. 2]{ballard_target_2012}.
% Der Ablauf gliedert sich dabei in drei wesentliche Schritte: die Formulierung eines belastbaren Business Case, die Ableitung der Allowable Cost sowie einen initialen Marktabgleich zur Verifizierung der generellen Machbarkeit.
% Am Ausgangspunkt dieser Phase steht der Business Case, dessen Entstehung abhängig von den Interessen und Zielen des Bauherren stark variieren kann\autocite[S.20]{hill_target_2017}.
% Hinsichtlich der Akteursbeteiligung ist Phase 0 durch eine Dominanz der Auftraggeber-Sphäre geprägt. Zwar können externe Berater oder Analysten zur Erstellung von Marktanalysen hinzugezogen werden, die Verantwortung für die Definition des Funktionsprogramms liegt jedoch exklusiv beim Bauherrn. Es gilt in dieser frühen Phase, das reine 'Leistungsprogramm' (Raumprogramm, technische Anforderungen) um die Dimension der 'Nutzerwerte' zu erweitern, die später als Entscheidungsgrundlage für Design-Alternativen dienen.

% Funktionsprogramm oder Leistungsprogramm 

% Ableitung der Allowable Cost als das was der Bauherr für den erwarteten Nutzen/Wert bereit ist zu zahlen\\
% Kostenschätzung - Estimated Cost (EC) meist basierend auf einem Benchmarking (historische Daten) oder parametrischen (Modell-)Simulationen \autocite[S.5]{ballard_target_2025}.\\
% Beide Ansätze sind ähnlich effektiv, wenn sie dem Projekt entsprechend gewählt und angewandt werden so Ballard (2025). Dementsprechend kann die Kostenschätzung mittels Modellsimulation bei Projekten mit verhältnismäßig hohem Innovationsgrad bessere Ergebnisse erzielen, als der Ansatz mit historischen Kostendaten und letztere umgekehrt bei häufiger Umgesetzten Projekttypen bzw. Designs eine verlässliche Kostenbasis liefern\autocite[S.5]{ballard_target_2025}.\\.

% Nachdem die beiden Größen AC und EC hinreichend genau definiert wurden erfolgt eine erste Validierung der Machbarkeit in Form einer einfachen Gap-Analyse. Dabei werden die Allowable Cost (AC) den Estimated Costs (EC) gegenübergestellt:

% \begin{itemize}[leftmargin=2em]
%     \item \textbf{Szenario A (EC $<$ AC)}: Das Projekt ist in der aktuellen Form nicht machbar. Anpassung der Ziele (Scope) oder Abbruch ("Revise or stop")
%     \item \textbf{Szenario B (EC $\le$ AC)}: Das Projekt ist machbar. Übergang zur Validierung.
% \end{itemize}

% Im Szenario A sind die Estimated Costs aus der Kostenschätzung höher als die vom Bauherren festgelegten Allowable Costs für das Projekt mit den zuvor definierten Anforderungen.

% Bauherr muss einschätzen EC = AC möglich?

% ja, EC = AC
% nein, Revision (Anpassung Scope) oder Abbruch. 


\clearpage



\FloatBarrier
\clearpage
 
\section{Phase 1 - Validierung und Zielkostenbestimmung}
\label{sec:4.3}




Vertragliche Synchronisation der Interessen

Eine valide Überprüfung des Business Case durch die ausführenden Unternehmen scheitert laut Ballard in der konventionellen Praxis häufig an der antagonistischen Vertragsstruktur und dem daraus resultierenden Misstrauen ('Adversarial Relationship'). Der Auftraggeber befürchtet hierbei opportunistisches Verhalten der Auftragnehmer.\autocite[vgl. S. 15]{ballard_target_2012}

In der Validierungsphase des TVD wird dieses Prinzipal-Agent-Problem durch den Mechanismus des Painsharing aufgelöst. Da eine Überschreitung der Kostenziele direkt die Gewinnmargen der Projektpartner reduziert ('Reduced Profit Margin'), werden die kommerziellen Interessen von Auftraggeber und Auftragnehmern synchronisiert ('Alignment of Interests'). Dies stellt sicher, dass die Validierung des Business Case nicht durch Eigeninteressen verzerrt wird, sondern auf einer realistischen Einschätzung der Machbarkeit basiert.\autocite[vgl. S. 15]{ballard_target_2012}















\clearpage

\section{Phase 2 - Wertorientierte Planung}
\label{sec:4.4}

Mit dem Abschluss der Validierungsphase (Phase 1) und der vertraglichen Fixierung der Zielkosten (Target Cost) ändert sich die Prozesslogik fundamental. Während Phase 1 die Machbarkeit prüft, fokussiert sich Phase 2 auf die operative Ausarbeitung einer qualitativen Lösung, die zwingend innerhalb des gesetzten Zielkostenrahmens verbleibt ("Design to Targets").
Der Planungsprozess wandelt sich hierbei von einer linearen Abarbeitung hin zu einem zyklischen, iterativen Steuerungsmodell.

\subsection{Zielsetzung und modelltheoretische Abgrenzung}
\label{sec:4.4.1}
Das primäre Ziel der Phase 2 ist die Überführung der funktionalen Anforderungen (aus Phase 0) in ein realisierungsfähiges Design, das Kostensicherheit bietet. Dieser Zustand wird im TVD als "Definitive Design" bezeichnet.

Es ist an dieser Stelle notwendig, eine differenzierte modelltheoretische Abgrenzung vorzunehmen. In der gelebten TVD-Praxis (insb. USA/IPD) verschwimmen Planung und Ausführung durch "Fast-Tracking" und frühe Fertigung komplett. In der deutschen Baupraxis existiert eine solche Integration zwar ebenfalls – beispielsweise in Generalunternehmer-Modellen oder bei funktionalen Ausschreibungen –, jedoch ist der "Standardfall" des deutschen Ordnungsrahmens durch das Trennungsprinzip geprägt.

Sowohl das Vergaberecht (VOB/A: Trennung von Planung und Bau) als auch das Preisrecht (HOAI: Honorierung abgeschlossener Planungsphasen) suggerieren eine sequenzielle Logik, bei der die Ausführung erst nach Abschluss der Planung beginnt. Selbst in partnerschaftlichen Modellen erfolgt die Einbindung der Bauunternehmen oft erst nach Abschluss der Entwurfsplanung (LPH 3), womit das wesentliche Potenzial der frühen Einflussnahme auf das Designkonzept ("Constructability") bereits eingeschränkt ist.

Für die vorliegende Untersuchung wird daher analytisch an der Trennung zwischen Phase 2 (Design) und Phase 3 (Realisierung) festgehalten. Dies dient dazu, die Reibungspunkte des TVD-Ansatzes mit dem kodifizierten deutschen Standardmodell (Einzelvergabe nach VOB, Planung nach HOAI) präzise herauszuarbeiten. Inhaltlich deckt die Phase 2 dieses Modells somit den gesamten Planungsraum der HOAI-Leistungsphasen 2 (Vorplanung) bis 5 (Ausführungsplanung) ab. Dies markiert die zentrale Verschiebung: Im TVD findet die werkplanerische Detaillierung nicht erst nach der Vergabe statt, sondern wird als integrative Leistung vorgezogen, um die Preissicherheit zu garantieren.


\subsection{Strukturierung des Budgets: Target Cost Allocation}
\label{sec:4.4.2}
Bevor die inhaltliche Planungsarbeit beginnt, muss das in Phase 1 ermittelte Gesamtbudget ("Project Target Cost") in steuerbare Teilbudgets zerlegt werden. Dieser Prozess der "Target Cost Allocation" transformiert das Budget von einer abstrakten Gesamtsumme in konkrete Handlungsanweisungen für die Planungsteams.

Anders als bei einer klassischen Kostenschätzung nach DIN 276 (Gewerke), erfolgt die Allokation im TVD systemorientiert auf die zu bildenden Cluster (vgl. Kap. 4.4.3). Das Budget wird typischerweise in einem Gegenstromverfahren verteilt:
\begin{itemize}
    \item \textbf{Top-Down:} Die Projektsteuerung gibt basierend auf Benchmarks grobe Zielwerte für die Cluster vor (z.B. "Zielwert Fassade: 2,5 Mio. €").
    \item \textbf{Bottom-Up:} Die Cluster-Teams prüfen diese Werte auf Plausibilität und melden Rückbedarf an.
\end{itemize}
Ein essenzieller Bestandteil der Allokation ist das "Contingency Management". Das Budget wird nicht zu 100\% auf die Cluster verteilt. Ein Teil verbleibt als zentraler Puffer ("Project Contingency") beim Bauherrn/Steuerungsteam, um unvorhergesehene Risiken abzudecken, ohne die Target Cost zu erhöhen.

\subsection{Organisatorische Struktur: Interdisziplinäre Cluster-Teams}
\label{sec:4.4.3}
Um die Komplexität des Gesamtprojekts handhabbar zu machen, wird die operative Projektorganisation in Phase 2 von einer gewerkespezifischen Trennung auf interdisziplinäre "Cluster-Teams" (Cross-Functional Teams) umgestellt.

\textbf{Zusammensetzung und Rollen}
Ein Cluster (z.B. "Hülle") vereint alle für dieses Teilsystem relevanten Kompetenzen an einem Tisch. Die typische Besetzung umfasst:
\begin{enumerate}
    \item \textbf{Cluster Lead (Planung):} Meist der Architekt oder Fachplaner, der die design-technische Führung übernimmt.
    \item \textbf{Construction Lead (Ausführung):} Ein Vertreter des ausführenden Unternehmens (z.B. Polier oder Projektleiter des Fassadenbauers).
    \item \textbf{Estimator (Kalkulation):} Ein Kostenschätzer, der permanenten Zugriff auf aktuelle Marktpreise hat.
\end{enumerate}

\textbf{Die Rolle der "Makers"}
Ein entscheidendes Novum des TVD-Prozesses ist die Integration der ausführenden Firmen ("Makers") bereits in dieser frühen Designphase. Ihre Rolle wandelt sich vom reinen Empfänger fertiger Pläne zum aktiven "Co-Creator". Sie bringen fertigungstechnisches Wissen ("Constructability") ein, um Lösungen zu finden, die funktional gleichwertig, aber günstiger zu fertigen sind.

> \textit{Transfer-Hinweis:} Diese frühe Integration steht im Konflikt zur konventionellen Vergabepraxis (VOB/A), die eine strikte Trennung von Planung und Ausführung sowie Produktneutralität fordert. (Diskussion in Kap. 5).

\subsection{Prozesslogik: Der zyklische Steuerungsloop (Continuous Estimating)}
\label{sec:4.4.4}
Das methodische Kernstück der Phase 2 ist das "Continuous Estimating". Um den im konventionellen Prozess üblichen "Blindflug" zwischen den Leistungsphasen zu vermeiden, wird der Planungsprozess in kurze, sich wiederholende Zyklen (Sprints) von typischerweise 3 bis 4 Wochen unterteilt.

Ein solcher Zyklus folgt einer festen Logik:
\begin{enumerate}
    \item \textbf{Design \& Innovation (Woche 1-2):} Die Cluster-Teams entwickeln Lösungsansätze. Hierbei wird oft parallel an mehreren Varianten gearbeitet (Set-Based Design), um den Lösungsraum nicht zu früh einzuschränken.
    \item \textbf{Pricing \& Feedback (Woche 3):} Die erarbeiteten Stände werden kalkuliert. Da die Baufirmen Teil des Teams sind, basiert dies auf echten Marktpreisen, nicht auf Datenbank-Mittelwerten. Unterstützt wird dies oft durch BIM-basierte Massenauszüge (Quantity Take-Off).
    \item \textbf{Steuerungs-Gate (Ende Woche 3):} Im zentralen "Big Room Meeting" werden die Kostenstände aller Cluster zusammengeführt.
\end{enumerate}

\textbf{Entscheidung am Gate:}
Das Gate fungiert als harte Qualitätsschranke. Liegt die Prognose eines Clusters über dem zugewiesenen Budget, darf nicht detailliert werden. Das Team muss "zurück ans Reißbrett", um das Design anzupassen (Design to Cost). Nur bei Einhaltung des Budgets erfolgt die Freigabe für den nächsten Detaillierungsgrad.

> \textit{Transfer-Hinweis:} Die HOAI honoriert den Planungserfolg am Phasenende. Iterative Schleifen zur Kostenoptimierung ("Rework") sind im linearen Leistungsbild oft nicht abgebildet und führen zu Honorardiskussionen. (Diskussion in Kap. 5).

\subsection{Entscheidungsfindung und Steuerungsmethodik}
\label{sec:4.4.5}
Damit die iterative Arbeitsweise zielgerichtet verläuft, stützt sich Phase 2 auf zwei spezifische Methoden zur Entscheidungsfindung:

\textbf{Set-Based Design (SBD)}
Anstatt sich intuitiv auf eine einzige Lösung festzulegen ("Point-Based Design") und diese linear auszuarbeiten, halten die Cluster-Teams bewusst mehrere Optionen parallel offen (Sets). Diese werden grob dimensioniert und kalkuliert. Optionen, die sich als technisch oder wirtschaftlich nicht tragfähig erweisen, werden sukzessive eliminiert, bis die optimale Lösung übrig bleibt. Dies vermeidet teure negative Iterationen in späten Phasen.

\textbf{Choosing by Advantages (CBA)}
Muss zwischen technisch gleichwertigen Varianten entschieden werden, kommt die Methode "Choosing by Advantages" zum Einsatz. Entscheidungen basieren dabei nicht auf einer Gewichtung von Faktoren (die subjektiv sein kann), sondern auf der Analyse der konkreten Vorteile ("Advantages") einer Option im Verhältnis zu ihren Kosten. Dies objektiviert den Entscheidungsprozess und macht ihn gegenüber dem Bauherrn dokumentierbar (z.B. mittels A3-Reports).

\subsection{Abschluss der Phase 2: Das Definitive Design}
\label{sec:4.4.6}
Der Abschluss der Phase 2 ist erreicht, wenn das "Definitive Design" vorliegt. Dieser Meilenstein unterscheidet sich qualitativ von einer klassischen Ausführungsplanung.

Das Design ist zu diesem Zeitpunkt nicht nur zeichnerisch dargestellt, sondern bereits "produktionsreif" (Production Ready). Durch die Mitwirkung der "Makers" sind Fertigungsdetails, Montageabläufe und Materialverfügbarkeiten bereits geklärt.
Das entscheidende Kriterium für den Übergang in Phase 3 ist jedoch die Preissicherheit: Das Team gibt ein verbindliches Commitment ab, dass das Design innerhalb der Target Cost realisierbar ist.

\textbf{Das finale Gate (Release for Construction):}
Erst wenn die Prognosekosten (Expected Cost) sicher unterhalb der Allowable Cost liegen, wird das Projekt zur physischen Realisierung freigegeben.

> \textit{Transfer-Hinweis:} Die Synchronisation von Planungsfreigabe und absoluter Preissicherheit ist in Einheitspreisverträgen kaum abzubilden, da Endkosten hier oft erst nach Aufmaß feststehen. (Diskussion in Kap. 5).







\clearpage

\section{Phase 3 - Realisierung und Abschluss}
\label{sec:4.5}


Mit dem Erreichen des Definitive Design und der Freigabe zur Ausführung ("Release for Construction") tritt das Projekt in die dritte und letzte Phase des Target Value Design ein. Im Gegensatz zu konventionellen Projekten, in denen während der Bauphase oft noch massive Umplanungen und Kostensteigerungen (Nachträge) auftreten, verlagert sich der Fokus in Phase 3 rein auf die exekutive Umsetzung des Geplanten.

\subsection{Prozessziel: Abweichungsfreie Umsetzung}
Das primäre Ziel der Phase 3 ist die physische Errichtung des Bauwerks unter strikter Einhaltung der in Phase 2 garantierten Parameter (Kosten, Zeit, Qualität). Da die "Makers" (ausführende Firmen) das Design selbst mitentwickelt haben, entfallen typische Bauablaufstörungen wie Probleme in der praktischen Umsetzung oder fehlende Detailinformationen weitgehend.
Die Planungsarbeit beschränkt sich in dieser Phase auf die Just-in-Time-Lieferung von logistischen Informationen oder minimalen Anpassungen an unvorhergesehene Baustellenbedingungen.

\subsection{Kostensteuerung: Cost Monitoring statt Cost Estimating}
Die Art der Kostensteuerung ändert sich fundamental. Während in Phase 2 noch aktiv kalkuliert und optimiert wurde ("Estimating"), geht es nun um das reine "Monitoring":
\begin{itemize}
    \item \textbf{Soll-Ist-Vergleich:} Die auflaufenden Kosten werden permanent gegen die Target Cost (bzw. die Allowable Cost) gespiegelt.
    \item \textbf{Gewinnsicherung:} Da das Budget fixiert ist, arbeiten die Cluster-Teams nun daran, durch Effizienz auf der Baustelle (z.B. durch Lean Construction Methoden wie Taktplanung) ihren Gewinn zu maximieren. Ein Unterschreiten der Kosten ist nun im direkten Interesse der Ausführenden (Pain/Gain-Sharing).
\end{itemize}

\subsection{Abschluss und Lernen}
Das TVD-Modell endet nicht mit der Schlüsselübergabe, sondern mit einer formalen Feedback-Schleife. Die tatsächlichen Endkosten werden final mit den Target Costs abgeglichen.
\begin{itemize}
    \item \textbf{Incentivierung:} Eventuelle Einsparungen ("Underrun") werden gemäß dem vorab definierten Schlüssel zwischen Bauherr und Risikopool ausgeschüttet.
    \item \textbf{Wissenssicherung:} Abweichungen (sowohl positive als auch negative) werden analysiert, um die Kennzahlen für zukünftige Projekte ("Benchmarks" für Phase 0) zu schärfen.
\end{itemize}

> \textit{Transfer-Hinweis:} In der deutschen VOB-Realität ist Phase 3 oft durch Nachtragsmanagement ("Claims") geprägt. Der TVD-Ansatz, dass der Preis fixiert ist und Nachträge durch die vorherige Integration ausgeschlossen sind, erfordert vertraglich eine völlig andere Risikoverteilung als den klassischen Einheitspreisvertrag. (Diskussion in Kap. 5).

