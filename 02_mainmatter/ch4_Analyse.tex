\chapter{Analyse und Modellierung}
\label{ch:analyse}
\section{Das Phasenmodell}
\label{sec:4.1}

Eine Betrachtung der einschlägigen Fachliteratur zeigt, dass Target Value Design (TVD) im Gegensatz zur deutschen \ac{HOAI} nur selten als lineares Prozessmodell mit starren Phasen dargestellt wird. Der Fokus der Forschung liegt primär auf den zugrundeliegenden Prinzipien, Rollenbildern und Entscheidungslogiken sowie auf der kulturellen Transformation der Zusammenarbeit.\autocite[vgl.]{zimina_target_2012}\, \autocite[vgl.]{ballard_target_2025}

Diese Abwesenheit eines einheitlichen und normierten Phasenmodells lässt sich auf die inhärente Logik des \ac{TVD}-Ansatzes zurückführen. Zum einen versteht sich \ac{TVD} in erster Linie als Managementphilosophie und nicht als formales Leistungsbild. Weiterhin wird \ac{TVD} in der Praxis meist als methodische Ergänzung in bestehende lokale Vertragsmodelle wie das US-amerikanische \ac{IPD} oder britische \acl{PA}-Modelle  \autocite[vgl.]{z.B. Fischer et al. 2017, Mossman oder LCI Guide}, wodurch es sich der jeweiligen Prozessstruktur anpasst, anstatt eine eigene Struktur vorzugeben. Zudem widerspricht eine rein lineare Abwicklung der zyklischen sowie iterativen Natur des \ac{TVD}-Prozesses, der maßgeblich von Rückkopplungsschleifen bestimmt wird.

Um im Rahmen dieser Arbeit dennoch untersuchen zu können, wie kompatibel \ac{TVD} mit den stark formalisierten deutschen Abläufen ist, muss ein idealtypisches Phasenmodell konstruiert werden. Dieses Modell dient nicht als neuer Standardvorschlag für die Praxis. Es fungiert vielmehr als analytisches Hilfsmittel, um Unterschiede und Gemeinsamkeiten messbar zu machen.

Für die weitere Untersuchung wird ein Modell mit vier Phasen zugrunde gelegt. Es verbindet das Prozess-Schema nach Ballard\,\autocite[vgl.]{ballard_target_2012} mit dem Phasenmodell des \acl{LCI}.\autocite[vgl.]{hill_target_2017} Diese Gliederung weist zudem deutliche Parallelen zum deutschen Modell der Integrierten Projektabwicklung (IPA) auf. Ein wesentlicher Unterschied liegt jedoch in der Gewichtung der Validierung: Während das hier genutzte TVD-Modell der Validierung aufgrund ihrer zentralen Steuerungsfunktion eine eigenständige Phase widmet (Phase 1), verortet das IPA-Modell diesen Prozessschritt integrativ zu Beginn der Planungsphase. Die zugrundeliegende Logik, dass keine Realisierung ohne gesicherte Zielkosten erfolgt, ist beiden Ansätzen jedoch gemein.\autocite[vgl. S.8]{haghsheno_ipa-report_2025}

Während das Modell des Lean Construction Institute die Phasen Design (Planung) und Construction (Ausführung) häufig unter dem Begriff \enquote{Value Delivery} zusammenfasst\autocite[vgl.]{hill_target_2017}, nimmt das vorgeschlagene Modell eine bewusste Trennung zwischen Phase 2 (Design) und Phase 3 (Realisierung) vor. Diese Unterscheidung erfolgt aus methodischen Gründen. Sie ermöglicht im späteren Transferkapitel einen präzisen Vergleich mit der in Deutschland durch \acs{VOB} und \ac{HOAI} verankerten Trennung von Planung und Ausführung (vgl. Abbildung \ref{fig:phasenvergleich}). Gleichzeitig bleibt die für \ac{TVD} essenzielle Phase der Validierung als eigenständiger Abschnitt erhalten.

Das zugrunde gelegte Modell gliedert sich demnach wie folgt:
\begin{itemize}[leftmargin=2em]
    \item \textbf{Phase 0}: Projektinitiierung (Project Definition)
    \item \textbf{Phase 1}: Validierung und Zielsetzung (Validation)
    \item \textbf{Phase 2}: Wertorientierte Planung (Design to Targets)
    \item \textbf{Phase 3}: Realisierung und Abschluss (Project Delivery)
\end{itemize}

\begin{figure}[htbp]
    \centering
     \fcolorbox{gray!50}{white}{%
    \includegraphics[width=\dimexpr\textwidth-2\fboxsep-2\fboxrule\relax]{05_figures/HOAI vs. TVD Phasen Vergleich .jpg}%
    }
    
    \caption[Vergleich HOAI und TVD]{Vergleich der HOAI-Leistungsphasen mit dem idealtypischen TVD-Phasenmodell (Quelle: Eigene Darstellung).}
    \label{fig:phasenvergleich}
\end{figure}

\clearpage

\section{Phase 0 - Projektdefinition und Wertbestimmung (Project Definition)}
\label{sec:4.2}

Die Phase der Projektdefinition (Phase 0) bildet das fundamentale Ausgangsniveau des Target Value Design. Anders als in konventionellen Planungsabläufen, bei denen der Entwurf oft frühzeitig auf physische Lösungsansätze fokussiert, konzentriert sich Phase 0 ausschließlich auf die Definition des \enquote{Warum} (Business Case) und des finanziellen \enquote{Wie viel} (Allowable Cost). 

Ziel dieser Phase ist es, den Wert aus Sicht des Bauherrn so präzise zu definieren, dass er in den nachfolgenden Phasen als harter Steuerungsmaßstab dienen kann. Design-Lösungen oder spezifische architektonische Ausformulierungen spielen zu diesem Zeitpunkt noch keine Rolle. Die Phase ist durch eine klare Dominanz der Auftraggeber-Sphäre geprägt: Der Bauherr muss, ggf. unterstützt durch Analysten, festlegen, welchen Nutzen das Projekt stiften soll und welches Investment dafür wirtschaftlich tragbar ist, noch bevor das eigentliche Planungsteam konstituiert wird \autocite[vgl.]{ballard_target_2025}.

Der Ablauf gliedert sich dabei in drei wesentliche Schritte: die Formulierung eines belastbaren Wertversprechens, die Ableitung des finanziellen Rahmens sowie einen initialen Marktabgleich zur Verifizierung der generellen Machbarkeit.

\subsection{Definition des Wertversprechens (Business Case \& Value)}
\label{sec:4.2.1}
Am Ausgangspunkt dieser Phase steht der Business Case, dessen Entstehung abhängig von den Interessen und Zielen des Bauherren stark variieren kann \autocite[vgl.]{hill_target_2017}. Während in der klassischen Bedarfsplanung (z.B. nach DIN 18205) häufig das \enquote{Raumprogramm} (Output) im Vordergrund steht, fordert der TVD-Ansatz eine Fokussierung auf den \enquote{Nutzen} (Outcome). Es gilt, das reine Leistungsprogramm um die Dimension der Nutzerwerte zu erweitern.

Eine besondere Herausforderung in der Definition des Wertversprechens liegt in der Differenzierung der Interessengruppen. Nach den Prinzipien des Lean Construction ist streng zwischen dem zahlenden Kunden (Paying Customer) und dem tatsächlichen Nutzer (End User) zu unterscheiden. Oft decken sich deren Ziele nicht vollständig (z.B. Minimierung der Investitionskosten vs. Optimierung der Arbeitsabläufe). Eine erfolgreiche Phase 0 muss daher zwingend die ‚Voice of the Customer‘ durch Nutzerworkshops oder Bedarfsanalysen integrieren. Werden die Endnutzerbedürfnisse in dieser frühen Phase ignoriert, führt dies in späteren Phasen häufig zu kostspieligen Änderungswünschen, die das Target Value Design destabilisieren können.\autocite[vgl.]{zimina_target_2012}

Diese Anforderungen werden als \textbf{Conditions of Satisfaction (CoS)} definiert. Sie beschreiben die Bedingungen, unter denen der Bauherr das Projekt als Erfolg wertet. Um späteren \enquote{Design Churn} (unnötige Planungsänderungen) zu vermeiden, ist hierbei eine strikte Priorisierung notwendig \autocite[vgl.]{zimina_target_2012}:
\begin{itemize}
    \item \textbf{Needs (Muss-Kriterien):} Diese Aspekte sind unverzichtbar für die Erfüllung des Business Case (z.B. Patientensicherheit im Krankenhaus, gesetzliche Vorgaben). Ein Nichterfüllen führt zum Scheitern des Projekts.
    \item \textbf{Wants (Wunsch-Kriterien):} Diese Aspekte steigern den Wert, sind jedoch verhandelbar (z.B. spezifische Ausstattungsqualitäten), falls das Budget dies erfordert.
\end{itemize}
Diese Unterscheidung bildet die Grundlage für spätere Value-Engineering-Entscheidungen im Designprozess.

\subsection{Ermittlung des finanziellen Rahmens (AC vs. EC)}
\label{sec:4.2.2}
Parallel zur Wertdefinition erfolgt die Festlegung der finanziellen Eckpfeiler. Hierbei prallen im TVD-Prozess zwei methodisch getrennte Sichtweisen aufeinander: die finanzielle Tragfähigkeit des Bauherrn und die Realität des Baumarktes.

\textbf{1. Allowable Cost (AC): Die Sicht des Bauherrn}\\
Die Allowable Cost repräsentieren den Betrag, den der Bauherr maximal investieren \textit{darf}, damit sein Business Case (z.B. Renditeerwartung, Refinanzierung) positiv bleibt. Diese Größe wird \enquote{top-down} ermittelt und ist explizit \textit{keine} Kostenschätzung des Gebäudes, sondern eine betriebswirtschaftliche Setzung \autocite[vgl.]{ballard_target_2012}.

\textbf{2. Estimated Cost (EC): Die Sicht des Marktes}\\
Um die Realisierbarkeit der Anforderungen zu prüfen, wird den AC eine Marktprognose gegenübergestellt (Estimated Cost). Da in Phase 0 noch kein detaillierter Entwurf vorliegt, kommen hierbei stochastische oder empirische Methoden zum Einsatz. Nach Ballard et al. (2025) sind zwei Ansätze zu unterscheiden:
\begin{itemize}
    \item \textbf{Benchmarking (Historische Daten):} Bei Standard-Typologien liefern Datenbanken aus abgewickelten Projekten Referenzwerte. Ballard et al. warnen jedoch davor, dass historische Daten oft die Ineffizienzen (Waste) vergangener Projekte beinhalten und daher eher als konservative Obergrenze denn als absolute Wahrheit zu betrachten sind.
    \item \textbf{Parametrische Simulation:} Bei komplexen Projekten mit hohem Innovationsgrad erzielen modellbasierte Simulationen oft präzisere Ergebnisse als reine Kennzahlenvergleiche \autocite[vgl.]{ballard_target_2025}.
\end{itemize}

Ein kritischer Erfolgsfaktor bei der Ermittlung der Estimated Cost ist die Qualität der zugrunde liegenden Daten. Ballard et al. (2025) weisen darauf hin, dass reines Benchmarking auf Basis historischer Kostendatenbanken (in Deutschland z.B. BKI) Risiken birgt. Da diese Daten den Durchschnitt vergangener – oft ineffizient abgewickelter – Projekte abbilden, besteht die Gefahr, dass Verschwendung (Waste) in die eigene Kostenschätzung ‚importiert‘ wird. Die Estimated Costs sollten daher nicht als unumstößliche Wahrheit, sondern als konservative Obergrenze betrachtet werden, die es durch Lean-Methoden zu unterbieten gilt. \autocite[vgl.]{ballard_target_2025}

\subsection{Initiale Machbarkeitsanalyse (The Gap)}
\label{sec:4.2.3}
Der Abschluss der Phase 0 bildet das erste entscheidende \enquote{Gate} im TVD-Prozess. Hierbei werden die Allowable Cost (AC) und die Estimated Costs (EC) in einer Gap-Analyse gegenübergestellt ($\Delta = EC - AC$). Das Ergebnis bestimmt das weitere Vorgehen:

\begin{itemize}
    \item \textbf{Szenario A ($EC \gg AC$):} Die Marktkosten übersteigen das Budget massiv (z.B. > 30\%). Das Projekt ist in der aktuellen Form unrealistisch. Es folgt die Entscheidung \enquote{Revise or Stop}: Entweder müssen die Anforderungen (CoS) drastisch reduziert oder der Business Case verworfen werden.
    \item \textbf{Szenario B ($EC > AC$):} Die Kosten liegen moderat über dem Budget (z.B. 5--15\%). Dies ist der \enquote{Idealfall} für den Start eines TVD-Projekts. Die Lücke definiert das \enquote{Stretch Goal}, das durch Innovation und Effizienzsteigerung im folgenden Designprozess geschlossen werden soll.
    \item \textbf{Szenario C ($EC \leq AC$):} Der Marktpreis liegt unter dem Budget. In diesem Fall sollten die Allowable Costs nach unten korrigiert werden, um unnötige Ausgaben (Verschwendung) zu vermeiden.
\end{itemize}

Es ist hervorzuheben, dass eine moderate Lücke (Szenario B) im TVD nicht als Planungsfehler, sondern als strategischer Treiber verstanden wird. Diese Differenz fungiert als \enquote{Stretch Goal} (ehrgeiziges Ziel). Sie erzeugt einen konstruktiven Innovationsdruck, der das spätere Planungsteam dazu zwingt, verlassene Pfade des \enquote{Business as Usual} zu verlassen und proaktiv nach effizienteren Lösungen zu suchen. Ohne dieses Spannungsfeld tendieren Projekte dazu, lediglich bestehende Standards zu reproduzieren.
Zudem bildet dieses \enquote{Gap} die ökonomische Basis für das in Phase 1 und 2 folgende Inzentivierungsmodell (\textit{Pain/Gain-Share}). Nur wenn eine Lücke existiert, die durch Optimierung geschlossen und unterboten werden kann, entsteht der finanzielle Spielraum (\enquote{Pot}), aus dem spätere Boni für das Planungsteam ausgeschüttet werden können.

Erst wenn diese prinzipielle Machbarkeit bestätigt ist, wird das Projekt freigegeben und die Rekrutierung des integrierten Teams eingeleitet.

\subsection{Implementierungshürden in der Projektinitiierung}
\label{sec:4.2.4}

Während der theoretische Prozess der Phase 0 eine rein an den Nutzerbedürfnissen orientierte Wertdefinition und daraus abgeleitete Zielkosten vorsieht, zeigt die Analyse der deutschen Baupraxis signifikante systemische Divergenzen. Die folgenden Aspekte verdeutlichen, warum die Festlegung valider \textit{Allowable Costs} in der Realität oft scheitert.

\minisec{Das \enquote{Nicht-ROI}-Dilemma öffentlicher Auftraggeber}
Die TVD-Methodik leitet die \textit{Allowable Cost} idealerweise aus einem Business Case ab ($Allowable Cost < Expected Revenue - Profit$) \autocite[vgl.]{ballard_target_2011}. Diese ökonomische Logik versagt jedoch bei öffentlichen Bauvorhaben (z.\,B. Schulen, Museen), denen kein monetärer Return on Investment (ROI) gegenübersteht.
Anstelle einer marktbedingten Obergrenze tritt hier oft eine politisch gesetzte Budgetrestriktion (Haushaltsmittel). Dies birgt die Gefahr, dass die \textit{Allowable Cost} nicht funktional hergeleitet, sondern als willkürliches \enquote{Cost-Cap} gesetzt wird \autocite[vgl.]{zimina_target_2012}. Fehlt der ökonomische Korrekturmechanismus des Marktes, droht die Zielkostendefinition von einer wertorientierten Steuerung zu einer reinen Budget-Verwaltung zu degenerieren.

\minisec{Deterministische Haushaltslogik vs. Probabilistische Steuerung}
TVD arbeitet in frühen Phasen definitionsgemäß mit Zielkorridoren und probabilistischen Kostenschätzungen (z.\,B. p85-Werten), um der Unsicherheit des Entwurfs Rechnung zu tragen \autocite[vgl.]{tommelein_ballard_2016}.
Das deutsche Haushaltsrecht (LHO) und die Verwaltungspraxis verlangen hingegen oft bereits vor Planungsbeginn einen deterministischen, skalaren Endwert (\enquote{Punktlandung}) zur Freigabe der Mittel \autocite[vgl.]{kalusche_projektmanagement_2020}. Dieser \enquote{Systembruch} zwingt Projektteams dazu, eine Scheingenauigkeit zu suggerieren, indem ein fixer \enquote{Guaranteed Maximum Price} (GMP) oder ein festes Budget simuliert wird, obwohl die Planungstiefe dies methodisch noch nicht zulässt.

\minisec{Eingeschränkter Lösungsraum durch hohe Regulierungsdichte}
Das TVD-Prinzip setzt voraus, dass bei der Zieldefinition in Phase 0 Variablen wie \textit{Scope} (Umfang) und \textit{Quality} verhandelbar bleiben, um das Kostenziel zu erreichen (Variabilität der Lösungen).
In der deutschen Baupraxis – insbesondere bei öffentlichen Auftraggebern – sind diese Parameter jedoch oft bereits vor Projektstart durch externe Vorgaben weitgehend determiniert. Gesetzliche Standards, Musterraumprogramme (z.\,B. im Schulbau) und strenge technische Richtlinien definieren faktisch eine \enquote{Untergrenze} der Leistung, die nicht unterschritten werden darf.
Für den Bauherrn bedeutet dies, dass der theoretische Lösungsraum für ein echtes \enquote{Design-to-Cost} massiv eingeschränkt ist. Die Kosten ergeben sich hier oft als Resultante aus den nicht-verhandelbaren Vorgaben, anstatt dass die Vorgaben an die erlaubten Kosten angepasst werden können.

\minisec{Das \enquote{Phantom-Nutzer}-Problem und die Abstraktion des Werts}
Oliva et al. (2024) weisen darauf hin, dass die für TVD essenzielle \textit{Value Generation} einen aktiven Nutzer voraussetzt \autocite[vgl.]{oliva_target_2024}. Im deutschen Investorenmodell (Büro/Wohnen) ist dieser Endnutzer zum Zeitpunkt der Zieldefinition oft unbekannt; im öffentlichen Bau wird er häufig durch administrative Stellvertretergremien (z.\,B. Bauämter) ersetzt, die keine direkte operative Entscheidungsgewalt über die Nutzungsprozesse haben \autocite[vgl.]{miron_target_2015}.
Ohne das direkte Feedback des \enquote{echten} Nutzers verkommt die Definition von \enquote{Wert} zu einer theoretischen Annahme der Planer. Dies führt zu dem Risiko, dass Zielkosten zwar eingehalten werden, das gebaute Ergebnis jedoch am tatsächlichen Bedarf vorbei optimiert wird.

\minisec{Das TOTEX-Vakuum: Investitionskostenfokus vs. Lebenszyklus}
Aktuelle TVD-Ansätze fokussieren in der Praxis stark auf die Einhaltung der Erstellungskosten (CAPEX) gemäß DIN 276. Eine ganzheitliche Betrachtung, die auch Betriebskosten (OPEX) und Nachhaltigkeitsziele in den Zielwert integriert (\enquote{Target Life Cycle Costing}), ist methodisch oft nicht verankert \autocite[vgl.]{ouma_target_2025}.
Dies führt in der Zieldefinition zu einem Zielkonflikt: Nachhaltige Investitionen (z.\,B. teurere Fassade, dafür geringere Kühlkosten), die sich erst im Betrieb amortisieren, fallen dem Kostendruck der Erstellungsphase zum Opfer (\enquote{Entfeinerung}), da der Zielwert keine Mechanismen für eine \textit{Total Expenditure} (TOTEX) Betrachtung bereitstellt.

\clearpage



\FloatBarrier
\clearpage
 
\section{Phase 1 - Validierung und Zielkostenbestimmung}
\label{sec:4.3}

Die Phase 1 markiert den operativen Startpunkt des Target Value Design. Während die vorangegangene Phase 0 durch die einseitige Definition der Rahmenbedingungen seitens Bauherrn geprägt war, erfolgt nun der Übergang in die integrierte Zusammenarbeit. Ziel dieser Phase ist es jedoch noch nicht, einen detaillierten Gebäudeentwurf zu erstellen. Vielmehr dient die Validierungsphase als vorgeschaltete technisch-ökonomische Machbarkeitsprüfung (\textit{Feasibility Validation}), welche die in Phase 0 getroffenen Annahmen (Business Case und Allowable Cost) verifiziert \autocite[vgl.]{ballard_target_2012}.

Die Kernaufgabe besteht darin, aus den abstrakten Anforderungen des Bauherrn ein funktionales Lösungskonzept (\textit{Basis of Design}) zu entwickeln und dieses durch Marktpreise zu plausibilisieren. Dieser Prozessschritt verankert das Prinzip der \enquote{Validierung vor der Detaillierung}, wonach bereits vor dem Einsatz signifikanter Ressourcen für die Ausarbeitung von Plänen der Nachweis erbracht werden muss, dass das Projekt für das verfügbare Budget realisierbar ist. Scheitert dieser Nachweis, sieht das TVD-Modell an dieser Stelle eine explizite \enquote{No-Go}-Entscheidung (Projektabbruch) vor \autocite[vgl.]{ballard_target_2012}, um die Gefahr späterer Kostenüberschreitungen und Fehlinvestitionen zu minimieren \autocite[vgl. S. 171]{do_target_2014}. Erst mit dem erfolgreichen Abschluss einer verbindlichen Zielkostenvereinbarung (\textit{Target Cost}) wird das Projekt für die eigentliche Planungsphase (Design to Targets) freigegeben \autocite[vgl.]{ballard_target_2025}.

\subsection{Teamstruktur und Wertdefinition}
\label{sec:4.3.1}

Der operative Beginn der Validierungsphase erfolgt durch die Zusammenstellung des Kernteams. Um die im \ac{TVD} geforderte Kostensicherheit bereits vor der detaillierten Planung zu gewährleisten, ist die frühzeitige Einbindung von Ausführungswissen zwingend erforderlich. Organisatorisch wird dies durch die Bildung von sogenannten \textbf{Clustern} (auch Systemgruppen oder \textit{Cross Functional Teams}) realisiert. Diese interdisziplinären Einheiten setzen sich aus Architekten, Fachplanern und Vertretern der ausführenden Schlüsselgewerke zusammen, wobei die Budgetverantwortung für spezifische Gebäudesysteme direkt auf diese Teams übertragen wird. \autocite[vgl. S.26]{ballard_target_2025}

% Hier sollte noch das wording "Kernteam" verarbeitet werden und betont werden dass dieses aus den Schlüsselgewerken besteht weche wiederrum entwurfsabhängig sind.

Im Gegensatz zur klassischen Projektabwicklung, bei der ausführende Firmen erst nach Abschluss der Planung vertraglich gebunden werden, integriert das TVD-Modell diese Akteure bereits in der Phase 1. Die vertragliche Grundlage hierfür bilden in der Regel vorgelagerte Dienstleistungsverträge (\textit{Preconstruction Services Agreements}), die eine Vergütung der Beratungs- und Kalkulationsleistungen vorsehen, ohne bereits die spätere Bauausführung fest zu beauftragen \autocite[vgl.]{heidemann_kooperative_2011}. Ziel dieser Struktur ist es, Planungs- und Kostenkompetenz in einer Entscheidungseinheit zu bündeln, um sofortiges Kostenfeedback zu ermöglichen.

Inhaltlich beginnt die Zusammenarbeit des Kernteams mit einer gemeinsamen Definition des Werteverständnisses (\textit{Value Definition}. Diese greift die Vorgaben des Bauherren \enquote{Stimme des Kunden} auf und definiert die funktionalen Anforderungen, losgelöst von der physischen Lösung. Essenziell ist dabei der Transfer des Verständnisses vom reinen Bau-Soll hin zum wirtschaftlichen Zweck der Investition (\textit{Business Case}). Das Team muss die strategischen Ziele und Wertvorstellungen des Kunden verstehen, um in der späteren Planung Konflikte autonom lösen zu können \autocite[vgl. S.13]{ballard_target_2025}.

Den Abschluss dieses Initialisierungsprozesses bildet die Überführung der subjektiven Kundenwünsche in objektive \textbf{\ac{CoS}}. Während die Wertvorstellungen oft abstrakt formuliert sind, stellen die CoS konkrete Kriterien dar (z.B. spezifische Anforderungen an Flexibilität, Akustik oder Energieeffizienz), die erfüllt sein müssen, damit das Projekt als erfolgreich betrachtet wird. Sie fungieren als qualitative Steuerungsgröße für alle weiteren Entscheidungen im Planungs- und Realisierungsprozess. Eine Kosteneinsparung, die zur Verletzung der \ac{CoS} führt, wird nicht als Effizienzgewinn, sondern als unzulässige Wertminderung betrachtet \autocite[vgl. S. 120]{ballard_target_2025}.

\subsection{Konzeptentwicklung und Validierung}
\label{sec:4.3.2}

Auf Basis der definierten Wertziele entwickeln die Cluster das sogenannte \textbf{Basis of Design} (BoD). Im Gegensatz zum klassischen Vorentwurf, der oft bereits spezifische architektonische Lösungen fixiert, definiert das BoD primär funktionale Qualitäten und Systementscheidungen, die notwendig sind, um die \textit{Conditions of Satisfaction} zu erfüllen. Es dient als technischer Nachweis, dass die Anforderungen des Kunden innerhalb des Budgets realisierbar sind, ohne bereits eine detaillierte Planung vorwegzunehmen \autocite[vgl.]{lci_guide_2016}\,\autocite[vgl. S.~13]{ballard_target_2025}.

Methodisch folgt dieser Prozess einer streng iterativen Logik. Anstatt Planung und Kalkulation sequenziell zu trennen, arbeiten Architekten und Kalkulatoren in kurzen Zyklen zusammen: Lösungsansätze werden skizziert und unmittelbar einer kostentechnischen Bewertung unterzogen. Dies ermöglicht ein sofortiges Feedback (\textit{Rapid Estimating}), wodurch Fehlentwicklungen frühzeitig erkannt und korrigiert werden können, noch bevor sie sich im Entwurf verfestigen. \autocite[vgl.]{zimina_target_2012}\,\autocite[vgl.]{ballard_target_2012}

Als zentrales Steuerungsinstrument dient hierbei das \textbf{Cost Modeling} (Kostenmodellierung). Da in dieser frühen Phase oft noch keine detaillierten Mengen für eine klassische Kalkulation vorliegen, werden parametrische Kostenmodelle genutzt. Diese basieren auf historischen Daten und identifizieren die wesentlichen Kostentreiber (\textit{Cost Drivers}) des Entwurfs. Das Ziel ist es, die Kostenentwicklung proaktiv zu prognostizieren, anstatt sie nur reaktiv zu erfassen. \autocite[vgl. S.~104]{ballard_target_2025}\,\autocite[vgl.]{ballard_target_2012}

Dabei gilt der fundamentale TVD-Grundsatz, dass die Kosten als unabhängige Entwurfsvariable betrachtet werden: Überschreitet ein technischer Lösungsansatz das anteilige Budget des Clusters, wird nicht das Budget erhöht, sondern die technische Lösung so lange variiert, bis sie kostenkonform ist. Das Design muss sich dem Budget anpassen, nicht umgekehrt. \autocite[vgl.]{zimina_target_2012}\,\autocite[vgl.]{ballard_target_2012}

Dieser Prozess mündet in einem kontinuierlichen \textbf{Validierungs-Check}. Das Team gleicht fortlaufend ab, ob das entwickelte \textit{Basis of Design} sowohl die qualitativen Anforderungen (CoS) erfüllt als auch die ökonomische Grenze der \textit{Allowable Cost} einhält. Nur wenn diese Kongruenz nachgewiesen ist, gilt das Konzept als validiert. \autocite[vgl.]{lci_guide_2016}\,\autocite[vgl. S.~15]{ballard_target_2025} 

\subsection{Budgetfreigabe und Zielkostenfestlegung}
\label{sec:4.3.3}

Die Ergebnisse der Validierungsphase werden abschließend im \textbf{Validierungsbericht} zusammengeführt. Dieses Dokument dient dem Bauherrn als fundierte Entscheidungsgrundlage: Es weist nach, dass das entwickelte \textit{Basis of Design} die funktionalen Anforderungen und \textit{Conditions of Satisfaction} erfüllt und innerhalb des gesteckten Kostenrahmens realisierbar ist. Auf dieser Basis trifft der Bauherr die formale \textbf{Budgetfreigabe} (\textit{Go / No-Go Decision}). Nur bei einem positiven Nachweis der Machbarkeit erfolgt der Startschuss für die nächste Projektphase, andernfalls wird das Projekt abgebrochen oder neu definiert, um Fehlinvestitionen (\textit{Sunk Costs}) zu vermeiden. \autocite[vgl. S. 15]{ballard_target_2025}\, \autocite[vgl.]{ballard_target_2012}

Mit der Projektfreigabe erfolgt die formale Fixierung der Ziele im \textbf{Target Value Statement}. Ein zentraler Aspekt ist dabei die Festlegung der \textbf{Basiszielkosten} (\textit{Target Cost}). Diese leiten sich aus den validierten Marktkosten (\textit{Estimated Cost}) ab, werden jedoch im TVD-Verfahren typischerweise unterhalb dieser Prognose angesetzt. Diese Differenz dient als bewusster Puffer gegen Unvorhergesehenes und schafft einen Innovationsdruck für das Team, die Kosten durch intelligentes Design weiter zu senken \autocite[vgl. S.~10]{ballard_target_2025}\,\autocite[vgl.]{zimina_target_2012}.

Zur Absicherung dieser Verpflichtung wird mit der Festlegung der Zielkosten auch der kommerzielle \textbf{Pain/Gain-Share-Mechanismus} aktiviert. Unterschreitet das Team durch Optimierungen in der Folgephase die Basiszielkosten, wird die Einsparung zwischen Bauherr und Planungsteam geteilt (\textit{Gain}). Überschreiten die Kosten hingegen den Zielwert, haftet das Team bis zu einer definierten Grenze mit seinem Risikopool (\textit{Pain}). Diese vertragliche Kopplung von Projekterfolg und Unternehmensgewinn stellt sicher, dass die Einhaltung der Zielkosten für alle Beteiligten nicht nur eine vertragliche Pflicht, sondern ein ökonomisches Eigeninteresse darstellt. \autocite[vgl.]{zimina_target_2012}\,\autocite[vgl.]{lci_guide_2016}

\subsection{Vergabe- und vertragsrechtliche Implementierungshürden}
\label{sec:4.3.4}

Die erfolgreiche Initialisierung der TVD-Phase 1 setzt die frühzeitige vertragliche Bindung der Schlüsselakteure voraus. In der deutschen Baupraxis trifft diese Anforderung jedoch auf die rigiden Schranken des Vergaberechts und die Systematik des Werkvertragsrechts, was zu folgenden strukturellen Konflikten führt.

\minisec{Das Wettbewerbsparadoxon: Preisfokus vs. Kompetenzauswahl}
TVD erfordert die Auswahl der Partner primär anhand von Qualifikationen, \enquote{Soft Skills} und der Bereitschaft zur Kollaboration. Für öffentliche Auftraggeber, die an das GWB und die VgV gebunden sind, stellt dies eine massive Hürde dar.
Das Vergaberecht fordert einen transparenten Wettbewerb um das \enquote{wirtschaftlichste Angebot} (§ 127 GWB). Da zum Zeitpunkt der TVD-Beauftragung oft noch keine detaillierte Leistungsbeschreibung vorliegt (da diese erst gemeinsam entwickelt werden soll), ist ein reiner Preiswettbewerb methodisch unmöglich. Die rechtssichere Operationalisierung von \enquote{Kooperationsfähigkeit} als Zuschlagskriterium erfordert aufwendige Verfahren (z.\,B. Wettbewerblicher Dialog), die viele öffentliche Auftraggeber aufgrund der Komplexität scheuen \autocite[vgl.]{heidemann_vergaberecht_2024}.

\minisec{Die \enquote{Vorbefasstheits-Falle} beim Early Contractor Involvement}
Die von Granja et al. (2023) geforderte frühzeitige Einbindung ausführender Unternehmen (ECI) als Berater in der frühen Phase birgt im deutschen Vergaberecht das Risiko der \enquote{Projektantenproblematik} (§ 7 VgV).
Ein Unternehmen, das in Phase 0/1 an der Zielkostenfindung mitgewirkt hat, verfügt über einen Wissensvorsprung. Um den Grundsatz der Gleichbehandlung im späteren Wettbewerb um die Bauleistung nicht zu verletzen, droht diesem Unternehmen der Ausschluss vom Verfahren. Dies führt zu einem systemischen Bruch: Das wertvolle Wissen, das in der Beratungsphase aufgebaut wurde, darf nicht nahtlos in die Realisierungsphase überführt werden, was den Kernnutzen von TVD konterkariert.

\minisec{Marktzutrittsbarrieren durch Digitalisierungsanforderungen (KMU-Lücke)}
Effizientes TVD setzt laut Ouma et al. (2025) datenbasierte Entscheidungsprozesse voraus (\enquote{Quantitatively Managed}), was in der Praxis den Einsatz von BIM und modellbasierten Kalkulationsmethoden bedingt.
Diese Anforderung steht im Spannungsfeld zur mittelständischen Struktur der deutschen Bauwirtschaft. Vielen kleineren Handwerksbetrieben (KMU) fehlen die Ressourcen und die digitale Reife für eine Teilnahme an solch hochintegrierten Prozessen. Die strikte Anwendung von TVD-Standards droht somit, lokale und spezialisierte Handwerksbetriebe vom Wettbewerb auszuschließen und eine Marktkonzentration auf Großkonzerne zu fördern, was politischen Zielsetzungen oft widerspricht.

\minisec{Dichotomie von Werkvertragsrecht und Dienstleistungsprozess}
Juristisch betrachtet ist TVD ein Dienstleistungsprozess (\enquote{Tätigwerden zur Erreichung eines Ziels}), während das deutsche Bauvertragsrecht (VOB/B) vom Typus des Werkvertrags geprägt ist, der einen konkreten Erfolg schuldet.
Klassische Verträge honorieren die Erstellung des Bauwerks, bieten aber kaum Mechanismen, um die intellektuelle Leistung der \enquote{Kostenoptimierung} oder \enquote{Variantenuntersuchung} zu vergüten, wenn diese nicht direkt in verbaute Masse mündet. Ohne vertragliche Sonderlösungen (z.\,B. Preconstruction Services Agreements) arbeiten Bauunternehmen in der TVD-Phase faktisch auf eigenes Risiko in der Hoffnung auf den späteren Auftrag, was keine stabile Basis für eine vertrauensvolle Zusammenarbeit darstellt.








% Vertragliche Synchronisation der Interessen

% Eine valide Überprüfung des Business Case durch die ausführenden Unternehmen scheitert laut Ballard in der konventionellen Praxis häufig an der antagonistischen Vertragsstruktur und dem daraus resultierenden Misstrauen ('Adversarial Relationship'). Der Auftraggeber befürchtet hierbei opportunistisches Verhalten der Auftragnehmer.\autocite[vgl. S. 15]{ballard_target_2012}

% In der Validierungsphase des TVD wird dieses Prinzipal-Agent-Problem durch den Mechanismus des Painsharing aufgelöst. Da eine Überschreitung der Kostenziele direkt die Gewinnmargen der Projektpartner reduziert ('Reduced Profit Margin'), werden die kommerziellen Interessen von Auftraggeber und Auftragnehmern synchronisiert ('Alignment of Interests'). Dies stellt sicher, dass die Validierung des Business Case nicht durch Eigeninteressen verzerrt wird, sondern auf einer realistischen Einschätzung der Machbarkeit basiert.\autocite[vgl. S. 15]{ballard_target_2012}















\clearpage

\section{Phase 2 - Wertorientierte Planung}
\label{sec:4.4}

Mit dem Abschluss der Validierungsphase (Phase 1) und der vertraglichen Fixierung der Zielkosten (Target Cost) ändert sich die Prozesslogik fundamental. Während Phase 1 die Machbarkeit prüft, fokussiert sich Phase 2 auf die operative Ausarbeitung einer qualitativen Lösung, die zwingend innerhalb des gesetzten Zielkostenrahmens verbleibt ("Design to Targets").
Der Planungsprozess wandelt sich hierbei von einer linearen Abarbeitung hin zu einem zyklischen, iterativen Steuerungsmodell.

\subsection{Zielsetzung und modelltheoretische Abgrenzung}
\label{sec:4.4.1}
Das primäre Ziel der Phase 2 ist die Überführung der funktionalen Anforderungen (aus Phase 0) in ein realisierungsfähiges Design, das Kostensicherheit bietet. Dieser Zustand wird im TVD als "Definitive Design" bezeichnet.

Es ist an dieser Stelle notwendig, eine differenzierte modelltheoretische Abgrenzung vorzunehmen. In der gelebten TVD-Praxis (insb. USA/IPD) verschwimmen Planung und Ausführung durch "Fast-Tracking" und frühe Fertigung komplett. In der deutschen Baupraxis existiert eine solche Integration zwar ebenfalls – beispielsweise in Generalunternehmer-Modellen oder bei funktionalen Ausschreibungen –, jedoch ist der "Standardfall" des deutschen Ordnungsrahmens durch das Trennungsprinzip geprägt.

Sowohl das Vergaberecht (VOB/A: Trennung von Planung und Bau) als auch das Preisrecht (HOAI: Honorierung abgeschlossener Planungsphasen) suggerieren eine sequenzielle Logik, bei der die Ausführung erst nach Abschluss der Planung beginnt. Selbst in partnerschaftlichen Modellen erfolgt die Einbindung der Bauunternehmen oft erst nach Abschluss der Entwurfsplanung (LPH 3), womit das wesentliche Potenzial der frühen Einflussnahme auf das Designkonzept ("Constructability") bereits eingeschränkt ist.

Für die vorliegende Untersuchung wird daher analytisch an der Trennung zwischen Phase 2 (Design) und Phase 3 (Realisierung) festgehalten. Dies dient dazu, die Reibungspunkte des TVD-Ansatzes mit dem kodifizierten deutschen Standardmodell (Einzelvergabe nach VOB, Planung nach HOAI) präzise herauszuarbeiten. Inhaltlich deckt die Phase 2 dieses Modells somit den gesamten Planungsraum der HOAI-Leistungsphasen 2 (Vorplanung) bis 5 (Ausführungsplanung) ab. Dies markiert die zentrale Verschiebung: Im TVD findet die werkplanerische Detaillierung nicht erst nach der Vergabe statt, sondern wird als integrative Leistung vorgezogen, um die Preissicherheit zu garantieren.


\subsection{Strukturierung des Budgets: Target Cost Allocation}
\label{sec:4.4.2}
Bevor die inhaltliche Planungsarbeit beginnt, muss das in Phase 1 ermittelte Gesamtbudget ("Project Target Cost") in steuerbare Teilbudgets zerlegt werden. Dieser Prozess der "Target Cost Allocation" transformiert das Budget von einer abstrakten Gesamtsumme in konkrete Handlungsanweisungen für die Planungsteams.

Anders als bei einer klassischen Kostenschätzung nach DIN 276 (Gewerke), erfolgt die Allokation im TVD systemorientiert auf die zu bildenden Cluster (vgl. Kap. 4.4.3). Das Budget wird typischerweise in einem Gegenstromverfahren verteilt:
\begin{itemize}
    \item \textbf{Top-Down:} Die Projektsteuerung gibt basierend auf Benchmarks grobe Zielwerte für die Cluster vor (z.B. Zielwert Fassade: 2,5 Mio. €).
    \item \textbf{Bottom-Up:} Die Cluster-Teams prüfen diese Werte auf Plausibilität und melden Rückbedarf an.
\end{itemize}
Ein essenzieller Bestandteil der Allokation ist das "Contingency Management". Das Budget wird nicht zu 100\% auf die Cluster verteilt. Ein Teil verbleibt als zentraler Puffer ("Project Contingency") beim Bauherrn/Steuerungsteam, um unvorhergesehene Risiken abzudecken, ohne die Target Cost zu erhöhen.

\subsection{Organisatorische Struktur: Interdisziplinäre Cluster-Teams}
\label{sec:4.4.3}
Um die Komplexität des Gesamtprojekts handhabbar zu machen, wird die operative Projektorganisation in Phase 2 von einer gewerkespezifischen Trennung auf interdisziplinäre "Cluster-Teams" (Cross-Functional Teams) umgestellt.

\textbf{Zusammensetzung und Rollen}
Ein Cluster (z.B. "Hülle") vereint alle für dieses Teilsystem relevanten Kompetenzen an einem Tisch. Die typische Besetzung umfasst:
\begin{enumerate}
    \item \textbf{Cluster Lead (Planung):} Meist der Architekt oder Fachplaner, der die design-technische Führung übernimmt.
    \item \textbf{Construction Lead (Ausführung):} Ein Vertreter des ausführenden Unternehmens (z.B. Polier oder Projektleiter des Fassadenbauers).
    \item \textbf{Estimator (Kalkulation):} Ein Kostenschätzer, der permanenten Zugriff auf aktuelle Marktpreise hat.
\end{enumerate}

\textbf{Die Rolle der "Makers"}
Ein entscheidendes Novum des TVD-Prozesses ist die Integration der ausführenden Firmen ("Makers") bereits in dieser frühen Designphase. Ihre Rolle wandelt sich vom reinen Empfänger fertiger Pläne zum aktiven "Co-Creator". Sie bringen fertigungstechnisches Wissen ("Constructability") ein, um Lösungen zu finden, die funktional gleichwertig, aber günstiger zu fertigen sind.

> \textit{Transfer-Hinweis:} Diese frühe Integration steht im Konflikt zur konventionellen Vergabepraxis (VOB/A), die eine strikte Trennung von Planung und Ausführung sowie Produktneutralität fordert. (Diskussion in Kap. 5).

\subsection{Prozesslogik: Der zyklische Steuerungsloop (Continuous Estimating)}
\label{sec:4.4.4}
Das methodische Kernstück der Phase 2 ist das "Continuous Estimating". Um den im konventionellen Prozess üblichen "Blindflug" zwischen den Leistungsphasen zu vermeiden, wird der Planungsprozess in kurze, sich wiederholende Zyklen (Sprints) von typischerweise 3 bis 4 Wochen unterteilt.

Ein solcher Zyklus folgt einer festen Logik:
\begin{enumerate}
    \item \textbf{Design \& Innovation (Woche 1-2):} Die Cluster-Teams entwickeln Lösungsansätze. Hierbei wird oft parallel an mehreren Varianten gearbeitet (Set-Based Design), um den Lösungsraum nicht zu früh einzuschränken.
    \item \textbf{Pricing \& Feedback (Woche 3):} Die erarbeiteten Stände werden kalkuliert. Da die Baufirmen Teil des Teams sind, basiert dies auf echten Marktpreisen, nicht auf Datenbank-Mittelwerten. Unterstützt wird dies oft durch BIM-basierte Massenauszüge (Quantity Take-Off).
    \item \textbf{Steuerungs-Gate (Ende Woche 3):} Im zentralen "Big Room Meeting" werden die Kostenstände aller Cluster zusammengeführt.
\end{enumerate}

\textbf{Entscheidung am Gate:}
Das Gate fungiert als harte Qualitätsschranke. Liegt die Prognose eines Clusters über dem zugewiesenen Budget, darf nicht detailliert werden. Das Team muss "zurück ans Reißbrett", um das Design anzupassen (Design to Cost). Nur bei Einhaltung des Budgets erfolgt die Freigabe für den nächsten Detaillierungsgrad.

> \textit{Transfer-Hinweis:} Die HOAI honoriert den Planungserfolg am Phasenende. Iterative Schleifen zur Kostenoptimierung ("Rework") sind im linearen Leistungsbild oft nicht abgebildet und führen zu Honorardiskussionen. (Diskussion in Kap. 5).

\subsection{Entscheidungsfindung und Steuerungsmethodik}
\label{sec:4.4.5}
Damit die iterative Arbeitsweise zielgerichtet verläuft, stützt sich Phase 2 auf zwei spezifische Methoden zur Entscheidungsfindung:

\textbf{Set-Based Design (SBD)}
Anstatt sich intuitiv auf eine einzige Lösung festzulegen ("Point-Based Design") und diese linear auszuarbeiten, halten die Cluster-Teams bewusst mehrere Optionen parallel offen (Sets). Diese werden grob dimensioniert und kalkuliert. Optionen, die sich als technisch oder wirtschaftlich nicht tragfähig erweisen, werden sukzessive eliminiert, bis die optimale Lösung übrig bleibt. Dies vermeidet teure negative Iterationen in späten Phasen.

\textbf{Choosing by Advantages (CBA)}
Muss zwischen technisch gleichwertigen Varianten entschieden werden, kommt die Methode "Choosing by Advantages" zum Einsatz. Entscheidungen basieren dabei nicht auf einer Gewichtung von Faktoren (die subjektiv sein kann), sondern auf der Analyse der konkreten Vorteile ("Advantages") einer Option im Verhältnis zu ihren Kosten. Dies objektiviert den Entscheidungsprozess und macht ihn gegenüber dem Bauherrn dokumentierbar (z.B. mittels A3-Reports).

\subsection{Abschluss der Phase 2: Das Definitive Design}
\label{sec:4.4.6}
Der Abschluss der Phase 2 ist erreicht, wenn das "Definitive Design" vorliegt. Dieser Meilenstein unterscheidet sich qualitativ von einer klassischen Ausführungsplanung.

Das Design ist zu diesem Zeitpunkt nicht nur zeichnerisch dargestellt, sondern bereits "produktionsreif" (Production Ready). Durch die Mitwirkung der "Makers" sind Fertigungsdetails, Montageabläufe und Materialverfügbarkeiten bereits geklärt.
Das entscheidende Kriterium für den Übergang in Phase 3 ist jedoch die Preissicherheit: Das Team gibt ein verbindliches Commitment ab, dass das Design innerhalb der Target Cost realisierbar ist.

\textbf{Das finale Gate (Release for Construction):}
Erst wenn die Prognosekosten (Expected Cost) sicher unterhalb der Allowable Cost liegen, wird das Projekt zur physischen Realisierung freigegeben.

> \textit{Transfer-Hinweis:} Die Synchronisation von Planungsfreigabe und absoluter Preissicherheit ist in Einheitspreisverträgen kaum abzubilden, da Endkosten hier oft erst nach Aufmaß feststehen. (Diskussion in Kap. 5).







\clearpage

\section{Phase 3 - Realisierung und Abschluss}
\label{sec:4.5}


Mit dem Erreichen des Definitive Design und der Freigabe zur Ausführung ("Release for Construction") tritt das Projekt in die dritte und letzte Phase des Target Value Design ein. Im Gegensatz zu konventionellen Projekten, in denen während der Bauphase oft noch massive Umplanungen und Kostensteigerungen (Nachträge) auftreten, verlagert sich der Fokus in Phase 3 rein auf die exekutive Umsetzung des Geplanten.

\subsection{Prozessziel: Abweichungsfreie Umsetzung}
Das primäre Ziel der Phase 3 ist die physische Errichtung des Bauwerks unter strikter Einhaltung der in Phase 2 garantierten Parameter (Kosten, Zeit, Qualität). Da die "Makers" (ausführende Firmen) das Design selbst mitentwickelt haben, entfallen typische Bauablaufstörungen wie Probleme in der praktischen Umsetzung oder fehlende Detailinformationen weitgehend.
Die Planungsarbeit beschränkt sich in dieser Phase auf die Just-in-Time-Lieferung von logistischen Informationen oder minimalen Anpassungen an unvorhergesehene Baustellenbedingungen.

\subsection{Kostensteuerung: Cost Monitoring statt Cost Estimating}
Die Art der Kostensteuerung ändert sich fundamental. Während in Phase 2 noch aktiv kalkuliert und optimiert wurde ("Estimating"), geht es nun um das reine "Monitoring":
\begin{itemize}
    \item \textbf{Soll-Ist-Vergleich:} Die auflaufenden Kosten werden permanent gegen die Target Cost (bzw. die Allowable Cost) gespiegelt.
    \item \textbf{Gewinnsicherung:} Da das Budget fixiert ist, arbeiten die Cluster-Teams nun daran, durch Effizienz auf der Baustelle (z.B. durch Lean Construction Methoden wie Taktplanung) ihren Gewinn zu maximieren. Ein Unterschreiten der Kosten ist nun im direkten Interesse der Ausführenden (Pain/Gain-Sharing).
\end{itemize}

\subsection{Abschluss und Lernen}
Das TVD-Modell endet nicht mit der Schlüsselübergabe, sondern mit einer formalen Feedback-Schleife. Die tatsächlichen Endkosten werden final mit den Target Costs abgeglichen.
\begin{itemize}
    \item \textbf{Incentivierung:} Eventuelle Einsparungen ("Underrun") werden gemäß dem vorab definierten Schlüssel zwischen Bauherr und Risikopool ausgeschüttet.
    \item \textbf{Wissenssicherung:} Abweichungen (sowohl positive als auch negative) werden analysiert, um die Kennzahlen für zukünftige Projekte ("Benchmarks" für Phase 0) zu schärfen.
\end{itemize}

> \textit{Transfer-Hinweis:} In der deutschen VOB-Realität ist Phase 3 oft durch Nachtragsmanagement ("Claims") geprägt. Der TVD-Ansatz, dass der Preis fixiert ist und Nachträge durch die vorherige Integration ausgeschlossen sind, erfordert vertraglich eine völlig andere Risikoverteilung als den klassischen Einheitspreisvertrag. (Diskussion in Kap. 5).

\clearpage