\chapter{Methodik}
\label{ch:methodik}

% Ziel: Das Vorgehen der Arbeit transparent und wissenschaftlich nachvollziehbar darlegen.
% Es wird der "Plan" für die Forschung beschrieben.

\section{Forschungsdesign und methodische Verortung}
\label{sec: 3.1}
%   - Ziel: Wahl der Methodik begründen und wissenschaftlich einordnen.
%   - Inhalt:
%     - Wiederholung der Forschungsfrage.
%     - Begründung für literaturbasierten, konzeptionell-analytischen Ansatz.
%     - Abgrenzung zu anderen Methoden (z.B. empirisch).
%     - Optional: Visualisierung des Vorgehens als einfaches Flussdiagramm.
\subsection{Herleitung des Forschungsbedarfs aus der Forschungsfrage}
\label{sec: 3.1.1}

%   Wiederholung der Forschungsfrage prägnant
%   Ableitung der notwendingen Schritte zur Beantwortung
%   


\subsection{Einordnung in das Forschungsparadigma: Ein qualitativ-konstruktiver Ansatz}
\label{sec: 3.1.2}

\subsection{Wahl des konkreten Forschungsdesigns: Die konzeptionell-analytische Arbeit}
\label{sec: 3.1.3}

  \subsection{Gütekriterien des qualitativen Forschungsdesigns}
\label{sec: 3.1.4}

\subsection{Visualisierung des Forschungsprozesses}
\label{sec: 3.1.5}

% Definition von Stilen, um den Code sauber zu halten
\tikzset{
  phase/.style={rectangle, rounded corners, draw=black, fill=gray!10,
                text width=0.85\textwidth, minimum height=1.5cm, text centered},
  pfeil/.style={-latex, thick}
}

\begin{figure}[hbt!]
  \makebox[\textwidth][c]{
    \begin{tikzpicture}
      \tikzset{
        phase/.style={rectangle, rounded corners, draw=black, fill=gray!10,
                      text width=0.9\textwidth, minimum height=1.5cm, text centered},
        pfeil/.style={-latex, thick}
      }

      \node[phase] (phase1) {
        \textbf{Phase 1: Forschungsdesign (Kap. 3.1)}\\
        \small Problemstellung, Methodische Verortung, Konzeptionell-analytischer Ansatz
      };

      \node[phase, below=0.8cm of phase1] (phase2) {
        \textbf{Phase 2: Literaturanalyse (Kap. 3.2)}\\
        \small Systematische Recherche, Qualitative Inhaltsanalyse, Identifikation von Bausteinen
      };

      \node[phase, below=0.8cm of phase2] (phase3) {
        \textbf{Phase 3: Modellkonstruktion (Kap. 3.3)}\\
        \small Synthese der Bausteine, Modellierung des idealtypischen TVD-Prozesses
      };

      \node[phase, below=0.8cm of phase3] (phase4) {
        \textbf{Phase 4: Kontextanalyse \& Transfer (Kap. 3.4)}\\
        \small Fit-Gap-Analyse (Vergleich mit HOAI/VOB), Ableitung des Transferleitfadens
      };

      \draw[pfeil] (phase1) -- (phase2);
      \draw[pfeil] (phase2) -- (phase3);
      \draw[pfeil] (phase3) -- (phase4);

    \end{tikzpicture}
  }
  \caption{Schematische Darstellung des vierphasigen Forschungsprozesses}
  \label{fig:forschungsprozess_tikz}
\end{figure}

\FloatBarrier

\section{Systematische Literaturrecherche und -analyse}
\label{sec: 3.2}

%   - Ziel: Das Vorgehen bei der Literatursuche und -auswahl exakt dokumentieren.
%   - Inhalt:

Für die systematische Literaturrecherche werden zunächst relevante Datenbanken identifiziert. Der Prozess wurde in drei Stufen gegliedert:

1. Größe - International \& Interdisziplinär
\begin{itemize}[leftmargin=2em]
    \item \textbf{Scopus}: Bietet eine extrem breite Abdeckung von ingenieurwissenschaftlichen und technischen Publikationen.
    \item \textbf{Web of Science}: Ist ähnlich umfassend wie Scopus und eine Standarddatenbank in der Wissenschaft.
\end{itemize}
2. Fachspezifische Datenbanken mit Relevanz für Bauingenieurwesen bzw. Baubetrieb
\begin{itemize}[leftmargin=2em]
    \item \textbf{Scopus}: Bietet eine extrem breite Abdeckung von ingenieurwissenschaftlichen und technischen Publikationen.
    \item \textbf{Web of Science}: Ist ähnlich umfassend wie Scopus und eine Standarddatenbank in der Wissenschaft.
\end{itemize}
3. Graue Literatur - Der Blick in die Praxis
\begin{itemize}[leftmargin=2em]
    \item \textbf{Scopus}: Bietet eine extrem breite Abdeckung von ingenieurwissenschaftlichen und technischen Publikationen.
    \item \textbf{Web of Science}: Ist ähnlich umfassend wie Scopus und eine Standarddatenbank in der Wissenschaft.
\end{itemize}

Die hier Angewandte Suchstrategie verwendet Schlüsselbegriffe bzw. Wörter und Kombinationen auf deutsch und englisch:

\begin{table}[hbt!]
  \centering
  \caption{Systematische Gliederung der Suchbegriffe und -operatoren}
  \label{tab:suchstrategie}
  \begin{tabularx}{\textwidth}{l >{\raggedright\arraybackslash}X >{\raggedright\arraybackslash}X l}
    \toprule
    \textbf{Such-Konzept} & \textbf{Englische Schlagwörter} & \textbf{Deutsche Schlagwörter} & \textbf{Operatoren} \\
    \midrule
    Kernthema & "Target Value Design", "TVD", "Target Costing" & "Target Value Design", "TVD", "Zielkostenmanagement" & \texttt{OR} \\
    \addlinespace
    Kontext & "construction", "building industry", "AEC" & "Bauwesen", "Bauwirtschaft", "Bauprojekte" & \texttt{OR} \\
    \addlinespace
    Spez. Kontext & "Germany", "German construction" & "deutsche Baupraxis", "Deutschland", "HOAI", "VOB" & \texttt{OR} \\
    \addlinespace
    Ziel (Prozess) & "process*", "method*", "framework", "implementation" & "Prozess*", "Methode*", "Anwendung", "Vorgehensmodell" & \texttt{OR}, \texttt{*} \\
    \bottomrule
  \end{tabularx}
\end{table}

\clearpage

Ergebnis der Literaturrecherche

\begin{figure}[hbt!]
  \centering
  
  % --- TIKZ CODE FÜR DAS PRISMA-FLUSSDIAGRAMM ---
  \begin{tikzpicture}[
    node distance=0.5cm and 1cm, % Vertikaler und horizontaler Abstand
    % Stil für die Haupt-Boxen im Prozess
    mainbox/.style={
      rectangle, 
      draw=black, 
      fill=gray!10, 
      text width=0.60\textwidth, 
      minimum height=1.5cm, 
      text centered
    },
    % Stil für die Boxen mit den Ausschlussgründen
    sidebox/.style={
      rectangle, 
      draw=black, 
      fill=gray!10, 
      text width=0.35\textwidth, 
      minimum height=1.5cm, 
      align=left
    },
    % Stil für die seitlichen Beschriftungen (Identifizierung, etc.)
    labelbox/.style={
      rotate=90, 
      font=\bfseries
    }
  ]

  % --- DEFINITION DER KNOTEN (BOXEN) ---
  
  % Stufe 1: Identifizierung
  \node[mainbox] (ident) {
    \textbf{Identifizierte Datensätze aus Datenbanken} \\ (Anzahl N = ...)
  };
  \node[labelbox, left=0.5cm of ident.west, anchor=center] {Identifizierung};

  % Stufe 2: Screening
  \node[mainbox, below=1.5cm of ident] (screen1) {
    \textbf{Datensätze nach Entfernung von Duplikaten} \\ (Anzahl N = ...)
  };
  \node[mainbox, below=1cm of screen1] (screen2) {
    \textbf{Geprüfte Datensätze (Titel/Abstract)} \\ (Anzahl N = ...)
  };
  \node[labelbox, left=0.5cm of screen1.west, anchor=east] {Screening};
  
  % Stufe 3: Eignung
  \node[mainbox, below=1.5cm of screen2] (elig) {
    \textbf{Gelesene Volltext-Artikel} \\ (Anzahl N = ...)
  };
  \node[labelbox, left=0.5cm of elig.west, anchor=center] {Eignung};

  % Stufe 4: Einschluss
  \node[mainbox, below=1.5cm of elig] (incl) {
    \textbf{In die Synthese eingeschlossene Studien} \\ (Anzahl N = ...)
  };
  \node[labelbox, left=0.5cm of incl.west, anchor=center] {Einschluss};

  % Ausschluss-Boxen
  \node[sidebox, right=of screen2] (excl1) {
    \textbf{Ausgeschlossene Datensätze:} \\ (Anzahl N = ...) \\
    \textit{Gründe: Falsches Fachgebiet, keine wiss. Quelle, etc.}
  };
  \node[sidebox, right=of elig] (excl2) {
    \textbf{Ausgeschlossene Volltext-Artikel:} \\ (Anzahl N = ...) \\
    \textit{Gründe: Methodische Schwächen, kein relevanter Fokus, etc.}
  };

  % --- ZEICHNEN DER PFEILE ---
  \draw[-latex, thick] (ident) -- (screen1);
  \draw[-latex, thick] (screen1) -- (screen2);
  \draw[-latex, thick] (screen2) -- (elig);
  \draw[-latex, thick] (elig) -- (incl);
  
  % Pfeile zu den Ausschluss-Boxen
  \draw[-latex, thick] (screen2) -- (excl1);
  \draw[-latex, thick] (elig) -- (excl2);

  \end{tikzpicture}
  % --- ENDE DES TIKZ-CODES ---

  \caption{PRISMA-Flussdiagramm des systematischen Literaturrechercheprozesses}
  \label{fig:prisma_flowchart}
\end{figure}

%     - Recherchestrategie: Genutzte Datenbanken (Scopus, etc.), Fachinstitutionen (GLCI, etc.).
%     - Suchbegriffe: Beispiele für Keyword-Kombinationen (DE/EN).
%     - Auswahlkriterien: Inklusions-/Exklusionskriterien (Relevanz, Aktualität, Qualität).
%     - Analysemethode: Qualitative Inhaltsanalyse zur Identifikation von Prozessbausteinen, Rollen, Werkzeugen.

\FloatBarrier

\section{Prozessbeschreibung und Modellierung}
\label{sec: 3.3}

%   - Ziel: Erklären, wie aus den Literatur-Fragmenten das idealtypische TVD-Modell konstruiert wird.
%   - Inhalt:
%     - Synthese: Zusammenführung der identifizierten Bausteine aus diversen Quellen.
%     - Modellierungsansatz: Begründung für die Wahl eines Prozessmodells (z.B. Flussdiagramm, Swimlane).
%     - Validierung: Logische Prüfung des Modells auf Konsistenz und Vollständigkeit basierend auf der Literatur.

\FloatBarrier

\section{Kontextanalyse und Ableitung des Transferleitfadens}
\label{sec: 3.4}

%   - Ziel: Den analytischen Vergleich und die Ableitung der Handlungsempfehlungen beschreiben.
%   - Inhalt:
%     - Gegenüberstellung: Systematischer Vergleich des TVD-Modells mit der deutschen Baupraxis (Basis: Kapitel 2.1).
%     - Analysefokus: Strukturelle, rechtliche (HOAI/VOB) und kulturelle Gegebenheiten.
%     - Identifikation: Ziel ist das Aufdecken von Reibungspunkten und Anpassungsbedarfen ("Fit-Gap-Analyse").
%     - Struktur des Leitfadens: Erläutern, wie die Ergebnisse zu konkreten Handlungsempfehlungen aufbereitet werden.

