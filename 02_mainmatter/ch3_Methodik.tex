\chapter{Forschungsdesign und methodisches Vorgehen}
\label{ch:3}

\section{Methodischer Ansatz}
\label{sec:methodik_ansatz}

Da für die Anwendung von Target Value Design (TVD) in der deutschen Baupraxis noch keine standardisierten Verfahren existieren, verfolgt die vorliegende Arbeit einen literaturbasierten, konzeptionell-analytischen Forschungsansatz. Ziel ist die Entwicklung eines adaptierten Prozessmodells. Hierfür wird nicht auf quantitative Daten zurückgegriffen, sondern bestehendes Expertenwissen aus der Literatur systematisch neu geordnet und auf den deutschen Kontext übertragen.

Das Vorgehen folgt dabei einem iterativen Prozess, der sich in vier Phasen gliedert:
\begin{enumerate}
    \item \textbf{Identifikation:} Bestimmung und Analyse maßgeblicher Leitquellen (Best Practice).
    \item \textbf{Synthese:} Konstruktion eines idealtypischen Soll-Modells.
    \item \textbf{Detaillierung:} Gezielte Ergänzungsrecherche zur Schließung von Prozesslücken.
    \item \textbf{Transfer:} Abgleich mit den Restriktionen der deutschen Baupraxis (Fit-Gap).
\end{enumerate}

Dieser methodische Ablauf und die Zuordnung der einzelnen Arbeitsschritte zu den Kapiteln der vorliegenden Arbeit sind in \cref{fig:forschungsprozess_tikz} schematisch visualisiert.

\tikzset{
  phase/.style={rectangle, rounded corners, draw=black, fill=gray!10,
                text width=0.85\textwidth, minimum height=1.5cm, text centered},
  pfeil/.style={-latex, thick}
}

\begin{figure}[hbt!]
  \makebox[\textwidth][c]{
    \begin{tikzpicture}
      \tikzset{
        phase/.style={rectangle, rounded corners, draw=black, fill=gray!10,
                      text width=0.9\textwidth, minimum height=1.5cm, text centered},
        pfeil/.style={-latex, thick}
      }

      % PHASE 1: Design
      \node[phase] (phase1) {
        \textbf{Phase 1: Forschungsdesign (Kap. 1 \& 3.1)}\\
        \small Problemdefinition, Zielsetzung, Methodische Verortung
      };

      % PHASE 2: Datengrundlage (angepasst auf deine neuen Kap 3.2/3.3)
      \node[phase, below=0.8cm of phase1] (phase2) {
        \textbf{Phase 2: Datenerhebung \& Analyse (Kap. 3.2 \& 3.3)}\\
        \small Identifikation von Leitquellen, Strukturierende Inhaltsanalyse
      };

      % PHASE 3: Modell (Das ist das Ergebnis in Kap 4)
      \node[phase, below=0.8cm of phase2] (phase3) {
        \textbf{Phase 3: Modellsynthese (Kap. 4)}\\
        \small Entwicklung des idealtypischen TVD-Prozesses (Phasen 0--3)
      };

      % PHASE 4: Transfer (Das ist die Diskussion in Kap 5)
      \node[phase, below=0.8cm of phase3] (phase4) {
        \textbf{Phase 4: Kontextanalyse \& Transfer (Kap. 5)}\\
        \small Fit-Gap-Analyse (VOB/HOAI), Diskussion der Dilemmata
      };

      \draw[pfeil] (phase1) -- (phase2);
      \draw[pfeil] (phase2) -- (phase3);
      \draw[pfeil] (phase3) -- (phase4);

    \end{tikzpicture}
  }
  \caption{Schematische Darstellung des vierphasigen Forschungsprozesses}
  \label{fig:forschungsprozess_tikz}
\end{figure}
\FloatBarrier

\section{Datengrundlage und Recherchestrategie}
\label{sec:datengrundlage}

Die Datengrundlage der Arbeit wurde in einem zweistufigen Verfahren ermittelt, das sich an der Logik des \textit{Purposive Sampling} (gezielte Auswahl) orientiert.
Die Wahl dieses selektiven Ansatzes ist der Heterogenität des Forschungsfeldes geschuldet: Selbst bei einer auf das Bauwesen eingegrenzten Suche liefert der Begriff \enquote{Target Value Design} in einschlägigen Datenbanken eine enorme Anzahl an Treffern. Häufig wird TVD in der Literatur lediglich schlagwortartig aufgegriffen oder in Kontexten verortet, die für eine systematische Prozessmodellierung im deutschen Bauwesen nur bedingt aussagekräftig sind.

Um in dieser Fülle an Informationen nicht den Fokus zu verlieren und stattdessen ein methodisch fundiertes Prozessmodell entwickeln zu können, wurde der Priorität auf die Identifikation autoritativer Quellen gelegt. Anstatt einer quantitativen Breitenanalyse erfolgte daher zunächst die Extraktion der theoretischen und methodischen Kernelemente anhand etablierter Standardwerke.

\subsection{Stufe 1: Identifikation von Leitquellen (Basis-Recherche)}
Im ersten Schritt wurden gezielt \enquote{Seminal Works} (Schlüsselpublikationen) identifiziert, die den internationalen und nationalen Diskurs maßgeblich prägen. Diese bilden das theoretische Rückgrat für die Grobstruktur des Prozessmodells:
\begin{itemize}
    \item \textbf{Theorie:} Die Grundlagenwerke von \textit{Glenn Ballard} und \textit{Greg Howell}, die die Ursprünge und Prinzipien des TVD definieren.
    \item \textbf{Methodik (International):} Die Prozessbeschreibungen und Leitfäden des \textit{Lean Construction Institute (LCI)}, die den aktuellen \enquote{State of the Art} in den USA abbilden.
    \item \textbf{Transfer (National):} Veröffentlichungen des \textit{German Lean Construction Institute (GLCI)} und des \textit{Kompetenzzentrums IPA}, die erste Adaptionsansätze für Deutschland beschreiben.
\end{itemize}

\subsection{Stufe 2: Problemzentrierte Ergänzungsrecherche}
Während der Modellierung zeigte sich, dass die Leitquellen oft auf einer abstrakten Ebene verbleiben oder sich auf anglo-amerikanische Rechtsräume beziehen. Eine rein generische Suche nach Begriffen wie \enquote{Target Value Design Germany} führte in den einschlägigen Datenbanken (\textit{Scopus}, \textit{Web of Science}, \textit{Google Scholar}) jedoch zu einer unüberschaubaren Menge an Treffern (\enquote{Information Overload}), die häufig nur oberflächliche Nennungen der Methode enthielten, ohne technische Details zu klären.

Um spezifische Lücken im Verständnis oder bei der Übertragung auf deutsche Normen (z.\,B. \enquote{Wie verhält sich das Target Costing zur DIN 276?}) präzise zu schließen, wurde daher eine blockbasierte Suchstrategie angewandt.

\minisec{Auswahl der Datenbanken}

Um ein ganzheitliches Bild zu gewährleisten, das sowohl den internationalen wissenschaftlichen Standard als auch die spezifischen deutschen Praxisanforderungen abdeckt, erfolgte die Recherche in drei kategorisierten Quellenarten:

\begin{itemize}
    \item \textbf{Wissenschaftliche Metadatenbanken (Scopus \& Web of Science):}
    Diese Datenbanken wurden primär genutzt, um qualitätsgesicherte, peer-reviewte Publikationen zum methodischen Kern von TVD zu identifizieren. Durch ihre breite Abdeckung ingenieurwissenschaftlicher Journals stellten sie sicher, dass der internationale \enquote{State of the Art} vollständig erfasst wurde.

    \item \textbf{Erweiterte wissenschaftliche Suchdienste (Google Scholar):}
    Ergänzend wurde Google Scholar herangezogen. Dies war insbesondere für den deutschen Kontext essenziell, da hierüber auch juristische Kommentare, Dissertationen und Tagungsbände (z.\,B. der \textit{IGLC}-Konferenzen) auffindbar sind, die in den klassischen englischsprachigen Datenbanken oft unterrepräsentiert sind.

    \item \textbf{Fachinstitutionen und Graue Literatur:}
    Da die praktische Umsetzung von TVD in Deutschland oft schneller voranschreitet als die akademische Aufarbeitung, wurden gezielt Publikationsserver relevanter Fachverbände durchsucht. Hierzu zählten insbesondere die Veröffentlichungen des \textit{Lean Construction Institute (LCI)}, des \textit{German Lean Construction Institute (GLCI)} sowie Berichte des \textit{Kompetenzzentrums IPA}. Diese Quellen lieferten wertvolle Einblicke in die aktuelle Anwendungspraxis und vertragliche Modellierungen.
\end{itemize}

\subsubsection*{Suchlogik und Kombinatorik}
Anstatt breiter Schlagworte wurden spezifische Suchstrings konstruiert, die das Kernthema (TVD) mit dem spezifischen Kontext (Deutschland/Recht) und dem Detailproblem verknüpfen. Die Suchbegriffe wurden mittels Boolescher Operatoren (\texttt{AND}, \texttt{OR}) nach folgendem Schema kombiniert:

\begin{equation*}
    \text{Suche} = \text{\textbf{Methode}} \quad \texttt{AND} \quad \text{\textbf{Kontext}} \quad \texttt{AND} \quad \text{\textbf{Detailproblem}}
\end{equation*}

Die verwendeten Schlagworte für diese Blöcke sind exemplarisch in \cref{fig:suchlogik_grafik} dargestellt.

\begin{figure}[hbt!]
\centering
\begin{tikzpicture}[
    node distance=0.5cm,
    % Stil für die Boxen
    block/.style={
        rectangle, 
        draw=black, 
        fill=gray!10, 
        text width=0.25\textwidth,  % schmaler: von 0.28 auf 0.25
        minimum height=4cm,
        align=center,
        rounded corners
    },
    % Stil für die Überschriften in den Boxen
    title/.style={
        font=\bfseries,
        anchor=north,
        yshift=-0.3cm
    },
    % Stil für den Inhalt
    content/.style={
        font=\footnotesize,
        anchor=center,
        text width=0.22\textwidth,  % angepasst
        align=center
    }
]

% Hintergrund-Box (grauer Rahmen) - Textbreite
\node[
    draw=gray!60,
    thick,
    fill=white,
    inner sep=0.4cm,
    minimum width=\textwidth,
    minimum height=7cm
] (background) at (0,0) {};

% --- Block A ---
\node[block] (A) at (-0.35\textwidth, 0.3cm) {};
\node[title] at (A.north) {Block A: Methode};
\node[content] at (A.center) {
    \enquote{Target Value Design}\\
    \enquote{TVD}\\
    \enquote{Target Costing}\\
    \enquote{Zielkosten}
};

% --- Block B ---
\node[block] (B) at (0, 0.3cm) {};
\node[title] at (B.north) {Block B: Kontext};
\node[content] at (B.center) {
    \enquote{Germany} / \enquote{Deutschland}\\
    \enquote{VOB} / \enquote{HOAI}\\
    \enquote{Public Sector}\\
    \enquote{Öffentliche Hand}
};

% --- Block C ---
\node[block] (C) at (0.35\textwidth, 0.3cm) {};
\node[title] at (C.north) {Block C: Detail};
\node[content] at (C.center) {
    \enquote{Leistungsphasen}\\
    \enquote{Vergabe} / \enquote{Procurement}\\
    \enquote{Incentivierung}\\
    \enquote{Zuschlagskriterien}
};

% --- Operatoren ---
\node[font=\bfseries\small ] at ($(A.east)!0.5!(B.west)$) {AND};
\node[font=\bfseries\small ] at ($(B.east)!0.5!(C.west)$) {AND};

% --- Ergebnis-Pfeil ---
\node[below=0.8cm of B, font=\bfseries] (result) {= Spezifischer Suchstring};
\draw[-latex, thick] (B.south) -- (result.north);
\draw[-latex, thick] (A.south) |- ++(0,-0.25) -| (result.north);
\draw[-latex, thick] (C.south) |- ++(0,-0.25) -| (result.north);

\end{tikzpicture}
\caption{Kombinatorik der Suchbausteine für die Detailrecherche}
\label{fig:suchlogik_grafik}
\end{figure}

\minisec{Anwendungsbeispiel}
Für die Klärung der vergaberechtlichen Zusammenhängen und \acl{TVD} wurde beispielsweise folgender Suchstring in den zuvor beschriebenen Datenbanken angewendet:



\begin{center}
\textbf{Exemplarischer Suchstring (Detailrecherche VOB):}\\[1.5em]

\begin{BVerbatim}
        ( "Target Value Design" OR "TVD" )
                        AND 
  ( "Vergaberecht" OR "VOB/A" OR "Öffentliche Hand" )
                        AND 
  ( "Eignung" OR "Zuschlag" OR "Wertungskriterien" )
\end{BVerbatim}
\end{center}

Durch dieses iterative, problemzentrierte Vorgehen konnte die relevante Fachliteratur (z.\,B. juristische Kommentare oder spezifische Konferenzbeiträge der \textit{IGLC}) effizient identifiziert werden, ohne durch irrelevante Treffer verfälscht zu werden. Die so gewonnenen Erkenntnisse flossen direkt in die Detaillierung des Prozessmodells (Kapitel 4) ein.

\section{Datenauswertung: Strukturierende Inhaltsanalyse}
\label{sec:inhaltsanalyse}

Die Auswertung der identifizierten Quellen erfolgte methodisch in Anlehnung an die Qualitative Inhaltsanalyse nach Mayring. 
Konkret wurde die strukturierenden Inhaltsanalyse verwendet. Ziel war es, bestimmte Aspekte aus dem Material herauszufiltern und systematisch zu ordnen.\autocite{mayring_qualitative_2022}

Hierfür wurde ein deduktives Kategoriensystem gebildet.\autocite[vgl. S.68]{mayring_qualitative_2022} Das bedeutet, die Quellen wurden gezielt nach vorab definierten Strukturmerkmalen durchsucht, die für das zu entwickelnde Prozessmodell essenziell sind. Die zentralen Analysekategorien waren:

\begin{enumerate}
    \item \textbf{Prozessphasen:} Welche zeitlichen Abschnitte und Meilensteine (z.\,B. \textit{Design Freeze}) werden definiert?
    \item \textbf{Rollen und Akteure:} Wer ist beteiligt und welche Verantwortung tragen diese Rollen (z.\,B. \textit{Cluster Leader}, \textit{Cost Manager})?
    \item \textbf{Methoden und Werkzeuge:} Welche operativen Instrumente kommen zum Einsatz (z.\,B. \textit{Set-Based Design}, \textit{Over-the-Shoulder Look})?
    \item \textbf{Steuerungslogik:} Wie werden Kosten und Werte definiert und überwacht?
\end{enumerate}

Durch diese Kategorisierung konnten die relevanten Bausteine aus den unterschiedlichen Quellen extrahiert und vergleichbar gemacht werden.

\section{Modellsynthese und Transfer}
\label{sec:synthese_transfer}

Auf Basis der kategorisierten Ergebnisse erfolgte die \textit{Modellsynthese} (\cref{ch:4}).
Die extrahierten Bausteine (Rollen, Phasen, Methoden) wurden logisch rekomponiert und in die Struktur der deutschen Leistungsphasen (HOAI) sowie der Projektstufen 0 bis 3 übersetzt. Es entstand ein \enquote{idealtypisches Prozessmodell}, das den methodisch optimalen Ablauf abbildet.

Abschließend wurde dieses Modell einer **qualitativen Fit-Gap-Analyse** unterzogen. Dabei wurde das entwickelte Soll-Modell systematisch mit den regulatorischen Vorgaben der VOB und des öffentlichen Haushaltsrechts abgeglichen, um strukturelle Passungen (\textit{Fits}) und Widersprüche (\textit{Gaps}) zu identifizieren. Diese Widersprüche bilden die Grundlage für die Diskussion der Dilemmata in Kapitel 5.





\FloatBarrier



% Aus der Kombination der in \cref{tab:suchstrategie} gezeigten Konzepte mittels \texttt{AND}-Verknüpfungen ergeben sich die folgenden finalen Suchstrings für die Anwendung in den Datenbanken:

% \textbf{Englischer Master-Suchstring:}
% \begin{verbatim}
% ( "Target Value Design" OR "TVD" OR "Target Costing" )
% AND
% ( "construction" OR "building industry" OR "AEC" )
% AND
% ( "process*" OR "method*" OR "framework" OR "implementation" )
% \end{verbatim}

% \textbf{Deutscher Master-Suchstring:}
% \begin{verbatim}
% ( "Target Value Design" OR "TVD" OR "Zielkostenmanagement" )
% AND
% ( "Bauwesen" OR "Bauwirtschaft" OR "Bauprojekte" )
% AND
% ( "Prozess*" OR "Methode*" OR "Anwendung" OR "Vorgehensmodell" )
% \end{verbatim}

% Diese Vorgehensweise stellt sicher, dass in jeder durchgeführten Suche alle relevanten Aspekte der Forschungsfrage abgedeckt sind und die Ergebnisse über die verschiedenen Datenbanken hinweg vergleichbar bleiben.

