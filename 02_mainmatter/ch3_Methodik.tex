\chapter{Methodik}
\label{ch:methodik}

% Ziel: Das Vorgehen der Arbeit transparent und wissenschaftlich nachvollziehbar darlegen.
% Es wird der "Plan" für die Forschung beschrieben.

\section{Forschungsdesign und methodische Verortung}
\label{sec: 3.1}
%   - Ziel: Wahl der Methodik begründen und wissenschaftlich einordnen.
%   - Inhalt:
%     - Wiederholung der Forschungsfrage.
%     - Begründung für literaturbasierten, konzeptionell-analytischen Ansatz.
%     - Abgrenzung zu anderen Methoden (z.B. empirisch).
%     - Optional: Visualisierung des Vorgehens als einfaches Flussdiagramm.
\subsection{Herleitung des Forschungsbedarfs aus der Forschungsfrage}
\label{sec: 3.1.1}

%   Wiederholung der Forschungsfrage prägnant
%   Ableitung der notwendingen Schritte zur Beantwortung
%   


\subsection{Einordnung in das Forschungsparadigma: Ein qualitativ-konstruktiver Ansatz}
\label{sec: 3.1.2}

\subsection{Wahl des konkreten Forschungsdesigns: Die konzeptionell-analytische Arbeit}
\label{sec: 3.1.3}

  \subsection{Gütekriterien des qualitativen Forschungsdesigns}
\label{sec: 3.1.4}

\subsection{Visualisierung des Forschungsprozesses}
\label{sec: 3.1.5}

% Definition von Stilen, um den Code sauber zu halten
\tikzset{
  phase/.style={rectangle, rounded corners, draw=black, fill=gray!10,
                text width=0.85\textwidth, minimum height=1.5cm, text centered},
  pfeil/.style={-latex, thick}
}

\begin{figure}[hbt!]
  \makebox[\textwidth][c]{
    \begin{tikzpicture}
      \tikzset{
        phase/.style={rectangle, rounded corners, draw=black, fill=gray!10,
                      text width=0.9\textwidth, minimum height=1.5cm, text centered},
        pfeil/.style={-latex, thick}
      }

      \node[phase] (phase1) {
        \textbf{Phase 1: Forschungsdesign (Kap. 3.1)}\\
        \small Problemstellung, Methodische Verortung, Konzeptionell-analytischer Ansatz
      };

      \node[phase, below=0.8cm of phase1] (phase2) {
        \textbf{Phase 2: Literaturanalyse (Kap. 3.2)}\\
        \small Systematische Recherche, Qualitative Inhaltsanalyse, Identifikation von Bausteinen
      };

      \node[phase, below=0.8cm of phase2] (phase3) {
        \textbf{Phase 3: Modellkonstruktion (Kap. 3.3)}\\
        \small Synthese der Bausteine, Modellierung des idealtypischen TVD-Prozesses
      };

      \node[phase, below=0.8cm of phase3] (phase4) {
        \textbf{Phase 4: Kontextanalyse \& Transfer (Kap. 3.4)}\\
        \small Fit-Gap-Analyse (Vergleich mit HOAI/VOB), Ableitung des Transferleitfadens
      };

      \draw[pfeil] (phase1) -- (phase2);
      \draw[pfeil] (phase2) -- (phase3);
      \draw[pfeil] (phase3) -- (phase4);

    \end{tikzpicture}
  }
  \caption{Schematische Darstellung des vierphasigen Forschungsprozesses}
  \label{fig:forschungsprozess_tikz}
\end{figure}


\section{Systematische Literaturrecherche und -analyse}
\label{sec: 3.2}

%   - Ziel: Das Vorgehen bei der Literatursuche und -auswahl exakt dokumentieren.
%   - Inhalt:
%     - Recherchestrategie: Genutzte Datenbanken (Scopus, etc.), Fachinstitutionen (GLCI, etc.).
%     - Suchbegriffe: Beispiele für Keyword-Kombinationen (DE/EN).
%     - Auswahlkriterien: Inklusions-/Exklusionskriterien (Relevanz, Aktualität, Qualität).
%     - Analysemethode: Qualitative Inhaltsanalyse zur Identifikation von Prozessbausteinen, Rollen, Werkzeugen.

\section{Prozessbeschreibung und Modellierung}
\label{sec: 3.3}

%   - Ziel: Erklären, wie aus den Literatur-Fragmenten das idealtypische TVD-Modell konstruiert wird.
%   - Inhalt:
%     - Synthese: Zusammenführung der identifizierten Bausteine aus diversen Quellen.
%     - Modellierungsansatz: Begründung für die Wahl eines Prozessmodells (z.B. Flussdiagramm, Swimlane).
%     - Validierung: Logische Prüfung des Modells auf Konsistenz und Vollständigkeit basierend auf der Literatur.

\section{Kontextanalyse und Ableitung des Transferleitfadens}
\label{sec: 3.4}

%   - Ziel: Den analytischen Vergleich und die Ableitung der Handlungsempfehlungen beschreiben.
%   - Inhalt:
%     - Gegenüberstellung: Systematischer Vergleich des TVD-Modells mit der deutschen Baupraxis (Basis: Kapitel 2.1).
%     - Analysefokus: Strukturelle, rechtliche (HOAI/VOB) und kulturelle Gegebenheiten.
%     - Identifikation: Ziel ist das Aufdecken von Reibungspunkten und Anpassungsbedarfen ("Fit-Gap-Analyse").
%     - Struktur des Leitfadens: Erläutern, wie die Ergebnisse zu konkreten Handlungsempfehlungen aufbereitet werden.

