\chapter{Methodik}
\label{ch:methodik}

% Ziel: Das Vorgehen der Arbeit transparent und wissenschaftlich nachvollziehbar darlegen.
% Es wird der "Plan" für die Forschung beschrieben.
\section{Forschungsdesign und methodische Verortung}
\label{sec: 3.1}
\subsection{Herleitung des Forschungsbedarfs aus der Forschungsfrage}
\label{sec: 3.1.1}

Wie in \cref{ch:einleitung} dargelegt, erfordert die Bewältigung der aktuellen Herausforderungen im Bauwesen neue Ansätze und Methoden. Die aktuell noch überwiegend praktizierte, konventionelle  Art der Projektabwicklung, welche vor allem von einer ausgeprägten Streitkultur charakterisiert wird, ist diesen großen Aufgaben offensichtlich nicht gewachsen, wie sich aus den aktuellen Zahlen und Entwicklung der Bauwirtschaft ableiten lässt.\\
Hier rücken aktuell alternative, partnerschaftliche Projektabwicklungsmodelle wie \ac{IPA} in den Fokus und finden auch in Deutschland zunehmend in Pilotprojekten Anwendung \autocite{haghsheno_ipa-report_2025}.\\
- TVD als Kernprozess/Methode zur Werk- und Kostensteuerung innerhalb des IPA-Rahmens\\
- Wiederholung der Forschungsfrage\\
- 

- Begründung für literaturbasierten, konzeptionell-analytischen Ansatz.\\
- Abgrenzung zu anderen Methoden (z.B. empirisch).\\
- Ableitung der notwendigen Schritte zur Beantwortung\\


\subsection{Einordnung in das Forschungsparadigma: Ein qualitativ-konstruktiver Ansatz}
\label{sec: 3.1.2}

\subsection{Wahl des konkreten Forschungsdesigns: Die konzeptionell-analytische Arbeit}
\label{sec: 3.1.3}

  \subsection{Gütekriterien des qualitativen Forschungsdesigns}
\label{sec: 3.1.4}

\subsection{Visualisierung des Forschungsprozesses}
\label{sec: 3.1.5}

% Definition von Stilen, um den Code sauber zu halten
\tikzset{
  phase/.style={rectangle, rounded corners, draw=black, fill=gray!10,
                text width=0.85\textwidth, minimum height=1.5cm, text centered},
  pfeil/.style={-latex, thick}
}

\begin{figure}[hbt!]
  \makebox[\textwidth][c]{
    \begin{tikzpicture}
      \tikzset{
        phase/.style={rectangle, rounded corners, draw=black, fill=gray!10,
                      text width=0.9\textwidth, minimum height=1.5cm, text centered},
        pfeil/.style={-latex, thick}
      }

      \node[phase] (phase1) {
        \textbf{Phase 1: Forschungsdesign (Kap. 3.1)}\\
        \small Problemstellung, Methodische Verortung, Konzeptionell-analytischer Ansatz
      };

      \node[phase, below=0.8cm of phase1] (phase2) {
        \textbf{Phase 2: Literaturanalyse (Kap. 3.2)}\\
        \small Systematische Recherche, Qualitative Inhaltsanalyse, Identifikation von Bausteinen
      };

      \node[phase, below=0.8cm of phase2] (phase3) {
        \textbf{Phase 3: Modellkonstruktion (Kap. 3.3)}\\
        \small Synthese der Bausteine, Modellierung des idealtypischen TVD-Prozesses
      };

      \node[phase, below=0.8cm of phase3] (phase4) {
        \textbf{Phase 4: Kontextanalyse \& Transfer (Kap. 3.4)}\\
        \small Fit-Gap-Analyse (Vergleich mit HOAI/VOB), Ableitung des Transferleitfadens
      };

      \draw[pfeil] (phase1) -- (phase2);
      \draw[pfeil] (phase2) -- (phase3);
      \draw[pfeil] (phase3) -- (phase4);

    \end{tikzpicture}
  }
  \caption{Schematische Darstellung des vierphasigen Forschungsprozesses}
  \label{fig:forschungsprozess_tikz}
\end{figure}

\FloatBarrier

\section{Systematische Literaturrecherche und -analyse}
\label{sec: 3.2}

Der Auswahlprozess relevanter Literatur stützt sich auf das international anerkannte Protokoll \acsf{PRISMA}. Dieses bildet als Richtlinie für Literatur-Reviews, den gesamten Prozess der von der Suche, über die Auswahl bis hin zur Präsentation von Studien ab und gewährleistet Transparenz (Reproduzierbarkeit), Vollständigkeit und Vergleichbarkeit für die Forschungsergebnisse \autocite{page_2021_prisma}. Der im \ac{PRISMA}-Protokoll definierte, mehrstufigen Filter- und Auswahlprozess dient hier als methodische Grundlage für die systematische Literaturrecherche und  in (\cref{fig:prisma_flowchart}) nachfolgend dargestellt.

\begin{figure}[hbt!]
  \centering
  \begin{tikzpicture}[
    node distance=0.5cm and 1cm, % Vertikaler und horizontaler Abstand
    % Stil für die Haupt-Boxen im Prozess
    mainbox/.style={
      rectangle, 
      draw=black, 
      fill=gray!10, 
      text width=0.50\textwidth, 
      minimum height=1.5cm, 
      text centered
    },
    % Stil für die Boxen mit den Ausschlussgründen
    sidebox/.style={
      rectangle, 
      draw=black, 
      fill=gray!10, 
      text width=0.35\textwidth, 
      minimum height=1.5cm, 
      align=left
    },
    % Stil für die seitlichen Beschriftungen (Identifizierung, etc.)
    labelbox/.style={
      rotate=90, 
      font=\bfseries
    }
  ]

  % --- DEFINITION DER KNOTEN (BOXEN) ---
  
  % Stufe 1: Identifizierung
  \node[mainbox] (ident) {
    \textbf{Identifizierte Datensätze aus Datenbanken} \\ (Anzahl N = ...)
  };
  \node[labelbox, left=0.5cm of ident.west, anchor=center] {Identifizierung};

  % Stufe 2: Screening
  \node[mainbox, below=1.5cm of ident] (screen1) {
    \textbf{Datensätze nach Entfernung von Duplikaten} \\ (Anzahl N = ...)
  };
  \node[mainbox, below=1cm of screen1] (screen2) {
    \textbf{Geprüfte Datensätze (Titel/Abstract)} \\ (Anzahl N = ...)
  };
  \node[labelbox, left=0.5cm of screen1.west, anchor=east] {Screening};
  
  % Stufe 3: Eignung
  \node[mainbox, below=1.5cm of screen2] (elig) {
    \textbf{Gelesene Volltext-Artikel} \\ (Anzahl N = ...)
  };
  \node[labelbox, left=0.5cm of elig.west, anchor=center] {Eignung};

  % Stufe 4: Einschluss
  \node[mainbox, below=1.5cm of elig] (incl) {
    \textbf{In die Synthese eingeschlossene Studien} \\ (Anzahl N = ...)
  };
  \node[labelbox, left=0.5cm of incl.west, anchor=center] {Einschluss};

  % Ausschluss-Boxen
  \node[sidebox, right=of screen2] (excl1) {
    \textbf{Ausgeschlossene Datensätze:} \\ (Anzahl N = ...) \\
    \textit{Gründe: Falsches Fachgebiet, keine wiss. Quelle, etc.}
  };
  \node[sidebox, right=of elig] (excl2) {
    \textbf{Ausgeschlossene Volltext-Artikel:} \\ (Anzahl N = ...) \\
    \textit{Gründe: Methodische Schwächen, kein relevanter Fokus, etc.}
  };

  % --- ZEICHNEN DER PFEILE ---
  \draw[-latex, thick] (ident) -- (screen1);
  \draw[-latex, thick] (screen1) -- (screen2);
  \draw[-latex, thick] (screen2) -- (elig);
  \draw[-latex, thick] (elig) -- (incl);
  
  % Pfeile zu den Ausschluss-Boxen
  \draw[-latex, thick] (screen2) -- (excl1);
  \draw[-latex, thick] (elig) -- (excl2);

  \end{tikzpicture}
  % --- ENDE DES TIKZ-CODES ---

  \caption{PRISMA-Flussdiagramm des systematischen Literaturrechercheprozesses}
  \label{fig:prisma_flowchart}
\end{figure}

\clearpage

\subsection{Schritt 1: Identifizierung}
Im ersten Schritt der Literaturrecherche wurden zunächst relevante zunächst relevante Datenbanken identifiziert. Dabei wurden die Quellen in drei Arten untergliedert:

1. Größe - International \& Interdisziplinär
\begin{itemize}[leftmargin=2em]
    \item \textbf{Scopus}: Bietet eine extrem breite Abdeckung von ingenieurwissenschaftlichen und technischen Publikationen.
    \item \textbf{Web of Science}: Ist ähnlich umfassend wie Scopus und eine Standarddatenbank in der Wissenschaft.
    \item \textbf{Google Scholar}:
    \item \textbf{Researchgate}:
    
\end{itemize}
2. Fachspezifische Datenbanken mit Relevanz für Bauingenieurwesen bzw. Baubetrieb
\begin{itemize}[leftmargin=2em]
    \item \textbf{Beispiel}: 
    \item \textbf{Beispiel 2}: 
\end{itemize}
3. Graue Literatur - Der Blick in die Praxis
\begin{itemize}[leftmargin=2em]
    \item \textbf{Beispiel 3}: 
    \item \textbf{Beispiel 4}: 
\end{itemize}

Diese initiale Sammlung stellt eine erste Grundlage für den nachfolgenden Auswahlprozess dar. Inwieweit diese Auswahl erschöpfende Ergebnisse liefert, wird, unter Berücksichtigung sowohl der inhaltlichen Divergenz der Datenbanken, als auch des zusätzlichen Erkenntnisgewinns durch neu hinzukommende Quellen, im Laufe des Filterprozesses überprüft.\\
Für die gewählten Datenbanken galt es im Folgenden, eine systematische und reproduzierbare Suchstrategie zu entwickeln. Um die Komplexität einer Vielzahl von Einzelabfragen zu reduzieren, wurde ein umfassender "Master-Suchstring"  für die jeweilige Sprache (Deutsch und Englisch) konzipiert. Dieser Ansatz kombiniert die zuvor definierten Schlüsselwörter mithilfe boolescher Operatoren zu einer einzigen, umfassenden Suchanfrage pro Datenbank.
Die zugrundeliegenden Suchkonzepte, die dazugehörigen Schlüsselbegriffe sowie die verwendeten Operatoren sind in \cref{tab:suchstrategie} systematisch gegliedert.

\begin{table}[hbt!]
  \centering
  \caption{Systematische Gliederung der Suchbegriffe und -operatoren}
  \label{tab:suchstrategie}
  \begin{tabularx}{\textwidth}{l >{\raggedright\arraybackslash}X >{\raggedright\arraybackslash}X l}
    \toprule
    \textbf{Such-Konzept} & \textbf{Englische Schlagwörter} & \textbf{Deutsche Schlagwörter} & \textbf{Operatoren} \\
    \midrule
    Kernthema & "Target Value Design", "TVD", "Target Costing" & "Target Value Design", "TVD", "Zielkostenmanagement" & \texttt{OR} \\
    \addlinespace
    Kontext & "construction", "building industry", "AEC" & "Bauwesen", "Bauwirtschaft", "Bauprojekte" & \texttt{OR} \\
    \addlinespace
    Spez. Kontext & "Germany", "German construction" & "deutsche Baupraxis", "Deutschland", "HOAI", "VOB" & \texttt{OR} \\
    \addlinespace
    Ziel (Prozess) & "process*", "method*", "framework", "implementation" & "Prozess*", "Methode*", "Anwendung", "Vorgehensmodell" & \texttt{OR}, \texttt{*} \\
    \bottomrule
  \end{tabularx}
\end{table}

\FloatBarrier

Aus der Kombination der in \cref{tab:suchstrategie} gezeigten Konzepte mittels \texttt{AND}-Verknüpfungen ergeben sich die folgenden finalen Suchstrings für die Anwendung in den Datenbanken:

\textbf{Englischer Master-Suchstring:}
\begin{verbatim}
( "Target Value Design" OR "TVD" OR "Target Costing" )
AND
( "construction" OR "building industry" OR "AEC" )
AND
( "process*" OR "method*" OR "framework" OR "implementation" )
\end{verbatim}

\textbf{Deutscher Master-Suchstring:}
\begin{verbatim}
( "Target Value Design" OR "TVD" OR "Zielkostenmanagement" )
AND
( "Bauwesen" OR "Bauwirtschaft" OR "Bauprojekte" )
AND
( "Prozess*" OR "Methode*" OR "Anwendung" OR "Vorgehensmodell" )
\end{verbatim}

Diese Vorgehensweise stellt sicher, dass in jeder durchgeführten Suche alle relevanten Aspekte der Forschungsfrage abgedeckt sind und die Ergebnisse über die verschiedenen Datenbanken hinweg vergleichbar bleiben.

\FloatBarrier

\subsection{Schritt 2: Screening}
Die im folgenden aufgeführten Einschränkungen sollen die Ergebnisse aus der initialen Suchanfrage filtern:

\textbf{Sprache:} \\
\textbf{Deutsch}: Zur Erfassung deutschsprachiger bzw. bereits im deutschen Kontext oder Blickwinkel entstandene Quellen.\\
\textbf{Englisch}: Essentiell für internationale Publikationen. Vor dem Hintergrund, dass der Ursprung und die wesentliche Entwicklung von \ac{TVD} und allen verwandten Themenfelder im anglo-amerikanischen Raum stattgefunden hat bzw. stattfindet \autocite{Quelle bzw. Verweis erforderlich?}, wird englisch in diesem Kontext als einzige wesentlich relevante Fremdsprache erachtet.\\

%     - Recherchestrategie: Genutzte Datenbanken (Scopus, etc.), Fachinstitutionen (GLCI, etc.).
%     - Suchbegriffe: Beispiele für Keyword-Kombinationen (DE/EN).
%     - Auswahlkriterien: Inklusions-/Exklusionskriterien (Relevanz, Aktualität, Qualität).
%     - Analysemethode: Qualitative Inhaltsanalyse zur Identifikation von Prozessbausteinen, Rollen, Werkzeugen.

\FloatBarrier
\clearpage

\section{Prozessbeschreibung und Modellierung}
\label{sec: 3.3}

%   - Ziel: Erklären, wie aus den Literatur-Fragmenten das idealtypische TVD-Modell konstruiert wird.
%   - Inhalt:
%     - Synthese: Zusammenführung der identifizierten Bausteine aus diversen Quellen.
%     - Modellierungsansatz: Begründung für die Wahl eines Prozessmodells (z.B. Flussdiagramm, Swimlane).
%     - Validierung: Logische Prüfung des Modells auf Konsistenz und Vollständigkeit basierend auf der Literatur.

\FloatBarrier

\section{Kontextanalyse und Ableitung des Transferleitfadens}
\label{sec: 3.4}

%   - Ziel: Den analytischen Vergleich und die Ableitung der Handlungsempfehlungen beschreiben.
%   - Inhalt:
%     - Gegenüberstellung: Systematischer Vergleich des TVD-Modells mit der deutschen Baupraxis (Basis: Kapitel 2.1).
%     - Analysefokus: Strukturelle, rechtliche (HOAI/VOB) und kulturelle Gegebenheiten.
%     - Identifikation: Ziel ist das Aufdecken von Reibungspunkten und Anpassungsbedarfen ("Fit-Gap-Analyse").
%     - Struktur des Leitfadens: Erläutern, wie die Ergebnisse zu konkreten Handlungsempfehlungen aufbereitet werden.

