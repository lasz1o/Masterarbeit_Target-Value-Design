\chapter{Methodik}
\label{ch:methodik}

% Ziel: Das Vorgehen der Arbeit transparent und wissenschaftlich nachvollziehbar darlegen.
% Es wird der "Plan" für die Forschung beschrieben.

\section{Forschungsdesign und methodische Verortung}
\label{sec: 3.1}

%   - Ziel: Wahl der Methodik begründen und wissenschaftlich einordnen.
%   - Inhalt:
%     - Wiederholung der Forschungsfrage.
%     - Begründung für literaturbasierten, konzeptionell-analytischen Ansatz.
%     - Abgrenzung zu anderen Methoden (z.B. empirisch).
%     - Optional: Visualisierung des Vorgehens als einfaches Flussdiagramm.

\section{Systematische Literaturrecherche und -analyse}
\label{sec: 3.2}

%   - Ziel: Das Vorgehen bei der Literatursuche und -auswahl exakt dokumentieren.
%   - Inhalt:
%     - Recherchestrategie: Genutzte Datenbanken (Scopus, etc.), Fachinstitutionen (GLCI, etc.).
%     - Suchbegriffe: Beispiele für Keyword-Kombinationen (DE/EN).
%     - Auswahlkriterien: Inklusions-/Exklusionskriterien (Relevanz, Aktualität, Qualität).
%     - Analysemethode: Qualitative Inhaltsanalyse zur Identifikation von Prozessbausteinen, Rollen, Werkzeugen.

\section{Prozessbeschreibung und Modellierung}
\label{sec: 3.3}

%   - Ziel: Erklären, wie aus den Literatur-Fragmenten das idealtypische TVD-Modell konstruiert wird.
%   - Inhalt:
%     - Synthese: Zusammenführung der identifizierten Bausteine aus diversen Quellen.
%     - Modellierungsansatz: Begründung für die Wahl eines Prozessmodells (z.B. Flussdiagramm, Swimlane).
%     - Validierung: Logische Prüfung des Modells auf Konsistenz und Vollständigkeit basierend auf der Literatur.

\section{Kontextanalyse und Ableitung des Transferleitfadens}
\label{sec: 3.4}

%   - Ziel: Den analytischen Vergleich und die Ableitung der Handlungsempfehlungen beschreiben.
%   - Inhalt:
%     - Gegenüberstellung: Systematischer Vergleich des TVD-Modells mit der deutschen Baupraxis (Basis: Kapitel 2.1).
%     - Analysefokus: Strukturelle, rechtliche (HOAI/VOB) und kulturelle Gegebenheiten.
%     - Identifikation: Ziel ist das Aufdecken von Reibungspunkten und Anpassungsbedarfen ("Fit-Gap-Analyse").
%     - Struktur des Leitfadens: Erläutern, wie die Ergebnisse zu konkreten Handlungsempfehlungen aufbereitet werden.