\chapter{Einleitung}
\label{ch:einleitung}
\section{Problemstellung und Relevanz}
\label{ch:1.1}

\section{Zielsetzung und Forschungsfrage}
\label{ch:1.2}

\section{Methodisches Vorgehen und Aufbau der Arbeit}
\label{ch: 1.3}

Die Arbeit beginnt im vorliegenden \cref{ch:einleitung} mit der Herleitung der Problemstellung sowie der Formulierung der zentralen Forschungsfrage.

Darauf aufbauend legt \cref{ch:grundlagen} die theoretischen Grundlagen und definiert die für die Untersuchung essenziellen Begrifflichkeiten.

Im Kapitel wird das methodische Vorgehen sowie das Forschungsdesign zur Beantwortung der Fragestellung detailliert erläutert.“

Den Schwerpunkt der Arbeit bildet \cref{ch:analyse}, in dem die eigentliche Analyse durchgeführt und die empirischen Ergebnisse präsentiert werden.

Anschließend diskutiert \cref{ch:transfer} die gewonnenen Erkenntnisse kritisch und ordnet sie in den aktuellen Forschungsstand ein.

Abgeschlossen wird die Untersuchung in \cref{ch:conclusion} mit einer Zusammenfassung der zentralen Ergebnisse sowie einem Ausblick auf weiteren Forschungsbedarf.

\clearpage