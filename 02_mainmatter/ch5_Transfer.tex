\chapter{Analyse systemischer Spannungsfelder und Implementierungsbarrieren}
\label{ch:5}

Kapitel 4 hat gezeigt, dass sich das Prozessmodell des Target Value Design (TVD) nicht ohne Weiteres auf Deutschland übertragen lässt. Die identifizierten Hürden sind keine Einzelfälle, sondern das Ergebnis eines Systemkonflikts.
Es prallen zwei Denkweisen aufeinander: Auf der einen Seite steht der TVD-Ansatz, der auf Flexibilität, Zusammenarbeit und Wahrscheinlichkeiten setzt. Auf der anderen Seite steht das deutsche System aus VOB, HOAI und Haushaltsrecht, das auf festen Preisen, strikter Trennung und detaillierten Vorgaben beharrt.

Um diese strukturellen Probleme nicht als unübersichtliche Liste von Einzelpunkten abzuarbeiten, werden sie im Folgenden in drei thematische Spannungsfelder (\enquote{Dilemmata}) gebündelt:

\begin{enumerate}
    \item \textbf{Das Beschaffungs-Dilemma (Marktzugang und Recht):}\\
    Wie lassen sich kooperative Teams frühzeitig bilden, wenn das Vergaberecht primär auf Preiswettbewerb statt auf Kompetenz ausgerichtet ist? (Fokus auf Phase 1).

    \item \textbf{Das Wert-Dilemma (Steuerung und Zielsystem):}\\
    Worauf wird optimiert, wenn es keinen ökonomischen ROI gibt und Budgets für Bau und Betrieb getrennt sind? Hier geht es um die Gefahr, dass kurzfristige Kostenoptimierung strukturell gegenüber langfristigen Qualitäts- und Nutzungszielen priorisiert wird (Fokus auf Phase 0 und 2).

    \item \textbf{Das Struktur-Dilemma (Ordnungsrahmen und Prozess):}\\
    Wo liegen die Grenzen der Machbarkeit? Dieses Cluster diskutiert Hürden, die selbst durch moderne Verträge (wie IPA) nicht gelöst werden können, weil sie im starren Genehmigungs- und Baurecht verankert sind (Fokus auf Phase 2 und 3).
\end{enumerate}

Ziel dieses Kapitels ist eine differenzierte Analyse: Welche Barrieren lassen sich durch neue Vertragsmodelle bereits senken und an welchen Stellen steht der gesetzliche Ordnungsrahmen einer echten TVD-Umsetzung weiterhin im Weg?

\section{Cluster 1: Das Beschaffungs-Dilemma}
\label{sec:cluster1}

Einer der erste und oft entscheidenden Stolpersteine bei der Implementierung von Target Value Design im deutschen Bauwesen liegt bereits vor dem eigentlichen Projektstart: in der Formierung des Teams. Während die TVD-Methodik auf der Prämisse basiert, dass die Schlüsselakteure (Planer und ausführende Unternehmen) frühzeitig und gemeinsam an der Lösungsfindung arbeiten, reglementiert das deutsche Vergaberecht den Marktzugang nach einer fundamental anderen Logik.

Dieses Spannungsfeld zwischen der Notwendigkeit einer frühen, kompetenzbasierten Teambindung und den gesetzlichen Anforderungen an Wettbewerb und Gleichbehandlung bildet das erste Diskussions-Cluster. Es wird im Folgenden anhand von drei Dimensionen untersucht: der juristischen Divergenz zwischen Preis- und Qualitätswettbewerb (5.1.1), der verfahrenstechnischen Hürde der Vorbefasstheit bei früher Einbindung (5.1.2) und den strukturellen Ausschlusskriterien für den Mittelstand (5.1.3).

\subsection{Das Transaktionskosten-Dilemma: Verfahrensökonomie und Skalierbarkeit}
\label{sec:5.1.1}

Die erste und grundlegendste Hürde bei der Initiierung eines TVD-Projekts im öffentlichen Sektor ist nicht die rechtliche Unmöglichkeit, sondern eine Diskrepanz zwischen dem methodisch erforderlichen Zeitpunkt der Beauftragung und der ökonomischen Verhältnismäßigkeit der verfügbaren Verfahren.
Das Paradoxon beginnt mit einer zeitlichen und logischen Divergenz: Wie in Abbildung \ref{fig:beschaffungsparadoxon} visualisiert, erfordert die TVD-Methodik zwingend, dass die Schlüsselakteure (Planer und ausführende Unternehmen) \textit{vor} Beginn der detaillierten Planung gebunden sind, um diese gemeinsam zu optimieren. Die Systematik der VOB/A hingegen ist historisch auf die Vergabe einer fertig geplanten Leistung zum Festpreis ausgelegt.

\begin{figure}[ht]
    \centering
    \fcolorbox{gray!50}{white}{%
        \begin{minipage}{\dimexpr\textwidth-2\fboxsep-2\fboxrule\relax}
            \centering
            \begin{tikzpicture}[
                >=Stealth,
                node distance=0cm,
                phase/.style={rectangle, draw=black, minimum height=1cm, text centered, font=\small},
                milestone/.style={circle, draw=red, fill=red!10, thick, minimum size=1.2cm, text width=1.5cm, align=center, font=\bfseries\footnotesize},
                arrow/.style={->, thick},
                scale=1,
                transform shape % Skaliert die Schrift mit
            ]
            % --- ZEITSTRAHL OBEN: TRADITIONELL (VOB) ---
            \node[anchor=west] at (0, 3.5) {\textbf{Traditionell (VOB/A): Sequenziell}};
            
            % Phasen (breitere Boxen für volle Breite)
            \node[phase, minimum width=5.5cm, fill=gray!10] (plan_vob) at (0, 2.2) {Planung (LPH 1-6)};
            \node[phase, minimum width=3.5cm, fill=gray!20, right=0cm of plan_vob] (tender_vob) {Ausschreibung};
            \node[phase, minimum width=5cm, fill=gray!30, right=0cm of tender_vob] (build_vob) {Bauausführung};
            
            % Meilenstein: Später Zuschlag
            \node[milestone, right=8.7cm of plan_vob.west, yshift=0cm] (award_vob) {Vergabe\\(Preis fix)};
            
            % Zeitachse
            \draw[arrow] (-0.5, 1.3) -- (12, 1.3);
            
            % --- ZEITSTRAHL UNTEN: TVD (INTEGRIERT) ---
            \node[anchor=west] at (0, -1.5) {\textbf{Target Value Design (TVD): Integriert}};
            
            % Phasen
            \node[phase, minimum width=2.8cm, fill=blue!10] (target) at (0, -2.8) {Zieldef.};
            \node[phase, minimum width=7.5cm, fill=blue!20, right=0cm of target] (tvd_phase) {TVD-Phase (Gemeinsame Planung)};
            \node[phase, minimum width=3.7cm, fill=blue!30, right=0cm of tvd_phase] (build_tvd) {Bauausführung};
            
            % Meilenstein: Früher Zuschlag
            \node[milestone, draw=blue, fill=blue!10, right=2.5cm of target.west] (award_tvd) {Vergabe\\(Team fix)};
            
            % Zeitachse
            \draw[arrow] (-0.5, -3.7) -- (12, -3.7);
            
            % --- KONFLIKT-VISUALISIERUNG ---
            % Verbindungslinie / Blitz
            \draw[<->, dashed, red, thick] (award_tvd.north) -- (award_vob.south) 
                node[midway, fill=white, text=red, font=\footnotesize\bfseries, align=center, draw=red, rounded corners] 
                {Das Beschaffungs-\\Paradoxon};
            
            \end{tikzpicture}
        \end{minipage}%
    }
    \caption{Divergenz der Vergabezeitpunkte: VOB-Modell vs. TVD-Modell}
    \label{fig:beschaffungsparadoxon}
\end{figure}

\minisec{Potenzielle Lösungsräume im Vergaberecht}
Betrachtet man den Rechtsrahmen isoliert, so scheint dieses Problem lösbar. Das Gesetz gegen Wettbewerbsbeschränkungen (§ 127 Abs. 1 GWB) erlaubt ausdrücklich die Berücksichtigung qualitativer Zuschlagskriterien neben dem Preis. Um die komplexe Auswahl eines TVD-Teams rechtssicher abzubilden, stellt der Gesetzgeber spezifische Verfahrensarten jenseits der starren \textit{Öffentlichen Ausschreibung} zur Verfügung.\autocite[vgl. S. 101]{noauthor_gesetz_2013}
Insbesondere das \textit{Verhandlungsverfahren mit Teilnahmewettbewerb} (§ 17 VgV) oder der noch komplexere \textit{Wettbewerbliche Dialog} (§ 18 VgV) bieten den rechtlichen Rahmen, um statt eines festen Preises (für eine noch unbekannte Lösung) die Kompetenz, die Methodenerfahrung und die Kooperationsbereitschaft der Bieter zu bewerten, bevor ein Vertrag geschlossen wird.\autocite[vgl. S. 14ff]{noauthor_verordnung_2016}

\minisec{Die Hürde der Transaktionskosten (Disproportionalität)}
Dass diese Verfahren in der Breite der öffentlichen Bauprojekte dennoch kaum Anwendung finden, ist auf ihre mangelnde ökonomische Effizienz bei Standardprojekten zurückzuführen. Wie Becker und Friedinger (2024) in ihrer Analyse zur Adaption von IPA-Modellen darlegen, verursachen diese komplexen Verfahrensarten massive \textbf{Transaktionskosten}, die weit über den Aufwand einer klassischen Submission hinausgehen \autocite{becker_adaptionen_2024}.
Die Kostentreiber sind hierbei vielschichtig:
\begin{itemize}
    \item \textbf{Juristischer Vorbereitungsaufwand:} Die Erstellung diskriminierungsfreier Eignungsmatrizen für \enquote{weiche} Kriterien (z.\,B. Bewertung eines fiktiven Workshops) erfordert spezialisierte juristische Beratung, um das Risiko von Rügen unterlegener Bieter zu minimieren.
    \item \textbf{Ressourcenbindung:} Mehrstufige Verhandlungsrunden binden über Monate hinweg hochqualifiziertes Personal in den Vergabestellen, das im operativen Tagesgeschäft fehlt.
    \item \textbf{Verfahrensdauer:} Die langen Fristen verzögern den Projektstart signifikant, was wiederum Finanzierungskosten treiben kann.
\end{itemize}

\minisec{Die Entstehung einer \enquote{Zwei-Klassen-Projektlandschaft}}
Für die öffentliche Hand ergibt sich daraus ein Konflikt mit dem haushaltsrechtlichen Grundsatz der Wirtschaftlichkeit (§ 7 BHO). Die Investition in das Vergabeverfahren muss in einem angemessenen Verhältnis zum Projektwert stehen.
Für Großprojekte (\enquote{Leuchttürme} > 50 Mio. €) amortisiert sich dieser Aufwand durch die späteren Projektoptimierungen. Für kleine und mittlere Projekte (Volumen < 10--15 Mio. €), die den Großteil der kommunalen Bauaufgaben (Schulen, Kitas, Wohnungsbau) ausmachen, ist dieser Aufwand jedoch unverhältnismäßig (\enquote{Over-Engineering} der Vergabe).
Becker und Friedinger (2024) konstatieren daher, dass die Komplexität des Vergaberechts faktisch als \enquote{Eintrittsbarriere} wirkt: TVD wird zu einer Methode für Eliten-Projekte, während die Breite der öffentlichen Infrastruktur aus Kostengründen in die VOB/A-Standardvergabe gezwungen wird, die eine echte frühe Einbindung strukturell verhindert \autocite{becker_adaptionen_2024}.

\minisec{Zwischenfazit: Die Verfahrenslücke}
Es lässt sich festhalten, dass das deutsche Vergabesystem zwar theoretisch Instrumente für TVD bereitstellt, diese aber für den Projektalltag \enquote{überdimensioniert} sind. Es fehlt ein niederschwelliger, rechtssicherer Prozess für die frühe Teambindung bei Standardprojekten.
Diese Lücke führt in der Praxis zu Ausweichbewegungen: Auftraggeber versuchen, die Komplexität zu reduzieren, indem sie vereinfachte Vertragsmodelle (\enquote{IPA-Light}) nutzen oder die Einbindung der Partner zeitlich nach hinten verschieben. Dass diese vermeintlichen Abkürzungen jedoch neue, methodische Risiken bergen, ist Gegenstand der folgenden Analyse.

\subsection{Methodische Defizite durch späte Einbindung (\enquote{IPA-Light})}
\label{sec:5.1.2}

Als Reaktion auf die in Abschnitt \ref{sec:5.1.1} dargelegten hohen Transaktionskosten etablierter Vergabeverfahren werden in der Fachliteratur und Praxis zunehmend Anpassungsstrategien diskutiert, die unter dem Begriff \enquote{IPA-Light} oder \enquote{vertragliche Adaptionen} subsumiert werden. Becker und Friedinger (2024) untersuchen hierbei Ansätze, wie Elemente der Integrierten Projektabwicklung in den Rechtsrahmen der VOB/A eingebettet werden können, ohne die volle Komplexität eines Allianzvertrages auszulösen \autocite{becker_adaptionen_2024}.

\minisec{Der Rückzug auf die späte Einbindung}
Der zentrale Hebel zur Reduktion der Verfahrenskomplexität besteht in diesen Modellen häufig darin, die vergaberechtlichen Risiken zu minimieren. Um dem administrativen Aufwand einer qualitativen Auswahl für eine sehr frühe Einbindung zu entgehen, tendieren öffentliche Auftraggeber dazu, die Einbindung der ausführenden Unternehmen zeitlich nach hinten zu verschieben \autocite[vgl. Diskussion der Adaptionsmodelle bei]{becker_adaptionen_2024}.

Treibender Faktor ist hierbei die Rechtsunsicherheit bezüglich der \textbf{Vorbefasstheit} gemäß § 7 VgV. Ein Unternehmen, das bereits in Leistungsphase 2 oder 3 beratend tätig ist, erlangt zwangsläufig einen Informationsvorsprung. Um dieses Unternehmen später rechtssicher mit der Bauausführung zu beauftragen, wären komplexe Ausgleichsmaßnahmen nötig, um den Wettbewerb nicht zu verzerren \autocite[vgl. § 7 VgV Rn. 20 ff.]{leinemann_vergabe_2021}.

Die \enquote{Light}-Lösung der Praxis besteht daher oft darin, diese Hürde zu umgehen, indem der Unternehmer erst zur Leistungsphase 5 (Ausführungsplanung) oder nach Abschluss der Genehmigungsplanung eingebunden wird. Die Logik lautet: Wer erst kommt, wenn die Planung steht, gilt nicht als vorbefasst \autocite[vgl. zur Kritik an späten Einbindungsmodellen]{breyer_alternative_2021}.

\minisec{Methodischer Widerspruch zum Design Freeze}
Diese Strategie der juristischen Risikovermeidung führt jedoch zu einer methodischen Entkernung des Target Value Design. Wie die theoretische Herleitung in Kapitel 4 gezeigt hat, basiert TVD essenziell auf der Einflussnahme auf die Kostentreiber \textit{während} der Entwurfsfindung (LPH 2--3). Granja et al. (2023) betonen, dass erfolgreiches TVD ein Verschmelzen der Grenzen zwischen Planung und Bau erfordert (\enquote{Boundary Spanning}) \autocite{granja_target_2023}.

Erfolgt die Einbindung im Rahmen eines vereinfachten Modells erst in LPH 5, sind die wesentlichen Weichenstellungen – wie Kubatur, Tragwerkskonzept oder Fassadensystem – bereits fixiert (\enquote{Design Freeze}). Der Unternehmer kann in dieser späten Phase nur noch die Ausführungslogistik optimieren oder Alternativangebote für Details unterbreiten, aber keine echten \textit{Design-to-Cost}-Entscheidungen mehr treffen, ohne die vorangegangene Planung und Genehmigung in Frage zu stellen.
Die Analyse zeigt somit, dass der Versuch, TVD durch \enquote{Light}-Modelle kompatibel zur VOB/A zu machen, oft das Kernmerkmal der Methode opfert: Das \textit{Early Contractor Involvement} (ECI) verkommt zu einem \textit{Late Contractor Involvement}, das zwar rechtssicher und kostengünstig zu vergeben ist, aber das Innovationspotenzial der Methode verfehlt.

\minisec{Zwischenfazit}
Es besteht ein direkter Zielkonflikt zwischen Verfahrensökonomie und methodischer Integrität. Die aktuellen Ansätze zur Vereinfachung (IPA-Light) lösen zwar das Kostenproblem der Vergabe (vgl. 5.1.1), schaffen dabei aber ein methodisches Vakuum, da sie die rechtzeitige Integration des Know-hows verhindern. Solange keine Lösung existiert, die sowohl einen geringen Vergabeaufwand als auch eine frühe Einbindung ermöglicht, bleiben adaptierte Modelle oft ein fauler Kompromiss. Neben diesen verfahrenstechnischen Hürden existiert jedoch noch eine weitere, strukturelle Barriere, die den Marktzugang erschwert: die mangelnde Investitionsfähigkeit der kleinteiligen Anbieterstruktur.

\subsection{Marktzutrittsbarrieren durch Struktur und Digitalisierungsgrad}
\label{sec:5.1.3}

Neben den vergaberechtlichen Hürden offenbart die Analyse eine tiefgreifende strukturelle Diskrepanz zwischen den Anforderungen der Methodik und der Realität der deutschen Anbieterstruktur. Target Value Design ist, wie Ouma et al. (2025) im Reifegradmodell darlegen, in seiner voll entwickelten Form ein datengetriebener Prozess (\enquote{Quantitatively Managed}). Er setzt voraus, dass Kostenkennwerte modellbasiert (BIM), transparent (Open Book) und in Echtzeit geteilt werden \autocite{ouma_target_2025}.

\minisec{Die digitale Kluft im Mittelstand}
Die deutsche Bauwirtschaft ist hingegen kleinteilig organisiert und durch das Handwerk geprägt. Über 90\,\% der Betriebe sind kleine und mittlere Unternehmen (KMU). Aktuelle Marktdaten, wie der \textit{BIM Monitor 2025} von BauInfoConsult, belegen eine signifikante \enquote{digitale Kluft}: Während Planungsbüros die BIM-Methodik zunehmend adaptieren, verfügen ausführende Handwerksbetriebe oft weder über die notwendige Software-Infrastruktur noch über die Prozessreife für eine modellbasierte Zusammenarbeit \autocite{bauinfoconsult_bim_2025}.
Für öffentliche Auftraggeber entsteht hier ein Zielkonflikt mit dem Gebot der Mittelstandsförderung (§ 97 Abs. 4 GWB). Eine konsequente TVD-Ausschreibung, die hohe digitale Kompetenzen (BIM Level 2/3) als Eignungskriterium fordert, wirkt faktisch als Marktzutrittsbarriere für lokale Betriebe.

\minisec{Kultureller Widerstand gegen die zuschlagsfreie Kalkulation}
Ein weiteres Hindernis ist die im deutschen Baumarkt verankerte Kalkulationssystematik. TVD erfordert nicht nur bloße Transparenz, sondern methodisch zwingend eine \enquote{zuschlagsfreie Kalkulation}.
Dies bedeutet, dass die Preise ausschließlich auf den tatsächlichen **Einzelkosten der Teilleistung** und den nachweisbaren Gemeinkosten basieren dürfen. Klassische Aufschläge für Wagnis, Gewinn oder Risikopuffer, die in der VOB-Kalkulation üblicherweise in die Einheitspreise eingerechnet werden, müssen hier explizit ausgeklammert und separat (z.\,B. über den Risikopool) vergütet werden.

Für viele KMU stellt dieser Paradigmenwechsel ein existenzielles Risiko dar. In einem Markt, der traditionell durch Preiskampf geprägt ist, basiert die Marge des Handwerkers oft auf einer internen Mischkalkulation und geschickten Einkaufskonditionen, die als Geschäftsgeheimnis gehütet werden. Die Forderung, diese \enquote{nackten Kosten} offenzulegen, erzeugt die Angst, gläsern zu werden und in zukünftigen Ausschreibungen erpressbar zu sein. Studien des Zentralverbands Deutsches Baugewerbe (ZDB) deuten darauf hin, dass diese Abwehrhaltung gegen die Offenlegung der wahren Kostenstruktur eines der Haupthindernisse für die Verbreitung partnerschaftlicher Modelle im Handwerk ist \autocite[vgl. S. 37]{zdb_geschaftsbericht_2024}.

\minisec{Asymmetrisches Investitionsrisiko}
Schließlich erfordert der Einstieg in TVD erhebliche Vorleistungen (\textit{Upfront Investment}). Um methodisch mitzuwirken, müssen Unternehmen in Schulungen (Lean Construction) und IT investieren. Da die Margen im Bauhauptgewerbe traditionell niedrig sind und, wie PwC (2024) analysiert, im aktuellen Krisenmodus Innovationsbudgets oft als erstes gekürzt werden, können sich oft nur kapitalstarke Großunternehmen dieses \enquote{Risikokapital} leisten. \autocite [vgl. S. 3ff]{berbner_bauindustrie_2024}
Es besteht die Gefahr einer \enquote{Elitisierung}: TVD-Projekte wären nur noch mit großen, überregionalen Generalunternehmern realisierbar. Dies schränkt nicht nur den Wettbewerb ein, sondern widerspricht oft dem politischen Willen kommunaler Auftraggeber, die regionale Wertschöpfung zu stärken.

\subsection{Lösungsansatz: Das zweistufige Bauteam-Modell als Brückentechnologie}
\label{sec:5.1.4}

Die Analyse der Defizite in den Abschnitten \ref{sec:5.1.1} bis \ref{sec:5.1.3} führt zu einer klaren Anforderungsmatrix für einen funktionierenden TVD-Beschaffungsprozess im öffentlichen Standardsegment. Gesucht wird ein Modell, das die Transaktionskosten der Vergabe niedrig hält (Lösung für \ref{sec:5.1.1}), eine Einbindung in Leistungsphase 2/3 ermöglicht, ohne vergaberechtliche \enquote{Vorbefasstheit} zu erzeugen (Lösung für \ref{sec:5.1.2}), und die Eintrittshürden für KMU senkt (Lösung für \ref{sec:5.1.3}).

Als synthesefähiger Lösungsansatz kristallisiert sich hierbei das Modell der **zweistufigen Vergabe**, in der deutschen Praxis oft als \enquote{Bauteam-Modell} bezeichnet, heraus. Konzepte hierzu wurden unter anderem vom Hauptverband der Deutschen Bauindustrie in Kooperation mit dem Deutschen Städtetag entwickelt, was ihre Akzeptanz sowohl auf Markt- als auch auf Auftraggeberseite unterstreicht \autocite{hauptverband_der_deutschen_bauindustrie_ev_partnerschaftliche_2020}.

\minisec{Funktionsweise der Zweistufigkeit}
Der Kern des Modells ist die vertragliche Trennung von Planungs-/Beratungsleistung und Bauausführung bei gleichzeitiger verfahrenstechnischer Koppelung.
In einer **ersten Stufe** schreibt der Auftraggeber lediglich die \textit{Pre-Construction Services} aus. Der Unternehmer wird hierbei nicht für das Bauwerk selbst, sondern zunächst für seine Beratungsleistung (Kalkulation, Optimierung, Baubarkeitsprüfung) in der Planungsphase vergütet.
Der Vertrag enthält jedoch eine **Option** auf die zweite Stufe (die Bauausführung). Diese Option wird nur gezogen, wenn am Ende der Planungsphase (Meilenstein: Design Freeze) das gemeinsam entwickelte Target Value (Zielkosten) eingehalten wird \autocite[vgl.]{hauptverband_der_deutschen_bauindustrie_ev_partnerschaftliche_2020}.

\minisec{Lösung des Transaktionskosten-Dilemmas}
Dieses Modell entschärft das in Abschnitt \ref{sec:5.1.1} beschriebene ökonomische Missverhältnis. Da der fixe Beauftragungsumfang der ersten Stufe finanziell gering ist (reines Beratungshonorar), sinkt das Risiko für beide Seiten. Der Wettbewerb verlagert sich vom reinen Endpreis (der noch unbekannt ist) hin zu den Zuschlagskriterien \enquote{Gemeinkosten}, \enquote{Gewinnmarge} und \enquote{Teamkompetenz}. Dies ist vergaberechtlich deutlich schlanker abzubilden als ein komplexer Allianzvertrag, da die Vertragsgrundlage weiterhin ein klassischer Werkvertrag (VOB/B) sein kann, dem lediglich eine partnerschaftliche Phase vorgeschaltet ist.

\minisec{Vermeidung der Vorbefasstheits-Falle}
Das Modell löst zudem das in Abschnitt \ref{sec:5.1.2} diskutierte Problem der späten Einbindung. Da der Bauunternehmer bereits \textit{durch} das Vergabeverfahren ausgewählt wurde (mit der Option auf Bau), ist er in der Planungsphase kein externer \enquote{Berater} (Projektant), sondern bereits der vertraglich gebundene Partner. Die Vorbefasstheit ist somit kein Störfaktor für einen \textit{späteren} Wettbewerb, da der Wettbewerb bereits \textit{stattgefunden} hat. Dies ermöglicht die methodisch zwingende Einbindung in Leistungsphase 3 (Design-to-Cost), ohne rechtliche Risiken einzugehen \autocite[vgl.]{greb_projektantenproblematik_2024}.

\minisec{Brücke für den Mittelstand}
Schließlich adressiert der Ansatz die in Abschnitt \ref{sec:5.1.3} identifizierten Markthürden. Für KMU ist das Bauteam-Modell attraktiv, da es kein \enquote{Alles-oder-Nichts}-Risiko darstellt. Die Beratungsleistung wird vergütet, auch wenn das Projekt später nicht gebaut wird. Zudem ist die Einstiegshürde geringer als bei IPA: Es wird kein gemeinsames Unternehmen (Mehrparteienvertrag) gegründet, und die \enquote{Open Book}-Anforderung beschränkt sich oft auf die Offenlegung der Nachunternehmer-Angebote in der zweiten Stufe. Dies mildert den kulturellen Widerstand gegen die komplette Offenlegung der eigenen Firmenkalkulation ab \autocite[vgl. S.37]{zdb_geschaftsbericht_2024}.

\minisec{Fazit Cluster 1}
Das zweistufige Bauteam-Modell fungiert somit als \enquote{Enabler} für Target Value Design in der Breite. Es ist der pragmatische Kompromiss, der die methodische Notwendigkeit der frühen Einbindung mit den rechtlichen und ökonomischen Restriktionen der öffentlichen Hand versöhnt. Es transformiert den Prozess von einer \enquote{Black Box}-Vergabe hin zu einer kooperativen Preisentwicklung, ohne die Sicherheitslinien des Vergaberechts zu verlassen.

\clearpage

\section{Cluster 2: Das Wert-Dilemma}
\label{sec:cluster2}

Die Kernfrage des Target Value Design lautet: \enquote{Wie generieren wir den maximalen Wert für den Kunden innerhalb der zulässigen Kosten?}
Diese scheinbar triviale Zielsetzung stößt im deutschen Kontext, insbesondere bei öffentlichen Hochbauprojekten, auf ein fundamentales Definitionsproblem. Während die Kostenseite (Target Cost) durch Haushaltsbudgets hart definiert ist, bleibt der \enquote{Wert} (Target Value) oft diffus. Es entsteht eine Asymmetrie, die dazu führt, dass die Methodik ihre Balance verliert. Dieses Cluster analysiert die Ursachen: das Fehlen eines monetären Business Case (5.2.1), die daraus resultierende Kostenzentrierung zu Lasten von Nutzer und Umwelt (5.2.2) sowie die strukturelle Vernachlässigung der Lebenszykluskosten (5.2.3).

\subsection{Das Vakuum der ökonomischen Sanktionierung und die Erosion der Zielkosten}
\label{sec:5.2.1}

In privatwirtschaftlichen Bauprojekten fungiert der \enquote{Business Case} als zentrales Steuerungs- und Sanktionsinstrument. Investitionsentscheidungen werden an einem erwarteten Return on Investment (ROI) gespiegelt; Kostensteigerungen sind nur dann rational, wenn sie durch höhere Erträge kompensiert werden können. Dieser ökonomische Rückkopplungsmechanismus verleiht Zielkosten eine tatsächlich bindende Funktion.

Im öffentlichen Hochbau, der primär der Daseinsvorsorge dient, existiert ein solcher Mechanismus nicht. Gebäude wie Schulen oder Verwaltungsbauten generieren keine monetäre Rendite, die in direkter Beziehung zu den Herstellungskosten steht. Der \enquote{Wert} öffentlicher Bauprojekte manifestiert sich vielmehr in gesellschaftlichen, funktionalen oder qualitativen Dimensionen, die sich nur begrenzt monetarisieren lassen \autocite[vgl. zur Problematik der Wertdefinition]{miron_target_2015}.

\minisec{Der Intention-Action-Gap}
Diese strukturelle Asymmetrie führt zu einem von Kozuch et al. (2024) beschriebenen \enquote{Intention-Action-Gap}: Zwar besteht auf politischer Ebene häufig ein expliziter Anspruch, qualitative Ziele wie Nachhaltigkeit, Nutzerkomfort oder langfristige Wirtschaftlichkeit zu verfolgen (\textit{Intention}). In der konkreten Vergabe- und Beschaffungspraxis dominiert jedoch der Anschaffungspreis als rechtssicherste und am leichtesten überprüfbare Entscheidungsgröße (\textit{Action}). Der \enquote{Wert} bleibt damit eine normative Zielvorstellung, während das Budget die einzige operative Restriktion darstellt \autocite[vgl.]{kozuch_nachhaltigkeit_2024}.

\minisec{Erosion durch Soft Budget Constraints}
Für Target Value Design ergibt sich daraus ein grundlegendes Spannungsfeld. Das methodische Kernprinzip von TVD – die Optimierung des Entwurfs innerhalb einer verbindlichen Kostenobergrenze – setzt voraus, dass die Zielkosten als nicht verhandelbare Randbedingung akzeptiert werden. In öffentlichen Bauprojekten wird diese Voraussetzung jedoch durch sogenannte \enquote{Soft Budget Constraints} unterlaufen. Historische Großprojekte wie der Flughafen Berlin Brandenburg zeigen, dass Kostenüberschreitungen zwar politische Konsequenzen haben können, jedoch selten zu einem Abbruch oder einer grundlegenden Infragestellung des Projekts führen. Stattdessen erfolgt im Regelfall eine Nachfinanzierung.

In einem solchen Kontext verlieren Zielkosten ihre disziplinierende Wirkung. Sie fungieren nicht länger als harte Steuerungsgröße, sondern werden selbst Teil eines politischen Aushandlungsprozesses. Die Gefahr besteht, dass Target Value Design unter diesen Bedingungen von einem kostensteuernden Entwurfsinstrument zu einem formalisierten, aber wirkungsarmen Verwaltungsprozess degeneriert. Die Innovationskraft von TVD, die gerade aus der Unausweichlichkeit der Kostenrestriktion entsteht, wird dadurch systematisch abgeschwächt.

\subsection{Die Methodenfalle: Kostenzentrierung (Cost-Centricity)}
\label{sec:5.2.2}

Die in Abschnitt \ref{sec:5.2.1} beschriebene Dominanz harter Budgetrestriktionen begünstigt im Projektverlauf eine Dynamik, die Miron et al. (2015) als \enquote{Cost-Centricity} von Target Value Design beschreiben. In ihrer Analyse zeigen die Autoren, dass die praktische Anwendung von TVD bislang primär über erzielte Kosteneinsparungen dokumentiert ist, während Beiträge zur tatsächlichen Wertgenerierung für den Auftraggeber nur unzureichend beschrieben, gemessen oder evaluiert werden \autocite{miron_target_2015}.

Diese Kostenzentrierung ist nicht als Fehlanwendung der Methode zu verstehen, sondern als strukturelle Konsequenz eines asymmetrischen Zielsystems. Während Zielkosten früh festgelegt, kontinuierlich überprüft und institutionell sanktioniert werden, bleibt der angestrebte Zielwert häufig vage, plural (\enquote{values} statt \enquote{value}) und ohne klare Bewertungslogik. In der Terminologie der TFV-Theorie nach Koskela (2000) ist festzustellen, dass TVD die Dimensionen \textit{Transformation} und \textit{Flow} methodisch adressiert, die Dimension \textit{Value} jedoch nicht in vergleichbarer Tiefe operationalisiert \autocite[vgl.]{koskela_theory_2000}.

\minisec{Dominanz der First Costs}
In Abwesenheit expliziter Wertdefinitionen und belastbarer Evaluationsmechanismen orientieren sich Projektteams zwangsläufig an der einzigen objektiv überprüfbaren Referenzgröße: den Herstellungskosten. Die intendierte Maximierung von Wert innerhalb eines Kostenrahmens reduziert sich damit faktisch auf die Minimierung der \enquote{First Costs}. Qualitative Aspekte, deren Nutzen sich erst in der Nutzung oder im Betrieb entfaltet, verlieren an Entscheidungsrelevanz \autocite[vgl.]{russell_smith_sustainable_2015}.

\minisec{Descoping als Symptom}
In der Projektpraxis manifestiert sich diese systemische Kostenzentrierung häufig im \enquote{Descoping}. Um die Zielkosten einzuhalten, werden Leistungsumfänge reduziert oder qualitative Standards abgesenkt. Besonders betroffen sind Nutzerinteressen und Nachhaltigkeitsaspekte, da diese zwar strategisch gewünscht, jedoch weder eindeutig definiert noch messbar in den Steuerungsprozess integriert sind \autocite[vgl.]{kozuch_nachhaltigkeit_2024}.
Damit bestätigt sich die von Miron et al. formulierte zentrale Herausforderung von Target Value Design: Ohne eine gleichwertige Methodik zur Erfassung, Bewertung und Rückkopplung von Wert bleibt die Kostenkontrolle der dominante Steuerungsmechanismus.

\subsection{Die TOTEX-Lücke: Strukturelle Diskrepanz zwischen Investition und Betrieb}
\label{sec:5.2.3}

Während die vorangegangenen Abschnitte Defizite der Wertdefinition und -steuerung im Planungsprozess analysierten, adressiert dieser Abschnitt eine strukturelle Barriere, die in der zeitlichen Organisation öffentlicher Bauprojekte liegt. Ziel des Target Value Design ist theoretisch die Maximierung des Kundenwertes über den gesamten Lebenszyklus eines Bauwerks. In der öffentlichen Baupraxis existiert jedoch eine institutionell verankerte Systemgrenze zwischen der Erstellungsphase (CAPEX) und der Nutzungsphase (OPEX).

\minisec{Dominanz der First Costs}
Russell-Smith et al. (2015) zeigen, dass herkömmliche TVD-Prozesse dazu tendieren, den Fokus auf die sogenannten \enquote{First Costs} zu verengen. Da Projektteams vertraglich, organisatorisch und inzentivierungsseitig primär an das Einhalten des Errichtungsbudgets gebunden sind, werden Entscheidungen systematisch zugunsten kurzfristiger Investitionseinsparungen getroffen. Ohne eine explizite Integration von Life Cycle Assessment (LCA) oder Life Cycle Costing (LCC) optimiert das System damit nicht den Gesamtwert des Bauwerks, sondern lediglich den Zeitpunkt der Schlüsselübergabe \autocite{russell_smith_sustainable_2015}.

Für öffentliche Bauherren bedeutet dies, dass Gebäude zwar formal \enquote{on budget} realisiert werden, jedoch durch geringere Investitionen in langlebige oder energieeffiziente Komponenten langfristig höhere Betriebs- und Instandhaltungskosten verursachen. Der eigentliche öffentliche Nutzen, der sich maßgeblich im Betrieb entfaltet, bleibt unzureichend berücksichtigt.

\minisec{Das Split-Incentive-Problem der Kameralistik}
Im deutschen Kontext wird diese Problematik durch die Logik der Kameralistik und der getrennten Budgetierung zusätzlich verschärft. Investive Ausgaben für den Bau und konsumtive Ausgaben für Betrieb und Instandhaltung werden häufig in unterschiedlichen organisatorischen Einheiten verantwortet. Zimina et al. (2012) beschreiben diese Konstellation als klassisches \enquote{Split-Incentive-Problem}: Die Instanz, die über die Investition entscheidet, profitiert nicht von Einsparungen im Betrieb \autocite[vgl.]{zimina_target_2012}.

\minisec{Rational erzwungene Suboptimierung}
Für das TVD-Team entsteht daraus ein struktureller Zielkonflikt. Investitionen mit höherem Initialaufwand – etwa in energieeffiziente Gebäudetechnik – erhöhen den Total Value des Bauwerks über den Lebenszyklus, gefährden jedoch unmittelbar das Target Cost-Ziel. Da das System ausschließlich die Einhaltung der Baukosten sanktioniert, während Betriebskosten erst Jahre später wirksam werden, wird die Suboptimierung des Gesamtobjekts nicht nur begünstigt, sondern rational erzwungen. Die fehlende TOTEX-Steuerung unterminiert damit einen zentralen Anspruch von Target Value Design im öffentlichen Sektor.

\subsection{Lösungsansatz: Die Operationalisierung des Wertes (Value Management)}
\label{sec:5.2.4}

Zwar sind die folgenden Herausforderungen typisch für jede Anwendung von Target Value Design, doch im öffentlichen Sektor wiegen sie deutlich schwerer. Aus bloßen methodischen Hürden werden hier oft feste strukturelle Barrieren. 
Die Ursache liegt im speziellen System der öffentlichen Hand: Budgets wirken in der Praxis oft weniger verbindlich (\enquote{weiche Budgetgrenzen}) und wirtschaftliche Anreize greifen nicht direkt, da Investitionskosten und späterer Nutzen organisatorisch voneinander entkoppelt sind. \autocite[vgl.]{kozuch_nachhaltigkeit_2024}\, \autocite[vgl.]{zimina_target_2012}

\minisec{Hierarchische Wert-Strukturierung}
Ein möglicher Ansatz, um das Vakuum des fehlenden Business Case zu füllen, besteht in der methodischen \enquote{Härtung der Wertziele}. Miron et al. (2015) schlagen hierfür vor, den Wertbegriff explizit zu strukturieren. Anstatt abstrakter Projektziele (\enquote{Hohe Nutzerzufriedenheit}) müssten konkrete Wert-Attribute (Value Attributes) definiert werden, die messbar oder zumindest eindeutig bewertbar sind (z.\,B. Nachhallzeiten, Reinigungsintervalle oder Flexibilitätsgrade).
Für den deutschen Kontext wäre hierbei eine stärkere Integration der Nutzwertanalyse in den TVD-Prozess denkbar. Neben den \enquote{Target Cost} (Zielkosten) könnte ein \enquote{Target Value Score} (Ziel-Nutzwert) als korrespondierende Steuerungsgröße etabliert werden. Dies würde dem Projektteam ermöglichen, qualitative Anforderungen verbindlicher gegen Kosteneinsparungen abzuwägen. \autocite[vgl.]{miron_target_2015}

\minisec{Reaktivierung der Entscheidungslogik (CBA)}
Um der Falle der Kostenzentrierung (vgl. Abschnitt \ref{sec:5.2.2}) entgegenzuwirken, ist weniger die Einführung neuer Methoden erforderlich als vielmehr die konsequente Anwendung der wertbasierten Entscheidungslogik, die dem Target Value Design bereits inhärent ist.
Insbesondere Verfahren wie\acf{CBA}, die im theoretischen TVD-Ansatz eine zentrale Rolle spielen, werden in der Praxis öffentlicher Bauprojekte häufig nur eingeschränkt oder informell genutzt.
Anstatt Planungsalternativen primär anhand der Investitionskosten zu vergleichen, erlaubt CBA eine explizite Gegenüberstellung von Vorteilen und Mehrkosten. Dadurch wird der öffentliche Auftraggeber gezwungen, seine Präferenzen transparent zu machen, und es wird verhindert, dass qualitative Aspekte reflexartig zugunsten kurzfristiger Kosteneinsparungen reduziert werden \autocite[vgl.]{miron_target_2015}.

\minisec{Integration von Lebenszyklus-Zielen (Sustainable TVD)}
Zur Überbrückung der TOTEX-Lücke (\ref{sec:5.2.3}) ist eine Erweiterung der Zielgrößen notwendig. Russell-Smith et al. (2015) sowie Olender und Rosen (2023) plädieren für ein **Sustainable Target Value Design**, bei dem neben dem Kostenziel ein verbindliches Budget für Lebenszyklusfaktoren (z.\,B. Energiebedarf oder CO\textsubscript{2}-Äquivalente) festgelegt wird.
Dass dies praktisch umsetzbar ist, belegen Silveira und Alves (2018) anhand von Fallstudien: Sie konnten nachweisen, dass TVD-Praktiken signifikant dazu beitragen, strenge Nachhaltigkeitsstandards kostenneutral umzusetzen, sofern sich das Team von der reinen Betrachtung der Anschaffungskosten (\enquote{First Cost}) löst und kollaborative Methoden wie das \textit{Set-Based Design} nutzt.
Für die deutsche Verwaltung würde dies bedeuten, die Wirtschaftlichkeitsuntersuchung nicht als statisches Dokument, sondern als dynamische Steuerungsgröße in den TVD-Prozess zu integrieren. \autocite[vgl.]{russell_smith_sustainable_2015}\, \autocite[vgl.]{olender_rosen_2023}\,\autocite[cgl.]{silveira_tvd_sustainable_2018}

\clearpage

\section{Einordnung: Entschärfte vs. persistente Strukturprobleme}
\label{sec:5.3_intro}

Die Diskussion um Hemmnisse von Target Value Design in Deutschland wird häufig von der Kritik an fragmentierten Vertragsstrukturen dominiert. Die klassische Trennung von Planung und Ausführung sowie die antagonistische Anreizlogik von Einheitspreisverträgen (VOB/B) gelten zurecht als fundamentale Barrieren für die kollaborative TVD-Methodik.
Diese Betrachtung greift jedoch zu kurz, da sie strukturelle Defizite (die durch neue Vertragsmodelle lösbar sind) mit systemischen Grenzen (die im Ordnungsrahmen verankert sind) vermischt.

Mit der zunehmenden Etablierung der Integrierten Projektabwicklung (IPA) im öffentlichen Sektor steht mittlerweile ein vertragliches Instrumentarium zur Verfügung, das viele der klassischen \enquote{TVD-Killer} – etwa die Haftungsproblematik oder das Honorarparadoxon – systematisch adressiert. Dieses Cluster fokussiert daher nicht auf die generellen Inkompatibilitäten des Standard-VOB-Vertrags. Stattdessen analysiert es jene persistenten Spannungsfelder, die selbst bei Einsatz eines idealen Mehrparteienvertrags bestehen bleiben.
Es wird die These vertreten, dass die verbleibende Fragilität von TVD in Deutschland weniger in der Vertragsgestaltung begründet liegt, sondern in der mangelnden Synchronisation zwischen der iterativen Projektmethodik und der linearen Logik des öffentlichen Genehmigungs- und Haushaltsrechts.

\subsection{TVD als implizit IPA-abhängige Methodik}
\label{sec:5.3.1}

In der Literatur werden häufig die Honorarordnung für Architekten und Ingenieure (HOAI) und das haftungsrechtliche Trennungsprinzip als Haupthindernisse für TVD angeführt. Diese Einschätzung ist im Kontext konventioneller Vergaben korrekt, verliert jedoch im Kontext partnerschaftlicher Modelle an Schärfe. Vielmehr muss konstatiert werden: Target Value Design ist in der deutschen Praxis faktisch an die Logik der Integrierten Projektabwicklung gekoppelt.

\minisec{Die Auflösung des Honorarparadoxons}
Das oft zitierte \enquote{Honorarparadoxon} – dass Planer in der HOAI für aufwendige Kostenoptimierungen bestraft werden, da ihr Honorar an die (sinkenden) Baukosten gekoppelt ist – stellt in einem Mehrparteienvertrag (IPA) kein Hindernis mehr dar. In solchen Modellen werden Planungsleistungen üblicherweise \enquote{at cost} (Selbstkosten) erstattet, zuzüglich eines Gewinnaufschlags, der an den Gesamterfolg des Projekts (Target Cost) gebunden ist. Die HOAI wirkt hier nur noch als preisrechtlicher Rahmen für die Mindestsätze, nicht mehr als Fehlanreizsystem. Die ökonomische Motivation des Planers wird durch das IPA-Modell harmonisiert: Er verdient mehr, wenn das Projekt günstiger wird.

\minisec{Funktionale Entschärfung der Haftung}
Ebenso verhält es sich mit der Schnittstellenproblematik. In der klassischen Einzelvergabe führt die iterative Zusammenarbeit (z.\,B. frühe Einbindung von Bauwissen in die Planung) zu diffusen Haftungsrisiken. Im IPA-Kontext wird dieses Risiko durch die \enquote{No-Blame}-Kultur und den vertraglichen Haftungsverzicht (außer bei Vorsatz/grober Fahrlässigkeit) funktional neutralisiert. Risiken werden nicht individuell, sondern kollektiv über den gemeinsamen Risikotopf (Risk Pool) getragen.

\minisec{Strukturelle Voraussetzung statt Schwäche}
Daraus folgt eine wichtige Neubewertung: Die Abhängigkeit von komplexen Vertragsmodellen ist keine \enquote{Schwäche} von TVD, sondern eine strukturelle Voraussetzung. Viele als \enquote{TVD-Probleme} diskutierte Aspekte sind in Wahrheit Probleme einer unvollständigen Anwendung ohne den passenden vertraglichen Container. Für die weitere Analyse dieses Clusters werden diese vertraglich lösbaren Themen daher als \enquote{entschärft} betrachtet. Der Fokus richtet sich nun auf den harten Kern der Probleme, der jenseits der Vertragsfreiheit der Parteien liegt: die regulatorische Prozesslogik.

\subsection{Persistentes Kernproblem: Phasentrennung und fehlendes Fast-Tracking}
\label{sec:5.3.2}

Während vertragliche Barrieren durch Modelle der Integrierten Projektabwicklung (IPA) weitgehend neutralisiert werden können, stößt die Anwendung von Target Value Design in Deutschland an eine härtere, externe Systemgrenze: die regulatorisch verankerte Trennung von Planung, Genehmigung und Ausführung. Dieses Spannungsfeld bleibt auch im optimalen Vertragsumfeld bestehen, da es nicht dem Privatrecht (Vertrag), sondern dem öffentlichen Baurecht (Ordnungsrahmen) entspringt.

Während die Phasentrennung in den vorangegangenen Clustern primär als organisations- und vergaberechtliches Problem der frühen Einbindung diskutiert wurde, wird sie im Folgenden bewusst auf einer anderen Ebene betrachtet: als systemische Inkompatibilität zwischen der iterativen Logik von Target Value Design und der fixierungsorientierten Logik des öffentlichen Genehmigungsrechts.

\minisec{Kollision der Logiken: Iteration vs. Fixierung}
Der methodische Kern von TVD basiert auf Prinzipien wie dem \textit{Set-Based Design} und dem \textit{Concurrent Engineering}. Entscheidungen sollen so lange wie möglich offengehalten werden (\enquote{Last Responsible Moment}), um Optimierungspotenziale zu wahren, während Planung und Ausführung sich zeitlich überlappen.
Diese Logik steht in fundamentalem Widerspruch zur deutschen Genehmigungs-Systematik. Das Baurecht setzt für die Erteilung einer Baugenehmigung eine \enquote{genehmigungsreife}, also in wesentlichen Teilen fixierte Planung voraus.

Das deutsche System erzwingt somit eine frühzeitige Festlegung (Design Freeze) zu einem Zeitpunkt, an dem TVD eigentlich noch Variabilität fordert. Auch ein IPA-Vertrag kann diese externe Restriktion nicht aufheben: Das Bauamt genehmigt keine \enquote{Lösungsräume}, sondern nur definierte Pläne.

\minisec{Die Grenzen des Fast-Trackings}
Besonders deutlich wird dies beim Thema \textit{Fast-Tracking} (Parallelisierung von Planen und Bauen). In anglo-amerikanischen Systemen, aus denen TVD stammt, ist es üblich, mit dem Rohbau zu beginnen, während der Ausbau noch geplant wird. In Deutschland ist dies zwar technisch möglich, aber prozessual risikobehaftet. Die Notwendigkeit einer abgeschlossenen Genehmigungsplanung vor Baubeginn (oder zumindest vor Beginn relevanter Teilabschnitte) wirkt als natürliche Bremse für die radikale Parallelisierung, die TVD zur Kosteneinhaltung oft benötigt.

\subsection{Teilbaugenehmigungen als unzureichender Ersatz für echte Parallelisierung}
\label{sec:5.3.3}

Um das beschriebene Dilemma zu umgehen, greifen Projekte in der Praxis häufig auf das Instrument der Teilbaugenehmigung (§\,74 Musterbauordnung) zurück. Dieses Vorgehen wird oft als Äquivalent zum Fast-Tracking missverstanden, stellt jedoch bei genauerer Analyse nur einen bürokratischen \enquote{Workaround} dar, der die methodischen Stärken von TVD schwächen kann.

\minisec{Erhöhung der Komplexität}
Zwar ermöglichen Teilbaugenehmigungen einen früheren Baubeginn (z.\,B. für Baugrube und Gründung). Sie erkaufen diesen Zeitgewinn jedoch durch eine Fragmentierung des Genehmigungsprozesses und eine Erhöhung der Schnittstellenkomplexität. Anstatt den Prozess zu verschlanken (Lean-Gedanke), wird der administrative Aufwand erhöht.

\minisec{Einschränkung des Lösungsraums}
Schwerwiegender ist der Einfluss auf die TVD-Methodik: Eine Teilbaugenehmigung zwingt das Team, bestimmte Parameter (z.\,B. Statik, Gebäudekubatur) zu einem sehr frühen Zeitpunkt verbindlich einzureichen und damit einzufrieren. Dies unterläuft den Ansatz des \textit{Set-Based Design}. Das Team muss Entscheidungen treffen, um den Verwaltungsakt zu bedienen, nicht weil es der \enquote{Last Responsible Moment} für den Projekterfolg erfordert. Faktisch führt der bürokratische Zwang zur Teilung des Genehmigungsprozesses dazu, dass Flexibilität – die wichtigste Ressource im Target Value Design – vorzeitig aufgegeben wird.


\subsection{Einordnung: Grenze der Vertragslogik}
\label{sec:5.3.4}

Die Analyse dieses Clusters führt zu einer differenzierten Bewertung der Machbarkeit von Target Value Design im öffentlichen Sektor. Die nahe liegende Annahme, dass die Einführung Integrierter Projektabwicklung (IPA) automatisch alle Strukturdefizite beseitigt, muss relativiert werden.

Zwar ist IPA ein notwendiger Enabler, um die internen Anreizsysteme des Projektteams (Vergütung, Haftung, Kooperation) auf die TVD-Logik auszurichten. Es ist jedoch kein hinreichender Enabler, um die externen Rahmenbedingungen zu flexibilisieren. Ein Mehrparteienvertrag ist ein privatrechtliches Instrument, das lediglich das Binnenverhältnis der Akteure regelt. Er besitzt keine juristische Kraft, um regulatorische Vorgaben des öffentlichen Baurechts – wie die Forderung nach einer fixierten Genehmigungsplanung – außer Kraft zu setzen.

Die verbleibende Barriere für eine konsequente Anwendung von TVD liegt somit nicht mehr in der Organisation der Projektbeteiligten, sondern im starren Ordnungsrahmen, der iterative Planungsprozesse institutionell benachteiligt. Solange das Genehmigungsrecht Linearität erzwingt, während die Projektmethodik auf Agilität setzt, bleibt TVD auch im IPA-Modell ein Kompromiss.

\subsection{Fazit Cluster 3: Von der Struktur- zur Systemfrage}
\label{sec:5.3.5}

Zusammenfassend zeigt sich, dass Target Value Design in Deutschland nicht primär an \enquote{weichen} Faktoren wie Kultur oder fehlendem Kooperationswillen scheitert, sondern an harten strukturellen Inkompatibilitäten.
Während klassische Defizite wie die Honorar- und Haftungsproblematik durch neue Vertragsmodelle (IPA) wirksam neutralisiert werden können, bleibt ein zentrales Spannungsfeld persistent: die mangelnde Vereinbarkeit von iterativer Wertentwicklung und sequenzieller Genehmigungslogik.

Die Implikation für die Praxis ist weitreichend: Die Weiterentwicklung von TVD im öffentlichen Bau kann nicht allein durch Pilotprojekte und Kulturwandel gelöst werden. Sie erfordert perspektivisch eine Anpassung der administrativen Prozesse – weg von der prüftechnischen Fixierung auf den Planungsstand, hin zu einer genehmigungsrechtlichen Toleranz für Lösungsräume. Ohne diese Evolution des Ordnungsrahmens bleibt TVD ein leistungsfähiger Motor, der in einem zu engen Chassis läuft.



























