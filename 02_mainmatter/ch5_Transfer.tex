\chapter{Analyse systemischer Spannungsfelder und Implementierungsbarrieren}
\label{ch:transfer}

Im vorangegangenen Kapitel wurde das Prozessmodell des Target Value Design (TVD) für den deutschen Kontext entwickelt. Bereits bei der Betrachtung der einzelnen Phasen hat sich gezeigt, dass eine direkte Übertragung der Lean-Methodik auf die hiesige Baupraxis schwierig ist. Die in Kapitel 4 beschriebenen \enquote{Sollbruchstellen} sind keine Einzelfälle, sondern weisen auf grundsätzliche Unterschiede in den Systemen hin.

Der Konflikt besteht im Kern zwischen zwei unterschiedlichen Denkweisen: Auf der einen Seite steht der TVD-Ansatz, der auf Wahrscheinlichkeiten, Zusammenarbeit und funktionalen Zielen basiert. Auf der anderen Seite steht das etablierte deutsche System aus VOB, HOAI und öffentlichem Haushaltsrecht, das feste Preise, eine strikte Trennung der Phasen und detaillierte Vorgaben verlangt.

Um die Vielzahl der gefundenen Hürden – vom fehlenden wirtschaftlichen Anreiz in Phase 0 bis zur Haftungsproblematik in Phase 3 – systematisch adressieren zu können, werden sie im Folgenden thematisch gebündelt. Eine reine Auflistung der 15 Punkte würde die Zusammenhänge vernachlässigen. Stattdessen werden die Hürden in drei große Spanungsfelder (Clustern) zusammengefasst, welche die Hauptprobleme bei der Einführung abbilden:

\begin{enumerate}
    \item \textbf{Das Beschaffungs-Dilemma (Marktzugang und Recht):} Wie lassen sich kooperative Teams bilden, wenn das Vergaberecht primär auf Wettbewerb und Preis ausgerichtet ist? (Diskussion der Punkte aus Phase 1).
    \item \textbf{Das Wert-Dilemma (Methodik und Steuerung):} Worauf wird optimiert, wenn klassische Werttreiber (wie Rendite) fehlen und Normen den Spielraum einschränken? (Diskussion der Punkte aus Phase 0 und 2).
    \item \textbf{Das Struktur-Dilemma (Anreize und Kultur):} Wie können Lernkurven und Optimierungen entstehen, wenn die aktuellen Honorar- und Haftungsregeln dies eher bestrafen? (Diskussion der Punkte aus Phase 2 und 3).
\end{enumerate}

Ziel dieses Kapitels ist es, innerhalb dieser drei Bereiche zu diskutieren, inwieweit neue Ansätze (wie die Integrierte Projektabwicklung/IPA) diese Barrieren bereits senken können und wo auch mit neuen Verträgen noch offene Fragen bleiben.

\section{Cluster 1: Das Beschaffungs-Dilemma}
\label{sec:cluster1}

Einer der erste und oft entscheidenden Stolpersteine bei der Implementierung von Target Value Design im deutschen Bauwesen liegt bereits vor dem eigentlichen Projektstart: in der Formierung des Teams. Während die TVD-Methodik auf der Prämisse basiert, dass die Schlüsselakteure (Planer und ausführende Unternehmen) frühzeitig und gemeinsam an der Lösungsfindung arbeiten, reglementiert das deutsche Vergaberecht den Marktzugang nach einer fundamental anderen Logik.

Dieses Spannungsfeld zwischen der Notwendigkeit einer frühen, kompetenzbasierten Teambindung und den gesetzlichen Anforderungen an Wettbewerb und Gleichbehandlung bildet das erste Diskussions-Cluster. Es wird im Folgenden anhand von drei Dimensionen untersucht: der juristischen Divergenz zwischen Preis- und Qualitätswettbewerb (5.1.1), der verfahrenstechnischen Hürde der Vorbefasstheit bei früher Einbindung (5.1.2) und den strukturellen Ausschlusskriterien für den Mittelstand (5.1.3).

\subsection{Das Transaktionskosten-Dilemma: Verfahrensökonomie und Skalierbarkeit}
\label{sec:5.1.1}

Die erste und grundlegendste Hürde bei der Initiierung eines TVD-Projekts im öffentlichen Sektor ist nicht die rechtliche Unmöglichkeit, sondern eine Diskrepanz zwischen dem methodisch erforderlichen Zeitpunkt der Beauftragung und der ökonomischen Verhältnismäßigkeit der verfügbaren Verfahren.
Das Paradoxon beginnt mit einer zeitlichen und logischen Divergenz: Wie in Abbildung \ref{fig:beschaffungsparadoxon} visualisiert, erfordert die TVD-Methodik zwingend, dass die Schlüsselakteure (Planer und ausführende Unternehmen) \textit{vor} Beginn der detaillierten Planung gebunden sind, um diese gemeinsam zu optimieren. Die Systematik der VOB/A hingegen ist historisch auf die Vergabe einer fertig geplanten Leistung zum Festpreis ausgelegt.

\begin{figure}[ht]
    \centering
    \fcolorbox{gray!50}{white}{%
        \begin{minipage}{\dimexpr\textwidth-2\fboxsep-2\fboxrule\relax}
            \centering
            \begin{tikzpicture}[
                >=Stealth,
                node distance=0cm,
                phase/.style={rectangle, draw=black, minimum height=1cm, text centered, font=\small},
                milestone/.style={circle, draw=red, fill=red!10, thick, minimum size=1.2cm, text width=1.5cm, align=center, font=\bfseries\footnotesize},
                arrow/.style={->, thick},
                scale=1,
                transform shape % Skaliert die Schrift mit
            ]
            % --- ZEITSTRAHL OBEN: TRADITIONELL (VOB) ---
            \node[anchor=west] at (0, 3.5) {\textbf{Traditionell (VOB/A): Sequenziell}};
            
            % Phasen (breitere Boxen für volle Breite)
            \node[phase, minimum width=5.5cm, fill=gray!10] (plan_vob) at (0, 2.2) {Planung (LPH 1-6)};
            \node[phase, minimum width=3.5cm, fill=gray!20, right=0cm of plan_vob] (tender_vob) {Ausschreibung};
            \node[phase, minimum width=5cm, fill=gray!30, right=0cm of tender_vob] (build_vob) {Bauausführung};
            
            % Meilenstein: Später Zuschlag
            \node[milestone, right=8.7cm of plan_vob.west, yshift=0cm] (award_vob) {Vergabe\\(Preis fix)};
            
            % Zeitachse
            \draw[arrow] (-0.5, 1.3) -- (12, 1.3);
            
            % --- ZEITSTRAHL UNTEN: TVD (INTEGRIERT) ---
            \node[anchor=west] at (0, -1.5) {\textbf{Target Value Design (TVD): Integriert}};
            
            % Phasen
            \node[phase, minimum width=2.8cm, fill=blue!10] (target) at (0, -2.8) {Zieldef.};
            \node[phase, minimum width=7.5cm, fill=blue!20, right=0cm of target] (tvd_phase) {TVD-Phase (Gemeinsame Planung)};
            \node[phase, minimum width=3.7cm, fill=blue!30, right=0cm of tvd_phase] (build_tvd) {Bauausführung};
            
            % Meilenstein: Früher Zuschlag
            \node[milestone, draw=blue, fill=blue!10, right=2.5cm of target.west] (award_tvd) {Vergabe\\(Team fix)};
            
            % Zeitachse
            \draw[arrow] (-0.5, -3.7) -- (12, -3.7);
            
            % --- KONFLIKT-VISUALISIERUNG ---
            % Verbindungslinie / Blitz
            \draw[<->, dashed, red, thick] (award_tvd.north) -- (award_vob.south) 
                node[midway, fill=white, text=red, font=\footnotesize\bfseries, align=center, draw=red, rounded corners] 
                {Das Beschaffungs-\\Paradoxon};
            
            \end{tikzpicture}
        \end{minipage}%
    }
    \caption{Divergenz der Vergabezeitpunkte: VOB-Modell vs. TVD-Modell}
    \label{fig:beschaffungsparadoxon}
\end{figure}

\minisec{Potenzielle Lösungsräume im Vergaberecht}
Betrachtet man den Rechtsrahmen isoliert, so scheint dieses Problem lösbar. Das Gesetz gegen Wettbewerbsbeschränkungen (§ 127 Abs. 1 GWB) erlaubt ausdrücklich die Berücksichtigung qualitativer Zuschlagskriterien neben dem Preis. Um die komplexe Auswahl eines TVD-Teams rechtssicher abzubilden, stellt der Gesetzgeber spezifische Verfahrensarten jenseits der starren \textit{Öffentlichen Ausschreibung} zur Verfügung.\autocite[vgl. S. 101]{noauthor_gesetz_2013}
Insbesondere das \textit{Verhandlungsverfahren mit Teilnahmewettbewerb} (§ 17 VgV) oder der noch komplexere \textit{Wettbewerbliche Dialog} (§ 18 VgV) bieten den rechtlichen Rahmen, um statt eines festen Preises (für eine noch unbekannte Lösung) die Kompetenz, die Methodenerfahrung und die Kooperationsbereitschaft der Bieter zu bewerten, bevor ein Vertrag geschlossen wird.\autocite[vgl. S. 14ff]{noauthor_verordnung_2016}

\minisec{Die Hürde der Transaktionskosten (Disproportionalität)}
Dass diese Verfahren in der Breite der öffentlichen Bauprojekte dennoch kaum Anwendung finden, ist auf ihre mangelnde ökonomische Effizienz bei Standardprojekten zurückzuführen. Wie Becker und Friedinger (2024) in ihrer Analyse zur Adaption von IPA-Modellen darlegen, verursachen diese komplexen Verfahrensarten massive \textbf{Transaktionskosten}, die weit über den Aufwand einer klassischen Submission hinausgehen \autocite{becker_adaptionen_2024}.
Die Kostentreiber sind hierbei vielschichtig:
\begin{itemize}
    \item \textbf{Juristischer Vorbereitungsaufwand:} Die Erstellung diskriminierungsfreier Eignungsmatrizen für \enquote{weiche} Kriterien (z.\,B. Bewertung eines fiktiven Workshops) erfordert spezialisierte juristische Beratung, um das Risiko von Rügen unterlegener Bieter zu minimieren.
    \item \textbf{Ressourcenbindung:} Mehrstufige Verhandlungsrunden binden über Monate hinweg hochqualifiziertes Personal in den Vergabestellen, das im operativen Tagesgeschäft fehlt.
    \item \textbf{Verfahrensdauer:} Die langen Fristen verzögern den Projektstart signifikant, was wiederum Finanzierungskosten treiben kann.
\end{itemize}

\minisec{Die Entstehung einer \enquote{Zwei-Klassen-Projektlandschaft}}
Für die öffentliche Hand ergibt sich daraus ein Konflikt mit dem haushaltsrechtlichen Grundsatz der Wirtschaftlichkeit (§ 7 BHO). Die Investition in das Vergabeverfahren muss in einem angemessenen Verhältnis zum Projektwert stehen.
Für Großprojekte (\enquote{Leuchttürme} > 50 Mio. €) amortisiert sich dieser Aufwand durch die späteren Projektoptimierungen. Für kleine und mittlere Projekte (Volumen < 10--15 Mio. €), die den Großteil der kommunalen Bauaufgaben (Schulen, Kitas, Wohnungsbau) ausmachen, ist dieser Aufwand jedoch unverhältnismäßig (\enquote{Over-Engineering} der Vergabe).
Becker und Friedinger (2024) konstatieren daher, dass die Komplexität des Vergaberechts faktisch als \enquote{Eintrittsbarriere} wirkt: TVD wird zu einer Methode für Eliten-Projekte, während die Breite der öffentlichen Infrastruktur aus Kostengründen in die VOB/A-Standardvergabe gezwungen wird, die eine echte frühe Einbindung strukturell verhindert \autocite{becker_adaptionen_2024}.

\minisec{Zwischenfazit: Die Verfahrenslücke}
Es lässt sich festhalten, dass das deutsche Vergabesystem zwar theoretisch Instrumente für TVD bereitstellt, diese aber für den Projektalltag \enquote{überdimensioniert} sind. Es fehlt ein niederschwelliger, rechtssicherer Prozess für die frühe Teambindung bei Standardprojekten.
Diese Lücke führt in der Praxis zu Ausweichbewegungen: Auftraggeber versuchen, die Komplexität zu reduzieren, indem sie vereinfachte Vertragsmodelle (\enquote{IPA-Light}) nutzen oder die Einbindung der Partner zeitlich nach hinten verschieben. Dass diese vermeintlichen Abkürzungen jedoch neue, methodische Risiken bergen, ist Gegenstand der folgenden Analyse.

\subsection{Methodische Defizite durch späte Einbindung (\enquote{IPA-Light})}
\label{sec:5.1.2}

Als Reaktion auf die in Abschnitt \ref{sec:5.1.1} dargelegten hohen Transaktionskosten etablierter Vergabeverfahren werden in der Fachliteratur und Praxis zunehmend Anpassungsstrategien diskutiert, die unter dem Begriff \enquote{IPA-Light} oder \enquote{vertragliche Adaptionen} subsumiert werden. Becker und Friedinger (2024) untersuchen hierbei Ansätze, wie Elemente der Integrierten Projektabwicklung in den Rechtsrahmen der VOB/A eingebettet werden können, ohne die volle Komplexität eines Allianzvertrages auszulösen \autocite{becker_adaptionen_2024}.

\minisec{Der Rückzug auf die späte Einbindung}
Der zentrale Hebel zur Reduktion der Verfahrenskomplexität besteht in diesen Modellen häufig darin, die vergaberechtlichen Risiken zu minimieren. Um dem administrativen Aufwand einer qualitativen Auswahl für eine sehr frühe Einbindung zu entgehen, tendieren öffentliche Auftraggeber dazu, die Einbindung der ausführenden Unternehmen zeitlich nach hinten zu verschieben \autocite[vgl. Diskussion der Adaptionsmodelle bei]{becker_adaptionen_2024}.

Treibender Faktor ist hierbei die Rechtsunsicherheit bezüglich der \textbf{Vorbefasstheit} gemäß § 7 VgV. Ein Unternehmen, das bereits in Leistungsphase 2 oder 3 beratend tätig ist, erlangt zwangsläufig einen Informationsvorsprung. Um dieses Unternehmen später rechtssicher mit der Bauausführung zu beauftragen, wären komplexe Ausgleichsmaßnahmen nötig, um den Wettbewerb nicht zu verzerren \autocite[vgl. § 7 VgV Rn. 20 ff.]{leinemann_vergabe_2021}.

Die \enquote{Light}-Lösung der Praxis besteht daher oft darin, diese Hürde zu umgehen, indem der Unternehmer erst zur Leistungsphase 5 (Ausführungsplanung) oder nach Abschluss der Genehmigungsplanung eingebunden wird. Die Logik lautet: Wer erst kommt, wenn die Planung steht, gilt nicht als vorbefasst \autocite[vgl. zur Kritik an späten Einbindungsmodellen]{breyer_alternative_2021}.

\minisec{Methodischer Widerspruch zum Design Freeze}
Diese Strategie der juristischen Risikovermeidung führt jedoch zu einer methodischen Entkernung des Target Value Design. Wie die theoretische Herleitung in Kapitel 4 gezeigt hat, basiert TVD essenziell auf der Einflussnahme auf die Kostentreiber \textit{während} der Entwurfsfindung (LPH 2--3). Granja et al. (2023) betonen, dass erfolgreiches TVD ein Verschmelzen der Grenzen zwischen Planung und Bau erfordert (\enquote{Boundary Spanning}) \autocite{granja_target_2023}.

Erfolgt die Einbindung im Rahmen eines vereinfachten Modells erst in LPH 5, sind die wesentlichen Weichenstellungen – wie Kubatur, Tragwerkskonzept oder Fassadensystem – bereits fixiert (\enquote{Design Freeze}). Der Unternehmer kann in dieser späten Phase nur noch die Ausführungslogistik optimieren oder Alternativangebote für Details unterbreiten, aber keine echten \textit{Design-to-Cost}-Entscheidungen mehr treffen, ohne die vorangegangene Planung und Genehmigung in Frage zu stellen.
Die Analyse zeigt somit, dass der Versuch, TVD durch \enquote{Light}-Modelle kompatibel zur VOB/A zu machen, oft das Kernmerkmal der Methode opfert: Das \textit{Early Contractor Involvement} (ECI) verkommt zu einem \textit{Late Contractor Involvement}, das zwar rechtssicher und kostengünstig zu vergeben ist, aber das Innovationspotenzial der Methode verfehlt.

\minisec{Zwischenfazit}
Es besteht ein direkter Zielkonflikt zwischen Verfahrensökonomie und methodischer Integrität. Die aktuellen Ansätze zur Vereinfachung (IPA-Light) lösen zwar das Kostenproblem der Vergabe (vgl. 5.1.1), schaffen dabei aber ein methodisches Vakuum, da sie die rechtzeitige Integration des Know-hows verhindern. Solange keine Lösung existiert, die sowohl einen geringen Vergabeaufwand als auch eine frühe Einbindung ermöglicht, bleiben adaptierte Modelle oft ein fauler Kompromiss. Neben diesen verfahrenstechnischen Hürden existiert jedoch noch eine weitere, strukturelle Barriere, die den Marktzugang erschwert: die mangelnde Investitionsfähigkeit der kleinteiligen Anbieterstruktur.

\subsection{Marktzutrittsbarrieren durch Struktur und Digitalisierungsgrad}
\label{sec:5.1.3}

Neben den vergaberechtlichen Hürden offenbart die Analyse eine tiefgreifende strukturelle Diskrepanz zwischen den Anforderungen der Methodik und der Realität der deutschen Anbieterstruktur. Target Value Design ist, wie Ouma et al. (2025) im Reifegradmodell darlegen, in seiner voll entwickelten Form ein datengetriebener Prozess (\enquote{Quantitatively Managed}). Er setzt voraus, dass Kostenkennwerte modellbasiert (BIM), transparent (Open Book) und in Echtzeit geteilt werden \autocite{ouma_target_2025}.

\minisec{Die digitale Kluft im Mittelstand}
Die deutsche Bauwirtschaft ist hingegen kleinteilig organisiert und durch das Handwerk geprägt. Über 90\,\% der Betriebe sind kleine und mittlere Unternehmen (KMU). Aktuelle Marktdaten, wie der \textit{BIM Monitor 2025} von BauInfoConsult, belegen eine signifikante \enquote{digitale Kluft}: Während Planungsbüros die BIM-Methodik zunehmend adaptieren, verfügen ausführende Handwerksbetriebe oft weder über die notwendige Software-Infrastruktur noch über die Prozessreife für eine modellbasierte Zusammenarbeit \autocite{bauinfoconsult_bim_2025}.
Für öffentliche Auftraggeber entsteht hier ein Zielkonflikt mit dem Gebot der Mittelstandsförderung (§ 97 Abs. 4 GWB). Eine konsequente TVD-Ausschreibung, die hohe digitale Kompetenzen (BIM Level 2/3) als Eignungskriterium fordert, wirkt faktisch als Marktzutrittsbarriere für lokale Betriebe.

\minisec{Kultureller Widerstand gegen die zuschlagsfreie Kalkulation}
Ein weiteres Hindernis ist die im deutschen Baumarkt verankerte Kalkulationssystematik. TVD erfordert nicht nur bloße Transparenz, sondern methodisch zwingend eine \enquote{zuschlagsfreie Kalkulation}.
Dies bedeutet, dass die Preise ausschließlich auf den tatsächlichen **Einzelkosten der Teilleistung** und den nachweisbaren Gemeinkosten basieren dürfen. Klassische Aufschläge für Wagnis, Gewinn oder Risikopuffer, die in der VOB-Kalkulation üblicherweise in die Einheitspreise eingerechnet werden, müssen hier explizit ausgeklammert und separat (z.\,B. über den Risikopool) vergütet werden.

Für viele KMU stellt dieser Paradigmenwechsel ein existenzielles Risiko dar. In einem Markt, der traditionell durch Preiskampf geprägt ist, basiert die Marge des Handwerkers oft auf einer internen Mischkalkulation und geschickten Einkaufskonditionen, die als Geschäftsgeheimnis gehütet werden. Die Forderung, diese \enquote{nackten Kosten} offenzulegen, erzeugt die Angst, gläsern zu werden und in zukünftigen Ausschreibungen erpressbar zu sein. Studien des Zentralverbands Deutsches Baugewerbe (ZDB) deuten darauf hin, dass diese Abwehrhaltung gegen die Offenlegung der wahren Kostenstruktur eines der Haupthindernisse für die Verbreitung partnerschaftlicher Modelle im Handwerk ist \autocite[vgl. S. 37]{zdb_geschaftsbericht_2024}.

\minisec{Asymmetrisches Investitionsrisiko}
Schließlich erfordert der Einstieg in TVD erhebliche Vorleistungen (\textit{Upfront Investment}). Um methodisch mitzuwirken, müssen Unternehmen in Schulungen (Lean Construction) und IT investieren. Da die Margen im Bauhauptgewerbe traditionell niedrig sind und, wie PwC (2024) analysiert, im aktuellen Krisenmodus Innovationsbudgets oft als erstes gekürzt werden, können sich oft nur kapitalstarke Großunternehmen dieses \enquote{Risikokapital} leisten. \autocite [vgl. S. 3ff]{berbner_bauindustrie_2024}
Es besteht die Gefahr einer \enquote{Elitisierung}: TVD-Projekte wären nur noch mit großen, überregionalen Generalunternehmern realisierbar. Dies schränkt nicht nur den Wettbewerb ein, sondern widerspricht oft dem politischen Willen kommunaler Auftraggeber, die regionale Wertschöpfung zu stärken.

\subsection{Lösungsansatz: Das zweistufige Bauteam-Modell als Brückentechnologie}
\label{sec:5.1.4}

Die Analyse der Defizite in den Abschnitten \ref{sec:5.1.1} bis \ref{sec:5.1.3} führt zu einer klaren Anforderungsmatrix für einen funktionierenden TVD-Beschaffungsprozess im öffentlichen Standardsegment. Gesucht wird ein Modell, das die Transaktionskosten der Vergabe niedrig hält (Lösung für \ref{sec:5.1.1}), eine Einbindung in Leistungsphase 2/3 ermöglicht, ohne vergaberechtliche \enquote{Vorbefasstheit} zu erzeugen (Lösung für \ref{sec:5.1.2}), und die Eintrittshürden für KMU senkt (Lösung für \ref{sec:5.1.3}).

Als synthesefähiger Lösungsansatz kristallisiert sich hierbei das Modell der **zweistufigen Vergabe**, in der deutschen Praxis oft als \enquote{Bauteam-Modell} bezeichnet, heraus. Konzepte hierzu wurden unter anderem vom Hauptverband der Deutschen Bauindustrie in Kooperation mit dem Deutschen Städtetag entwickelt, was ihre Akzeptanz sowohl auf Markt- als auch auf Auftraggeberseite unterstreicht \autocite{hauptverband_der_deutschen_bauindustrie_ev_partnerschaftliche_2020}.

\minisec{Funktionsweise der Zweistufigkeit}
Der Kern des Modells ist die vertragliche Trennung von Planungs-/Beratungsleistung und Bauausführung bei gleichzeitiger verfahrenstechnischer Koppelung.
In einer **ersten Stufe** schreibt der Auftraggeber lediglich die \textit{Pre-Construction Services} aus. Der Unternehmer wird hierbei nicht für das Bauwerk selbst, sondern zunächst für seine Beratungsleistung (Kalkulation, Optimierung, Baubarkeitsprüfung) in der Planungsphase vergütet.
Der Vertrag enthält jedoch eine **Option** auf die zweite Stufe (die Bauausführung). Diese Option wird nur gezogen, wenn am Ende der Planungsphase (Meilenstein: Design Freeze) das gemeinsam entwickelte Target Value (Zielkosten) eingehalten wird \autocite[vgl.]{hauptverband_der_deutschen_bauindustrie_ev_partnerschaftliche_2020}.

\minisec{Lösung des Transaktionskosten-Dilemmas}
Dieses Modell entschärft das in Abschnitt \ref{sec:5.1.1} beschriebene ökonomische Missverhältnis. Da der fixe Beauftragungsumfang der ersten Stufe finanziell gering ist (reines Beratungshonorar), sinkt das Risiko für beide Seiten. Der Wettbewerb verlagert sich vom reinen Endpreis (der noch unbekannt ist) hin zu den Zuschlagskriterien \enquote{Gemeinkosten}, \enquote{Gewinnmarge} und \enquote{Teamkompetenz}. Dies ist vergaberechtlich deutlich schlanker abzubilden als ein komplexer Allianzvertrag, da die Vertragsgrundlage weiterhin ein klassischer Werkvertrag (VOB/B) sein kann, dem lediglich eine partnerschaftliche Phase vorgeschaltet ist.

\minisec{Vermeidung der Vorbefasstheits-Falle}
Das Modell löst zudem das in Abschnitt \ref{sec:5.1.2} diskutierte Problem der späten Einbindung. Da der Bauunternehmer bereits \textit{durch} das Vergabeverfahren ausgewählt wurde (mit der Option auf Bau), ist er in der Planungsphase kein externer \enquote{Berater} (Projektant), sondern bereits der vertraglich gebundene Partner. Die Vorbefasstheit ist somit kein Störfaktor für einen \textit{späteren} Wettbewerb, da der Wettbewerb bereits \textit{stattgefunden} hat. Dies ermöglicht die methodisch zwingende Einbindung in Leistungsphase 3 (Design-to-Cost), ohne rechtliche Risiken einzugehen \autocite[vgl.]{greb_projektantenproblematik_2024}.

\minisec{Brücke für den Mittelstand}
Schließlich adressiert der Ansatz die in Abschnitt \ref{sec:5.1.3} identifizierten Markthürden. Für KMU ist das Bauteam-Modell attraktiv, da es kein \enquote{Alles-oder-Nichts}-Risiko darstellt. Die Beratungsleistung wird vergütet, auch wenn das Projekt später nicht gebaut wird. Zudem ist die Einstiegshürde geringer als bei IPA: Es wird kein gemeinsames Unternehmen (Mehrparteienvertrag) gegründet, und die \enquote{Open Book}-Anforderung beschränkt sich oft auf die Offenlegung der Nachunternehmer-Angebote in der zweiten Stufe. Dies mildert den kulturellen Widerstand gegen die komplette Offenlegung der eigenen Firmenkalkulation ab \autocite[vgl. S.37]{zdb_geschaftsbericht_2024}.

\minisec{Fazit Cluster 1}
Das zweistufige Bauteam-Modell fungiert somit als \enquote{Enabler} für Target Value Design in der Breite. Es ist der pragmatische Kompromiss, der die methodische Notwendigkeit der frühen Einbindung mit den rechtlichen und ökonomischen Restriktionen der öffentlichen Hand versöhnt. Es transformiert den Prozess von einer \enquote{Black Box}-Vergabe hin zu einer kooperativen Preisentwicklung, ohne die Sicherheitslinien des Vergaberechts zu verlassen.

\clearpage

\section{Cluster 2: Das Wert-Dilemma}
\label{sec:cluster2}

Die Kernfrage des Target Value Design lautet: \enquote{Wie generieren wir den maximalen Wert für den Kunden innerhalb der zulässigen Kosten?}
Diese scheinbar triviale Zielsetzung stößt im deutschen Kontext, insbesondere bei öffentlichen Hochbauprojekten, auf ein fundamentales Definitionsproblem. Während die Kostenseite (Target Cost) durch Haushaltsbudgets hart definiert ist, bleibt der \enquote{Wert} (Target Value) oft diffus. Es entsteht eine Asymmetrie, die dazu führt, dass die Methodik ihre Balance verliert. Dieses Cluster analysiert die Ursachen: das Fehlen eines monetären Business Case (5.2.1), die daraus resultierende Kostenzentrierung zu Lasten von Nutzer und Umwelt (5.2.2) sowie die strukturelle Vernachlässigung der Lebenszykluskosten (5.2.3).

\subsection{Das Vakuum der ökonomischen Sanktionierung und die Erosion der Zielkosten}
\label{sec:5.2.1}

In privatwirtschaftlichen Bauprojekten fungiert der \enquote{Business Case} als zentrales Steuerungs- und Sanktionsinstrument. Investitionsentscheidungen werden an einem erwarteten Return on Investment (ROI) gespiegelt; Kostensteigerungen sind nur dann rational, wenn sie durch höhere Erträge kompensiert werden können. Dieser ökonomische Rückkopplungsmechanismus verleiht Zielkosten eine tatsächlich bindende Funktion.

Im öffentlichen Hochbau, der primär der Daseinsvorsorge dient, existiert ein solcher Mechanismus nicht. Gebäude wie Schulen oder Verwaltungsbauten generieren keine monetäre Rendite, die in direkter Beziehung zu den Herstellungskosten steht. Der \enquote{Wert} öffentlicher Bauprojekte manifestiert sich vielmehr in gesellschaftlichen, funktionalen oder qualitativen Dimensionen, die sich nur begrenzt monetarisieren lassen \autocite[vgl. zur Problematik der Wertdefinition]{miron_target_2015}.

\minisec{Der Intention-Action-Gap}
Diese strukturelle Asymmetrie führt zu einem von Kozuch et al. (2024) beschriebenen \enquote{Intention-Action-Gap}: Zwar besteht auf politischer Ebene häufig ein expliziter Anspruch, qualitative Ziele wie Nachhaltigkeit, Nutzerkomfort oder langfristige Wirtschaftlichkeit zu verfolgen (\textit{Intention}). In der konkreten Vergabe- und Beschaffungspraxis dominiert jedoch der Anschaffungspreis als rechtssicherste und am leichtesten überprüfbare Entscheidungsgröße (\textit{Action}). Der \enquote{Wert} bleibt damit eine normative Zielvorstellung, während das Budget die einzige operative Restriktion darstellt \autocite[vgl.]{kozuch_nachhaltigkeit_2024}.

\minisec{Erosion durch Soft Budget Constraints}
Für Target Value Design ergibt sich daraus ein grundlegendes Spannungsfeld. Das methodische Kernprinzip von TVD – die Optimierung des Entwurfs innerhalb einer verbindlichen Kostenobergrenze – setzt voraus, dass die Zielkosten als nicht verhandelbare Randbedingung akzeptiert werden. In öffentlichen Bauprojekten wird diese Voraussetzung jedoch durch sogenannte \enquote{Soft Budget Constraints} unterlaufen. Historische Großprojekte wie der Flughafen Berlin Brandenburg zeigen, dass Kostenüberschreitungen zwar politische Konsequenzen haben können, jedoch selten zu einem Abbruch oder einer grundlegenden Infragestellung des Projekts führen. Stattdessen erfolgt im Regelfall eine Nachfinanzierung.

In einem solchen Kontext verlieren Zielkosten ihre disziplinierende Wirkung. Sie fungieren nicht länger als harte Steuerungsgröße, sondern werden selbst Teil eines politischen Aushandlungsprozesses. Die Gefahr besteht, dass Target Value Design unter diesen Bedingungen von einem kostensteuernden Entwurfsinstrument zu einem formalisierten, aber wirkungsarmen Verwaltungsprozess degeneriert. Die Innovationskraft von TVD, die gerade aus der Unausweichlichkeit der Kostenrestriktion entsteht, wird dadurch systematisch abgeschwächt.

\subsection{Die Methodenfalle: Kostenzentrierung (Cost-Centricity)}
\label{sec:5.2.2}

Die in Abschnitt \ref{sec:5.2.1} beschriebene Dominanz harter Budgetrestriktionen begünstigt im Projektverlauf eine Dynamik, die Miron et al. (2015) als \enquote{Cost-Centricity} von Target Value Design beschreiben. In ihrer Analyse zeigen die Autoren, dass die praktische Anwendung von TVD bislang primär über erzielte Kosteneinsparungen dokumentiert ist, während Beiträge zur tatsächlichen Wertgenerierung für den Auftraggeber nur unzureichend beschrieben, gemessen oder evaluiert werden \autocite{miron_target_2015}.

Diese Kostenzentrierung ist nicht als Fehlanwendung der Methode zu verstehen, sondern als strukturelle Konsequenz eines asymmetrischen Zielsystems. Während Zielkosten früh festgelegt, kontinuierlich überprüft und institutionell sanktioniert werden, bleibt der angestrebte Zielwert häufig vage, plural (\enquote{values} statt \enquote{value}) und ohne klare Bewertungslogik. In der Terminologie der TFV-Theorie nach Koskela (2000) ist festzustellen, dass TVD die Dimensionen \textit{Transformation} und \textit{Flow} methodisch adressiert, die Dimension \textit{Value} jedoch nicht in vergleichbarer Tiefe operationalisiert \autocite[vgl.]{koskela_theory_2000}.

\minisec{Dominanz der First Costs}
In Abwesenheit expliziter Wertdefinitionen und belastbarer Evaluationsmechanismen orientieren sich Projektteams zwangsläufig an der einzigen objektiv überprüfbaren Referenzgröße: den Herstellungskosten. Die intendierte Maximierung von Wert innerhalb eines Kostenrahmens reduziert sich damit faktisch auf die Minimierung der \enquote{First Costs}. Qualitative Aspekte, deren Nutzen sich erst in der Nutzung oder im Betrieb entfaltet, verlieren an Entscheidungsrelevanz \autocite[vgl.]{russell_smith_sustainable_2015}.

\minisec{Descoping als Symptom}
In der Projektpraxis manifestiert sich diese systemische Kostenzentrierung häufig im \enquote{Descoping}. Um die Zielkosten einzuhalten, werden Leistungsumfänge reduziert oder qualitative Standards abgesenkt. Besonders betroffen sind Nutzerinteressen und Nachhaltigkeitsaspekte, da diese zwar strategisch gewünscht, jedoch weder eindeutig definiert noch messbar in den Steuerungsprozess integriert sind \autocite[vgl.]{kozuch_nachhaltigkeit_2024}.
Damit bestätigt sich die von Miron et al. formulierte zentrale Herausforderung von Target Value Design: Ohne eine gleichwertige Methodik zur Erfassung, Bewertung und Rückkopplung von Wert bleibt die Kostenkontrolle der dominante Steuerungsmechanismus.

\subsection{Die TOTEX-Lücke: Strukturelle Diskrepanz zwischen Investition und Betrieb}
\label{sec:5.2.3}

Während die vorangegangenen Abschnitte Defizite der Wertdefinition und -steuerung im Planungsprozess analysierten, adressiert dieser Abschnitt eine strukturelle Barriere, die in der zeitlichen Organisation öffentlicher Bauprojekte liegt. Ziel des Target Value Design ist theoretisch die Maximierung des Kundenwertes über den gesamten Lebenszyklus eines Bauwerks. In der öffentlichen Baupraxis existiert jedoch eine institutionell verankerte Systemgrenze zwischen der Erstellungsphase (CAPEX) und der Nutzungsphase (OPEX).

\minisec{Dominanz der First Costs}
Russell-Smith et al. (2015) zeigen, dass herkömmliche TVD-Prozesse dazu tendieren, den Fokus auf die sogenannten \enquote{First Costs} zu verengen. Da Projektteams vertraglich, organisatorisch und inzentivierungsseitig primär an das Einhalten des Errichtungsbudgets gebunden sind, werden Entscheidungen systematisch zugunsten kurzfristiger Investitionseinsparungen getroffen. Ohne eine explizite Integration von Life Cycle Assessment (LCA) oder Life Cycle Costing (LCC) optimiert das System damit nicht den Gesamtwert des Bauwerks, sondern lediglich den Zeitpunkt der Schlüsselübergabe \autocite{russell_smith_sustainable_2015}.

Für öffentliche Bauherren bedeutet dies, dass Gebäude zwar formal \enquote{on budget} realisiert werden, jedoch durch geringere Investitionen in langlebige oder energieeffiziente Komponenten langfristig höhere Betriebs- und Instandhaltungskosten verursachen. Der eigentliche öffentliche Nutzen, der sich maßgeblich im Betrieb entfaltet, bleibt unzureichend berücksichtigt.

\minisec{Das Split-Incentive-Problem der Kameralistik}
Im deutschen Kontext wird diese Problematik durch die Logik der Kameralistik und der getrennten Budgetierung zusätzlich verschärft. Investive Ausgaben für den Bau und konsumtive Ausgaben für Betrieb und Instandhaltung werden häufig in unterschiedlichen organisatorischen Einheiten verantwortet. Zimina et al. (2012) beschreiben diese Konstellation als klassisches \enquote{Split-Incentive-Problem}: Die Instanz, die über die Investition entscheidet, profitiert nicht von Einsparungen im Betrieb \autocite[vgl.]{zimina_target_2012}.

\minisec{Rational erzwungene Suboptimierung}
Für das TVD-Team entsteht daraus ein struktureller Zielkonflikt. Investitionen mit höherem Initialaufwand – etwa in energieeffiziente Gebäudetechnik – erhöhen den Total Value des Bauwerks über den Lebenszyklus, gefährden jedoch unmittelbar das Target Cost-Ziel. Da das System ausschließlich die Einhaltung der Baukosten sanktioniert, während Betriebskosten erst Jahre später wirksam werden, wird die Suboptimierung des Gesamtobjekts nicht nur begünstigt, sondern rational erzwungen. Die fehlende TOTEX-Steuerung unterminiert damit einen zentralen Anspruch von Target Value Design im öffentlichen Sektor.

\subsection{Lösungsansatz: Die Operationalisierung des Wertes (Value Management)}
\label{sec:5.2.4}

Die vorangegangene Analyse legt nahe, dass die Anwendung von Target Value Design im öffentlichen Sektor auf eine strukturelle Asymmetrie trifft: Während die Variable \enquote{Kosten} (Euro) exakt definiert ist, bleibt die Variable \enquote{Wert} (Nutzen) häufig interpretationsfähig. Es fehlt die natürliche Markt-Regulierung durch den ROI (siehe Abschnitt \ref{sec:5.2.1}) sowie eine ganzheitliche Kostenklammer über den Lebenszyklus (siehe Abschnitt \ref{sec:5.2.3}).
Um das beschriebene Wert-Dilemma aufzulösen, bedarf es daher keiner singulären Maßnahme, sondern eines methodischen Baukastens, der an den identifizierten Bruchstellen ansetzt.

\minisec{Hierarchische Wert-Strukturierung}
Ein möglicher Ansatz, um das Vakuum des fehlenden Business Case zu füllen, besteht in der methodischen \textbf{Härtung der Wertziele}. Miron et al. (2015) schlagen hierfür vor, den Wertbegriff explizit zu strukturieren. Anstatt abstrakter Projektziele (\enquote{Hohe Nutzerzufriedenheit}) müssten konkrete **Wert-Attribute** (Value Attributes) definiert werden, die messbar oder zumindest eindeutig bewertbar sind (z.\,B. Nachhallzeiten, Reinigungsintervalle oder Flexibilitätsgrade).
Für den deutschen Kontext wäre hierbei eine stärkere Integration der **Nutzwertanalyse** in den TVD-Prozess denkbar. Neben den \enquote{Target Cost} (Zielkosten) könnte ein \enquote{Target Value Score} (Ziel-Nutzwert) als korrespondierende Steuerungsgröße etabliert werden. Dies würde dem Projektteam ermöglichen, qualitative Anforderungen verbindlicher gegen Kosteneinsparungen abzuwägen \autocite[vgl.]{miron_target_2015}.

\minisec{Erweiterte Entscheidungsmethodik (CBA)}
Um der Falle der Kostenzentrierung (\ref{sec:5.2.2}) entgegenzuwirken, erscheint zudem eine Anpassung der Entscheidungslogik im Designprozess sinnvoll. Statt Planungsalternativen primär anhand der Investitionskosten zu vergleichen, könnten Verfahren wie **Choosing by Advantages (CBA)** herangezogen werden.
Hierbei wird eine kostenintensivere Lösung (z.\,B. eine langlebigere Fassade) nicht zwangsläufig verworfen, sondern ihr \enquote{Vorteil} wird explizit ins Verhältnis zu den Mehrkosten gesetzt. Dies zwingt den öffentlichen Auftraggeber, seine Präferenzen offenzulegen, und verhindert das automatische \enquote{Descoping} von Qualität zugunsten kurzfristiger Einsparungen.

\minisec{Integration von Lebenszyklus-Zielen (Sustainable TVD)}
Zur Überbrückung der TOTEX-Lücke (\ref{sec:5.2.3}) ist eine Erweiterung der Zielgrößen notwendig. Russell-Smith et al. (2015) sowie Olender und Rosen (2023) plädieren für ein **Sustainable Target Value Design**, bei dem neben dem Kostenziel ein verbindliches Budget für Lebenszyklusfaktoren (z.\,B. Energiebedarf oder CO\textsubscript{2}-Äquivalente) festgelegt wird.
Dass dies praktisch umsetzbar ist, belegen Silveira und Alves (2018) anhand von Fallstudien: Sie konnten nachweisen, dass TVD-Praktiken signifikant dazu beitragen, strenge Nachhaltigkeitsstandards kostenneutral umzusetzen, sofern sich das Team von der reinen Betrachtung der Anschaffungskosten (\enquote{First Cost}) löst und kollaborative Methoden wie das \textit{Set-Based Design} nutzt.
Für die deutsche Verwaltung würde dies bedeuten, die Wirtschaftlichkeitsuntersuchung nicht als statisches Dokument, sondern als dynamische Steuerungsgröße in den TVD-Prozess zu integrieren \autocite[vgl.]{russell_smith_sustainable_2015, olender_rosen_2023, silveira_tvd_sustainable_2018}.

\clearpage


\section{Cluster 3: Das Struktur-Dilemma (Anreize und Kultur)}
\label{sec:cluster3}

Selbst wenn es gelingt, ein kompetentes Team vertraglich zu binden (Cluster 1) und wertorientierte Ziele zu definieren (Cluster 2), trifft die operative Umsetzung des Target Value Design auf den Widerstand der gewachsenen Branchenstruktur. Das dritte Diskussions-Cluster widmet sich daher den \enquote{weichen} Faktoren der Zusammenarbeit – Anreizsystemen, Haftungsregimen und der Projektkultur – die durch \enquote{harte} gesetzliche Rahmenbedingungen (HOAI, VOB) zementiert werden.

Die Analyse zeigt, dass das deutsche Bauwesen als \enquote{adversariales System} (gegensätzlich) konstruiert ist. Die Akteure optimieren ihren eigenen Vorteil oft auf Kosten des Gesamtprojekts. TVD erfordert hingegen ein \enquote{kollaboratives System}. Dieser Kulturwandel scheitert in der Praxis nicht am Unwillen der Beteiligten, sondern an systemischen Fehlanreizen: Wer sich im aktuellen deutschen Rechtsrahmen kooperativ verhält, geht oft ein unkalkulierbares ökonomisches oder juristisches Risiko ein.

\subsection{Das Honorar-Paradoxon: Ökonomische Bestrafung von Effizienz}
\label{sec:5.3.1}

Ein zentraler Pfeiler des TVD ist das Prinzip des \enquote{Frontloading}: Durch einen erhöhten Planungsaufwand in frühen Phasen sollen teure Änderungen in der Ausführung vermieden und die Baukosten insgesamt gesenkt werden. Betrachtet man die Honorarstruktur der HOAI (Honorarordnung für Architekten und Ingenieure), offenbart sich eine diametrale Fehlsteuerung, die in der Literatur als \enquote{Fee Paradox} bezeichnet wird.

\minisec{Die Kopplung an die anrechenbaren Kosten}
Gemäß § 33 HOAI bemisst sich das Honorar der Planer maßgeblich nach den \enquote{anrechenbaren Kosten} des Bauwerks. Diese Mechanik war ursprünglich gedacht, um Planer vor Preisdumping zu schützen, wirkt im TVD-Kontext jedoch kontraproduktiv.
Gelingt es einem Architekten durch kreatives TVD (z.\,B. intelligente Rasterung, Materialsubstitution), die Baukosten von 10 Mio. € auf 8 Mio. € zu senken, sinkt seine eigene Vergütungsgrundlage proportional. Er \enquote{erkauft} den Erfolg des Bauherrn mit einem eigenen Umsatzverlust.

\minisec{Das Scheren-Problem (Aufwand vs. Ertrag)}
Gleichzeitig steigt im TVD der Aufwand für den Planer massiv an: Die Teilnahme an wöchentlichen Cluster-Meetings, die iterative Erstellung von Varianten und die Koordination mit ausführenden Firmen erfordern deutlich mehr Stunden als ein klassischer linearer Entwurf.
Wie Abbildung \ref{fig:honorar_paradoxon} verdeutlicht, öffnet sich eine Schere: Während der Aufwand steigt (durch TVD-Prozesse), sinkt der Ertrag (durch Kostenreduktion). Ein rational handelnder Planer hat im HOAI-System somit keinen extrinsischen Anreiz, TVD ernsthaft zu betreiben. Ohne alternative Vergütungsmodelle (z.\,B. \enquote{Cost+Fee} mit Bonus-Malus-Regelung), die im öffentlichen Recht schwer durchsetzbar sind, bleibt TVD eine idealistische Forderung, die an der ökonomischen Realität der Planungsbüros scheitert \autocite[vgl.]{kalusche_projektmanagement_2020}.

\begin{figure}[ht]
    \centering
    \begin{tikzpicture}[scale=1.0, >=Stealth]
        % Achsen
        \draw[->] (0,0) -- (8,0) node[right] {Projekterfolg (Kostensenkung)};
        \draw[->] (0,0) -- (0,5) node[above] {Finanzieller Ertrag / Aufwand};

        % Kurve 1: TVD-Aufwand (steigt)
        \draw[red, thick] (0,1.5) -- (7,4.5) node[right] {Planungsaufwand (TVD)};
        
        % Kurve 2: HOAI-Honorar (fällt)
        \draw[blue, thick] (0,4) -- (7,1) node[right] {Honorar (HOAI-Basis)};

        % Fläche dazwischen (Verlustzone)
        \fill[gray!20] (2.5, 2.5) -- (7, 4.5) -- (7, 1) -- cycle;
        \node at (5.5, 2.8) [align=center, font=\small\bfseries] {Ökonomische\\Fehlanreiz-Zone};

        % Beschriftung Startpunkt
        \node[align=center, below] at (0,-0.2) {Start Budget};
        \node[align=center, below] at (7,-0.2) {Ziel erreicht\\(-20\% Kosten)};

    \end{tikzpicture}
    \caption{Das Honorar-Paradoxon: Divergenz von Aufwand und Vergütung bei Kostenoptimierung nach HOAI}
    \label{fig:honorar_paradoxon}
\end{figure}

\subsection{Die Haftungsmauer: Individualschuld vs. Kollektiventscheidung}
\label{sec:5.3.2}

Ein weiteres strukturelles Hindernis liegt im deutschen Haftungsrecht. TVD propagiert eine \enquote{No-Blame}-Kultur, in der Entscheidungen im \textit{Big Room} gemeinsam getroffen werden (\enquote{Best for Project}). Kommt es jedoch zu einem Bauschaden, kennt das BGB und die VOB fast ausschließlich die Individualhaftung.

\minisec{Versicherungsrechtliche Inkompatibilität}
Berufshaftpflichtversicherungen decken Schäden nur dann ab, wenn ein klares Verschulden des Versicherungsnehmers nachweisbar ist. In einem integrierten TVD-Team, in dem der Rohbauer dem Architekten in die Statik redet und der Architekt dem TGA-Planer in die Dimensionierung, verschwimmen die Verantwortlichkeiten.
Im Schadensfall führt dies zu massiven Regressstreitigkeiten (\enquote{Wer hat die Entscheidung im Meeting protokolliert?}). Aus Angst vor diesem Szenario neigen deutsche Akteure dazu, trotz Lippenbekenntnissen zur Kooperation, im Hintergrund eine \enquote{Cover Your Ass}-Strategie zu fahren. Entscheidungen werden verzögert, Bedenkenanmeldungen präventiv verschickt und Protokolle juristisch abgesichert.
Diese \enquote{Defensiv-Planung} lähmt die Agilität des TVD-Prozesses. Solange keine projektbezogenen Mehrparteien-Versicherungen (Project Insurance) etabliert sind, die das gesamte Team als Einheit versichern, bleibt die Angst vor der persönlichen Haftung der stärkste Gegner der offenen Kooperation \autocite[vgl.]{breyer_partnerschaftliche_2023}.

\subsection{Diskontinuität durch Phasentrennung und Schnittstellen}
\label{sec:5.3.3}

Der ideale TVD-Prozess ist ein \textit{Continuous Flow}: Das Wissen wächst stetig, der Detaillierungsgrad nimmt zu. Das deutsche Leistungsphasenmodell (LPH 1--9) organisiert das Bauen hingegen als \enquote{Wasserfall-Modell} mit harten Zäsuren. Besonders die Schnittstelle zwischen Genehmigungsplanung (LPH 4) und Ausführungsplanung (LPH 5) wirkt als Bruchstelle.

Oftmals endet mit der LPH 4 die Beauftragung des Entwurfsarchitekten oder sie wird aus Kostengründen an ein anderes Büro oder einen Generalunternehmer übergeben. Im TVD-Kontext ist dies fatal: Das gesamte implizite Wissen darüber, \textit{warum} bestimmte Entscheidungen in der Zielkostenphase so getroffen wurden (die \enquote{Design Intent}), geht beim Übergabeprozess verloren. Der neue Akteur beginnt oft, den Entwurf zu \enquote{optimieren} (d.h. zu verbilligen), ohne die funktionalen Hintergründe zu kennen. Die im TVD mühsam erarbeitete Wert-Balance wird durch die fragmentierte Prozesskette wieder zerstört.

\subsection{Fehlendes Organizational Learning und Datenverlust}
\label{sec:5.3.4}

Abschließend verhindert die Struktur der deutschen Bauwirtschaft die für TVD essenzielle Lernkurve. Ouma et al. (2025) definieren \enquote{Organizational Learning} als die höchste Reifegradstufe: Projektdaten fließen zurück in eine Datenbank, um die Zielkostenschätzung des nächsten Projekts zu präzisieren.

Da Bauprojekte in Deutschland als \enquote{Unikate} organisiert sind und Teams sich nach Abschluss (§ 12 VOB/B Abnahme) sofort auflösen (\textit{Ad-hoc-Teams}), findet dieser Rückfluss nicht statt. Es gibt keinen kommerziellen Anreiz für einen Bauunternehmer, seine echten Kalkulationsdaten (\enquote{Urkalkulation}) mit dem Bauherrn zu teilen, da diese sein Betriebsgeheimnis und seinen Wettbewerbsvorteil für die nächste Ausschreibung darstellen.
Das System bleibt somit \enquote{gedächtnislos}. Jedes TVD-Projekt beginnt wieder bei Null, anstatt auf validierten Kennwerten aufzubauen. Ohne diesen Datenpool bleibt die Zielkostenfindung (Phase 0) ein Ratespiel, was die methodische Integrität des gesamten Ansatzes schwächt.

\clearpage