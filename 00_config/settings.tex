% ===== ZEILENABSTAND =====
\onehalfspacing                    % 1,5-facher Zeilenabstand (Standard für Masterarbeiten)

% ===== KOPF- UND FUSSZEILEN =====
\pagestyle{scrheadings}
\clearpairofpagestyles
\ohead{\headmark}                  % Kapitelname rechts oben
\ofoot{\pagemark}                  % Seitenzahl rechts unten

%% ===== DEUTSCHE BESCHRIFTUNGEN FÜR QUERVERWEISE =====
\crefname{figure}{Abbildung}{Abbildungen}
\crefname{table}{Tabelle}{Tabellen}
\crefname{equation}{Gleichung}{Gleichungen}
\crefname{chapter}{Kapitel}{Kapitel}
\crefname{section}{Abschnitt}{Abschnitte}
\crefname{subsection}{Unterabschnitt}{Unterabschnitte}

% ===== EINHEITEN-KONFIGURATION (DEUTSCH) =====
\sisetup{
    locale=DE,                     % Deutsche Notation
    per-mode=fraction,             % Bruchdarstellung für Einheiten (m/s)
    output-decimal-marker={,},     % Komma statt Punkt
    group-separator={.},           % Tausender-Punkte (1.000 statt 1,000)
    group-minimum-digits=4         % Gruppierung ab 4 Stellen
}

% ===== FARBEN DEFINIEREN =====
\definecolor{tu-green}{RGB}{99,154,58}      % TU Dortmund Grün
\definecolor{tu-blue}{RGB}{0,69,136}        % TU Dortmund Blau  
\definecolor{highlight}{RGB}{220,50,50}     % Rot für Markierungen
\definecolor{gray-light}{RGB}{245,245,245}  % Hellgrau für Tabellen

% ===== GRAFIK-EINSTELLUNGEN =====
\graphicspath{{05_figures/}}                % Standardpfad für alle Bilder

% ===== TIKZ/PGFPLOTS KONFIGURATION =====
\usetikzlibrary{positioning,shapes,arrows,calc}
\pgfplotsset{compat=1.18}

% Standardfarben für Diagramme
\pgfplotscreateplotcyclelist{tu-colors}{
    {tu-green, mark=*},
    {tu-blue, mark=square*},
    {highlight, mark=triangle*},
}

% ===== TABELLEN-STYLING =====
% Professionelle Tabellen mit booktabs
\setlength{\heavyrulewidth}{1.2pt}
\setlength{\lightrulewidth}{0.6pt}

% ===== LISTEN-KONFIGURATION =====
\setlist[itemize]{nosep, left=0pt}          % Kompakte Aufzählungen
\setlist[enumerate]{nosep, left=0pt}        % Kompakte Nummerierungen

% ===== ABKÜRZUNGSVERZEICHNIS-STIL =====
\renewcommand*{\acsfont}[1]{\textsc{#1}}    % Abkürzungen in Kapitälchen

% ===== ZUSÄTZLICHE HYPERLINK-EINSTELLUNGEN =====
% (Grundeinstellungen bereits in packages.tex definiert)
\hypersetup{
    bookmarksnumbered=true,        % Nummerierte Bookmarks im PDF
    bookmarksopen=true,            % Bookmarks aufgeklappt
    pdfstartview=FitH             % PDF startet mit Seitenbreite
}

% ===== QUELLENANGABEN KONFIGURATION =====
% Anpassungen für deutsches Literaturverzeichnis
\DefineBibliographyStrings{ngerman}{
    andothers = {et\,al\adddot},
    pages = {S\adddot},
    page = {S\adddot}
}

% ===== CAPTION-STYLING =====
\captionsetup{
    format=plain,
    font=small,
    labelfont=bf,
    textfont=it,
    justification=centering,
    singlelinecheck=false
}

% ===== ZUSÄTZLICHE ABSTÄNDE =====
\setlength{\parindent}{0pt}        % Keine Einrückung bei Absätzen
\setlength{\parskip}{6pt}          % Abstand zwischen Absätzen