% ===== ENCODING UND SPRACHE =====
\usepackage[utf8]{inputenc}        % UTF-8 für Umlaute
\usepackage[T1]{fontenc}           % Bessere Fontdarstellung
\usepackage[ngerman]{babel}        % Deutsche Silbentrennung, Datumsformat

\usepackage{lmodern}               % Moderne Latin Modern Schrift
\usepackage{microtype}

\usepackage{soul}         % farbiges markieren von Text
\sethlcolor{yellow!35}    % Farbe

% ===== SEITENLAYOUT =====
\usepackage[left=3cm,right=2.5cm,top=3cm,bottom=3cm]{geometry}
\usepackage{setspace}              % Zeilenabstand-Kontrolle
\usepackage[headsepline]{scrlayer-scrpage}  % Kopf-/Fußzeilen

% ===== MATHEMATIK UND EINHEITEN =====
\usepackage{amsmath,amssymb,amsthm}
\usepackage{mathtools}             % Erweiterte Mathe-Tools
\usepackage{siunitx}               % Korrekte Einheiten (€, m², kg, etc.)

% ===== TABELLEN =====
\usepackage{booktabs}              % Professionelle Tabellen (\toprule, \midrule, \bottomrule)
\usepackage{longtable}             % Mehrseitige Tabellen
\usepackage{array}                 % Erweiterte Tabellenfunktionen
\usepackage{multirow}              % Mehrzeilige Tabellenzellen
\usepackage{tabularx}              % Automatische Spaltenbreiten

% ===== GRAFIKEN UND ABBILDUNGEN =====
\usepackage{graphicx}              % Bilder einbinden (PDF, PNG, JPG)
\usepackage{subcaption}            % Unterabbildungen (a), (b), (c)
\usepackage{tikz}                  % Technische Zeichnungen und Diagramme
\usepackage{pgfplots}              % Professionelle Plots und Diagramme
\usepackage{xcolor}                % Farbdefinitionen

% ===== LITERATURVERZEICHNIS =====
\usepackage[
    backend=biber,
    %style=authoryear-icomp,  
    style=numeric, 
    maxbibnames=4,
    maxcitenames=2,
    autocite=footnote,
    date=year,
    giveninits=true,         % Vornamen Abkürzen
    dateuncertain=true,
    eprint=false,
    url=false,
    doi=false,
    isbn=false,
    sorting=nyt
]{biblatex}
\addbibresource{08_bibliography/references.bib}

% ===== VERWEISE UND LINKS =====
\usepackage[hidelinks]{hyperref}   % Anklickbare Links (unsichtbar gedruckt)
\usepackage{cleveref}              % Intelligente Querverweise (\cref{})

% ===== LISTEN UND AUFZÄHLUNGEN =====
\usepackage{enumitem}              % Bessere Kontrolle über Listen

% ===== ANFÜHRUNGSZEICHEN UND ZITATE =====
\usepackage{csquotes}              % Korrekte deutsche Anführungszeichen

% ===== ABKÜRZUNGSVERZEICHNIS =====
\usepackage{acronym}               % Automatisches Abkürzungsverzeichnis

% ===== SPEZIELLE BAUINGENIEUR-PAKETE =====
\usepackage{textcomp}              % Zusätzliche Symbole
\usepackage{gensymb}               % Grad-Symbol und andere

\usepackage{tikz}
\usetikzlibrary{shapes.geometric, arrows.meta, positioning}

% BPMN Stil-Definitionen (in deine Präambel oder vor das Diagramm)
\tikzset{
    % Start Event (Kreis)
    startevent/.style={circle, draw=black, thick, minimum size=1cm, fill=white},
    % End Event (dicker Kreis)
    endevent/.style={circle, draw=black, line width=3pt, minimum size=1cm, fill=white},
    % Task (abgerundetes Rechteck)
    task/.style={rectangle, draw=black, thick, rounded corners=3pt, minimum width=3cm, minimum height=1cm, align=center, fill=white},
    % Gateway (Raute)
    gateway/.style={diamond, draw=black, thick, minimum size=1.2cm, fill=white, inner sep=0pt},
    % Subprocess
    subprocess/.style={rectangle, draw=black, thick, rounded corners=3pt, minimum width=3cm, minimum height=1cm, align=center, fill=white},
    % Pfeile
    flow/.style={-{Latex[length=3mm]}, thick},
    % Pool
    pool/.style={rectangle, draw=black, very thick, minimum width=15cm, minimum height=8cm},
    % Lane
    lane/.style={rectangle, draw=black, thick}
}

% Platzierung von Diagrammen und Tabellen
\usepackage{placeins}

\usepackage{pdfpages}              % PDFs einbinden