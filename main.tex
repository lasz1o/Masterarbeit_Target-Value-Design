\documentclass[
12pt,
a4paper,
oneside,
ngerman,
]{scrbook}

% === KONFIGURATION LADEN (richtige Reihenfolge!) ===
% ===== ENCODING UND SPRACHE =====
\usepackage[utf8]{inputenc}        % UTF-8 für Umlaute
\usepackage[T1]{fontenc}           % Bessere Fontdarstellung
\usepackage[ngerman]{babel}        % Deutsche Silbentrennung, Datumsformat

\usepackage{lmodern}               % Moderne Latin Modern Schrift
\usepackage{microtype}

\usepackage{soul}         % farbiges markieren von Text
\sethlcolor{yellow!35}    % Farbe

% ===== SEITENLAYOUT =====
\usepackage[left=3cm,right=2.5cm,top=3cm,bottom=3cm]{geometry}
\usepackage{setspace}              % Zeilenabstand-Kontrolle
\usepackage[headsepline]{scrlayer-scrpage}  % Kopf-/Fußzeilen

% ===== MATHEMATIK UND EINHEITEN =====
\usepackage{amsmath,amssymb,amsthm}
\usepackage{mathtools}             % Erweiterte Mathe-Tools
\usepackage{siunitx}               % Korrekte Einheiten (€, m², kg, etc.)

% ===== TABELLEN =====
\usepackage{booktabs}              % Professionelle Tabellen (\toprule, \midrule, \bottomrule)
\usepackage{longtable}             % Mehrseitige Tabellen
\usepackage{array}                 % Erweiterte Tabellenfunktionen
\usepackage{multirow}              % Mehrzeilige Tabellenzellen
\usepackage{tabularx}              % Automatische Spaltenbreiten

% ===== GRAFIKEN UND ABBILDUNGEN =====
\usepackage{graphicx}              % Bilder einbinden (PDF, PNG, JPG)
\usepackage{subcaption}            % Unterabbildungen (a), (b), (c)
\usepackage{tikz}                  % Technische Zeichnungen und Diagramme
\usepackage{pgfplots}              % Professionelle Plots und Diagramme
\usepackage{xcolor}                % Farbdefinitionen

% ===== LITERATURVERZEICHNIS =====
\usepackage[
    backend=biber,
    %style=authoryear-icomp,  
    style=numeric, 
    maxbibnames=4,
    maxcitenames=2,
    autocite=footnote,
    date=year,
    giveninits=true,         % Vornamen Abkürzen
    dateuncertain=true,
    eprint=false,
    url=false,
    doi=false,
    isbn=false,
    sorting=nyt
]{biblatex}
\addbibresource{08_bibliography/references.bib}

% ===== VERWEISE UND LINKS =====
\usepackage[hidelinks]{hyperref}   % Anklickbare Links (unsichtbar gedruckt)
\usepackage{cleveref}              % Intelligente Querverweise (\cref{})

% ===== LISTEN UND AUFZÄHLUNGEN =====
\usepackage{enumitem}              % Bessere Kontrolle über Listen

% ===== ANFÜHRUNGSZEICHEN UND ZITATE =====
\usepackage{csquotes}              % Korrekte deutsche Anführungszeichen

% ===== ABKÜRZUNGSVERZEICHNIS =====
\usepackage{acronym}               % Automatisches Abkürzungsverzeichnis

% ===== SPEZIELLE BAUINGENIEUR-PAKETE =====
\usepackage{textcomp}              % Zusätzliche Symbole
\usepackage{gensymb}               % Grad-Symbol und andere

\usepackage{tikz}
\usetikzlibrary{shapes.geometric, arrows.meta, positioning}

% BPMN Stil-Definitionen (in deine Präambel oder vor das Diagramm)
\tikzset{
    % Start Event (Kreis)
    startevent/.style={circle, draw=black, thick, minimum size=1cm, fill=white},
    % End Event (dicker Kreis)
    endevent/.style={circle, draw=black, line width=3pt, minimum size=1cm, fill=white},
    % Task (abgerundetes Rechteck)
    task/.style={rectangle, draw=black, thick, rounded corners=3pt, minimum width=3cm, minimum height=1cm, align=center, fill=white},
    % Gateway (Raute)
    gateway/.style={diamond, draw=black, thick, minimum size=1.2cm, fill=white, inner sep=0pt},
    % Subprocess
    subprocess/.style={rectangle, draw=black, thick, rounded corners=3pt, minimum width=3cm, minimum height=1cm, align=center, fill=white},
    % Pfeile
    flow/.style={-{Latex[length=3mm]}, thick},
    % Pool
    pool/.style={rectangle, draw=black, very thick, minimum width=15cm, minimum height=8cm},
    % Lane
    lane/.style={rectangle, draw=black, thick}
}

% Platzierung von Diagrammen und Tabellen
\usepackage{placeins}

\usepackage{pdfpages}              % PDFs einbinden        % Zuerst Pakete
% ===== ZEILENABSTAND =====
\onehalfspacing                    % 1,5-facher Zeilenabstand (Standard für Masterarbeiten)

% ===== KAPITEL-ABSTÄNDE ANPASSEN =====
\renewcommand*{\chapterheadstartvskip}{\vspace*{-\topskip}}

% ===== KAPITEL-ABSTÄNDE ANPASSEN =====
\renewcommand*{\chapterheadstartvskip}{\vspace*{-\topskip}}

% ===== KOPF- UND FUSSZEILEN =====
\pagestyle{scrheadings}
\clearpairofpagestyles

% Normale Seiten
\ohead{\headmark}                  % Kapitelname rechts oben
\ofoot[\pagemark]{\pagemark}      % Seitenzahl rechts unten (WICHTIG: auch bei plain!)

%% ===== DEUTSCHE BESCHRIFTUNGEN FÜR QUERVERWEISE =====
\crefname{figure}{Abbildung}{Abbildungen}
\crefname{table}{Tabelle}{Tabellen}
\crefname{equation}{Gleichung}{Gleichungen}
\crefname{chapter}{Kapitel}{Kapitel}
\crefname{section}{Abschnitt}{Abschnitte}
\crefname{subsection}{Unterabschnitt}{Unterabschnitte}

% Deutsche Übersetzungen für Zitate
\DefineBibliographyStrings{ngerman}{ 
   andothers = {et\addabbrvspace al\adddot},
   andmore   = {et\addabbrvspace al\adddot},
}

% ===== FARBEN DEFINIEREN =====
\definecolor{tu-green}{RGB}{99,154,58}      % TU Dortmund Grün
\definecolor{tu-blue}{RGB}{0,69,136}        % TU Dortmund Blau  
\definecolor{highlight}{RGB}{220,50,50}     % Rot für Markierungen
\definecolor{gray-light}{RGB}{245,245,245}  % Hellgrau für Tabellen

% ===== GRAFIK-EINSTELLUNGEN =====
\graphicspath{{05_figures/}}                % Standardpfad für alle Bilder

% ===== TIKZ/PGFPLOTS KONFIGURATION =====
\usetikzlibrary{positioning,shapes,arrows,calc}
\pgfplotsset{compat=1.18}

% Standardfarben für Diagramme
\pgfplotscreateplotcyclelist{tu-colors}{
    {tu-green, mark=*},
    {tu-blue, mark=square*},
    {highlight, mark=triangle*},
}

% ===== TABELLEN-STYLING =====
% Professionelle Tabellen mit booktabs
\setlength{\heavyrulewidth}{1.2pt}
\setlength{\lightrulewidth}{0.6pt}

% ===== LISTEN-KONFIGURATION =====
\setlist[itemize]{nosep, left=0pt}          % Kompakte Aufzählungen
\setlist[enumerate]{nosep, left=0pt}        % Kompakte Nummerierungen

% ===== ABKÜRZUNGSVERZEICHNIS-STIL =====
\renewcommand*{\acsfont}[1]{\textsc{#1}}    % Abkürzungen in Kapitälchen

% ===== ZUSÄTZLICHE HYPERLINK-EINSTELLUNGEN =====
% (Grundeinstellungen bereits in packages.tex definiert)
\hypersetup{
    bookmarksnumbered=true,        % Nummerierte Bookmarks im PDF
    bookmarksopen=true,            % Bookmarks aufgeklappt
    pdfstartview=FitH             % PDF startet mit Seitenbreite
}

% ===== QUELLENANGABEN KONFIGURATION =====
% Anpassungen für deutsches Literaturverzeichnis
\DefineBibliographyStrings{ngerman}{
    andothers = {et\,al\adddot},
    pages = {S\adddot},
    page = {S\adddot}
}

% ===== CAPTION-STYLING =====
\captionsetup{
    format=plain,
    font=footnotesize,
    labelfont=bf,
    textfont=it,
    justification=raggedright,
    singlelinecheck=false
}

% ===== ZUSÄTZLICHE ABSTÄNDE =====
\setlength{\parindent}{0pt}        % Keine Einrückung bei Absätzen
\setlength{\parskip}{6pt}          % Abstand zwischen Absätzen

% ===== FUSSNOTENEINSTELLUNGEN =====
% Fußnoten durchlaufend über Kapitel hinweg nummerieren
\counterwithout{footnote}{chapter}

% Schriftgröße der Fußnoten anpassen
\setkomafont{footnote}{\footnotesize}  % oder \scriptsize, \tiny, etc.        % Dann Einstellungen  
% ===== GRUNDDATEN DER ARBEIT =====
\title{Target Value Design im Bauwesen -- Strukturierte Analyse und prozessorientierte 
Darstellung im Kontext der deutschen Baupraxis}

\author{Lasse Krisztian}
\date{\today}

% ===== PERSÖNLICHE DATEN =====
\newcommand{\matrikelnummer}{184406}
\newcommand{\emailadresse}{lasse.krisztian@tu-dortmund.de}

% ===== UNIVERSITÄTSDATEN =====
\newcommand{\universitaet}{Technische Universität Dortmund}
\newcommand{\fakultaet}{Fakultät für Architektur und Bauingenieurwesen}
\newcommand{\lehrstuhl}{Lehrstuhl für Baubetrieb und Bauprozessmanagement}
\newcommand{\studiengang}{Bauingenieurwesen Vertiefung Baubetrieb}
\newcommand{\abschluss}{Master of Science (M.Sc.)}

% ===== BETREUUNG =====
\newcommand{\professor}{Univ.-Prof. Dr.-Ing. Mike Gralla}
\newcommand{\erstbetreuer}{Franziska Blennemann, M.Sc.}
\newcommand{\zweitbetreuer}{Dipl.-Ing. Tim Brandt}

% ===== TERMINE =====
\newcommand{\abgabedatum}{30. November 2025}
\newcommand{\bearbeitungszeitraum}{Juli 2025 -- November 2025}

% ===== THEMA UND SCHWERPUNKTE =====
\newcommand{\themabereich}{Target Value Design}
\newcommand{\untertitel}{Eine Analyse moderner Kostensteuerungsmethoden im Bauwesen}
\newcommand{\schlagwoerter}{Target Value Design, Lean Construction, Integrated Project Delivery, Bauwirtschaft}

% ===== PDF-METADATEN =====
\hypersetup{
    pdftitle={Target Value Design im Bauingenieurwesen},
    pdfauthor={Lasse Krisztian},
    pdfsubject={Masterarbeit Bauingenieurwesen},
    pdfkeywords={Target Value Design, Lean Construction, Bauwirtschaft, IPD},
    pdfcreator={LaTeX with Overleaf},
    pdfproducer={LaTeX}
}        % Zuletzt Metadaten

\begin{document}

\pagenumbering{gobble}  % Schaltet Seitenzahlen komplett aus
\begin{titlepage}
    \begin{center} % Zentriert den gesamten Inhalt horizontal
    
    % --- BLOCK 1: KOPFBEREICH ---
    \includegraphics[width=0.7\textwidth]{tud_logo_cmyk}\\
    {\large \fakultaet}\\[0.2cm]
    {\normalsize \lehrstuhl}
    
    \vspace{1.5cm} 
    
    % --- BLOCK 2: TITEL DER ARBEIT ---
    \makeatletter
    \parbox{0.9\textwidth}{\centering\Large\bfseries\@title}
    \makeatother
    
    \vspace{1.5cm} % Fester Abstand unter dem Titel
    
    % --- BLOCK 3: ART DER ARBEIT ---
    {\large Masterarbeit}\\
    {\normalsize zur Erlangung des akademischen Grades}\\
    {\normalsize \textbf{\abschluss}}\\
    {\normalsize im Studiengang \textbf{\studiengang}}
    
      \vspace{0.5cm} % Fester Abstand unter dem Titel
    
    % --- BLOCK 4: VERFASSER, BETREUER, DATUM ---
    \rule{0.8\textwidth}{0.1pt} % Feine horizontale Linie
    \vspace{1cm}
    
    \makeatletter
   
    \begin{tabular}{r@{\hspace{1.5em}}l}
        \textbf{Verfasser:} & \@author \\
        \textbf{Matrikelnummer:} & \matrikelnummer \\
        \textbf{E-Mail:} & \emailadresse \\[0.5cm]
        
        \textbf{Erstprüfer:} & \professor \\
        \textbf{Erstbetreuerin:} & \erstbetreuer \\
        \textbf{Zweitbetreuer:} & \zweitbetreuer \\[0.5cm]
        
        \textbf{Bearbeitungszeitraum:} & \bearbeitungszeitraum \\
        \textbf{Abgabedatum:} & \abgabedatum \\
    \end{tabular}
    \makeatother
    
    \vspace{0.5cm}
   {\normalsize Dortmund, den \today}

    \end{center}
\end{titlepage}

% Leere Rückseite für doppelseitigen Druck
\thispagestyle{empty}
\cleardoublepage



% === VORSPANN (römische Seitenzahlen) ===
\frontmatter
\tableofcontents
\listoffigures
\addcontentsline{toc}{chapter}{Abbildungsverzeichnis}
\thispagestyle{plain}
\cleardoublepage
\thispagestyle{plain}  % Zeigt Seitennummer im Abbildungsverzeichnis
% \listoftables
% \addcontentsline{toc}{chapter}{Tabellenverzeichnis}
% \thispagestyle{plain}
% \cleardoublepage
\chapter{Abkürzungsverzeichnis}

% Short-first nur bei der ersten Stelle; danach gilt "benutzt".
\newcommand{\acsf}[1]{\acs{#1} (\acl{#1})\acused{#1}}

\begin{acronym}[PRISMA]  % Längste Abkürzung für Spaltenbreite
    \acro{AHO}{Ausschuss der Verbände und Kammern der Ingenieure und Architekten für die Honorarordnung}
    \acro{BIM}{Building Information Modeling}
    \acro{CoS}{Conditions of Satisfaction}
    \acro{IPA}{Integrierte Projektabwicklung}
    \acro{IPD}{Integrated Project Delivery}
    \acro{LC}{Lean Construction}
    \acro{LCI}{Lean Construction Institute}
    \acro{HOAI}{Honorarordnung für Architekten und Ingenieure}
    \acro{PA}{Project Alliancing}
    \acro{PRISMA}{Preferred Reporting Items for Systematic Reviews and Meta-Analyses}
    \acro{TVD}{Target Value Design}
    \acro{VOB}{Vergabe- und Vertragsordnung für Bauleistungen}
\end{acronym}

% Seitenzahl des Vorspanns speichern
\newcounter{savepage}
\setcounter{savepage}{\value{page}}

% === HAUPTTEIL (arabische Seitenzahlen) ===
\mainmatter
\chapter{Einleitung}
\label{ch:einleitung}
\section{Motivation und Zielsetzung der Arbeit}
\label{ch:1.1}

\section{Forschungslücke und Forschungsfrage}
\label{ch:1.2}

\section{Abgrenzung der Arbeit}
\label{ch:1.3}

- erforderlich?

\section{Aufbau der Arbeit}
\label{ch: 1.4}

\chapter{Grundlagen}
\label{ch:grundlagen}
% Ziel: Ziel: Der Leser soll die Struktur, die Rollenverteilung und die prozessualen Schwachstellen der traditionellen Projektabwicklung verstehen. Dieses Kapitel etabliert das "Problem", für das Ihre Arbeit eine Lösung präsentiert.

\section{Konventionelle Projektabwicklung in der deutschen Baupraxis}
\label{sec: 2.1}
Ziel:

\clearpage

\section{Integrierte Projektabwicklungsmodelle als alternativer Ansatz}
\label{sec: 2.2}

% Ziel: Der Leser soll eine Einführung in die Entstehung und Entwicklung integrierter Projekabwicklung bekommen.

\subsection{Der Lean-Ansatz als Fundament integrierter Projektabwicklung}
\label{sec:2.2.1}

% Ziel: Vermittlung der Lean-Pilosophie als Ursprungsgedanke und der Weiterentwicklung über Lean Construction zu kollaborativen Projekabwicklungsmodellen

\textit{Einleitung - Hinführung zur Notwendigkeit neuer Ansätze nach der Kritik an der konventionellen Projektabwicklung (aus Abschnitt 2.1). Die Philosophie des Lean Managements bietet hierfür das Fundament.}

Die im vorangegangenen Abschnitt aufgezeigten Schwachstellen konventioneller, sequenzieller Projektabwicklung sind keineswegs ein neues Phänomen oder eine reine Eigenheit der Bauwirtschaft. Vergleichbare Herausforderungen zeigten sich bereits nach dem Zweiten Weltkrieg in der japanischen Automobilindustrie. Die dort vorherrschenden Produktionsmethoden waren von Ineffizienz und Verschwendung geprägt, was angesichts knapper Ressourcen nicht tragbar war. Als Reaktion darauf entwickelte der Ingenieur Taiichi Ohno bei Toyota einen grundlegend neuen Managementansatz, der heute als das Toyota-Produktionssystem bekannt ist und das Fundament der Lean-Philosophie bildet.

Ursprung des Ansatzes (Lean Management) Skizzieren
- Toyota-Produktionssystem\autocite[]{ohno_toyota-produktionssystem_2013} als Ursprung
- Kernideen und Prinzipien : Wertmaximierung, Verschwendungsreduktion, People First (checken), Kaizen

Transfer auf die Baubranche durch Lean Construction
- Landesweite Aufmerksamkeit in Japan nach der Ölkrise 1973
- Während sich die 
- Internationale Aufmerksamkeit durch 

Ursprung und Entwicklung von Projetct Alliancing, IPD und IPA bis heute

\subsection{Prinzipien und Merkmale der Integrierten Projektabwicklung (IPA)}
\label{sec:2.2.2}

\subsection{Internationale Erfolge und Status Quo in Deutschland}
\label{sec:2.2.3}

\clearpage

\section{Target Value Design (TVD) als Kernmethode}
\label{sec: 2.3}

%Ziel:

\subsection{TVD - Prozess oder Methode}
\label{sec: 2.3.1}

% =======================================================
% FÜR ABSCHNITT 2.3.1: TVD - Prozess oder Methode
% =======================================================
%https://service.tu-dortmund.de/group/intra/ausstattung-lokaler-arbeitsplatze
%   - Ziel: Eine präzise und trennscharfe Definition von TVD für die
%     vorliegende Arbeit herleiten. Es soll geklärt werden, ob TVD als
%     Prozess, als Methode oder als beides zu verstehen ist, um eine
%     eindeutige Basis für die nachfolgende Analyse zu schaffen.
%
%   - Einstieg: Feststellung, dass die Begriffe "Prozess" und "Methode"
%     in der Literatur zu TVD oft unscharf oder synonym verwendet werden.
%
%   - Begriffsdefinition "Methode": Ein systematisches, geplantes Vorgehen,
%     das auf Prinzipien, Regeln und Werkzeugen basiert, um ein Ziel zu erreichen.
%
%   - Begriffsdefinition "Prozess": Eine logische und zeitliche Abfolge von
%     miteinander verknüpften Aktivitäten zur Umwandlung eines Inputs in einen Output.
%
%   - Anwendung auf TVD:
%     - TVD ist mehr als ein reiner Prozess, da es eine bestimmte Denkweise
%       ("Target First"), Prinzipien (Kollaboration) und Werkzeuge voraussetzt.
%     - Diese übergeordnete Denkweise wird jedoch erst durch einen konkreten,
%       strukturierten Ablauf (Prozess) in der Praxis anwendbar.
%
%   - Fazit & Positionierung: Für diese Arbeit wird TVD als eine
%     *MANAGEMENT-METHODE* verstanden, die durch einen
%     _STRUKTURIERTEN PROZESS_ operationalisiert und umgesetzt wird.
%     Diese Definition legitimiert die nachfolgende "prozessorientierte
%     Darstellung" der Methode.
%
% =======================================================

Zur Hinleitung auf die in \cref{ch:methodik} beschriebene, methodische Vorgehensweise soll an dieser Stelle auf das in dieser Arbeit zu Grunde liegende Verständnis von \ac{TVD} als Prozess bzw. Methode eingegangen werden.\\
In der Literatur ist die Abgrenzung zwischen Prozess und Methode in der aktuellen Diskussion um TVD unscharf. Folgt man dem Ergebnis einer einfachen Stichwortsuche in einschlägigen wissenschaftlichen Datenbanken (z.B. reseachgate.net) so wird schnell deutlich, das in der überwiegenden Mehrheit der Literatur (ca. 70\%) Target Value Design als Prozess verstanden wird.
Glenn Ballard, der als einer Gründer des \ac{TVD} gilt, bezeichnet letzteres hingegen häufig als Managementansatz bzw. Management-Methode \autocite[]{}.

In der allgemeinen Literatur wird ein Prozess als 
\chapter{Methodik}
\label{ch:methodik}

\section{Abschnitt}
\label{sec: 3}
\chapter{Analyse und Modellierung}
\label{ch:analyse}
\section{Das Phasenmodell}
\label{sec:4.1}

Eine Betrachtung der einschlägigen Fachliteratur zeigt, dass Target Value Design (TVD) im Gegensatz zur deutschen \ac{HOAI} nur selten als lineares Prozessmodell mit starren Phasen dargestellt wird. Der Fokus der Forschung liegt primär auf den zugrundeliegenden Prinzipien, Rollenbildern und Entscheidungslogiken sowie auf der kulturellen Transformation der Zusammenarbeit.\autocite[vgl.]{zimina_target_2012}\, \autocite[vgl.]{ballard_target_2025}

Diese Abwesenheit eines einheitlichen und normierten Phasenmodells lässt sich auf die inhärente Logik des \ac{TVD}-Ansatzes zurückführen. Zum einen versteht sich \ac{TVD} in erster Linie als Managementphilosophie und nicht als formales Leistungsbild. Weiterhin wird \ac{TVD} in der Praxis meist als methodische Ergänzung in bestehende lokale Vertragsmodelle wie das US-amerikanische \ac{IPD} oder britische \acl{PA}-Modelle  \autocite[vgl.]{z.B. Fischer et al. 2017, Mossman oder LCI Guide}, wodurch es sich der jeweiligen Prozessstruktur anpasst, anstatt eine eigene Struktur vorzugeben. Zudem widerspricht eine rein lineare Abwicklung der zyklischen sowie iterativen Natur des \ac{TVD}-Prozesses, der maßgeblich von Rückkopplungsschleifen bestimmt wird.

Um im Rahmen dieser Arbeit dennoch untersuchen zu können, wie kompatibel \ac{TVD} mit den stark formalisierten deutschen Abläufen ist, muss ein idealtypisches Phasenmodell konstruiert werden. Dieses Modell dient nicht als neuer Standardvorschlag für die Praxis. Es fungiert vielmehr als analytisches Hilfsmittel, um Unterschiede und Gemeinsamkeiten messbar zu machen.

Für die weitere Untersuchung wird ein Modell mit vier Phasen zugrunde gelegt. Es verbindet das Prozess-Schema nach Ballard\,\autocite[vgl.]{ballard_target_2012} mit dem Phasenmodell des \acl{LCI}.\autocite[vgl.]{hill_target_2017} Diese Gliederung weist zudem deutliche Parallelen zum deutschen Modell der Integrierten Projektabwicklung (IPA) auf. Ein wesentlicher Unterschied liegt jedoch in der Gewichtung der Validierung: Während das hier genutzte TVD-Modell der Validierung aufgrund ihrer zentralen Steuerungsfunktion eine eigenständige Phase widmet (Phase 1), verortet das IPA-Modell diesen Prozessschritt integrativ zu Beginn der Planungsphase. Die zugrundeliegende Logik, dass keine Realisierung ohne gesicherte Zielkosten erfolgt, ist beiden Ansätzen jedoch gemein.\autocite[vgl. S.8]{haghsheno_ipa-report_2025}

Während das Modell des Lean Construction Institute die Phasen Design (Planung) und Construction (Ausführung) häufig unter dem Begriff \enquote{Value Delivery} zusammenfasst\autocite[vgl.]{hill_target_2017}, nimmt das vorgeschlagene Modell eine bewusste Trennung zwischen Phase 2 (Design) und Phase 3 (Realisierung) vor. Diese Unterscheidung erfolgt aus methodischen Gründen. Sie ermöglicht im späteren Transferkapitel einen präzisen Vergleich mit der in Deutschland durch \acs{VOB} und \ac{HOAI} verankerten Trennung von Planung und Ausführung (vgl. Abbildung \ref{fig:phasenvergleich}). Gleichzeitig bleibt die für \ac{TVD} essenzielle Phase der Validierung als eigenständiger Abschnitt erhalten.

Das zugrunde gelegte Modell gliedert sich demnach wie folgt:
\begin{itemize}[leftmargin=2em]
    \item \textbf{Phase 0}: Projektinitiierung (Project Definition)
    \item \textbf{Phase 1}: Validierung und Zielsetzung (Validation)
    \item \textbf{Phase 2}: Wertorientierte Planung (Design to Targets)
    \item \textbf{Phase 3}: Realisierung und Abschluss (Project Delivery)
\end{itemize}

\begin{figure}[htbp]
    \centering
     \fcolorbox{gray!50}{white}{%
    \includegraphics[width=\dimexpr\textwidth-2\fboxsep-2\fboxrule\relax]{05_figures/HOAI vs. TVD Phasen Vergleich .jpg}%
    }
    
    \caption[Vergleich HOAI und TVD]{Vergleich der HOAI-Leistungsphasen mit dem idealtypischen TVD-Phasenmodell (Quelle: Eigene Darstellung).}
    \label{fig:phasenvergleich}
\end{figure}

\clearpage

\section{Phase 0 - Projektdefinition und Wertbestimmung (Project Definition)}
\label{sec:4.2}

Die Phase der Projektdefinition (Phase 0) bildet das fundamentale Ausgangsniveau des Target Value Design. Anders als in konventionellen Planungsabläufen, bei denen der Entwurf oft frühzeitig auf physische Lösungsansätze fokussiert, konzentriert sich Phase 0 ausschließlich auf die Definition des \enquote{Warum} (Business Case) und des finanziellen \enquote{Wie viel} (Allowable Cost). 

Ziel dieser Phase ist es, den Wert aus Sicht des Bauherrn so präzise zu definieren, dass er in den nachfolgenden Phasen als harter Steuerungsmaßstab dienen kann. Design-Lösungen oder spezifische architektonische Ausformulierungen spielen zu diesem Zeitpunkt noch keine Rolle. Die Phase ist durch eine klare Dominanz der Auftraggeber-Sphäre geprägt: Der Bauherr muss, ggf. unterstützt durch Analysten, festlegen, welchen Nutzen das Projekt stiften soll und welches Investment dafür wirtschaftlich tragbar ist, noch bevor das eigentliche Planungsteam konstituiert wird \autocite[vgl.]{ballard_target_2025}.

Der Ablauf gliedert sich dabei in drei wesentliche Schritte: die Formulierung eines belastbaren Wertversprechens, die Ableitung des finanziellen Rahmens sowie einen initialen Marktabgleich zur Verifizierung der generellen Machbarkeit.

\subsection{Definition des Wertversprechens (Business Case \& Value)}
\label{sec:4.2.1}
Am Ausgangspunkt dieser Phase steht der Business Case, dessen Entstehung abhängig von den Interessen und Zielen des Bauherren stark variieren kann \autocite[vgl.]{hill_target_2017}. Während in der klassischen Bedarfsplanung (z.B. nach DIN 18205) häufig das \enquote{Raumprogramm} (Output) im Vordergrund steht, fordert der TVD-Ansatz eine Fokussierung auf den \enquote{Nutzen} (Outcome). Es gilt, das reine Leistungsprogramm um die Dimension der Nutzerwerte zu erweitern.

Eine besondere Herausforderung in der Definition des Wertversprechens liegt in der Differenzierung der Interessengruppen. Nach den Prinzipien des Lean Construction ist streng zwischen dem zahlenden Kunden (Paying Customer) und dem tatsächlichen Nutzer (End User) zu unterscheiden. Oft decken sich deren Ziele nicht vollständig (z.B. Minimierung der Investitionskosten vs. Optimierung der Arbeitsabläufe). Eine erfolgreiche Phase 0 muss daher zwingend die ‚Voice of the Customer‘ durch Nutzerworkshops oder Bedarfsanalysen integrieren. Werden die Endnutzerbedürfnisse in dieser frühen Phase ignoriert, führt dies in späteren Phasen häufig zu kostspieligen Änderungswünschen, die das Target Value Design destabilisieren können.\autocite[vgl.]{zimina_target_2012}

Diese Anforderungen werden als \textbf{Conditions of Satisfaction (CoS)} definiert. Sie beschreiben die Bedingungen, unter denen der Bauherr das Projekt als Erfolg wertet. Um späteren \enquote{Design Churn} (unnötige Planungsänderungen) zu vermeiden, ist hierbei eine strikte Priorisierung notwendig \autocite[vgl.]{zimina_target_2012}:
\begin{itemize}
    \item \textbf{Needs (Muss-Kriterien):} Diese Aspekte sind unverzichtbar für die Erfüllung des Business Case (z.B. Patientensicherheit im Krankenhaus, gesetzliche Vorgaben). Ein Nichterfüllen führt zum Scheitern des Projekts.
    \item \textbf{Wants (Wunsch-Kriterien):} Diese Aspekte steigern den Wert, sind jedoch verhandelbar (z.B. spezifische Ausstattungsqualitäten), falls das Budget dies erfordert.
\end{itemize}
Diese Unterscheidung bildet die Grundlage für spätere Value-Engineering-Entscheidungen im Designprozess.

\subsection{Ermittlung des finanziellen Rahmens (AC vs. EC)}
\label{sec:4.2.2}
Parallel zur Wertdefinition erfolgt die Festlegung der finanziellen Eckpfeiler. Hierbei prallen im TVD-Prozess zwei methodisch getrennte Sichtweisen aufeinander: die finanzielle Tragfähigkeit des Bauherrn und die Realität des Baumarktes.

\textbf{1. Allowable Cost (AC): Die Sicht des Bauherrn}\\
Die Allowable Cost repräsentieren den Betrag, den der Bauherr maximal investieren \textit{darf}, damit sein Business Case (z.B. Renditeerwartung, Refinanzierung) positiv bleibt. Diese Größe wird \enquote{top-down} ermittelt und ist explizit \textit{keine} Kostenschätzung des Gebäudes, sondern eine betriebswirtschaftliche Setzung \autocite[vgl.]{ballard_target_2012}.

\textbf{2. Estimated Cost (EC): Die Sicht des Marktes}\\
Um die Realisierbarkeit der Anforderungen zu prüfen, wird den AC eine Marktprognose gegenübergestellt (Estimated Cost). Da in Phase 0 noch kein detaillierter Entwurf vorliegt, kommen hierbei stochastische oder empirische Methoden zum Einsatz. Nach Ballard et al. (2025) sind zwei Ansätze zu unterscheiden:
\begin{itemize}
    \item \textbf{Benchmarking (Historische Daten):} Bei Standard-Typologien liefern Datenbanken aus abgewickelten Projekten Referenzwerte. Ballard et al. warnen jedoch davor, dass historische Daten oft die Ineffizienzen (Waste) vergangener Projekte beinhalten und daher eher als konservative Obergrenze denn als absolute Wahrheit zu betrachten sind.
    \item \textbf{Parametrische Simulation:} Bei komplexen Projekten mit hohem Innovationsgrad erzielen modellbasierte Simulationen oft präzisere Ergebnisse als reine Kennzahlenvergleiche \autocite[vgl.]{ballard_target_2025}.
\end{itemize}

Ein kritischer Erfolgsfaktor bei der Ermittlung der Estimated Cost ist die Qualität der zugrunde liegenden Daten. Ballard et al. (2025) weisen darauf hin, dass reines Benchmarking auf Basis historischer Kostendatenbanken (in Deutschland z.B. BKI) Risiken birgt. Da diese Daten den Durchschnitt vergangener – oft ineffizient abgewickelter – Projekte abbilden, besteht die Gefahr, dass Verschwendung (Waste) in die eigene Kostenschätzung ‚importiert‘ wird. Die Estimated Costs sollten daher nicht als unumstößliche Wahrheit, sondern als konservative Obergrenze betrachtet werden, die es durch Lean-Methoden zu unterbieten gilt. \autocite[vgl.]{ballard_target_2025}

\subsection{Initiale Machbarkeitsanalyse (The Gap)}
\label{sec:4.2.3}
Der Abschluss der Phase 0 bildet das erste entscheidende \enquote{Gate} im TVD-Prozess. Hierbei werden die Allowable Cost (AC) und die Estimated Costs (EC) in einer Gap-Analyse gegenübergestellt ($\Delta = EC - AC$). Das Ergebnis bestimmt das weitere Vorgehen:

\begin{itemize}
    \item \textbf{Szenario A ($EC \gg AC$):} Die Marktkosten übersteigen das Budget massiv (z.B. > 30\%). Das Projekt ist in der aktuellen Form unrealistisch. Es folgt die Entscheidung \enquote{Revise or Stop}: Entweder müssen die Anforderungen (CoS) drastisch reduziert oder der Business Case verworfen werden.
    \item \textbf{Szenario B ($EC > AC$):} Die Kosten liegen moderat über dem Budget (z.B. 5--15\%). Dies ist der \enquote{Idealfall} für den Start eines TVD-Projekts. Die Lücke definiert das \enquote{Stretch Goal}, das durch Innovation und Effizienzsteigerung im folgenden Designprozess geschlossen werden soll.
    \item \textbf{Szenario C ($EC \leq AC$):} Der Marktpreis liegt unter dem Budget. In diesem Fall sollten die Allowable Costs nach unten korrigiert werden, um unnötige Ausgaben (Verschwendung) zu vermeiden.
\end{itemize}

Es ist hervorzuheben, dass eine moderate Lücke (Szenario B) im TVD nicht als Planungsfehler, sondern als strategischer Treiber verstanden wird. Diese Differenz fungiert als \enquote{Stretch Goal} (ehrgeiziges Ziel). Sie erzeugt einen konstruktiven Innovationsdruck, der das spätere Planungsteam dazu zwingt, verlassene Pfade des \enquote{Business as Usual} zu verlassen und proaktiv nach effizienteren Lösungen zu suchen. Ohne dieses Spannungsfeld tendieren Projekte dazu, lediglich bestehende Standards zu reproduzieren.
Zudem bildet dieses \enquote{Gap} die ökonomische Basis für das in Phase 1 und 2 folgende Inzentivierungsmodell (\textit{Pain/Gain-Share}). Nur wenn eine Lücke existiert, die durch Optimierung geschlossen und unterboten werden kann, entsteht der finanzielle Spielraum (\enquote{Pot}), aus dem spätere Boni für das Planungsteam ausgeschüttet werden können.

Erst wenn diese prinzipielle Machbarkeit bestätigt ist, wird das Projekt freigegeben und die Rekrutierung des integrierten Teams eingeleitet.

\subsection{Implementierungshürden in der Projektinitiierung}
\label{sec:4.2.4}

Während der theoretische Prozess der Phase 0 eine rein an den Nutzerbedürfnissen orientierte Wertdefinition und daraus abgeleitete Zielkosten vorsieht, zeigt die Analyse der deutschen Baupraxis signifikante systemische Divergenzen. Die folgenden Aspekte verdeutlichen, warum die Festlegung valider \textit{Allowable Costs} in der Realität oft scheitert.

\minisec{Das \enquote{Nicht-ROI}-Dilemma öffentlicher Auftraggeber}
Die TVD-Methodik leitet die \textit{Allowable Cost} idealerweise aus einem Business Case ab ($Allowable Cost < Expected Revenue - Profit$) \autocite[vgl.]{ballard_target_2011}. Diese ökonomische Logik versagt jedoch bei öffentlichen Bauvorhaben (z.\,B. Schulen, Museen), denen kein monetärer Return on Investment (ROI) gegenübersteht.
Anstelle einer marktbedingten Obergrenze tritt hier oft eine politisch gesetzte Budgetrestriktion (Haushaltsmittel). Dies birgt die Gefahr, dass die \textit{Allowable Cost} nicht funktional hergeleitet, sondern als willkürliches \enquote{Cost-Cap} gesetzt wird \autocite[vgl.]{zimina_target_2012}. Fehlt der ökonomische Korrekturmechanismus des Marktes, droht die Zielkostendefinition von einer wertorientierten Steuerung zu einer reinen Budget-Verwaltung zu degenerieren.

\minisec{Deterministische Haushaltslogik vs. Probabilistische Steuerung}
TVD arbeitet in frühen Phasen definitionsgemäß mit Zielkorridoren und probabilistischen Kostenschätzungen (z.\,B. p85-Werten), um der Unsicherheit des Entwurfs Rechnung zu tragen \autocite[vgl.]{tommelein_ballard_2016}.
Das deutsche Haushaltsrecht (LHO) und die Verwaltungspraxis verlangen hingegen oft bereits vor Planungsbeginn einen deterministischen, skalaren Endwert (\enquote{Punktlandung}) zur Freigabe der Mittel \autocite[vgl.]{kalusche_projektmanagement_2020}. Dieser \enquote{Systembruch} zwingt Projektteams dazu, eine Scheingenauigkeit zu suggerieren, indem ein fixer \enquote{Guaranteed Maximum Price} (GMP) oder ein festes Budget simuliert wird, obwohl die Planungstiefe dies methodisch noch nicht zulässt.

\minisec{Eingeschränkter Lösungsraum durch hohe Regulierungsdichte}
Das TVD-Prinzip setzt voraus, dass bei der Zieldefinition in Phase 0 Variablen wie \textit{Scope} (Umfang) und \textit{Quality} verhandelbar bleiben, um das Kostenziel zu erreichen (Variabilität der Lösungen).
In der deutschen Baupraxis – insbesondere bei öffentlichen Auftraggebern – sind diese Parameter jedoch oft bereits vor Projektstart durch externe Vorgaben weitgehend determiniert. Gesetzliche Standards, Musterraumprogramme (z.\,B. im Schulbau) und strenge technische Richtlinien definieren faktisch eine \enquote{Untergrenze} der Leistung, die nicht unterschritten werden darf.
Für den Bauherrn bedeutet dies, dass der theoretische Lösungsraum für ein echtes \enquote{Design-to-Cost} massiv eingeschränkt ist. Die Kosten ergeben sich hier oft als Resultante aus den nicht-verhandelbaren Vorgaben, anstatt dass die Vorgaben an die erlaubten Kosten angepasst werden können.

\minisec{Das \enquote{Phantom-Nutzer}-Problem und die Abstraktion des Werts}
Oliva et al. (2024) weisen darauf hin, dass die für TVD essenzielle \textit{Value Generation} einen aktiven Nutzer voraussetzt \autocite[vgl.]{oliva_target_2024}. Im deutschen Investorenmodell (Büro/Wohnen) ist dieser Endnutzer zum Zeitpunkt der Zieldefinition oft unbekannt; im öffentlichen Bau wird er häufig durch administrative Stellvertretergremien (z.\,B. Bauämter) ersetzt, die keine direkte operative Entscheidungsgewalt über die Nutzungsprozesse haben \autocite[vgl.]{miron_target_2015}.
Ohne das direkte Feedback des \enquote{echten} Nutzers verkommt die Definition von \enquote{Wert} zu einer theoretischen Annahme der Planer. Dies führt zu dem Risiko, dass Zielkosten zwar eingehalten werden, das gebaute Ergebnis jedoch am tatsächlichen Bedarf vorbei optimiert wird.

\minisec{Das TOTEX-Vakuum: Investitionskostenfokus vs. Lebenszyklus}
Aktuelle TVD-Ansätze fokussieren in der Praxis stark auf die Einhaltung der Erstellungskosten (CAPEX) gemäß DIN 276. Eine ganzheitliche Betrachtung, die auch Betriebskosten (OPEX) und Nachhaltigkeitsziele in den Zielwert integriert (\enquote{Target Life Cycle Costing}), ist methodisch oft nicht verankert \autocite[vgl.]{ouma_target_2025}.
Dies führt in der Zieldefinition zu einem Zielkonflikt: Nachhaltige Investitionen (z.\,B. teurere Fassade, dafür geringere Kühlkosten), die sich erst im Betrieb amortisieren, fallen dem Kostendruck der Erstellungsphase zum Opfer (\enquote{Entfeinerung}), da der Zielwert keine Mechanismen für eine \textit{Total Expenditure} (TOTEX) Betrachtung bereitstellt.

\clearpage



\FloatBarrier
\clearpage
 
\section{Phase 1 - Validierung und Zielkostenbestimmung}
\label{sec:4.3}

Die Phase 1 markiert den operativen Startpunkt des Target Value Design. Während die vorangegangene Phase 0 durch die einseitige Definition der Rahmenbedingungen seitens Bauherrn geprägt war, erfolgt nun der Übergang in die integrierte Zusammenarbeit. Ziel dieser Phase ist es jedoch noch nicht, einen detaillierten Gebäudeentwurf zu erstellen. Vielmehr dient die Validierungsphase als vorgeschaltete technisch-ökonomische Machbarkeitsprüfung (\textit{Feasibility Validation}), welche die in Phase 0 getroffenen Annahmen (Business Case und Allowable Cost) verifiziert \autocite[vgl.]{ballard_target_2012}.

Die Kernaufgabe besteht darin, aus den abstrakten Anforderungen des Bauherrn ein funktionales Lösungskonzept (\textit{Basis of Design}) zu entwickeln und dieses durch Marktpreise zu plausibilisieren. Dieser Prozessschritt verankert das Prinzip der \enquote{Validierung vor der Detaillierung}, wonach bereits vor dem Einsatz signifikanter Ressourcen für die Ausarbeitung von Plänen der Nachweis erbracht werden muss, dass das Projekt für das verfügbare Budget realisierbar ist. Scheitert dieser Nachweis, sieht das TVD-Modell an dieser Stelle eine explizite \enquote{No-Go}-Entscheidung (Projektabbruch) vor \autocite[vgl.]{ballard_target_2012}, um die Gefahr späterer Kostenüberschreitungen und Fehlinvestitionen zu minimieren \autocite[vgl. S. 171]{do_target_2014}. Erst mit dem erfolgreichen Abschluss einer verbindlichen Zielkostenvereinbarung (\textit{Target Cost}) wird das Projekt für die eigentliche Planungsphase (Design to Targets) freigegeben \autocite[vgl.]{ballard_target_2025}.

\subsection{Teamstruktur und Wertdefinition}
\label{sec:4.3.1}

Der operative Beginn der Validierungsphase erfolgt durch die Zusammenstellung des Kernteams. Um die im \ac{TVD} geforderte Kostensicherheit bereits vor der detaillierten Planung zu gewährleisten, ist die frühzeitige Einbindung von Ausführungswissen zwingend erforderlich. Organisatorisch wird dies durch die Bildung von sogenannten \textbf{Clustern} (auch Systemgruppen oder \textit{Cross Functional Teams}) realisiert. Diese interdisziplinären Einheiten setzen sich aus Architekten, Fachplanern und Vertretern der ausführenden Schlüsselgewerke zusammen, wobei die Budgetverantwortung für spezifische Gebäudesysteme direkt auf diese Teams übertragen wird. \autocite[vgl. S.26]{ballard_target_2025}

% Hier sollte noch das wording "Kernteam" verarbeitet werden und betont werden dass dieses aus den Schlüsselgewerken besteht weche wiederrum entwurfsabhängig sind.

Im Gegensatz zur klassischen Projektabwicklung, bei der ausführende Firmen erst nach Abschluss der Planung vertraglich gebunden werden, integriert das TVD-Modell diese Akteure bereits in der Phase 1. Die vertragliche Grundlage hierfür bilden in der Regel vorgelagerte Dienstleistungsverträge (\textit{Preconstruction Services Agreements}), die eine Vergütung der Beratungs- und Kalkulationsleistungen vorsehen, ohne bereits die spätere Bauausführung fest zu beauftragen \autocite[vgl.]{heidemann_kooperative_2011}. Ziel dieser Struktur ist es, Planungs- und Kostenkompetenz in einer Entscheidungseinheit zu bündeln, um sofortiges Kostenfeedback zu ermöglichen.

Inhaltlich beginnt die Zusammenarbeit des Kernteams mit einer gemeinsamen Definition des Werteverständnisses (\textit{Value Definition}. Diese greift die Vorgaben des Bauherren \enquote{Stimme des Kunden} auf und definiert die funktionalen Anforderungen, losgelöst von der physischen Lösung. Essenziell ist dabei der Transfer des Verständnisses vom reinen Bau-Soll hin zum wirtschaftlichen Zweck der Investition (\textit{Business Case}). Das Team muss die strategischen Ziele und Wertvorstellungen des Kunden verstehen, um in der späteren Planung Konflikte autonom lösen zu können \autocite[vgl. S.13]{ballard_target_2025}.

Den Abschluss dieses Initialisierungsprozesses bildet die Überführung der subjektiven Kundenwünsche in objektive \textbf{\ac{CoS}}. Während die Wertvorstellungen oft abstrakt formuliert sind, stellen die CoS konkrete Kriterien dar (z.B. spezifische Anforderungen an Flexibilität, Akustik oder Energieeffizienz), die erfüllt sein müssen, damit das Projekt als erfolgreich betrachtet wird. Sie fungieren als qualitative Steuerungsgröße für alle weiteren Entscheidungen im Planungs- und Realisierungsprozess. Eine Kosteneinsparung, die zur Verletzung der \ac{CoS} führt, wird nicht als Effizienzgewinn, sondern als unzulässige Wertminderung betrachtet \autocite[vgl. S. 120]{ballard_target_2025}.

\subsection{Konzeptentwicklung und Validierung}
\label{sec:4.3.2}

Auf Basis der definierten Wertziele entwickeln die Cluster das sogenannte \textbf{Basis of Design} (BoD). Im Gegensatz zum klassischen Vorentwurf, der oft bereits spezifische architektonische Lösungen fixiert, definiert das BoD primär funktionale Qualitäten und Systementscheidungen, die notwendig sind, um die \textit{Conditions of Satisfaction} zu erfüllen. Es dient als technischer Nachweis, dass die Anforderungen des Kunden innerhalb des Budgets realisierbar sind, ohne bereits eine detaillierte Planung vorwegzunehmen \autocite[vgl.]{lci_guide_2016}\,\autocite[vgl. S.~13]{ballard_target_2025}.

Methodisch folgt dieser Prozess einer streng iterativen Logik. Anstatt Planung und Kalkulation sequenziell zu trennen, arbeiten Architekten und Kalkulatoren in kurzen Zyklen zusammen: Lösungsansätze werden skizziert und unmittelbar einer kostentechnischen Bewertung unterzogen. Dies ermöglicht ein sofortiges Feedback (\textit{Rapid Estimating}), wodurch Fehlentwicklungen frühzeitig erkannt und korrigiert werden können, noch bevor sie sich im Entwurf verfestigen. \autocite[vgl.]{zimina_target_2012}\,\autocite[vgl.]{ballard_target_2012}

Als zentrales Steuerungsinstrument dient hierbei das \textbf{Cost Modeling} (Kostenmodellierung). Da in dieser frühen Phase oft noch keine detaillierten Mengen für eine klassische Kalkulation vorliegen, werden parametrische Kostenmodelle genutzt. Diese basieren auf historischen Daten und identifizieren die wesentlichen Kostentreiber (\textit{Cost Drivers}) des Entwurfs. Das Ziel ist es, die Kostenentwicklung proaktiv zu prognostizieren, anstatt sie nur reaktiv zu erfassen. \autocite[vgl. S.~104]{ballard_target_2025}\,\autocite[vgl.]{ballard_target_2012}

Dabei gilt der fundamentale TVD-Grundsatz, dass die Kosten als unabhängige Entwurfsvariable betrachtet werden: Überschreitet ein technischer Lösungsansatz das anteilige Budget des Clusters, wird nicht das Budget erhöht, sondern die technische Lösung so lange variiert, bis sie kostenkonform ist. Das Design muss sich dem Budget anpassen, nicht umgekehrt. \autocite[vgl.]{zimina_target_2012}\,\autocite[vgl.]{ballard_target_2012}

Dieser Prozess mündet in einem kontinuierlichen \textbf{Validierungs-Check}. Das Team gleicht fortlaufend ab, ob das entwickelte \textit{Basis of Design} sowohl die qualitativen Anforderungen (CoS) erfüllt als auch die ökonomische Grenze der \textit{Allowable Cost} einhält. Nur wenn diese Kongruenz nachgewiesen ist, gilt das Konzept als validiert. \autocite[vgl.]{lci_guide_2016}\,\autocite[vgl. S.~15]{ballard_target_2025} 

\subsection{Budgetfreigabe und Zielkostenfestlegung}
\label{sec:4.3.3}

Die Ergebnisse der Validierungsphase werden abschließend im \textbf{Validierungsbericht} zusammengeführt. Dieses Dokument dient dem Bauherrn als fundierte Entscheidungsgrundlage: Es weist nach, dass das entwickelte \textit{Basis of Design} die funktionalen Anforderungen und \textit{Conditions of Satisfaction} erfüllt und innerhalb des gesteckten Kostenrahmens realisierbar ist. Auf dieser Basis trifft der Bauherr die formale \textbf{Budgetfreigabe} (\textit{Go / No-Go Decision}). Nur bei einem positiven Nachweis der Machbarkeit erfolgt der Startschuss für die nächste Projektphase, andernfalls wird das Projekt abgebrochen oder neu definiert, um Fehlinvestitionen (\textit{Sunk Costs}) zu vermeiden. \autocite[vgl. S. 15]{ballard_target_2025}\, \autocite[vgl.]{ballard_target_2012}

Mit der Projektfreigabe erfolgt die formale Fixierung der Ziele im \textbf{Target Value Statement}. Ein zentraler Aspekt ist dabei die Festlegung der \textbf{Basiszielkosten} (\textit{Target Cost}). Diese leiten sich aus den validierten Marktkosten (\textit{Estimated Cost}) ab, werden jedoch im TVD-Verfahren typischerweise unterhalb dieser Prognose angesetzt. Diese Differenz dient als bewusster Puffer gegen Unvorhergesehenes und schafft einen Innovationsdruck für das Team, die Kosten durch intelligentes Design weiter zu senken \autocite[vgl. S.~10]{ballard_target_2025}\,\autocite[vgl.]{zimina_target_2012}.

Zur Absicherung dieser Verpflichtung wird mit der Festlegung der Zielkosten auch der kommerzielle \textbf{Pain/Gain-Share-Mechanismus} aktiviert. Unterschreitet das Team durch Optimierungen in der Folgephase die Basiszielkosten, wird die Einsparung zwischen Bauherr und Planungsteam geteilt (\textit{Gain}). Überschreiten die Kosten hingegen den Zielwert, haftet das Team bis zu einer definierten Grenze mit seinem Risikopool (\textit{Pain}). Diese vertragliche Kopplung von Projekterfolg und Unternehmensgewinn stellt sicher, dass die Einhaltung der Zielkosten für alle Beteiligten nicht nur eine vertragliche Pflicht, sondern ein ökonomisches Eigeninteresse darstellt. \autocite[vgl.]{zimina_target_2012}\,\autocite[vgl.]{lci_guide_2016}

\subsection{Vergabe- und vertragsrechtliche Implementierungshürden}
\label{sec:4.3.4}

Die erfolgreiche Initialisierung der TVD-Phase 1 setzt die frühzeitige vertragliche Bindung der Schlüsselakteure voraus. In der deutschen Baupraxis trifft diese Anforderung jedoch auf die rigiden Schranken des Vergaberechts und die Systematik des Werkvertragsrechts, was zu folgenden strukturellen Konflikten führt.

\minisec{Das Wettbewerbsparadoxon: Preisfokus vs. Kompetenzauswahl}
TVD erfordert die Auswahl der Partner primär anhand von Qualifikationen, \enquote{Soft Skills} und der Bereitschaft zur Kollaboration. Für öffentliche Auftraggeber, die an das GWB und die VgV gebunden sind, stellt dies eine massive Hürde dar.
Das Vergaberecht fordert einen transparenten Wettbewerb um das \enquote{wirtschaftlichste Angebot} (§ 127 GWB). Da zum Zeitpunkt der TVD-Beauftragung oft noch keine detaillierte Leistungsbeschreibung vorliegt (da diese erst gemeinsam entwickelt werden soll), ist ein reiner Preiswettbewerb methodisch unmöglich. Die rechtssichere Operationalisierung von \enquote{Kooperationsfähigkeit} als Zuschlagskriterium erfordert aufwendige Verfahren (z.\,B. Wettbewerblicher Dialog), die viele öffentliche Auftraggeber aufgrund der Komplexität scheuen \autocite[vgl.]{heidemann_vergaberecht_2024}.

\minisec{Die \enquote{Vorbefasstheits-Falle} beim Early Contractor Involvement}
Die von Granja et al. (2023) geforderte frühzeitige Einbindung ausführender Unternehmen (ECI) als Berater in der frühen Phase birgt im deutschen Vergaberecht das Risiko der \enquote{Projektantenproblematik} (§ 7 VgV).
Ein Unternehmen, das in Phase 0/1 an der Zielkostenfindung mitgewirkt hat, verfügt über einen Wissensvorsprung. Um den Grundsatz der Gleichbehandlung im späteren Wettbewerb um die Bauleistung nicht zu verletzen, droht diesem Unternehmen der Ausschluss vom Verfahren. Dies führt zu einem systemischen Bruch: Das wertvolle Wissen, das in der Beratungsphase aufgebaut wurde, darf nicht nahtlos in die Realisierungsphase überführt werden, was den Kernnutzen von TVD konterkariert.

\minisec{Marktzutrittsbarrieren durch Digitalisierungsanforderungen (KMU-Lücke)}
Effizientes TVD setzt laut Ouma et al. (2025) datenbasierte Entscheidungsprozesse voraus (\enquote{Quantitatively Managed}), was in der Praxis den Einsatz von BIM und modellbasierten Kalkulationsmethoden bedingt.
Diese Anforderung steht im Spannungsfeld zur mittelständischen Struktur der deutschen Bauwirtschaft. Vielen kleineren Handwerksbetrieben (KMU) fehlen die Ressourcen und die digitale Reife für eine Teilnahme an solch hochintegrierten Prozessen. Die strikte Anwendung von TVD-Standards droht somit, lokale und spezialisierte Handwerksbetriebe vom Wettbewerb auszuschließen und eine Marktkonzentration auf Großkonzerne zu fördern, was politischen Zielsetzungen oft widerspricht.

\minisec{Dichotomie von Werkvertragsrecht und Dienstleistungsprozess}
Juristisch betrachtet ist TVD ein Dienstleistungsprozess (\enquote{Tätigwerden zur Erreichung eines Ziels}), während das deutsche Bauvertragsrecht (VOB/B) vom Typus des Werkvertrags geprägt ist, der einen konkreten Erfolg schuldet.
Klassische Verträge honorieren die Erstellung des Bauwerks, bieten aber kaum Mechanismen, um die intellektuelle Leistung der \enquote{Kostenoptimierung} oder \enquote{Variantenuntersuchung} zu vergüten, wenn diese nicht direkt in verbaute Masse mündet. Ohne vertragliche Sonderlösungen (z.\,B. Preconstruction Services Agreements) arbeiten Bauunternehmen in der TVD-Phase faktisch auf eigenes Risiko in der Hoffnung auf den späteren Auftrag, was keine stabile Basis für eine vertrauensvolle Zusammenarbeit darstellt.








% Vertragliche Synchronisation der Interessen

% Eine valide Überprüfung des Business Case durch die ausführenden Unternehmen scheitert laut Ballard in der konventionellen Praxis häufig an der antagonistischen Vertragsstruktur und dem daraus resultierenden Misstrauen ('Adversarial Relationship'). Der Auftraggeber befürchtet hierbei opportunistisches Verhalten der Auftragnehmer.\autocite[vgl. S. 15]{ballard_target_2012}

% In der Validierungsphase des TVD wird dieses Prinzipal-Agent-Problem durch den Mechanismus des Painsharing aufgelöst. Da eine Überschreitung der Kostenziele direkt die Gewinnmargen der Projektpartner reduziert ('Reduced Profit Margin'), werden die kommerziellen Interessen von Auftraggeber und Auftragnehmern synchronisiert ('Alignment of Interests'). Dies stellt sicher, dass die Validierung des Business Case nicht durch Eigeninteressen verzerrt wird, sondern auf einer realistischen Einschätzung der Machbarkeit basiert.\autocite[vgl. S. 15]{ballard_target_2012}















\clearpage

\section{Phase 2 - Wertorientierte Planung}
\label{sec:4.4}

Mit dem Abschluss der Validierungsphase (Phase 1) und der vertraglichen Fixierung der Zielkosten (Target Cost) ändert sich die Prozesslogik fundamental. Während Phase 1 die Machbarkeit prüft, fokussiert sich Phase 2 auf die operative Ausarbeitung einer qualitativen Lösung, die zwingend innerhalb des gesetzten Zielkostenrahmens verbleibt ("Design to Targets").
Der Planungsprozess wandelt sich hierbei von einer linearen Abarbeitung hin zu einem zyklischen, iterativen Steuerungsmodell.

\subsection{Zielsetzung und modelltheoretische Abgrenzung}
\label{sec:4.4.1}
Das primäre Ziel der Phase 2 ist die Überführung der funktionalen Anforderungen (aus Phase 0) in ein realisierungsfähiges Design, das Kostensicherheit bietet. Dieser Zustand wird im TVD als "Definitive Design" bezeichnet.

Es ist an dieser Stelle notwendig, eine differenzierte modelltheoretische Abgrenzung vorzunehmen. In der gelebten TVD-Praxis (insb. USA/IPD) verschwimmen Planung und Ausführung durch "Fast-Tracking" und frühe Fertigung komplett. In der deutschen Baupraxis existiert eine solche Integration zwar ebenfalls – beispielsweise in Generalunternehmer-Modellen oder bei funktionalen Ausschreibungen –, jedoch ist der "Standardfall" des deutschen Ordnungsrahmens durch das Trennungsprinzip geprägt.

Sowohl das Vergaberecht (VOB/A: Trennung von Planung und Bau) als auch das Preisrecht (HOAI: Honorierung abgeschlossener Planungsphasen) suggerieren eine sequenzielle Logik, bei der die Ausführung erst nach Abschluss der Planung beginnt. Selbst in partnerschaftlichen Modellen erfolgt die Einbindung der Bauunternehmen oft erst nach Abschluss der Entwurfsplanung (LPH 3), womit das wesentliche Potenzial der frühen Einflussnahme auf das Designkonzept ("Constructability") bereits eingeschränkt ist.

Für die vorliegende Untersuchung wird daher analytisch an der Trennung zwischen Phase 2 (Design) und Phase 3 (Realisierung) festgehalten. Dies dient dazu, die Reibungspunkte des TVD-Ansatzes mit dem kodifizierten deutschen Standardmodell (Einzelvergabe nach VOB, Planung nach HOAI) präzise herauszuarbeiten. Inhaltlich deckt die Phase 2 dieses Modells somit den gesamten Planungsraum der HOAI-Leistungsphasen 2 (Vorplanung) bis 5 (Ausführungsplanung) ab. Dies markiert die zentrale Verschiebung: Im TVD findet die werkplanerische Detaillierung nicht erst nach der Vergabe statt, sondern wird als integrative Leistung vorgezogen, um die Preissicherheit zu garantieren.


\subsection{Strukturierung des Budgets: Target Cost Allocation}
\label{sec:4.4.2}
Bevor die inhaltliche Planungsarbeit beginnt, muss das in Phase 1 ermittelte Gesamtbudget ("Project Target Cost") in steuerbare Teilbudgets zerlegt werden. Dieser Prozess der "Target Cost Allocation" transformiert das Budget von einer abstrakten Gesamtsumme in konkrete Handlungsanweisungen für die Planungsteams.

Anders als bei einer klassischen Kostenschätzung nach DIN 276 (Gewerke), erfolgt die Allokation im TVD systemorientiert auf die zu bildenden Cluster (vgl. Kap. 4.4.3). Das Budget wird typischerweise in einem Gegenstromverfahren verteilt:
\begin{itemize}
    \item \textbf{Top-Down:} Die Projektsteuerung gibt basierend auf Benchmarks grobe Zielwerte für die Cluster vor (z.B. Zielwert Fassade: 2,5 Mio. €).
    \item \textbf{Bottom-Up:} Die Cluster-Teams prüfen diese Werte auf Plausibilität und melden Rückbedarf an.
\end{itemize}
Ein essenzieller Bestandteil der Allokation ist das "Contingency Management". Das Budget wird nicht zu 100\% auf die Cluster verteilt. Ein Teil verbleibt als zentraler Puffer ("Project Contingency") beim Bauherrn/Steuerungsteam, um unvorhergesehene Risiken abzudecken, ohne die Target Cost zu erhöhen.

\subsection{Organisatorische Struktur: Interdisziplinäre Cluster-Teams}
\label{sec:4.4.3}
Um die Komplexität des Gesamtprojekts handhabbar zu machen, wird die operative Projektorganisation in Phase 2 von einer gewerkespezifischen Trennung auf interdisziplinäre "Cluster-Teams" (Cross-Functional Teams) umgestellt.

\textbf{Zusammensetzung und Rollen}
Ein Cluster (z.B. "Hülle") vereint alle für dieses Teilsystem relevanten Kompetenzen an einem Tisch. Die typische Besetzung umfasst:
\begin{enumerate}
    \item \textbf{Cluster Lead (Planung):} Meist der Architekt oder Fachplaner, der die design-technische Führung übernimmt.
    \item \textbf{Construction Lead (Ausführung):} Ein Vertreter des ausführenden Unternehmens (z.B. Polier oder Projektleiter des Fassadenbauers).
    \item \textbf{Estimator (Kalkulation):} Ein Kostenschätzer, der permanenten Zugriff auf aktuelle Marktpreise hat.
\end{enumerate}

\textbf{Die Rolle der "Makers"}
Ein entscheidendes Novum des TVD-Prozesses ist die Integration der ausführenden Firmen ("Makers") bereits in dieser frühen Designphase. Ihre Rolle wandelt sich vom reinen Empfänger fertiger Pläne zum aktiven "Co-Creator". Sie bringen fertigungstechnisches Wissen ("Constructability") ein, um Lösungen zu finden, die funktional gleichwertig, aber günstiger zu fertigen sind.

> \textit{Transfer-Hinweis:} Diese frühe Integration steht im Konflikt zur konventionellen Vergabepraxis (VOB/A), die eine strikte Trennung von Planung und Ausführung sowie Produktneutralität fordert. (Diskussion in Kap. 5).

\subsection{Prozesslogik: Der zyklische Steuerungsloop (Continuous Estimating)}
\label{sec:4.4.4}
Das methodische Kernstück der Phase 2 ist das "Continuous Estimating". Um den im konventionellen Prozess üblichen "Blindflug" zwischen den Leistungsphasen zu vermeiden, wird der Planungsprozess in kurze, sich wiederholende Zyklen (Sprints) von typischerweise 3 bis 4 Wochen unterteilt.

Ein solcher Zyklus folgt einer festen Logik:
\begin{enumerate}
    \item \textbf{Design \& Innovation (Woche 1-2):} Die Cluster-Teams entwickeln Lösungsansätze. Hierbei wird oft parallel an mehreren Varianten gearbeitet (Set-Based Design), um den Lösungsraum nicht zu früh einzuschränken.
    \item \textbf{Pricing \& Feedback (Woche 3):} Die erarbeiteten Stände werden kalkuliert. Da die Baufirmen Teil des Teams sind, basiert dies auf echten Marktpreisen, nicht auf Datenbank-Mittelwerten. Unterstützt wird dies oft durch BIM-basierte Massenauszüge (Quantity Take-Off).
    \item \textbf{Steuerungs-Gate (Ende Woche 3):} Im zentralen "Big Room Meeting" werden die Kostenstände aller Cluster zusammengeführt.
\end{enumerate}

\textbf{Entscheidung am Gate:}
Das Gate fungiert als harte Qualitätsschranke. Liegt die Prognose eines Clusters über dem zugewiesenen Budget, darf nicht detailliert werden. Das Team muss "zurück ans Reißbrett", um das Design anzupassen (Design to Cost). Nur bei Einhaltung des Budgets erfolgt die Freigabe für den nächsten Detaillierungsgrad.

> \textit{Transfer-Hinweis:} Die HOAI honoriert den Planungserfolg am Phasenende. Iterative Schleifen zur Kostenoptimierung ("Rework") sind im linearen Leistungsbild oft nicht abgebildet und führen zu Honorardiskussionen. (Diskussion in Kap. 5).

\subsection{Entscheidungsfindung und Steuerungsmethodik}
\label{sec:4.4.5}
Damit die iterative Arbeitsweise zielgerichtet verläuft, stützt sich Phase 2 auf zwei spezifische Methoden zur Entscheidungsfindung:

\textbf{Set-Based Design (SBD)}
Anstatt sich intuitiv auf eine einzige Lösung festzulegen ("Point-Based Design") und diese linear auszuarbeiten, halten die Cluster-Teams bewusst mehrere Optionen parallel offen (Sets). Diese werden grob dimensioniert und kalkuliert. Optionen, die sich als technisch oder wirtschaftlich nicht tragfähig erweisen, werden sukzessive eliminiert, bis die optimale Lösung übrig bleibt. Dies vermeidet teure negative Iterationen in späten Phasen.

\textbf{Choosing by Advantages (CBA)}
Muss zwischen technisch gleichwertigen Varianten entschieden werden, kommt die Methode "Choosing by Advantages" zum Einsatz. Entscheidungen basieren dabei nicht auf einer Gewichtung von Faktoren (die subjektiv sein kann), sondern auf der Analyse der konkreten Vorteile ("Advantages") einer Option im Verhältnis zu ihren Kosten. Dies objektiviert den Entscheidungsprozess und macht ihn gegenüber dem Bauherrn dokumentierbar (z.B. mittels A3-Reports).

\subsection{Abschluss der Phase 2: Das Definitive Design}
\label{sec:4.4.6}
Der Abschluss der Phase 2 ist erreicht, wenn das "Definitive Design" vorliegt. Dieser Meilenstein unterscheidet sich qualitativ von einer klassischen Ausführungsplanung.

Das Design ist zu diesem Zeitpunkt nicht nur zeichnerisch dargestellt, sondern bereits "produktionsreif" (Production Ready). Durch die Mitwirkung der "Makers" sind Fertigungsdetails, Montageabläufe und Materialverfügbarkeiten bereits geklärt.
Das entscheidende Kriterium für den Übergang in Phase 3 ist jedoch die Preissicherheit: Das Team gibt ein verbindliches Commitment ab, dass das Design innerhalb der Target Cost realisierbar ist.

\textbf{Das finale Gate (Release for Construction):}
Erst wenn die Prognosekosten (Expected Cost) sicher unterhalb der Allowable Cost liegen, wird das Projekt zur physischen Realisierung freigegeben.

> \textit{Transfer-Hinweis:} Die Synchronisation von Planungsfreigabe und absoluter Preissicherheit ist in Einheitspreisverträgen kaum abzubilden, da Endkosten hier oft erst nach Aufmaß feststehen. (Diskussion in Kap. 5).







\clearpage

\section{Phase 3 - Realisierung und Abschluss}
\label{sec:4.5}


Mit dem Erreichen des Definitive Design und der Freigabe zur Ausführung ("Release for Construction") tritt das Projekt in die dritte und letzte Phase des Target Value Design ein. Im Gegensatz zu konventionellen Projekten, in denen während der Bauphase oft noch massive Umplanungen und Kostensteigerungen (Nachträge) auftreten, verlagert sich der Fokus in Phase 3 rein auf die exekutive Umsetzung des Geplanten.

\subsection{Prozessziel: Abweichungsfreie Umsetzung}
Das primäre Ziel der Phase 3 ist die physische Errichtung des Bauwerks unter strikter Einhaltung der in Phase 2 garantierten Parameter (Kosten, Zeit, Qualität). Da die "Makers" (ausführende Firmen) das Design selbst mitentwickelt haben, entfallen typische Bauablaufstörungen wie Probleme in der praktischen Umsetzung oder fehlende Detailinformationen weitgehend.
Die Planungsarbeit beschränkt sich in dieser Phase auf die Just-in-Time-Lieferung von logistischen Informationen oder minimalen Anpassungen an unvorhergesehene Baustellenbedingungen.

\subsection{Kostensteuerung: Cost Monitoring statt Cost Estimating}
Die Art der Kostensteuerung ändert sich fundamental. Während in Phase 2 noch aktiv kalkuliert und optimiert wurde ("Estimating"), geht es nun um das reine "Monitoring":
\begin{itemize}
    \item \textbf{Soll-Ist-Vergleich:} Die auflaufenden Kosten werden permanent gegen die Target Cost (bzw. die Allowable Cost) gespiegelt.
    \item \textbf{Gewinnsicherung:} Da das Budget fixiert ist, arbeiten die Cluster-Teams nun daran, durch Effizienz auf der Baustelle (z.B. durch Lean Construction Methoden wie Taktplanung) ihren Gewinn zu maximieren. Ein Unterschreiten der Kosten ist nun im direkten Interesse der Ausführenden (Pain/Gain-Sharing).
\end{itemize}

\subsection{Abschluss und Lernen}
Das TVD-Modell endet nicht mit der Schlüsselübergabe, sondern mit einer formalen Feedback-Schleife. Die tatsächlichen Endkosten werden final mit den Target Costs abgeglichen.
\begin{itemize}
    \item \textbf{Incentivierung:} Eventuelle Einsparungen ("Underrun") werden gemäß dem vorab definierten Schlüssel zwischen Bauherr und Risikopool ausgeschüttet.
    \item \textbf{Wissenssicherung:} Abweichungen (sowohl positive als auch negative) werden analysiert, um die Kennzahlen für zukünftige Projekte ("Benchmarks" für Phase 0) zu schärfen.
\end{itemize}

> \textit{Transfer-Hinweis:} In der deutschen VOB-Realität ist Phase 3 oft durch Nachtragsmanagement ("Claims") geprägt. Der TVD-Ansatz, dass der Preis fixiert ist und Nachträge durch die vorherige Integration ausgeschlossen sind, erfordert vertraglich eine völlig andere Risikoverteilung als den klassischen Einheitspreisvertrag. (Diskussion in Kap. 5).

\clearpage
\chapter{Analyse systemischer Spannungsfelder und Implementierungsbarrieren}
\label{ch:5}

Kapitel 4 hat gezeigt, dass sich das Prozessmodell des Target Value Design (TVD) nicht ohne Weiteres auf Deutschland übertragen lässt. Die identifizierten Hürden sind keine Einzelfälle, sondern das Ergebnis eines Systemkonflikts.
Es prallen zwei Denkweisen aufeinander: Auf der einen Seite steht der TVD-Ansatz, der auf Flexibilität, Zusammenarbeit und Wahrscheinlichkeiten setzt. Auf der anderen Seite steht das deutsche System aus VOB, HOAI und Haushaltsrecht, das auf festen Preisen, strikter Trennung und detaillierten Vorgaben beharrt.

Um diese strukturellen Probleme nicht als unübersichtliche Liste von Einzelpunkten abzuarbeiten, werden sie im Folgenden in drei thematische Spannungsfelder (\enquote{Dilemmata}) gebündelt:

\begin{enumerate}
    \item \textbf{Das Beschaffungs-Dilemma (Marktzugang und Recht):}\\
    Wie lassen sich kooperative Teams frühzeitig bilden, wenn das Vergaberecht primär auf Preiswettbewerb statt auf Kompetenz ausgerichtet ist? (Fokus auf Phase 1).

    \item \textbf{Das Wert-Dilemma (Steuerung und Zielsystem):}\\
    Worauf wird optimiert, wenn es keinen ökonomischen ROI gibt und Budgets für Bau und Betrieb getrennt sind? Hier geht es um die Gefahr, dass kurzfristige Kostenoptimierung strukturell gegenüber langfristigen Qualitäts- und Nutzungszielen priorisiert wird (Fokus auf Phase 0 und 2).

    \item \textbf{Das Struktur-Dilemma (Ordnungsrahmen und Prozess):}\\
    Wo liegen die Grenzen der Machbarkeit? Dieses Cluster diskutiert Hürden, die selbst durch moderne Verträge (wie IPA) nicht gelöst werden können, weil sie im starren Genehmigungs- und Baurecht verankert sind (Fokus auf Phase 2 und 3).
\end{enumerate}

Ziel dieses Kapitels ist eine differenzierte Analyse: Welche Barrieren lassen sich durch neue Vertragsmodelle bereits senken und an welchen Stellen steht der gesetzliche Ordnungsrahmen einer echten TVD-Umsetzung weiterhin im Weg?

\section{Cluster 1: Das Beschaffungs-Dilemma}
\label{sec:cluster1}

Einer der erste und oft entscheidenden Stolpersteine bei der Implementierung von Target Value Design im deutschen Bauwesen liegt bereits vor dem eigentlichen Projektstart: in der Formierung des Teams. Während die TVD-Methodik auf der Prämisse basiert, dass die Schlüsselakteure (Planer und ausführende Unternehmen) frühzeitig und gemeinsam an der Lösungsfindung arbeiten, reglementiert das deutsche Vergaberecht den Marktzugang nach einer fundamental anderen Logik.

Dieses Spannungsfeld zwischen der Notwendigkeit einer frühen, kompetenzbasierten Teambindung und den gesetzlichen Anforderungen an Wettbewerb und Gleichbehandlung bildet das erste Diskussions-Cluster. Es wird im Folgenden anhand von drei Dimensionen untersucht: der juristischen Divergenz zwischen Preis- und Qualitätswettbewerb (5.1.1), der verfahrenstechnischen Hürde der Vorbefasstheit bei früher Einbindung (5.1.2) und den strukturellen Ausschlusskriterien für den Mittelstand (5.1.3).

\subsection{Das Transaktionskosten-Dilemma: Verfahrensökonomie und Skalierbarkeit}
\label{sec:5.1.1}

Die erste und grundlegendste Hürde bei der Initiierung eines TVD-Projekts im öffentlichen Sektor ist nicht die rechtliche Unmöglichkeit, sondern eine Diskrepanz zwischen dem methodisch erforderlichen Zeitpunkt der Beauftragung und der ökonomischen Verhältnismäßigkeit der verfügbaren Verfahren.
Das Paradoxon beginnt mit einer zeitlichen und logischen Divergenz: Wie in Abbildung \ref{fig:beschaffungsparadoxon} visualisiert, erfordert die TVD-Methodik zwingend, dass die Schlüsselakteure (Planer und ausführende Unternehmen) \textit{vor} Beginn der detaillierten Planung gebunden sind, um diese gemeinsam zu optimieren. Die Systematik der VOB/A hingegen ist historisch auf die Vergabe einer fertig geplanten Leistung zum Festpreis ausgelegt.

\begin{figure}[ht]
    \centering
    \fcolorbox{gray!50}{white}{%
        \begin{minipage}{\dimexpr\textwidth-2\fboxsep-2\fboxrule\relax}
            \centering
            \begin{tikzpicture}[
                >=Stealth,
                node distance=0cm,
                phase/.style={rectangle, draw=black, minimum height=1cm, text centered, font=\small},
                milestone/.style={circle, draw=red, fill=red!10, thick, minimum size=1.2cm, text width=1.5cm, align=center, font=\bfseries\footnotesize},
                arrow/.style={->, thick},
                scale=1,
                transform shape % Skaliert die Schrift mit
            ]
            % --- ZEITSTRAHL OBEN: TRADITIONELL (VOB) ---
            \node[anchor=west] at (0, 3.5) {\textbf{Traditionell (VOB/A): Sequenziell}};
            
            % Phasen (breitere Boxen für volle Breite)
            \node[phase, minimum width=5.5cm, fill=gray!10] (plan_vob) at (0, 2.2) {Planung (LPH 1-6)};
            \node[phase, minimum width=3.5cm, fill=gray!20, right=0cm of plan_vob] (tender_vob) {Ausschreibung};
            \node[phase, minimum width=5cm, fill=gray!30, right=0cm of tender_vob] (build_vob) {Bauausführung};
            
            % Meilenstein: Später Zuschlag
            \node[milestone, right=8.7cm of plan_vob.west, yshift=0cm] (award_vob) {Vergabe\\(Preis fix)};
            
            % Zeitachse
            \draw[arrow] (-0.5, 1.3) -- (12, 1.3);
            
            % --- ZEITSTRAHL UNTEN: TVD (INTEGRIERT) ---
            \node[anchor=west] at (0, -1.5) {\textbf{Target Value Design (TVD): Integriert}};
            
            % Phasen
            \node[phase, minimum width=2.8cm, fill=blue!10] (target) at (0, -2.8) {Zieldef.};
            \node[phase, minimum width=7.5cm, fill=blue!20, right=0cm of target] (tvd_phase) {TVD-Phase (Gemeinsame Planung)};
            \node[phase, minimum width=3.7cm, fill=blue!30, right=0cm of tvd_phase] (build_tvd) {Bauausführung};
            
            % Meilenstein: Früher Zuschlag
            \node[milestone, draw=blue, fill=blue!10, right=2.5cm of target.west] (award_tvd) {Vergabe\\(Team fix)};
            
            % Zeitachse
            \draw[arrow] (-0.5, -3.7) -- (12, -3.7);
            
            % --- KONFLIKT-VISUALISIERUNG ---
            % Verbindungslinie / Blitz
            \draw[<->, dashed, red, thick] (award_tvd.north) -- (award_vob.south) 
                node[midway, fill=white, text=red, font=\footnotesize\bfseries, align=center, draw=red, rounded corners] 
                {Das Beschaffungs-\\Paradoxon};
            
            \end{tikzpicture}
        \end{minipage}%
    }
    \caption{Divergenz der Vergabezeitpunkte: VOB-Modell vs. TVD-Modell}
    \label{fig:beschaffungsparadoxon}
\end{figure}

\minisec{Potenzielle Lösungsräume im Vergaberecht}
Betrachtet man den Rechtsrahmen isoliert, so scheint dieses Problem lösbar. Das Gesetz gegen Wettbewerbsbeschränkungen (§ 127 Abs. 1 GWB) erlaubt ausdrücklich die Berücksichtigung qualitativer Zuschlagskriterien neben dem Preis. Um die komplexe Auswahl eines TVD-Teams rechtssicher abzubilden, stellt der Gesetzgeber spezifische Verfahrensarten jenseits der starren \textit{Öffentlichen Ausschreibung} zur Verfügung.\autocite[vgl. S. 101]{noauthor_gesetz_2013}
Insbesondere das \textit{Verhandlungsverfahren mit Teilnahmewettbewerb} (§ 17 VgV) oder der noch komplexere \textit{Wettbewerbliche Dialog} (§ 18 VgV) bieten den rechtlichen Rahmen, um statt eines festen Preises (für eine noch unbekannte Lösung) die Kompetenz, die Methodenerfahrung und die Kooperationsbereitschaft der Bieter zu bewerten, bevor ein Vertrag geschlossen wird.\autocite[vgl. S. 14ff]{noauthor_verordnung_2016}

\minisec{Die Hürde der Transaktionskosten (Disproportionalität)}
Dass diese Verfahren in der Breite der öffentlichen Bauprojekte dennoch kaum Anwendung finden, ist auf ihre mangelnde ökonomische Effizienz bei Standardprojekten zurückzuführen. Wie Becker und Friedinger (2024) in ihrer Analyse zur Adaption von IPA-Modellen darlegen, verursachen diese komplexen Verfahrensarten massive \textbf{Transaktionskosten}, die weit über den Aufwand einer klassischen Submission hinausgehen \autocite{becker_adaptionen_2024}.
Die Kostentreiber sind hierbei vielschichtig:
\begin{itemize}
    \item \textbf{Juristischer Vorbereitungsaufwand:} Die Erstellung diskriminierungsfreier Eignungsmatrizen für \enquote{weiche} Kriterien (z.\,B. Bewertung eines fiktiven Workshops) erfordert spezialisierte juristische Beratung, um das Risiko von Rügen unterlegener Bieter zu minimieren.
    \item \textbf{Ressourcenbindung:} Mehrstufige Verhandlungsrunden binden über Monate hinweg hochqualifiziertes Personal in den Vergabestellen, das im operativen Tagesgeschäft fehlt.
    \item \textbf{Verfahrensdauer:} Die langen Fristen verzögern den Projektstart signifikant, was wiederum Finanzierungskosten treiben kann.
\end{itemize}

\minisec{Die Entstehung einer \enquote{Zwei-Klassen-Projektlandschaft}}
Für die öffentliche Hand ergibt sich daraus ein Konflikt mit dem haushaltsrechtlichen Grundsatz der Wirtschaftlichkeit (§ 7 BHO). Die Investition in das Vergabeverfahren muss in einem angemessenen Verhältnis zum Projektwert stehen.
Für Großprojekte (\enquote{Leuchttürme} > 50 Mio. €) amortisiert sich dieser Aufwand durch die späteren Projektoptimierungen. Für kleine und mittlere Projekte (Volumen < 10--15 Mio. €), die den Großteil der kommunalen Bauaufgaben (Schulen, Kitas, Wohnungsbau) ausmachen, ist dieser Aufwand jedoch unverhältnismäßig (\enquote{Over-Engineering} der Vergabe).
Becker und Friedinger (2024) konstatieren daher, dass die Komplexität des Vergaberechts faktisch als \enquote{Eintrittsbarriere} wirkt: TVD wird zu einer Methode für Eliten-Projekte, während die Breite der öffentlichen Infrastruktur aus Kostengründen in die VOB/A-Standardvergabe gezwungen wird, die eine echte frühe Einbindung strukturell verhindert \autocite{becker_adaptionen_2024}.

\minisec{Zwischenfazit: Die Verfahrenslücke}
Es lässt sich festhalten, dass das deutsche Vergabesystem zwar theoretisch Instrumente für TVD bereitstellt, diese aber für den Projektalltag \enquote{überdimensioniert} sind. Es fehlt ein niederschwelliger, rechtssicherer Prozess für die frühe Teambindung bei Standardprojekten.
Diese Lücke führt in der Praxis zu Ausweichbewegungen: Auftraggeber versuchen, die Komplexität zu reduzieren, indem sie vereinfachte Vertragsmodelle (\enquote{IPA-Light}) nutzen oder die Einbindung der Partner zeitlich nach hinten verschieben. Dass diese vermeintlichen Abkürzungen jedoch neue, methodische Risiken bergen, ist Gegenstand der folgenden Analyse.

\subsection{Methodische Defizite durch späte Einbindung (\enquote{IPA-Light})}
\label{sec:5.1.2}

Als Reaktion auf die in Abschnitt \ref{sec:5.1.1} dargelegten hohen Transaktionskosten etablierter Vergabeverfahren werden in der Fachliteratur und Praxis zunehmend Anpassungsstrategien diskutiert, die unter dem Begriff \enquote{IPA-Light} oder \enquote{vertragliche Adaptionen} subsumiert werden. Becker und Friedinger (2024) untersuchen hierbei Ansätze, wie Elemente der Integrierten Projektabwicklung in den Rechtsrahmen der VOB/A eingebettet werden können, ohne die volle Komplexität eines Allianzvertrages auszulösen \autocite{becker_adaptionen_2024}.

\minisec{Der Rückzug auf die späte Einbindung}
Der zentrale Hebel zur Reduktion der Verfahrenskomplexität besteht in diesen Modellen häufig darin, die vergaberechtlichen Risiken zu minimieren. Um dem administrativen Aufwand einer qualitativen Auswahl für eine sehr frühe Einbindung zu entgehen, tendieren öffentliche Auftraggeber dazu, die Einbindung der ausführenden Unternehmen zeitlich nach hinten zu verschieben \autocite[vgl. Diskussion der Adaptionsmodelle bei]{becker_adaptionen_2024}.

Treibender Faktor ist hierbei die Rechtsunsicherheit bezüglich der \textbf{Vorbefasstheit} gemäß § 7 VgV. Ein Unternehmen, das bereits in Leistungsphase 2 oder 3 beratend tätig ist, erlangt zwangsläufig einen Informationsvorsprung. Um dieses Unternehmen später rechtssicher mit der Bauausführung zu beauftragen, wären komplexe Ausgleichsmaßnahmen nötig, um den Wettbewerb nicht zu verzerren \autocite[vgl. § 7 VgV Rn. 20 ff.]{leinemann_vergabe_2021}.

Die \enquote{Light}-Lösung der Praxis besteht daher oft darin, diese Hürde zu umgehen, indem der Unternehmer erst zur Leistungsphase 5 (Ausführungsplanung) oder nach Abschluss der Genehmigungsplanung eingebunden wird. Die Logik lautet: Wer erst kommt, wenn die Planung steht, gilt nicht als vorbefasst \autocite[vgl. zur Kritik an späten Einbindungsmodellen]{breyer_alternative_2021}.

\minisec{Methodischer Widerspruch zum Design Freeze}
Diese Strategie der juristischen Risikovermeidung führt jedoch zu einer methodischen Entkernung des Target Value Design. Wie die theoretische Herleitung in Kapitel 4 gezeigt hat, basiert TVD essenziell auf der Einflussnahme auf die Kostentreiber \textit{während} der Entwurfsfindung (LPH 2--3). Granja et al. (2023) betonen, dass erfolgreiches TVD ein Verschmelzen der Grenzen zwischen Planung und Bau erfordert (\enquote{Boundary Spanning}) \autocite{granja_target_2023}.

Erfolgt die Einbindung im Rahmen eines vereinfachten Modells erst in LPH 5, sind die wesentlichen Weichenstellungen – wie Kubatur, Tragwerkskonzept oder Fassadensystem – bereits fixiert (\enquote{Design Freeze}). Der Unternehmer kann in dieser späten Phase nur noch die Ausführungslogistik optimieren oder Alternativangebote für Details unterbreiten, aber keine echten \textit{Design-to-Cost}-Entscheidungen mehr treffen, ohne die vorangegangene Planung und Genehmigung in Frage zu stellen.
Die Analyse zeigt somit, dass der Versuch, TVD durch \enquote{Light}-Modelle kompatibel zur VOB/A zu machen, oft das Kernmerkmal der Methode opfert: Das \textit{Early Contractor Involvement} (ECI) verkommt zu einem \textit{Late Contractor Involvement}, das zwar rechtssicher und kostengünstig zu vergeben ist, aber das Innovationspotenzial der Methode verfehlt.

\minisec{Zwischenfazit}
Es besteht ein direkter Zielkonflikt zwischen Verfahrensökonomie und methodischer Integrität. Die aktuellen Ansätze zur Vereinfachung (IPA-Light) lösen zwar das Kostenproblem der Vergabe (vgl. 5.1.1), schaffen dabei aber ein methodisches Vakuum, da sie die rechtzeitige Integration des Know-hows verhindern. Solange keine Lösung existiert, die sowohl einen geringen Vergabeaufwand als auch eine frühe Einbindung ermöglicht, bleiben adaptierte Modelle oft ein fauler Kompromiss. Neben diesen verfahrenstechnischen Hürden existiert jedoch noch eine weitere, strukturelle Barriere, die den Marktzugang erschwert: die mangelnde Investitionsfähigkeit der kleinteiligen Anbieterstruktur.

\subsection{Marktzutrittsbarrieren durch Struktur und Digitalisierungsgrad}
\label{sec:5.1.3}

Neben den vergaberechtlichen Hürden offenbart die Analyse eine tiefgreifende strukturelle Diskrepanz zwischen den Anforderungen der Methodik und der Realität der deutschen Anbieterstruktur. Target Value Design ist, wie Ouma et al. (2025) im Reifegradmodell darlegen, in seiner voll entwickelten Form ein datengetriebener Prozess (\enquote{Quantitatively Managed}). Er setzt voraus, dass Kostenkennwerte modellbasiert (BIM), transparent (Open Book) und in Echtzeit geteilt werden \autocite{ouma_target_2025}.

\minisec{Die digitale Kluft im Mittelstand}
Die deutsche Bauwirtschaft ist hingegen kleinteilig organisiert und durch das Handwerk geprägt. Über 90\,\% der Betriebe sind kleine und mittlere Unternehmen (KMU). Aktuelle Marktdaten, wie der \textit{BIM Monitor 2025} von BauInfoConsult, belegen eine signifikante \enquote{digitale Kluft}: Während Planungsbüros die BIM-Methodik zunehmend adaptieren, verfügen ausführende Handwerksbetriebe oft weder über die notwendige Software-Infrastruktur noch über die Prozessreife für eine modellbasierte Zusammenarbeit \autocite{bauinfoconsult_bim_2025}.
Für öffentliche Auftraggeber entsteht hier ein Zielkonflikt mit dem Gebot der Mittelstandsförderung (§ 97 Abs. 4 GWB). Eine konsequente TVD-Ausschreibung, die hohe digitale Kompetenzen (BIM Level 2/3) als Eignungskriterium fordert, wirkt faktisch als Marktzutrittsbarriere für lokale Betriebe.

\minisec{Kultureller Widerstand gegen die zuschlagsfreie Kalkulation}
Ein weiteres Hindernis ist die im deutschen Baumarkt verankerte Kalkulationssystematik. TVD erfordert nicht nur bloße Transparenz, sondern methodisch zwingend eine \enquote{zuschlagsfreie Kalkulation}.
Dies bedeutet, dass die Preise ausschließlich auf den tatsächlichen **Einzelkosten der Teilleistung** und den nachweisbaren Gemeinkosten basieren dürfen. Klassische Aufschläge für Wagnis, Gewinn oder Risikopuffer, die in der VOB-Kalkulation üblicherweise in die Einheitspreise eingerechnet werden, müssen hier explizit ausgeklammert und separat (z.\,B. über den Risikopool) vergütet werden.

Für viele KMU stellt dieser Paradigmenwechsel ein existenzielles Risiko dar. In einem Markt, der traditionell durch Preiskampf geprägt ist, basiert die Marge des Handwerkers oft auf einer internen Mischkalkulation und geschickten Einkaufskonditionen, die als Geschäftsgeheimnis gehütet werden. Die Forderung, diese \enquote{nackten Kosten} offenzulegen, erzeugt die Angst, gläsern zu werden und in zukünftigen Ausschreibungen erpressbar zu sein. Studien des Zentralverbands Deutsches Baugewerbe (ZDB) deuten darauf hin, dass diese Abwehrhaltung gegen die Offenlegung der wahren Kostenstruktur eines der Haupthindernisse für die Verbreitung partnerschaftlicher Modelle im Handwerk ist \autocite[vgl. S. 37]{zdb_geschaftsbericht_2024}.

\minisec{Asymmetrisches Investitionsrisiko}
Schließlich erfordert der Einstieg in TVD erhebliche Vorleistungen (\textit{Upfront Investment}). Um methodisch mitzuwirken, müssen Unternehmen in Schulungen (Lean Construction) und IT investieren. Da die Margen im Bauhauptgewerbe traditionell niedrig sind und, wie PwC (2024) analysiert, im aktuellen Krisenmodus Innovationsbudgets oft als erstes gekürzt werden, können sich oft nur kapitalstarke Großunternehmen dieses \enquote{Risikokapital} leisten. \autocite [vgl. S. 3ff]{berbner_bauindustrie_2024}
Es besteht die Gefahr einer \enquote{Elitisierung}: TVD-Projekte wären nur noch mit großen, überregionalen Generalunternehmern realisierbar. Dies schränkt nicht nur den Wettbewerb ein, sondern widerspricht oft dem politischen Willen kommunaler Auftraggeber, die regionale Wertschöpfung zu stärken.

\subsection{Lösungsansatz: Das zweistufige Bauteam-Modell als Brückentechnologie}
\label{sec:5.1.4}

Die Analyse der Defizite in den Abschnitten \ref{sec:5.1.1} bis \ref{sec:5.1.3} führt zu einer klaren Anforderungsmatrix für einen funktionierenden TVD-Beschaffungsprozess im öffentlichen Standardsegment. Gesucht wird ein Modell, das die Transaktionskosten der Vergabe niedrig hält (Lösung für \ref{sec:5.1.1}), eine Einbindung in Leistungsphase 2/3 ermöglicht, ohne vergaberechtliche \enquote{Vorbefasstheit} zu erzeugen (Lösung für \ref{sec:5.1.2}), und die Eintrittshürden für KMU senkt (Lösung für \ref{sec:5.1.3}).

Als synthesefähiger Lösungsansatz kristallisiert sich hierbei das Modell der **zweistufigen Vergabe**, in der deutschen Praxis oft als \enquote{Bauteam-Modell} bezeichnet, heraus. Konzepte hierzu wurden unter anderem vom Hauptverband der Deutschen Bauindustrie in Kooperation mit dem Deutschen Städtetag entwickelt, was ihre Akzeptanz sowohl auf Markt- als auch auf Auftraggeberseite unterstreicht \autocite{hauptverband_der_deutschen_bauindustrie_ev_partnerschaftliche_2020}.

\minisec{Funktionsweise der Zweistufigkeit}
Der Kern des Modells ist die vertragliche Trennung von Planungs-/Beratungsleistung und Bauausführung bei gleichzeitiger verfahrenstechnischer Koppelung.
In einer **ersten Stufe** schreibt der Auftraggeber lediglich die \textit{Pre-Construction Services} aus. Der Unternehmer wird hierbei nicht für das Bauwerk selbst, sondern zunächst für seine Beratungsleistung (Kalkulation, Optimierung, Baubarkeitsprüfung) in der Planungsphase vergütet.
Der Vertrag enthält jedoch eine **Option** auf die zweite Stufe (die Bauausführung). Diese Option wird nur gezogen, wenn am Ende der Planungsphase (Meilenstein: Design Freeze) das gemeinsam entwickelte Target Value (Zielkosten) eingehalten wird \autocite[vgl.]{hauptverband_der_deutschen_bauindustrie_ev_partnerschaftliche_2020}.

\minisec{Lösung des Transaktionskosten-Dilemmas}
Dieses Modell entschärft das in Abschnitt \ref{sec:5.1.1} beschriebene ökonomische Missverhältnis. Da der fixe Beauftragungsumfang der ersten Stufe finanziell gering ist (reines Beratungshonorar), sinkt das Risiko für beide Seiten. Der Wettbewerb verlagert sich vom reinen Endpreis (der noch unbekannt ist) hin zu den Zuschlagskriterien \enquote{Gemeinkosten}, \enquote{Gewinnmarge} und \enquote{Teamkompetenz}. Dies ist vergaberechtlich deutlich schlanker abzubilden als ein komplexer Allianzvertrag, da die Vertragsgrundlage weiterhin ein klassischer Werkvertrag (VOB/B) sein kann, dem lediglich eine partnerschaftliche Phase vorgeschaltet ist.

\minisec{Vermeidung der Vorbefasstheits-Falle}
Das Modell löst zudem das in Abschnitt \ref{sec:5.1.2} diskutierte Problem der späten Einbindung. Da der Bauunternehmer bereits \textit{durch} das Vergabeverfahren ausgewählt wurde (mit der Option auf Bau), ist er in der Planungsphase kein externer \enquote{Berater} (Projektant), sondern bereits der vertraglich gebundene Partner. Die Vorbefasstheit ist somit kein Störfaktor für einen \textit{späteren} Wettbewerb, da der Wettbewerb bereits \textit{stattgefunden} hat. Dies ermöglicht die methodisch zwingende Einbindung in Leistungsphase 3 (Design-to-Cost), ohne rechtliche Risiken einzugehen \autocite[vgl.]{greb_projektantenproblematik_2024}.

\minisec{Brücke für den Mittelstand}
Schließlich adressiert der Ansatz die in Abschnitt \ref{sec:5.1.3} identifizierten Markthürden. Für KMU ist das Bauteam-Modell attraktiv, da es kein \enquote{Alles-oder-Nichts}-Risiko darstellt. Die Beratungsleistung wird vergütet, auch wenn das Projekt später nicht gebaut wird. Zudem ist die Einstiegshürde geringer als bei IPA: Es wird kein gemeinsames Unternehmen (Mehrparteienvertrag) gegründet, und die \enquote{Open Book}-Anforderung beschränkt sich oft auf die Offenlegung der Nachunternehmer-Angebote in der zweiten Stufe. Dies mildert den kulturellen Widerstand gegen die komplette Offenlegung der eigenen Firmenkalkulation ab \autocite[vgl. S.37]{zdb_geschaftsbericht_2024}.

\minisec{Fazit Cluster 1}
Das zweistufige Bauteam-Modell fungiert somit als \enquote{Enabler} für Target Value Design in der Breite. Es ist der pragmatische Kompromiss, der die methodische Notwendigkeit der frühen Einbindung mit den rechtlichen und ökonomischen Restriktionen der öffentlichen Hand versöhnt. Es transformiert den Prozess von einer \enquote{Black Box}-Vergabe hin zu einer kooperativen Preisentwicklung, ohne die Sicherheitslinien des Vergaberechts zu verlassen.

\clearpage

\section{Cluster 2: Das Wert-Dilemma}
\label{sec:cluster2}

Die Kernfrage des Target Value Design lautet: \enquote{Wie generieren wir den maximalen Wert für den Kunden innerhalb der zulässigen Kosten?}
Diese scheinbar triviale Zielsetzung stößt im deutschen Kontext, insbesondere bei öffentlichen Hochbauprojekten, auf ein fundamentales Definitionsproblem. Während die Kostenseite (Target Cost) durch Haushaltsbudgets hart definiert ist, bleibt der \enquote{Wert} (Target Value) oft diffus. Es entsteht eine Asymmetrie, die dazu führt, dass die Methodik ihre Balance verliert. Dieses Cluster analysiert die Ursachen: das Fehlen eines monetären Business Case (5.2.1), die daraus resultierende Kostenzentrierung zu Lasten von Nutzer und Umwelt (5.2.2) sowie die strukturelle Vernachlässigung der Lebenszykluskosten (5.2.3).

\subsection{Das Vakuum der ökonomischen Sanktionierung und die Erosion der Zielkosten}
\label{sec:5.2.1}

In privatwirtschaftlichen Bauprojekten fungiert der \enquote{Business Case} als zentrales Steuerungs- und Sanktionsinstrument. Investitionsentscheidungen werden an einem erwarteten Return on Investment (ROI) gespiegelt; Kostensteigerungen sind nur dann rational, wenn sie durch höhere Erträge kompensiert werden können. Dieser ökonomische Rückkopplungsmechanismus verleiht Zielkosten eine tatsächlich bindende Funktion.

Im öffentlichen Hochbau, der primär der Daseinsvorsorge dient, existiert ein solcher Mechanismus nicht. Gebäude wie Schulen oder Verwaltungsbauten generieren keine monetäre Rendite, die in direkter Beziehung zu den Herstellungskosten steht. Der \enquote{Wert} öffentlicher Bauprojekte manifestiert sich vielmehr in gesellschaftlichen, funktionalen oder qualitativen Dimensionen, die sich nur begrenzt monetarisieren lassen \autocite[vgl. zur Problematik der Wertdefinition]{miron_target_2015}.

\minisec{Der Intention-Action-Gap}
Diese strukturelle Asymmetrie führt zu einem von Kozuch et al. (2024) beschriebenen \enquote{Intention-Action-Gap}: Zwar besteht auf politischer Ebene häufig ein expliziter Anspruch, qualitative Ziele wie Nachhaltigkeit, Nutzerkomfort oder langfristige Wirtschaftlichkeit zu verfolgen (\textit{Intention}). In der konkreten Vergabe- und Beschaffungspraxis dominiert jedoch der Anschaffungspreis als rechtssicherste und am leichtesten überprüfbare Entscheidungsgröße (\textit{Action}). Der \enquote{Wert} bleibt damit eine normative Zielvorstellung, während das Budget die einzige operative Restriktion darstellt \autocite[vgl.]{kozuch_nachhaltigkeit_2024}.

\minisec{Erosion durch Soft Budget Constraints}
Für Target Value Design ergibt sich daraus ein grundlegendes Spannungsfeld. Das methodische Kernprinzip von TVD – die Optimierung des Entwurfs innerhalb einer verbindlichen Kostenobergrenze – setzt voraus, dass die Zielkosten als nicht verhandelbare Randbedingung akzeptiert werden. In öffentlichen Bauprojekten wird diese Voraussetzung jedoch durch sogenannte \enquote{Soft Budget Constraints} unterlaufen. Historische Großprojekte wie der Flughafen Berlin Brandenburg zeigen, dass Kostenüberschreitungen zwar politische Konsequenzen haben können, jedoch selten zu einem Abbruch oder einer grundlegenden Infragestellung des Projekts führen. Stattdessen erfolgt im Regelfall eine Nachfinanzierung.

In einem solchen Kontext verlieren Zielkosten ihre disziplinierende Wirkung. Sie fungieren nicht länger als harte Steuerungsgröße, sondern werden selbst Teil eines politischen Aushandlungsprozesses. Die Gefahr besteht, dass Target Value Design unter diesen Bedingungen von einem kostensteuernden Entwurfsinstrument zu einem formalisierten, aber wirkungsarmen Verwaltungsprozess degeneriert. Die Innovationskraft von TVD, die gerade aus der Unausweichlichkeit der Kostenrestriktion entsteht, wird dadurch systematisch abgeschwächt.

\subsection{Die Methodenfalle: Kostenzentrierung (Cost-Centricity)}
\label{sec:5.2.2}

Die in Abschnitt \ref{sec:5.2.1} beschriebene Dominanz harter Budgetrestriktionen begünstigt im Projektverlauf eine Dynamik, die Miron et al. (2015) als \enquote{Cost-Centricity} von Target Value Design beschreiben. In ihrer Analyse zeigen die Autoren, dass die praktische Anwendung von TVD bislang primär über erzielte Kosteneinsparungen dokumentiert ist, während Beiträge zur tatsächlichen Wertgenerierung für den Auftraggeber nur unzureichend beschrieben, gemessen oder evaluiert werden \autocite{miron_target_2015}.

Diese Kostenzentrierung ist nicht als Fehlanwendung der Methode zu verstehen, sondern als strukturelle Konsequenz eines asymmetrischen Zielsystems. Während Zielkosten früh festgelegt, kontinuierlich überprüft und institutionell sanktioniert werden, bleibt der angestrebte Zielwert häufig vage, plural (\enquote{values} statt \enquote{value}) und ohne klare Bewertungslogik. In der Terminologie der TFV-Theorie nach Koskela (2000) ist festzustellen, dass TVD die Dimensionen \textit{Transformation} und \textit{Flow} methodisch adressiert, die Dimension \textit{Value} jedoch nicht in vergleichbarer Tiefe operationalisiert \autocite[vgl.]{koskela_theory_2000}.

\minisec{Dominanz der First Costs}
In Abwesenheit expliziter Wertdefinitionen und belastbarer Evaluationsmechanismen orientieren sich Projektteams zwangsläufig an der einzigen objektiv überprüfbaren Referenzgröße: den Herstellungskosten. Die intendierte Maximierung von Wert innerhalb eines Kostenrahmens reduziert sich damit faktisch auf die Minimierung der \enquote{First Costs}. Qualitative Aspekte, deren Nutzen sich erst in der Nutzung oder im Betrieb entfaltet, verlieren an Entscheidungsrelevanz \autocite[vgl.]{russell_smith_sustainable_2015}.

\minisec{Descoping als Symptom}
In der Projektpraxis manifestiert sich diese systemische Kostenzentrierung häufig im \enquote{Descoping}. Um die Zielkosten einzuhalten, werden Leistungsumfänge reduziert oder qualitative Standards abgesenkt. Besonders betroffen sind Nutzerinteressen und Nachhaltigkeitsaspekte, da diese zwar strategisch gewünscht, jedoch weder eindeutig definiert noch messbar in den Steuerungsprozess integriert sind \autocite[vgl.]{kozuch_nachhaltigkeit_2024}.
Damit bestätigt sich die von Miron et al. formulierte zentrale Herausforderung von Target Value Design: Ohne eine gleichwertige Methodik zur Erfassung, Bewertung und Rückkopplung von Wert bleibt die Kostenkontrolle der dominante Steuerungsmechanismus.

\subsection{Die TOTEX-Lücke: Strukturelle Diskrepanz zwischen Investition und Betrieb}
\label{sec:5.2.3}

Während die vorangegangenen Abschnitte Defizite der Wertdefinition und -steuerung im Planungsprozess analysierten, adressiert dieser Abschnitt eine strukturelle Barriere, die in der zeitlichen Organisation öffentlicher Bauprojekte liegt. Ziel des Target Value Design ist theoretisch die Maximierung des Kundenwertes über den gesamten Lebenszyklus eines Bauwerks. In der öffentlichen Baupraxis existiert jedoch eine institutionell verankerte Systemgrenze zwischen der Erstellungsphase (CAPEX) und der Nutzungsphase (OPEX).

\minisec{Dominanz der First Costs}
Russell-Smith et al. (2015) zeigen, dass herkömmliche TVD-Prozesse dazu tendieren, den Fokus auf die sogenannten \enquote{First Costs} zu verengen. Da Projektteams vertraglich, organisatorisch und inzentivierungsseitig primär an das Einhalten des Errichtungsbudgets gebunden sind, werden Entscheidungen systematisch zugunsten kurzfristiger Investitionseinsparungen getroffen. Ohne eine explizite Integration von Life Cycle Assessment (LCA) oder Life Cycle Costing (LCC) optimiert das System damit nicht den Gesamtwert des Bauwerks, sondern lediglich den Zeitpunkt der Schlüsselübergabe \autocite{russell_smith_sustainable_2015}.

Für öffentliche Bauherren bedeutet dies, dass Gebäude zwar formal \enquote{on budget} realisiert werden, jedoch durch geringere Investitionen in langlebige oder energieeffiziente Komponenten langfristig höhere Betriebs- und Instandhaltungskosten verursachen. Der eigentliche öffentliche Nutzen, der sich maßgeblich im Betrieb entfaltet, bleibt unzureichend berücksichtigt.

\minisec{Das Split-Incentive-Problem der Kameralistik}
Im deutschen Kontext wird diese Problematik durch die Logik der Kameralistik und der getrennten Budgetierung zusätzlich verschärft. Investive Ausgaben für den Bau und konsumtive Ausgaben für Betrieb und Instandhaltung werden häufig in unterschiedlichen organisatorischen Einheiten verantwortet. Zimina et al. (2012) beschreiben diese Konstellation als klassisches \enquote{Split-Incentive-Problem}: Die Instanz, die über die Investition entscheidet, profitiert nicht von Einsparungen im Betrieb \autocite[vgl.]{zimina_target_2012}.

\minisec{Rational erzwungene Suboptimierung}
Für das TVD-Team entsteht daraus ein struktureller Zielkonflikt. Investitionen mit höherem Initialaufwand – etwa in energieeffiziente Gebäudetechnik – erhöhen den Total Value des Bauwerks über den Lebenszyklus, gefährden jedoch unmittelbar das Target Cost-Ziel. Da das System ausschließlich die Einhaltung der Baukosten sanktioniert, während Betriebskosten erst Jahre später wirksam werden, wird die Suboptimierung des Gesamtobjekts nicht nur begünstigt, sondern rational erzwungen. Die fehlende TOTEX-Steuerung unterminiert damit einen zentralen Anspruch von Target Value Design im öffentlichen Sektor.

\subsection{Lösungsansatz: Die Operationalisierung des Wertes (Value Management)}
\label{sec:5.2.4}

Zwar sind die folgenden Herausforderungen typisch für jede Anwendung von Target Value Design, doch im öffentlichen Sektor wiegen sie deutlich schwerer. Aus bloßen methodischen Hürden werden hier oft feste strukturelle Barrieren. 
Die Ursache liegt im speziellen System der öffentlichen Hand: Budgets wirken in der Praxis oft weniger verbindlich (\enquote{weiche Budgetgrenzen}) und wirtschaftliche Anreize greifen nicht direkt, da Investitionskosten und späterer Nutzen organisatorisch voneinander entkoppelt sind. \autocite[vgl.]{kozuch_nachhaltigkeit_2024}\, \autocite[vgl.]{zimina_target_2012}

\minisec{Hierarchische Wert-Strukturierung}
Ein möglicher Ansatz, um das Vakuum des fehlenden Business Case zu füllen, besteht in der methodischen \enquote{Härtung der Wertziele}. Miron et al. (2015) schlagen hierfür vor, den Wertbegriff explizit zu strukturieren. Anstatt abstrakter Projektziele (\enquote{Hohe Nutzerzufriedenheit}) müssten konkrete Wert-Attribute (Value Attributes) definiert werden, die messbar oder zumindest eindeutig bewertbar sind (z.\,B. Nachhallzeiten, Reinigungsintervalle oder Flexibilitätsgrade).
Für den deutschen Kontext wäre hierbei eine stärkere Integration der Nutzwertanalyse in den TVD-Prozess denkbar. Neben den \enquote{Target Cost} (Zielkosten) könnte ein \enquote{Target Value Score} (Ziel-Nutzwert) als korrespondierende Steuerungsgröße etabliert werden. Dies würde dem Projektteam ermöglichen, qualitative Anforderungen verbindlicher gegen Kosteneinsparungen abzuwägen. \autocite[vgl.]{miron_target_2015}

\minisec{Reaktivierung der Entscheidungslogik (CBA)}
Um der Falle der Kostenzentrierung (vgl. Abschnitt \ref{sec:5.2.2}) entgegenzuwirken, ist weniger die Einführung neuer Methoden erforderlich als vielmehr die konsequente Anwendung der wertbasierten Entscheidungslogik, die dem Target Value Design bereits inhärent ist.
Insbesondere Verfahren wie\acf{CBA}, die im theoretischen TVD-Ansatz eine zentrale Rolle spielen, werden in der Praxis öffentlicher Bauprojekte häufig nur eingeschränkt oder informell genutzt.
Anstatt Planungsalternativen primär anhand der Investitionskosten zu vergleichen, erlaubt CBA eine explizite Gegenüberstellung von Vorteilen und Mehrkosten. Dadurch wird der öffentliche Auftraggeber gezwungen, seine Präferenzen transparent zu machen, und es wird verhindert, dass qualitative Aspekte reflexartig zugunsten kurzfristiger Kosteneinsparungen reduziert werden \autocite[vgl.]{miron_target_2015}.

\minisec{Integration von Lebenszyklus-Zielen (Sustainable TVD)}
Zur Überbrückung der TOTEX-Lücke (\ref{sec:5.2.3}) ist eine Erweiterung der Zielgrößen notwendig. Russell-Smith et al. (2015) sowie Olender und Rosen (2023) plädieren für ein **Sustainable Target Value Design**, bei dem neben dem Kostenziel ein verbindliches Budget für Lebenszyklusfaktoren (z.\,B. Energiebedarf oder CO\textsubscript{2}-Äquivalente) festgelegt wird.
Dass dies praktisch umsetzbar ist, belegen Silveira und Alves (2018) anhand von Fallstudien: Sie konnten nachweisen, dass TVD-Praktiken signifikant dazu beitragen, strenge Nachhaltigkeitsstandards kostenneutral umzusetzen, sofern sich das Team von der reinen Betrachtung der Anschaffungskosten (\enquote{First Cost}) löst und kollaborative Methoden wie das \textit{Set-Based Design} nutzt.
Für die deutsche Verwaltung würde dies bedeuten, die Wirtschaftlichkeitsuntersuchung nicht als statisches Dokument, sondern als dynamische Steuerungsgröße in den TVD-Prozess zu integrieren. \autocite[vgl.]{russell_smith_sustainable_2015}\, \autocite[vgl.]{olender_rosen_2023}\,\autocite[cgl.]{silveira_tvd_sustainable_2018}

\clearpage

\section{Einordnung: Entschärfte vs. persistente Strukturprobleme}
\label{sec:5.3_intro}

Die Diskussion um Hemmnisse von Target Value Design in Deutschland wird häufig von der Kritik an fragmentierten Vertragsstrukturen dominiert. Die klassische Trennung von Planung und Ausführung sowie die antagonistische Anreizlogik von Einheitspreisverträgen (VOB/B) gelten zurecht als fundamentale Barrieren für die kollaborative TVD-Methodik.
Diese Betrachtung greift jedoch zu kurz, da sie strukturelle Defizite (die durch neue Vertragsmodelle lösbar sind) mit systemischen Grenzen (die im Ordnungsrahmen verankert sind) vermischt.

Mit der zunehmenden Etablierung der Integrierten Projektabwicklung (IPA) im öffentlichen Sektor steht mittlerweile ein vertragliches Instrumentarium zur Verfügung, das viele der klassischen \enquote{TVD-Killer} – etwa die Haftungsproblematik oder das Honorarparadoxon – systematisch adressiert. Dieses Cluster fokussiert daher nicht auf die generellen Inkompatibilitäten des Standard-VOB-Vertrags. Stattdessen analysiert es jene persistenten Spannungsfelder, die selbst bei Einsatz eines idealen Mehrparteienvertrags bestehen bleiben.
Es wird die These vertreten, dass die verbleibende Fragilität von TVD in Deutschland weniger in der Vertragsgestaltung begründet liegt, sondern in der mangelnden Synchronisation zwischen der iterativen Projektmethodik und der linearen Logik des öffentlichen Genehmigungs- und Haushaltsrechts.

\subsection{TVD als implizit IPA-abhängige Methodik}
\label{sec:5.3.1}

In der Literatur werden häufig die Honorarordnung für Architekten und Ingenieure (HOAI) und das haftungsrechtliche Trennungsprinzip als Haupthindernisse für TVD angeführt. Diese Einschätzung ist im Kontext konventioneller Vergaben korrekt, verliert jedoch im Kontext partnerschaftlicher Modelle an Schärfe. Vielmehr muss konstatiert werden: Target Value Design ist in der deutschen Praxis faktisch an die Logik der Integrierten Projektabwicklung gekoppelt.

\minisec{Die Auflösung des Honorarparadoxons}
Das oft zitierte \enquote{Honorarparadoxon} – dass Planer in der HOAI für aufwendige Kostenoptimierungen bestraft werden, da ihr Honorar an die (sinkenden) Baukosten gekoppelt ist – stellt in einem Mehrparteienvertrag (IPA) kein Hindernis mehr dar. In solchen Modellen werden Planungsleistungen üblicherweise \enquote{at cost} (Selbstkosten) erstattet, zuzüglich eines Gewinnaufschlags, der an den Gesamterfolg des Projekts (Target Cost) gebunden ist. Die HOAI wirkt hier nur noch als preisrechtlicher Rahmen für die Mindestsätze, nicht mehr als Fehlanreizsystem. Die ökonomische Motivation des Planers wird durch das IPA-Modell harmonisiert: Er verdient mehr, wenn das Projekt günstiger wird.

\minisec{Funktionale Entschärfung der Haftung}
Ebenso verhält es sich mit der Schnittstellenproblematik. In der klassischen Einzelvergabe führt die iterative Zusammenarbeit (z.\,B. frühe Einbindung von Bauwissen in die Planung) zu diffusen Haftungsrisiken. Im IPA-Kontext wird dieses Risiko durch die \enquote{No-Blame}-Kultur und den vertraglichen Haftungsverzicht (außer bei Vorsatz/grober Fahrlässigkeit) funktional neutralisiert. Risiken werden nicht individuell, sondern kollektiv über den gemeinsamen Risikotopf (Risk Pool) getragen.

\minisec{Strukturelle Voraussetzung statt Schwäche}
Daraus folgt eine wichtige Neubewertung: Die Abhängigkeit von komplexen Vertragsmodellen ist keine \enquote{Schwäche} von TVD, sondern eine strukturelle Voraussetzung. Viele als \enquote{TVD-Probleme} diskutierte Aspekte sind in Wahrheit Probleme einer unvollständigen Anwendung ohne den passenden vertraglichen Container. Für die weitere Analyse dieses Clusters werden diese vertraglich lösbaren Themen daher als \enquote{entschärft} betrachtet. Der Fokus richtet sich nun auf den harten Kern der Probleme, der jenseits der Vertragsfreiheit der Parteien liegt: die regulatorische Prozesslogik.

\subsection{Persistentes Kernproblem: Phasentrennung und fehlendes Fast-Tracking}
\label{sec:5.3.2}

Während vertragliche Barrieren durch Modelle der Integrierten Projektabwicklung (IPA) weitgehend neutralisiert werden können, stößt die Anwendung von Target Value Design in Deutschland an eine härtere, externe Systemgrenze: die regulatorisch verankerte Trennung von Planung, Genehmigung und Ausführung. Dieses Spannungsfeld bleibt auch im optimalen Vertragsumfeld bestehen, da es nicht dem Privatrecht (Vertrag), sondern dem öffentlichen Baurecht (Ordnungsrahmen) entspringt.

Während die Phasentrennung in den vorangegangenen Clustern primär als organisations- und vergaberechtliches Problem der frühen Einbindung diskutiert wurde, wird sie im Folgenden bewusst auf einer anderen Ebene betrachtet: als systemische Inkompatibilität zwischen der iterativen Logik von Target Value Design und der fixierungsorientierten Logik des öffentlichen Genehmigungsrechts.

\minisec{Kollision der Logiken: Iteration vs. Fixierung}
Der methodische Kern von TVD basiert auf Prinzipien wie dem \textit{Set-Based Design} und dem \textit{Concurrent Engineering}. Entscheidungen sollen so lange wie möglich offengehalten werden (\enquote{Last Responsible Moment}), um Optimierungspotenziale zu wahren, während Planung und Ausführung sich zeitlich überlappen.
Diese Logik steht in fundamentalem Widerspruch zur deutschen Genehmigungs-Systematik. Das Baurecht setzt für die Erteilung einer Baugenehmigung eine \enquote{genehmigungsreife}, also in wesentlichen Teilen fixierte Planung voraus.

Das deutsche System erzwingt somit eine frühzeitige Festlegung (Design Freeze) zu einem Zeitpunkt, an dem TVD eigentlich noch Variabilität fordert. Auch ein IPA-Vertrag kann diese externe Restriktion nicht aufheben: Das Bauamt genehmigt keine \enquote{Lösungsräume}, sondern nur definierte Pläne.

\minisec{Die Grenzen des Fast-Trackings}
Besonders deutlich wird dies beim Thema \textit{Fast-Tracking} (Parallelisierung von Planen und Bauen). In anglo-amerikanischen Systemen, aus denen TVD stammt, ist es üblich, mit dem Rohbau zu beginnen, während der Ausbau noch geplant wird. In Deutschland ist dies zwar technisch möglich, aber prozessual risikobehaftet. Die Notwendigkeit einer abgeschlossenen Genehmigungsplanung vor Baubeginn (oder zumindest vor Beginn relevanter Teilabschnitte) wirkt als natürliche Bremse für die radikale Parallelisierung, die TVD zur Kosteneinhaltung oft benötigt.

\subsection{Teilbaugenehmigungen als unzureichender Ersatz für echte Parallelisierung}
\label{sec:5.3.3}

Um das beschriebene Dilemma zu umgehen, greifen Projekte in der Praxis häufig auf das Instrument der Teilbaugenehmigung (§\,74 Musterbauordnung) zurück. Dieses Vorgehen wird oft als Äquivalent zum Fast-Tracking missverstanden, stellt jedoch bei genauerer Analyse nur einen bürokratischen \enquote{Workaround} dar, der die methodischen Stärken von TVD schwächen kann.

\minisec{Erhöhung der Komplexität}
Zwar ermöglichen Teilbaugenehmigungen einen früheren Baubeginn (z.\,B. für Baugrube und Gründung). Sie erkaufen diesen Zeitgewinn jedoch durch eine Fragmentierung des Genehmigungsprozesses und eine Erhöhung der Schnittstellenkomplexität. Anstatt den Prozess zu verschlanken (Lean-Gedanke), wird der administrative Aufwand erhöht.

\minisec{Einschränkung des Lösungsraums}
Schwerwiegender ist der Einfluss auf die TVD-Methodik: Eine Teilbaugenehmigung zwingt das Team, bestimmte Parameter (z.\,B. Statik, Gebäudekubatur) zu einem sehr frühen Zeitpunkt verbindlich einzureichen und damit einzufrieren. Dies unterläuft den Ansatz des \textit{Set-Based Design}. Das Team muss Entscheidungen treffen, um den Verwaltungsakt zu bedienen, nicht weil es der \enquote{Last Responsible Moment} für den Projekterfolg erfordert. Faktisch führt der bürokratische Zwang zur Teilung des Genehmigungsprozesses dazu, dass Flexibilität – die wichtigste Ressource im Target Value Design – vorzeitig aufgegeben wird.


\subsection{Einordnung: Grenze der Vertragslogik}
\label{sec:5.3.4}

Die Analyse dieses Clusters führt zu einer differenzierten Bewertung der Machbarkeit von Target Value Design im öffentlichen Sektor. Die nahe liegende Annahme, dass die Einführung Integrierter Projektabwicklung (IPA) automatisch alle Strukturdefizite beseitigt, muss relativiert werden.

Zwar ist IPA ein notwendiger Enabler, um die internen Anreizsysteme des Projektteams (Vergütung, Haftung, Kooperation) auf die TVD-Logik auszurichten. Es ist jedoch kein hinreichender Enabler, um die externen Rahmenbedingungen zu flexibilisieren. Ein Mehrparteienvertrag ist ein privatrechtliches Instrument, das lediglich das Binnenverhältnis der Akteure regelt. Er besitzt keine juristische Kraft, um regulatorische Vorgaben des öffentlichen Baurechts – wie die Forderung nach einer fixierten Genehmigungsplanung – außer Kraft zu setzen.

Die verbleibende Barriere für eine konsequente Anwendung von TVD liegt somit nicht mehr in der Organisation der Projektbeteiligten, sondern im starren Ordnungsrahmen, der iterative Planungsprozesse institutionell benachteiligt. Solange das Genehmigungsrecht Linearität erzwingt, während die Projektmethodik auf Agilität setzt, bleibt TVD auch im IPA-Modell ein Kompromiss.

\subsection{Fazit Cluster 3: Von der Struktur- zur Systemfrage}
\label{sec:5.3.5}

Zusammenfassend zeigt sich, dass Target Value Design in Deutschland nicht primär an \enquote{weichen} Faktoren wie Kultur oder fehlendem Kooperationswillen scheitert, sondern an harten strukturellen Inkompatibilitäten.
Während klassische Defizite wie die Honorar- und Haftungsproblematik durch neue Vertragsmodelle (IPA) wirksam neutralisiert werden können, bleibt ein zentrales Spannungsfeld persistent: die mangelnde Vereinbarkeit von iterativer Wertentwicklung und sequenzieller Genehmigungslogik.

Die Implikation für die Praxis ist weitreichend: Die Weiterentwicklung von TVD im öffentlichen Bau kann nicht allein durch Pilotprojekte und Kulturwandel gelöst werden. Sie erfordert perspektivisch eine Anpassung der administrativen Prozesse – weg von der prüftechnischen Fixierung auf den Planungsstand, hin zu einer genehmigungsrechtlichen Toleranz für Lösungsräume. Ohne diese Evolution des Ordnungsrahmens bleibt TVD ein leistungsfähiger Motor, der in einem zu engen Chassis läuft.




























\chapter{Fazit und Ausblick}
\label{ch:conclusion}

\section{Zusammenfassung der Ergebnisse}
\label{sec:summary}

\section{Weiterer Forschungsbedarf}
\label{sec:further research}

- Infrastrukturbau

% === NACHSPANN (römische Seitenzahlen fortgesetzt) ===
\backmatter
% Römische Nummerierung manuell fortsetzen
\setcounter{page}{\value{savepage}}
\renewcommand{\thepage}{\Roman{page}}

% Literaturverzeichnis (wird ins Inhaltsverzeichnis aufgenommen)
\printbibliography[title=Literaturverzeichnis]
\addcontentsline{toc}{chapter}{Literaturverzeichnis}



\cleardoublepage

% Eidesstattliche Erklärung als PDF einfügen
% Die Überschrift ist bereits im PDF enthalten
\addcontentsline{toc}{chapter}{Eidesstattliche Erklärung}
\includepdf[pages=-, pagecommand={\thispagestyle{plain}}]{03_backmatter/eidesstattliche_erklaerung_signed.pdf}


\end{document}