
\begin{tikzpicture}[node distance=1.5cm, auto]

% Knoten vertikal anordnen
\node[startevent] (start) {};
\node[task, below=of start] (task1) {Entwicklung\\ Projekt Business-Case};
\node[task, below=of task1] (task2) {Definition des\\Werteverständnis};
\node[gateway, below=of task2] (gateway2) {×};

% Verzweigungen horizontal
\node[task, below left=1.5cm and 2cm of gateway2] (task3) {Ablehnung\\senden};
\node[task, below right=1.5cm and 2cm of gateway2] (task4) {Bestellung\\bestätigen};

\node[endevent, below=of task3] (end1) {};
\node[endevent, below=of task4] (end2) {};

% Label
\node[right=0.3cm of start] {\small Start};
\node[right=0.3cm of end1] {\small Ende};
\node[right=0.3cm of end2] {\small Ende};
\node[right=0.3cm of gateway2] {\small Verfügbar?};

% Gerade Verbindungen
\draw[flow] (start) -- (task1);
\draw[flow] (task1) -- (task2);
\draw[flow] (task2) -- (gateway2);

% RECHTWINKLIGE PFEILE MIT ANKERPUNKTEN (löst das Verspringen!)
% Erst vertikal nach unten, dann horizontal mit Ankerpunkten
\draw[flow] (gateway2.south) -- ++(0,-0.5) -| node[left, pos=0.75] {\small Nein} (task3.north);
\draw[flow] (gateway2.south) -- ++(0,-0.5) -| node[right, pos=0.75] {\small Ja} (task4.north);

% Gerade Pfeile nach unten
\draw[flow] (task3) -- (end1);
\draw[flow] (task4) -- (end2);

\end{tikzpicture}

\FloatBarrier

\clearpage
\begin{tikzpicture}[node distance=1.5cm and 0.5cm, auto]

% Definiere die Breite und Höhe der Lanes
\def\lanewidth{5.15}
\def\laneheight{20}
\def\headerheight{1.2}

% ===== LANE HINTERGRÜNDE =====
% Spalte 1 (Links) - Auftraggeber
\fill[gray!5] (0,0) rectangle (\lanewidth,-\laneheight);
% Spalte 2 (Mitte) - Kernteam
\fill[gray!10] (\lanewidth,0) rectangle (2*\lanewidth,-\laneheight);
% Spalte 3 (Rechts) - Cluster-Teams
\fill[gray!5] (2*\lanewidth,0) rectangle (3*\lanewidth,-\laneheight);

% ===== TRENNLINIEN =====
% Horizontale Linie unter Header
\draw[very thick] (0,0) -- (3*\lanewidth,0);
\draw[very thick] (0,-\headerheight) -- (3*\lanewidth,-\headerheight);
% Vertikale Linien
\draw[very thick] (0,0) -- (0,-\laneheight);
\draw[very thick] (\lanewidth,0) -- (\lanewidth,-\laneheight);
\draw[very thick] (2*\lanewidth,0) -- (2*\lanewidth,-\laneheight);
\draw[very thick] (3*\lanewidth,0) -- (3*\lanewidth,-\laneheight);
% Untere Linie
\draw[very thick] (0,-\laneheight) -- (3*\lanewidth,-\laneheight);

% ===== HEADER BESCHRIFTUNGEN =====
% Spalte 1
\node[align=center, font=\bfseries] at (\lanewidth/2,-0.6) {Bauherr (AG)};

% Spalte 2
\node[align=center, font=\bfseries] at (\lanewidth+\lanewidth/2,-0.6) {Kernteam /\\ Objektplanung};

% Spalte 3
\node[align=center, font=\bfseries] at (2*\lanewidth+\lanewidth/2,-0.6) {Cluster-Teams};

% ===== BPMN ELEMENTE (Beispiel) =====
% Spalte 1 - Auftraggeber
\node[startevent] at (\lanewidth/2,-2) (start) {};
\node[task, below=of start] (task1) {Nutzerwerte & Projektziele definieren};
\node[task, below=of task1] (task2) {Freigabe\\erteilen};

% Spalte 2 - Kernteam
\node[task] at (\lanewidth+\lanewidth/2,-3) (task3) {Planung\\erstellen};
\node[gateway, below=of task3] (gateway1) {×};
\node[task, below=of gateway1] (task4) {Koordination\\durchführen};

% Spalte 3 - Cluster-Teams
\node[task] at (2*\lanewidth+\lanewidth/2,-4.5) (task5) {Umsetzung\\durchführen};
\node[endevent, below=of task5] (end1) {};

% ===== VERBINDUNGEN =====
% Innerhalb Spalte 1
\draw[flow] (start) -- (task1);
\draw[flow] (task1) -- (task2);

% Von Spalte 1 zu Spalte 2 (gestrichelt für Message Flow)
\draw[flow, dashed] (task1.east) -- (task3.west);

% Innerhalb Spalte 2
\draw[flow] (task3) -- (gateway1);
\draw[flow] (gateway1) -- (task4);

% Von Spalte 2 zu Spalte 3
\draw[flow, dashed] (gateway1.east) -| node[above, pos=0.25] {\small Ja} (task5.north);

% Von Spalte 2 zu Spalte 1
\draw[flow, dashed] (task4.west) -- (task2.east);

% Innerhalb Spalte 3
\draw[flow] (task5) -- (end1);

\end{tikzpicture}

\FloatBarrier
\clearpage

% In deinem Dokument:
\begin{tikzpicture}[node distance=1.5cm and 0.5cm, auto]

% Definiere individuelle Breiten für jede Lane
\def\laneonewidth{6}      % Breite Spalte 1 (Auftraggeber)
\def\lanetwowidth{6}      % Breite Spalte 2 (Kernteam)
\def\lanethreewidth{3}    % Breite Spalte 3 (Cluster-Teams)
\def\laneheight{22}       % Höhe der Lanes (verlängert für mehr Tasks)
\def\headerheight{1.2}    % Höhe des Headers

% Berechnete Positionen
\pgfmathsetmacro{\laneoneend}{\laneonewidth}
\pgfmathsetmacro{\lanetwoend}{\laneonewidth+\lanetwowidth}
\pgfmathsetmacro{\lanethreeend}{\laneonewidth+\lanetwowidth+\lanethreewidth}

% ===== LANE HINTERGRÜNDE =====
\fill[gray!5] (0,0) rectangle (\laneoneend,-\laneheight);
\fill[gray!10] (\laneoneend,0) rectangle (\lanetwoend,-\laneheight);
\fill[gray!5] (\lanetwoend,0) rectangle (\lanethreeend,-\laneheight);

% ===== TRENNLINIEN =====
\draw[very thick] (0,0) -- (\lanethreeend,0);
\draw[very thick] (0,-\headerheight) -- (\lanethreeend,-\headerheight);
\draw[very thick] (0,0) -- (0,-\laneheight);
\draw[very thick] (\laneoneend,0) -- (\laneoneend,-\laneheight);
\draw[very thick] (\lanetwoend,0) -- (\lanetwoend,-\laneheight);
\draw[very thick] (\lanethreeend,0) -- (\lanethreeend,-\laneheight);
\draw[very thick] (0,-\laneheight) -- (\lanethreeend,-\laneheight);

% ===== HEADER BESCHRIFTUNGEN =====
\node[align=center, font=\bfseries] at (\laneonewidth/2,-0.6) {Auftraggeber (Bauherr)};
\node[align=center, font=\bfseries] at (\laneoneend+\lanetwowidth/2,-0.6) {Kernteam};
\node[align=center, font=\bfseries] at (\lanetwoend+\lanethreewidth/2,-0.6) {Cluster-\\Teams};

% ===== SPALTE 1: AUFTRAGGEBER =====
% 1. Start-Event
\node[startevent] at (\laneonewidth/2,-2,5) (start) {};
\node[=0.3cm of start, text width=4cm, font=\small] {Projektbedarf festgestellt};

% 2. Task: Nutzerwerte & Projektziele definieren
\node[task, below=1.2cm of start, text width=4cm, align=center] (task1) {Nutzerwerte \&\\Projektziele definieren};

% 3. Task: Business Case & Allowable Cost ermitteln
\node[task, below=1.2cm of task1, text width=4cm, align=center] (task2) {Business Case \&\\Allowable Cost ermitteln};

% 4. Gateway: Business Case positiv?
\node[gateway, below=1.2cm of task2] (gateway1) {×};
\node[right=0.3cm of gateway1, font=\small] {Business Case positiv?};

% Nein → Abbruch
\node[endevent, left=1cm of gateway1] (end_abbruch) {};
\node[below=0.1cm of end_abbruch, font=\small] {Abbruch};

% 5. Task: Kompetenzbasiertes Auswahlverfahren vorbereiten
\node[task, below=1.2cm of gateway1, text width=4cm, align=center] (task3) {Kompetenzbasiertes\\Auswahlverfahren\\vorbereiten};

% 6. Task: Auswahlgespräche führen & Entscheidung
\node[task, below=1.2cm of task3, text width=4cm, align=center] (task4) {Auswahlgespräche führen\\\& Entscheidung};

% ===== SPALTE 2: KERNTEAM =====
% 7. Task: Vertragsunterzeichnung / Onboarding
\node[task, text width=4cm, align=center] at (\laneoneend+\lanetwowidth/2,-15) (task5) {Vertragsunterzeichnung\\/ Onboarding};

% 8. End-Event: Kernteam ist formiert
\node[endevent, below=1.5cm of task5] (end_kernteam) {};

% ===== SPALTE 3: CLUSTER-TEAMS =====

% ===== VERBINDUNGEN =====
% Spalte 1 interne Flows
\draw[flow] (start) -- (task1);
\draw[flow] (task1) -- (task2);
\draw[flow] (task2) -- (gateway1);

% Gateway Verzweigungen
\draw[flow] (gateway1.west) -| node[above, pos=0.15] {\small Nein} (end_abbruch);
\draw[flow] (gateway1.south) -- node[right, pos=0.1] {\small Ja} (task3);
\draw[flow] (task3) -- (task4);

% Von Spalte 1 zu Spalte 2 (gestrichelt = Message Flow)
\draw[flow, dashed] (task4.east) -| node[above, pos=0.15] {\small Beschreibung} (task5.west);

% Spalte 2 interne Flows
\draw[flow] (task5) -- (end_kernteam);

\end{tikzpicture}